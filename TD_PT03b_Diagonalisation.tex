\catcode`@=11\relax

\input LD@Maths@TD.tex

\vglue-10mm\rightline{PT\hfill TD 3 :  Diagonalisation\hfill}
\bigskip
\bigskip

%\tikzstyle{object}=[inner sep=0.5em,text depth=-0.2em,fill=white!20]
\tikzstyle{operator}=[rectangle,rounded corners, draw,text height=1em,text depth=0.2em,inner sep=0.4em,fill=white!20]
%\tikzstyle{fork}=[diamond, draw,text height=1.2em,text depth=0.2em,fill=white!20]
%\tikzstyle{line}=[draw,line width=0.5ex]
%\tikzstyle{thinline}=[draw,line width=0.2ex]

\noindent
\centerline{\font\SvgText=cmr8\relax
\tikzstyle{every node}=[inner sep=2pt]
\tikzpicture
\pgfdeclarelayer{background}
\pgfsetlayers{background,main}
\node [object] (a) {Matrice $A$} ;
\node [operator,right of=a, node distance=2.7cm] (b) {\eightpts$\det(A-\lambda I_n)$} ;
\draw [line] (a)--(b) ; 
\node[object, right of=b,node distance=3cm] (c) {\eightpts Polynome $P_A$} ;
\draw[line,->] (b) -- (c) ;
\node[fork,node distance=2cm,below of=c] (m) {\eightpts$\forall i:m_i=n_i$} ;
\node[below of=m,node distance=2.2cm] (prep) {};
\path (m)-- node[midway] (mo) {oui} (prep) ;
\draw [line] (m)--(mo) ; \draw [line,->] (mo)--(prep);
\node[below of=prep,node distance=0.2cm] (p) {$=$} ;
\node[right of=p] (q) {$P$} ;
\node[above of=q,node distance=0.3cm] (abq) {};
\node[right of=q] (r) {$D$} ;
\node[above of=r,node distance=0.3cm] (abr) {};
\node[operator,node distance=1cm,font=\texttt,below of=r] (t) {\eightpts Inversion} ;
\node[right of=r] (s) {$P^{-1}$} ;
\node[left of=p,label=180:{\qquad}] (pp) {$A$} ;
\node[node distance=0.3cm,left of=pp] (ppp) {} ;
%\draw[->,thinline,draw=red] (a)  .. controls +(down:4cm) and +(left:2cm) .. (ppp.west) ;
\node[node distance=0.03cm,above of=q] (qq) {} ;
\node[node distance=0.02cm,above of=s] (ss) {} ;
\draw[<-,thinline,red] (ss.south east) .. controls +(315:1cm) and +(right:1cm) .. (t);
\draw[->,thinline,red] (qq.south west) .. controls +(225:1cm) and +(left:1cm) .. (t);
\node[node distance=4cm,snake=saw,thinline,left of=m, text width=2cm] (n) {\eightpts Trigonaliser $A$} ;
\path (m)-- node[midway] (mn) {\eightpts non } (n) ;
\draw [line] (m)--(mn) ; \draw [line,->] (mn)--(n);
\node[operator,node distance=2.8cm,right of=c] (d) {\eightpts Factorisation} ;
\draw [line](c)--(d) ;
\node[object,node distance=3cm,right of=d] (f) {\eightpts multiplicit\'e $m_i$ } ;
\draw[->,line] (d) -- (f) ;
%\draw[->,thinline,draw=red] (f.south)  -- (abr.north east) ;
\node[object,node distance=1cm,below of=f] (g) {\eightpts Valeur propre $\lambda_i$} ;
\draw[->,line] (d) -- (g) ;
%\draw[->,thinline, draw=red] (g.south)  -- (abr.north east) ;
\node[object,node distance=2.8cm,right of=m] (k) {\eightpts Dimension $n_i$} ;
\node[operator,node distance=3cm,right of=k,text width=2.4cm,text height=1.8em] (h) {\eightpts\quad R\'esolution de\pn$(A-\lambda_iI_n)X=0$} ;
\node[object,node distance=1.2cm,below of=h] (j) {$\!\!\!$\eightpts Espace propre $E_i\!\!\!$} ;
\draw[->,line] (h) -- (j) ;
\draw [line] (g) -- (h) ;
\draw[->,thinline] (h) -- (k) ;
\draw [line] (k) -- (m) ;
\node[object,node distance=3.5cm,left of=j] (l) {\eightpts Base $B_i$} ;
\draw[->,thinline] (h) -- (l) ;
\draw[->,line] (j) -- (l) ;
\draw[->,line] (l) -- (k) ;
\draw[->,thinline,draw=red] (l.south west) -- (abq.north east) ;
\pgfonlayer{background}
%	\draw [snake=saw,line] (pp.south west) rectangle (s.north east);
	\node [forbidden sign,line width=0.5ex, draw=red,fill=white] at  (mn) {\eightpts \qquad\quad} ;
\endpgfonlayer
\endtikzpicture
}%
\medskip
\centerline{Pour Diagonaliser une matrice diagonalisable}
\medskip

\noindent
{\bf Etape 1 : }Ecrire la relation $P(\lambda)=\det(A-\lambda\hbox{I}_n)$, en dessinant la matrice. \medskip\noindent
{\it Au besoin, chercher des racines \'evidentes $\lambda$ de $P$ (m\'ethode standard) et trouver des relations de d\'ependance lin\'eaire entre les colonnes (m\'ethode Olus), pour ces valeurs de $\lambda$. }
\medskip
\noindent
{\bf Etape 2 : }Calculer et factoriser $P$. En d\'eduire les valeurs propres $\lambda$ et leur multiplicit\'e~$m_\lambda$.\medskip\noindent
{\it M\'ethode Olus : ajouter \`a une colonne une combinaison lin\'eaire des autres colonnes, conform\'ement aux relations de d\'ependances lin\'eaires pr\'ec\'edemment trouv\'ees, pour factoriser plus rapidement. }
\medskip
\noindent
{\bf Etape 3 : }Pour chaque valeur propre $\lambda$, r\'esoudre le syst\`eme $(A-\lambda\hbox{I}_n)X=0$. En d\'eduire une base~$\sc B_\lambda$ (comportant $m_\lambda$ vecteurs) de l'espace vectoriel des solutions. \medskip\noindent
{\it M\'ethode Olus, A chaque relation de d\'ependance entre colonnes pr\'ec\'edemment trouv\'ee, correspond une solution (un vecteur propre non nul)}
\medskip
\noindent
{\bf Etape 4 : } Ecrire la relation $A=PDP^{-1}$ ainsi que la matrice diagonale $D$ dont les coefficients sont les valeurs propres trouv\'ees. \medskip\noindent
{\it Si vous avez des valeurs propres $\lambda$, $\mu$ et $\gamma$ de multiplicit\'e respective $3$, $1$ et $2$, votre matrice $D$ sera }
$$
D=\pmatrix{\lambda&&&&&\cr
&\lambda&&&&\cr
&&\lambda&&&\cr
&&&\mu&&\cr
&&&&\gamma&\cr
&&&&&\gamma}
$$
Ecrire la matrice $P$ dont les colonnes sont les vecteurs des bases $\sc B_ \lambda$, en respectant les contraintes : \pn
{\bf a) chaque vecteur n'apparait qu'une fois dans la matrice $P$. \pn b) Sur une m\^eme colonne de $P$ et de $D$ se trouvent un vecteur propre et sa valeur propre correspondante. }\medskip\noindent
{\it Reprenant l'exemple pr\'ec\'edant, Si vous obtenez $\sc B_\lambda=\{u_1,u_2,u_3\}$, $\sc B_\mu=\{v\}$ et $\sc B_\gamma=\{w_1,v_2\}$, votre matrice $P$ sera la matrice dont les colonnes sont }
$$
P=\pmatrix{u_1,u_2,u_3,v,w_1,w_2}\qquad\hbox{\rm ou \'egalement par exemple }\qquad P=\pmatrix{u_3,u_1,u_2,v,w_2,w_1}
$$
{\bf Etape 5 : }Au besoin, inverser la matrice $P$ pour obtenir la matrice $P^{-1}$. \pn
{\it on \'evitera cette \'etape lorsque cela est possible car elle est lourde en calculs}. 
\goodbreak


\hrule
\centerline{Diagonalisation\strut}
\hrule\medskip

\Exercice{PTfa}%2008
\bigskip
\Exercice{PTams}%2008
\bigskip
%\Exercice{PTamr}%2007
\Exercice{PTlq}%2008
\bigskip
\Exercice{PTfk}%2008
%\Exercice{PTajc}%2007
\bigskip
\Exercice{PTajd}%

\bigskip
\Exercice{PTbt}%
\bigskip


\Exercice{PTaja}%
\bigskip
\Exercice{PTajb}%
\bigskip


\Exercice{PTalv}%
\bigskip
\Exercice{PTfs}%

\bigskip


% 
% \Chapter Indications, Indications.
% 
% \LD@Exo@Indication@Display
% 
% \Chapter Notions, Notions.
% 
% \LD@Exo@Notion@Display

\Chapter Solutions, Solutions.

\LD@Exo@Sol@Display

\bye









