%\def\Variables{MathsVariables}%
\catcode`@=11\relax
\def\Api{Mathematicon@Api}%
%%%% Newif
%\def\@firstofone#1{#1}

\newif\ifexonumber
%%%% Switches
\exonumberfalse
\catcode`@=11\relax
\input LD@Header.tex
\input LD.tex
\input LD@Typesetting.tex
\input LD@Exercices.tex
\input LD@Exercices.tex
\def\LD@Exercice@Display@Code{\eightpts}%
% debug tikz
\let\@firstofone\pgfutil@firstofone
\let\@ifnextchar\pgfutil@ifnextchar

\olspept
\DefineRGBcolor F0F9E3=VLGreen.
\DefineRGBcolor E5F9D1=LGreen.
\DefineRGBcolor DAF9BE=TGreen.
\DefineRGBcolor 5DA93B=Green.
\DefineRGBcolor F6DCCA=VLRed.
\DefineRGBcolor F6D4BD=LRed.
\DefineRGBcolor DAF9BE=TRed.
\DefineRGBcolor B5F9A1=TTRed.
\DefineRGBcolor F6B080=Red.
\DefineRGBcolor F9F5E3=VLOrange.
\DefineRGBcolor F9F5D0=LOrange.
\DefineRGBcolor DAF9BE=TOrange.
\DefineRGBcolor B5F9A1=TTOrange.
\DefineRGBcolor D7A93B=Orange.
\DefineRGBcolor EEEEEE=VLBlack.
\DefineRGBcolor DDDDDD=LBlack.
\DefineRGBcolor CCCCCC=TBlack.
\DefineRGBcolor B5F9A1=TTBlack.
\DefineRGBcolor 000000=Black.

\def\transparent{%
	\CS long\EC\def\Demonstration##1\CQFD{}%
}%
%\transparent
\def\Students{%
	\DefineRGBcolor FFFFFF=VLGreen.
	\DefineRGBcolor FFFFFF=LGreen.
	\DefineRGBcolor FFFFFF=TGreen.
	\DefineRGBcolor 000000=Green.
	\DefineRGBcolor FFFFFF=VLRed.
	\DefineRGBcolor FFFFFF=LRed.
	\DefineRGBcolor FFFFFF=TRed.
	\DefineRGBcolor FFFFFF=TTRed.
	\DefineRGBcolor 000000=Red.
	\DefineRGBcolor FFFFFF=VLOrange.
	\DefineRGBcolor FFFFFF=LOrange.
	\DefineRGBcolor FFFFFF=TOrange.
	\DefineRGBcolor FFFFFF=TTOrange.
	\DefineRGBcolor 000000=Orange.
	\DefineRGBcolor FFFFFFF=VLBlack.
	\DefineRGBcolor FFFFFF=LBlack.
	\DefineRGBcolor FFFFFF=TBlack.
	\DefineRGBcolor FFFFFF=TTBlack.
	\DefineRGBcolor 000000=Black.
}
\Students
\def\red{}
\def\blue{}
\def\Red#1{#1}%%%% Fix this !
\def\Blue#1{#1}%
\def\Font #1@#2pt{\font\olbi=cmr10\olbi}
\font\SvgText=cmr10\relax
%
%
%\catcode`@=11\relax
%\def\Api{Mathematicon@Api}%
%
%\input LD@Header.tex
%\input LD.tex
%\input LD@Exercices.tex
%\input LD@Typesetting.tex
%
%\catcode`@=11\relax
\font\LD@Font@Arial="Arial" at 10pt
%%%%%%%%%%%%%%%%%%%%%%%%%%%%%%%%%%%%%%%%%%%%%%%%%%%%%%%%%%%%%%%%%%
%															%
%						Métrique des courbes							%
%															%
%%%%%%%%%%%%%%%%%%%%%%%%%%%%%%%%%%%%%%%%%%%%%%%%%%%%%%%%%%%%%%%%%%
\newcount\LD@Count@Temp
\def\LD@Exercice@Display@Code{}%%\LD@Option@@Label\qquad\eightpts}%
\def\LD@Exercice@Display@Code@Post{%
	\ifcsname LD@Exo@@Solution\endcsname
		\unless\ifx\LD@Exo@@Solution\LD@Empty
			\pn{\eightpts Solution : \eightpts \LD@Exo@@Solution}%
		\fi
	\fi
}%
\def\LD@Display#1{%
	\LD@Count@Temp=#1\relax
	\ifcase\LD@Count@Temp
	\or
	Math. Sup.
	\or
	Math. Sp\'e
	\else
	\fi
}%
\newcount\LD@Exo@Total\LD@Exo@Total=0\relax


%%% TD 11.
\centerline{Rappels utiles sur limites, DL, \'equivalents, fonctions n\'egligeables}
\hrule
\medskip
\bigskip
\epsfbox{Courgr1.eps}\hfill\null\pn
\centerline{Figure 1. \qquad Support de l'astro\"\i de $\Phi:t\mapsto(\cos^3t,\sin^3t)$ pour $I=\ob R$}

\hfill\epsfbox{courbure.eps}\hfill\null\pn
\centerline{Figure 2. \qquad Centre et cercle 
de courbure de $t\mapsto(\cos^3t,\sin^3t)$ au point $M(\pi/3)$. }

\noindent\epsfxsize 6cm
\epsfbox{hel.ps}
\epsfxsize 5cm
\epsfbox{hell.eps}\medskip
\centerline{ Figures 3 \& 4 : allure 
et comportement local de l'h\'elice $t\mapsto \b(t,\cos(\pi t),\sin(\pi t)\b)$.}
\bigskip
\epsfxsize=6cm\noindent
\hfill\epsfbox{env.eps}\hfill\null
\medskip
\centerline{Figure 5. Enveloppe de la famille de droite $\Delta_t:x\cos t+y\sin t=\cos t\sin t$.}
\bigskip
\medskip\epsfxsize=5cm
\epsfysize=5cm
\noindent\hfill\epsfbox{Card.eps}\hfill\epsfxsize=5cm
\epsfysize=5cm
\epsfbox{Dell.eps}\hfill\null
\medskip
\bigskip
\noindent\hfill
\epsfbox{Dcercle.eps}\hfill\null
\medskip
\centerline{Figure 8. D\'eveloppantes de cercle.}
\bigskip
\medskip\noindent
\hfill\epsfbox{Cycloide.eps}\hfill\null
\medskip
\centerline{Figure 9. 
Cyclo\"\i de $t\mapsto(t-\sin t,1-\cos t)$.} 
\bigskip
\medskip\epsfxsize=6cm
\noindent\hfill\epsfbox{Ellip.eps}\hfill\null
\medskip
\bye

