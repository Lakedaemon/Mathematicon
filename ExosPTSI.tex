%%%%%     Banque d'exercices  PTSI 2004
%
\catcode`@=11\relax

\exo [Level=1,Fight=1,Learn=0,Field=\NombresComplexes,Type=\Colles,Origin=] a. 
Simplifier les expressions suivantes : 
$$
\eqalign{
\ds A&:=\cos\Q({\pi\F10}\W)+\cos\Q({4\pi\F10}\W)+\cos\Q({6\pi\F10}\W)+\cos\Q({9\pi\F10}\W),
\cr
\ds B&:=\cos^2\Q({\pi\F 10}\W)+\cos^2\Q({4\pi\F10}\W)+\cos^2\Q({6\pi\F10}\W)+\cos^2\Q({9\pi\F10}\W).
}
$$ 


\exo [Level=1,Fight=1,Learn=0,Field=\NombresComplexes,Type=\Colles,Origin=] b. 
Trouver les solutions réelles de l'équation 
$$
\ds\tan\Q(2x-{3\pi\F2}\W)=\tan\Q({\pi\F4}-x\W).\eqdef{Eqexob}\qquad
$$ 

\exo [Level=1,Fight=1,Learn=0,Field=\NombresComplexes,Type=\Colles,Origin=] c. 
Trouver les solutions réelles de l'équation 
$$
\sin\Q(x+{\pi\F4}\W)=\cos\Q(2x-{\pi\F3}\W).\eqdef{Eqexoc}\qquad
$$ 

\exo [Level=1,Fight=1,Learn=0,Field=\NombresComplexes,Type=\Colles,Origin=] d. 
Résoudre dans $\ob R$ l'équation $\ds\cos\Q(3x-{\pi\F5}\W)=\cos\Q(x+{4\pi\F5}\W)$.

\exo [Level=1,Fight=1,Learn=0,Field=\NombresComplexes,Type=\Colles,Origin=] e. 
Résoudre dans $\ob R$ l'équation $\tan(x)\tan(2x)=1$. 

\exo [Level=1,Fight=1,Learn=0,Field=\NombresComplexes,Type=\Colles,Origin=] f. 
Résoudre dans $\ob R$ l'équation $-2\cos^2x+\sin x+1=0$. 

\exo [Level=1,Fight=2,Learn=1,Field=\NombresComplexes,Type=\Colles,Origin=] g. 
Résoudre dans $\ob R$ l'équation $\tan x=2\cos x^2$. 

\exo [Level=1,Fight=1,Learn=1,Field=\NombresComplexes,Type=\Colles,Origin=] h. 
Calculer une valeur exacte de $\ds\sin{2\pi\F12}$, de $\ds\cos{7\pi\F12}$ et de $\ds\tan{7\pi\F12}$. 

\exo [Level=1,Fight=0,Learn=0,Field=\NombresComplexes,Type=\Cours,Origin=] i. 
Exprimer $\sin(3x)$, $\cos(3x)$ et $\tan(3x)$ respectivement en fonction de $\sin x$, $\cos x$ et $\tan x$. 

\exo [Level=1,Fight=1,Learn=-1,Field=\NombresComplexes,Type=\Others,Origin=] j. 
Calculer le nombre $\ds{1-\cos x\F2\sin x}$ en fonction de $\tan{x\F2}$. 

\exo [Level=1,Fight=0,Learn=0,Field=\NombresComplexes,Type=\Cours,Origin=] k. 
Mettre sous forme trigonométrique les nombres complexes suivants : 
$$
-3-3i, \qquad (1+i)^n,\qquad-\cos\theta+i\sin\theta,\qquad\sin\theta+i\cos\theta\qquad(\theta\in\ob R).
$$

\exo [Level=1,Fight=0,Learn=0,Field=\NombresComplexes,Type=\TravauxDirigés,Origin=] l. 
Pour $x\in\ob R$, déterminer les racines carrées des nombres suivants : 
$$
\e^{i\pi\F5},\qquad i,\qquad 1+i, \qquad 1+2i,\qquad2i-7,\cos(2x)+i\sin(2x). 
$$

\exo [Origin=\Lakedaemon,Level=1,Fight=1,Learn=1,Type=\Colles,Field=\Trigonométrie] m. 
Résoudre dans $\ob R$ l'équation $\cos x-\sqrt3\sin x=1$. 

\exo [Origin=\Lakedaemon,Level=1,Fight=1,Learn=1,Type=\Colles,Field=\Trigonométrie] n. 
Résoudre dans $\ob R$ l'équation $\cos(2x)+\sqrt3\sin(2x)=\sqrt2$. 

\exo [Level=1,Fight=2,Learn=1,Field=\NombresComplexes,Type=\Colles,Origin=] o. 
Pour $x\in\ob R$, linéariser $\cos^4x$ et $\sin^4x\cos^3x$. % Bourrin

\exo [Level=1,Fight=1,Learn=1,Field=\NombresComplexes,Type=\Colles,Origin=] p. 
Résoudre dans $\ob C$ l'équation $z^6-(1-i)z^3-i=0$.

\exo [Level=1,Fight=1,Learn=1,Field=\NombresComplexes,Type=\Colles,Origin=] q. 
Soit $\theta\in\ob R$. Résoudre dans $\ob C$ l'équation $z^4+2\cos(\theta)(1+\cos\theta)z^2+(1+\cos\theta)^2=0$.

\exo [Level=1,Fight=0,Learn=0,Field=\NombresComplexes,Type=\Cours,Origin=] r. 
Résoudre dans $\ob C$ l'équation $iz^2+(1-5i)z+6i-2=0$. 

\exo [Level=1,Fight=1,Learn=1,Field=\NombresComplexes,Type=\Exercices,Origin=] s. 
Soit $n\ge1$ un entier. Résoudre dans $\ob C$ l'équation $(z+1)^n=(z-1)^n$. 

\exo [Level=1,Fight=1,Learn=1,Field=\NombresComplexes,Type=\Exercices,Origin=] t. 
Soit $n\ge1$ un entier. Résoudre dans $\ob C$ l'équation $(z^2+1)^n-(z-i)^{2n}=0$. 

\exo [Level=1,Fight=1,Learn=1,Field=\NombresComplexes,Type=\Colles,Origin=] u. 
Résoudre dans $\ob C$ l'équation $\ds z^6={1+i\sqrt3\F1-i\sqrt3}$. 

\exo [Level=1,Fight=1,Learn=1,Field=\NombresComplexes,Type=\Cours,Origin=] v. 
Pour $x\in\ob R$ et $n\in\ob N$, calculer la somme $\sum_{k=0}^n\Q({n\atop k}\W)\cos(kx)$. 
 
\exo [Level=1,Fight=2,Learn=2,Field=\NombresComplexes,Type=\Exercices,Origin=] w. 
Pour $x\in\ob R$ et $n\in\ob N$, calculer la somme $\sum_{k=0}^n\cos^3(kx)$. 

\exo [Level=1,Fight=3,Learn=2,Field=\NombresComplexes,Type=\Colles,Origin=\MP] x. 
Pour $x\in\ob R$ et $n\in\ob N$, calculer la somme $\sum_{k=0}^nk\cos(kx)$.

\exo [Level=1,Fight=1,Learn=1,Field=\NombresComplexes,Type=\TravauxDirigés,Origin=] y. 
a) Mettre sous forme trigonométrique les nombres $\ds z_1:={\sqrt6-i\sqrt2\F2}$, 
$z_2:=1-i$ et $Z:=\ds{z_1\F z_2}$.\pn
b) En déduire l'expression de $\ds\cos{\pi\F12}$ et $\ds\sin{\pi\F12}$, à l'aide de radicaux. \pn
c) Résoudre dans $\ob R$ l'équation $(\sqrt6+\sqrt2)\cos x+(\sqrt6-\sqrt2)\sin x=2$. 

\exo [Level=1,Fight=2,Learn=1,Field=\NombresComplexes,Type=\Exercices,Origin=] z. 
Pour quels nombres complexes $z$, les points $A$, $B$, $C$ d'affixes $z$, $z^2$ et $z^4$ sont-ils alignés ?

\exo [Level=1,Fight=2,Learn=1,Field=\NombresComplexes,Type=\Colles,Origin=] aa. 
Déterminer les nombres complexes $z$ pour lesquels les points d'affixe $z$, $z^2$ et $z^3$ 
forment un triangle rectangle. 

\exo [Level=1,Fight=2,Learn=2,Field=\NombresComplexes,Type=\TravauxDirigés,Origin=] ab. 
A tout point $M$ d'affixe $z$, on associe le point $M'$ d'affixe $\ds z'={z-1\F1-\ol z}$. \pn
a) Etablir que $|z'|=1$, montrer que le nombre $\ds{z'-1\F z-1}$ est réel et que $\ds{z'+1\F z-1}$ est imaginaire pur. \pn
b) Le point $M$ étant donné, en déduire une construction géométrique du point $M'$. 

\exo [Level=1,Fight=2,Learn=1,Field=\NombresComplexes,Type=\Colles,Origin=] ac. 
Déterminer l'ensemble $E$ des points $M$ d'affixe $z$ tels que les points $I$ et $M'$ d'affixe $i$ et $iz$ soient alignés avec $M$. 
Déterminer alors l'ensemble des points $M'$, lorsque $M$ décrit $E$. 

\exo [Level=1,Fight=1,Learn=1,Field=\NombresComplexes,Type=\Exercices,Origin=] ad. 
Déterminer l'image de l'ensemble $\{z\in\ob C:|z|=1\hbox{ et }z\neq1\}$ par l'application $\ds z\mapsto {1\F 1-z}$. 

\exo [Level=1,Fight=0,Learn=0,Field=\NombresComplexes,Type=\Cours,Origin=] ae. 
Etablir la formule du parallélogramme, c'est-à-dire que  : 
$$
\forall (z,s)\in\ob C^2, \quad|z+s|^2+|z-s|^2=2|z|^2+2|s|^2. 
$$
 
\exo [Level=1,Fight=1,Learn=1,Type=\TravauxDirigés,Field=\EquationsDifférentiellesAVariablesSéparables,Origin=\Lakedaemon] af. 
Résoudre l'équation différentielle $y'=y^2$. 

\exo [Level=1,Fight=0,Learn=1,Field=\Trigonométrie,Type=\TravauxDirigés,Origin=] ag. 
Soient $\ds(\alpha,\beta)\in\Q]0,{\pi\F2}\W[^2$ tels que $\ds\tan\alpha={1\F2}$ et $\ds\tan\beta={1\F3}$. \pn
Calculer $\tan(\alpha+\beta)$ et en déduire la valeur de $\alpha+\beta$. 

\exo [Level=1,Fight=1,Learn=1,Field=\NombresComplexes,Type=\Exercices,Origin=] ah. 
Du calcul de $(1+i)(\sqrt3+i)$ déduire $\ds\cos{\pi\F12}$ et $\ds\sin{\pi\F12}$. 

\exo [Level=1,Fight=0,Learn=0,Field=\NombresComplexes,Type=\Exercices,Origin=] ai. 
Calculer $\cos (5x)$ en fonction de $\cos x$ et en déduire $\ds\cos {\pi\F10}$. 

\exo [Level=1,Fight=1,Learn=0,Field=\NombresComplexes,Type=\Cours,Origin=] aj. 
Pour $\ds0\le \theta\le{\pi\F2}$ résoudre dans $\ob C$ l'équation $z^6-2z^3\cos\theta+1=0$. 

\exo [Level=1,Fight=1,Learn=0,Field=\NombresComplexes,Type=\Colles,Origin=] ak. 
Résoudre dans $\ob C$ l'équation $z^4-(5-14i)z^2-2(5i+12)=0$. 

\exo [Level=1,Fight=1,Learn=1,Field=\NombresComplexes,Type=\Colles,Origin=] al. 
Soit $n\ge1$ un entier. Résoudre dans $\ob C$ l'équation $z^{2n}-2z^n\cos\theta+1=0$ 

\exo [Level=1,Fight=0,Learn=0,Field=\NombresComplexes,Type=\TravauxDirigés,Origin=] am. 
Résoudre dans $\ob C$ l'équation $\ds z^8={1+i\F\sqrt3-i}$. 

\exo [Level=1,Fight=2,Learn=1,Field=\NombresComplexes,Type=\Colles,Origin=] an. 
Pour $n\ge1$ et $-\ds{\pi\F2}<\theta<{\pi\F2}$, résoudre dans $\ob C$ 
l'équation $\ds\Q({1-iz\F1+iz}\W)^n={1-i\tan\theta\F1+i\tan\theta}$.

\exo [Level=1,Fight=2,Learn=2,Field=\NombresComplexes,Type=\TravauxDirigés,Origin=] ao. 
Soit $n\ge0$ un entier fixé. Calculer les sommes suivantes : 
$$
S:=\sum_{0\le 3k\le n}\Q({n\atop k}\W),\qquad T:=\sum_{0\le 3k+1\le n}\Q({n\atop k}\W)
\qquad\hbox{et}\qquad U:=\sum_{0\le 3k+2\le n}\Q({n\atop k}\W).
$$

\exo [Level=1,Fight=1,Learn=1,Field=\NombresComplexes,Type=\Exercices,Origin=] ap. 
Pour $n\in\ob N$, calculer la somme $\sum_{0\le 2k\le n}\Q({n\atop 2k}\W)(-1)^k$. 

\exo [Level=1,Fight=2,Learn=1,Field=\Trigonométrie|\Polynômes,Type=\Colles,Origin=] aq. Le but de cet exercice est la résolution dans $\ob R$ de l'équation 
$$
\sin(7x)-\sin(3x)-\sin(x)=0.\leqno{(E)}
$$ 
a) Montrer qu'il existe un polynôme $P(X)=aX^3+bX^2+cX+d$ tel que 
$$
\forall x\in\ob R, \qquad \sin(7x)-\sin(3x)-\sin(x)=\sin(x)P\b(\sin^2(x)\b). 
$$ 
b) Trouver une racine $r$ évidente du polynôme $P$ et diviser $P$ par $X-r$. \pn
c) Factoriser le polynôme $P$. \pn
d) En déduire les solutions de l'équation $(E)$. 

\exo [Level=1,Fight=1,Learn=1,Field=\NombresComplexes,Type=\Exercices,Origin=] ar. 
Résoudre dans $\ob R$ l'équation $3\tan x=2\cos x$. 

\exo [Level=1,Fight=0,Learn=0,Field=\Trigonométrie,Type=\Colles,Origin=] as. 
Résoudre dans $\ob R$ l'équation $\ds\tan\Q(3x-{\pi\F5}\W)=\tan\Q(x+{4\pi\F5}\W)$. 

\exo [Level=1,Fight=1,Learn=1,Field=\NombresComplexes,Type=\Colles,Origin=] at. 
Trouver les couples $(x,y)$ de nombres réels vérifant le système $\ds\Q\{\eqalign{
&\tan x+\tan y=1,
\cr
&\tan(x+y)={4\F3},
}\W.$. 


\exo [Level=1,Fight=2,Learn=1,Field=\NombresComplexes,Type=\Exercices,Origin=] au. 
Pour $n\ge0$ et $x\in\ob R$, calculer la somme $\sum_{k=0}^n\cos(kx)\cos^kx$. 

\exo [Level=1,Fight=1,Learn=1,Field=\NombresComplexes,Type=\Colles,Origin=] av. 
Pour $n\ge1$ et $x\in\ob R$, calculer $\sum_{k=0}^n\cos^2(kx)$. 

\exo [Level=1,Fight=2,Learn=1,Field=\NombresComplexes,Type=\Colles,Origin=] aw. 
Résoudre dans $\ob C$ l'équation $z^2+8i=|z|^2-2$. 

\exo [Level=1,Fight=0,Learn=0,Field=\NombresComplexes,Type=\Cours,Origin=] ax. 
Résoudre dans $\ob C$ l'équation $(1-i)z^2-(5-i)z+10=0$. 

\exo [Level=1,Fight=1,Learn=0,Field=\NombresComplexes,Type=\Colles,Origin=] ay. 
Résoudre dans $\ob C$ l'équation $z^4-(5-14i)z^2-2(5i+12)=0$. 

\exo [Level=1,Fight=3,Learn=1,Field=\NombresComplexes|\GéométriePlane,Type=\Colles,Origin=] az. 
Montrer qu'un triangle $ABC$ est équilatéral si, et seulement si, les affixes $a$, $b$ et $c$ de ses sommets vérifient 
$$
a^2+b^2+c^2=ab+ac+bc. 
$$

\exo [Level=1,Fight=1,Learn=1,Field=\NombresComplexes|\GéométriePlane,Type=\Colles,Origin=] ba. 
Dans le plan complexe, on considère trois points $A(2+i)$, $B(-1+3i)$ et $C(-2)$. \pn
a) Déterminer l'affixe du point $D$ pour lequel $ABCD$ est un parallélogramme. \pn
b) Calculer l'affixe de l'iso-barycentre $G$ de $ABCD$. \pn
c) Déterminer le couple $(\alpha,\beta)$ de nombres complexes tel que l'application $z\mapsto\alpha z+\beta$ 
corresponde à la similitude directe de centre $G$ envoyant $A$ sur $B$. 

\exo [Level=1,Fight=0,Learn=0,Field=\NombresComplexes|\GéométriePlane,Type=\Exercices,Origin=] bb. 
Déterminer et tracer l'ensemble des points $M$du plan dont l'affixe $z$ vérifie
$$
\Q|{z-1\F z+1}\W|=2.
$$

\exo [Level=1,Fight=1,Learn=0,Field=\NombresComplexes|\GéométriePlane,Type=\Exercices,Origin=] bc. 
Déterminer et tracer l'ensemble des points $M$du plan dont l'affixe $z$ vérifie
$$
\arg\Q({z-i\F z+i}\W)\equiv-{\pi\F4}\quad[2\pi].
$$

\exo [Level=1,Fight=0,Learn=0,Field=\NombresComplexes,Type=\Exercices,Origin=] bc. Déterminer les racines $5^{\hbox{\sevenrm ièmes}}$ du nombre $1+i\sqrt3$.

\exo [Level=1,Fight=0,Learn=0,Field=\NombresComplexes,Type=\Exercices,Origin=] bd. 
Trouver sans calcul les formes algébriques et trigonométriques des racines cubiques de $-8$. 

\exo [Level=1,Fight=1,Learn=1,Field=\NombresComplexes|\GéométriePlane,Type=\Exercices,Origin=] be. 
Dans le plan, quel est l'ensemble des points d'affixe $z$ tel que $\ds {z+1\F z}$ soit imaginaire pur. 


\exo [Level=1,Fight=2,Learn=2,Field=\NombresComplexes,Type=\TravauxDirigés,Origin=] bf. 
Pour $(\theta,\theta')\in\ob R^2$, déterminer le module et 
l'argument de $\e^{i\theta}+\e^{i\theta'}$ \hbox{ et } $\e^{i\theta}-\e^{i\theta'}$. \pn
En déduire la forme trigonométrique de $\ds {1-u\F1+u}$ quand $u\neq-1$ est de module $1$. 

\exo [Level=1,Fight=1,Learn=1,Field=\NombresComplexes,Type=\Exercices,Origin=] bg. 
Pour $x\in\ob R$ et $n\in\ob N$, calculer la somme $\ds \sum_{k=0}^n\sin(kx)$. 

\exo [Level=1,Fight=0,Learn=0,Field=\NombresComplexes,Type=\TravauxDirigés,Origin=] bh. Mettre les nombres suivants sous la forme algébrique $x+iy$ : 
$$
a:=(1+i)(1+2i),\qquad b:=i(\sqrt5-i)(i-1),\qquad c:=(1+2i)^3, \qquad d:={18+26i\F -2+2i},  \qquad f:={1\F2+i}-{1\F3-i}.
$$

\exo [Level=1,Fight=0,Learn=0,Field=\NombresComplexes,Type=\TravauxDirigés,Origin=] bi. 
Mettre les nombres suivants sous forme trigonométrique :
$$
a:=\sqrt3+i, \qquad b:=5+3i, \qquad c:={7\F\sqrt2+i}, \qquad{3-4i\F(1+i)^2}
$$

\exo [Level=1,Fight=1,Learn=1,Field=\NombresComplexes,Type=\Exercices,Origin=] bj. 
Pour $z\in\ob C$, prouver que $|\re z|+|\im z|\le |z|\sqrt2$. 

\exo [Level=1,Fight=0,Learn=0,Field=\NombresComplexes,Type=\Exercices,Origin=] bk. 
Montrer que $|z+i|=|z-i|$ si, et seulement si, $z\in\ob R$. 

\exo [Level=1,Fight=1,Learn=1,Field=\Trigonométrie,Type=\Exercices,Origin=\Capaces] bl. 
Pour $(x,y)\in\ob R^2$, démontrer que  
$$
\eqalign{
\cos(x+y)\cos(x-y)&=\cos^2(x)-\sin^2(y)=\cos^2(y)-\sin^2(x),\cr
&={1-\tan^2(x)\tan^2(y)\F(1+\tan^2x)(1+\tan^2y)} \hbox{ si les tangentes sont définies }
}
$$

\exo [Level=1,Fight=1,Learn=0,Field=\Trigonométrie,Type=\Exercices,Origin=\Capaces] bm. 
Soient $(x,y,z)\in\ob R^2$. Lorsque les tangentes sont définies, prouver que 
$$
\tan(x)+\tan(y)+\tan(z)-\tan(x)\tan(y)\tan(z)={\sin(x+y+z)\F\cos(x)\cos(y)\cos(z)}.
$$

\exo [Level=1,Fight=2,Learn=0,Field=\Trigonométrie,Type=\Exercices,Origin=\Capaces] bn. 
Pour $(x,y,z)\in\ob R^3$, prouver que  le nombre
$\Q(\cos x-\cos(y-z)\W)\Q(\cos x-\cos(y+z)\W)$ est égal à 
$$
\cos^2x+\cos^2y+\cos^2z-2\cos(x)\cos(y)\cos(z)-1.
$$
Puis, en déduire que 
$$
\cos^2(x-y)+\cos^2(y-z)+\cos^2(z-x)-2\cos(x-y)\cos(y-z)\cos(z-x)=1.
$$

\exo [Level=1,Fight=1,Learn=0,Field=\NombresComplexes,Type=\Colles,Origin=] bo. 
Montrer que deux nombres complexes $a$ et $b$ ont même module si, et seulement s'il existe $\lambda\in\ob R$ tel que 
$a+b=\lambda i(a-b)$. 

\exo [Level=1,Fight=2,Learn=0,Field=\NombresComplexes,Type=\Others,Origin=] bp. 
Soit $(a,b)\in\ob C^2$. L'équation $z=a\overline z+b$ admet-elle une solution ?

\exo  [Origin=\Lakedaemon,Level=1,Fight=2,Learn=1,Type=\Colles,Field=\NombresComplexes|\Trigonométrie] bq. 
Pour $a\in\ob R$, trouver les solutions $(x,y)\in\ob R^2$ du système 
$$
\Q\{\eqalign{
\cos a+\cos(a+x)+\cos(a+y)=0
\cr
\sin a+\sin(a+x)+\sin(a+y)=0
}\W.
$$ 

\exo [Level=1,Fight=1,Learn=1,Field=\NombresComplexes,Type=\Exercices,Origin=] br. 
Montrer que la fonction $f(x):=\cos^4(x)+\cos^4(x+{\pi\F4})+\cos^4(x+{\pi\F2})+\cos^4(x+{3\pi\F4})$ est constante sur $\ob R$. 

\exo [Level=1,Fight=2,Learn=2,Field=\NombresComplexes,Type=\Exercices,Origin=] bs. 
On pose $u:=\cos{2\pi\F9}$, $v:=\cos{4\pi\F9}$ et $w:=\cos{8\pi\F9}$. \pn
Calculer $u+v+w$, $uv+vw+uw$ et $uvw$. 

\exo [Level=1,Fight=2,Learn=2,Field=\NombresComplexes,Type=\Exercices,Origin=] bt. 
On pose $u:=\cos{2\pi\F 7}$, $v:=\cos{4\pi\F7}$ et $w:=\cos{6\pi\F7}$. \pn
Calculer $u+v+w$, $uv+vw+uw$ et $uvw$. 

\exo [Level=1,Fight=3,Learn=1,Field=\Fonctions,Type=\Exercices,Origin=] bu. 
Trouver les couples $(x,y)$ d'éléments de $\ob N^*$ vérifiant $x^y=y^x$.

\exo [Level=1,Fight=1,Learn=1,Field=\Fonctions,Type=\Exercices,Origin=] bv. 
Pour $(a,b,x)\in\ob R^3$, calculer les sommes 
$$
C(a,b):=\sum_{0\le p\le n}\ch(a+pb), \qquad  S(a,b):=\sum_{0\le p\le b}\sh(a+pb)\qquad\hbox{et}\qquad D(x):=\sum_{0\le p\le n}p\ch(px).
$$

\exo [Level=1,Fight=1,Learn=0,Field=\Fonctions,Type=\Exercices,Origin=] bw. 
Étudier la fonction $\ds f:x\mapsto\arcsin\Q({2x\F1+x^2}\W)$. 

\exo [Level=1,Fight=1,Learn=0,Field=\Fonctions,Type=\Exercices,Origin=] bx. 
Étudier la fonction $\ds f:x\mapsto\arctan\b(\sqrt x\b)$. 

\exo [Level=1,Fight=2,Learn=1,Field=\Fonctions,Type=\Exercices,Origin=] by. 
Étudier la fonction $\ds f:x\mapsto\argth\Q({1+3\th(x)\F3+\th(x)}\W)$. 

\exo [Level=1,Fight=2,Learn=1,Field=\Fonctions,Type=\Exercices,Origin=] bz. 
Étudier la fonction $f:x\mapsto\argth\b(4x^3-3x\b)$. 

\exo [Level=1,Fight=1,Learn=1,Field=\Suites|\TrigonométrieHyperbolique,Type=\Exercices,Origin=] ca. 
Étudier la suite $u_n:=\th(1)+\th(2)+\cdots+\th(n)-\ln\b(\ch(n)\b)$. 

\exo [Level=1,Fight=1,Learn=0,Field=\TrigonométrieHyperbolique,Type=\Exercices,Origin=] cb. 
Pour $x\in\ob R$, calculer $\ds\prod_{1\le p\le n}\ch\Q({x\F 2^p}\W)$. 

\exo [Level=1,Fight=1,Learn=1,Field=\TrigonométrieHyperbolique,Type=\Exercices,Origin=] cc. 
Calculer $\ds\sum_{1\le p\le n}2^{-p}\th\Q({x\F2^p}\W)$ pour $x\in\ob R$ après avoir montré, dans un premier temps, que 
$$
\forall x\in\ob R^*, \qquad \th(x)={2\F\th(2x)}-{1\F \th(x)}.
$$

\exo  [Level=1,Fight=3,Learn=1,Field=\Fonctions|\Trigonométrie,Type=\Exercices,Origin=] cd. 
Trouver les solutions $f\in\sc C\b(\Q[0,1\W[,\ob R\b)$ de l'équation fonctionnelle
$$
\forall x\in\Q[0,1\W[,\qquad f\Q({2x\F1+x^2}\W)=(1+x^2)f(x).
$$

\exo [Level=1,Fight=1,Learn=0,Field=\Fonctions,Type=\Exercices,Origin=] ce. 
Trouver les couples $(x,y)$ d'éléments de $\Q]0,+\infty\W[$ vérifiant 
$\ds 
\Q\{\eqalign{
&xy=256,\cr
&7(\log_yx+\log_xy)=50.
}\W.
$

\exo [Level=1,Fight=2,Learn=1,Field=\DéveloppementsLimités,Type=\Exercices,Origin=] cf. 
Calculer $\ds \lim_{n\to\infty}\Q({\arctan(n+1)\F\arctan n}\W)^{n^2}$. 

\exo [Level=2,Fight=3,Learn=1,Field=\Séries,Type=\Exercices,Origin=] cg. 
Convergence et somme de $\ds\sum_{k=0}^{+\infty}\arcsin\Q(\sin\Q(2\pi(4+\sqrt{15})^k\W)\W)$. 

\exo [Level=2,Fight=2,Learn=1,Field=\Séries,Type=\Exercices,Origin=] ch. 
Convergence de la série de terme général $u_n=\arcsin\Q({n^2\F n^2+1}\W)-\arcsin{n^2+\F n^2+2}$. 

\exo [Level=1,Fight=3,Learn=1,Field=\Fonctions,Type=\Exercices,Origin=] ci. 
Trouver toutes les applications $f:\ob R\to\ob R$, dérivables en $0$, vérifiant 
$$
\forall x\in\ob R, \qquad f(2x)={2f(x)\F1+f(x)^2}.
$$

\exo [Level=1,Fight=1,Learn=1,Field=\Trigonométrie,Type=\Exercices,Origin=] cj. 
Montrer que $\ds1+{1\F\cos(2x)}={\tan(2x)\F\tan(x)}$ et en déduire une expression simple de $\ds
\prod_{k=0}^n\Q(1+{1\F\cos(2^kx)}\W).
$

\exo [Level=1,Fight=2,Learn=1,Field=\Fonctions,Type=\Exercices,Origin=] ck. 
Étudier la fonction $\ds x\mapsto\arctan\Q({1+x\F1-x}\W)$. 

\exo [Level=1,Fight=1,Learn=1,Field=\TrigonométrieHyperbolique,Type=\Exercices,Origin=] cl. Résoudre dans $\ob R$ l'équation $\ch x+2\sh x=3$. 

\exo [Level=1,Fight=2,Learn=1,Field=\Fonctions,Type=\Exercices,Origin=] cm. 
Étudier la fonction $\ds x\mapsto\argsh\Q({x^2-1\F2x}\W)$. 

\exo [Level=1,Fight=2,Learn=1,Field=\Fonctions,Type=\Exercices,Origin=] cn. 
Calculer $u:=\ds\arctan\Q({1\F2}\W)+\arctan\Q({1\F3}\W)$ puis $\ds v:=\arctan(2)+\arctan(3)$. 

\exo [Level=1,Fight=2,Learn=1,Field=\Fonctions,Type=\Exercices,Origin=] co. 
Calculer $u:=\ds\arctan\Q({1\F2}\W)+\arctan\Q({1\F5}\W)+\arctan\Q({1\F8}\W)$ puis $\ds v:=\arctan(2)+\arctan(5)+\arctan(8)$. 

\exo [Level=1,Fight=1,Learn=0,Field=\Fonctions,Type=\Exercices,Origin=] cp. 
Résoudre dans $\ob R$ l'équation $\ds\arctan(2x)+\arctan(3x)=-{\pi\F4}$.

\exo [Level=1,Fight=1,Learn=0,Field=\Fonctions,Type=\Exercices,Origin=] cq. 
Montrer que l'équation 
$$
\arctan(x-1)+\arctan(x)+\arctan(x+1)={\pi\F2}
$$
admet une unique solution, qui appartient à l'intervalle $\Q]0,1\W[$. En déduire cette solution. 

\exo [Level=1,Fight=2,Learn=1,Field=\Fonctions,Type=\TravauxDirigés,Origin=]cr. 
Étudier la fonction $f:x\mapsto\arccos\b(\cos(x)\b)$. 

\exo [Level=1,Fight=1,Learn=0,Field=\Récurrences,Type=\Exercices,Origin=\Capaces] cs. 
Pour $n\ge0$, prouver que $\ds\sum_{k=0}^n{k\F(k+1)!}=1-{1\F(n+1)!}$. 

\exo [Level=1,Fight=1,Learn=0,Field=\Récurrences,Type=\Exercices,Origin=\Capaces] ct. 
Pour $n\ge0$, montrer que $\ds\sum_{k=0}^nk!k=(n+1)!-1$. 
 
\exo [Level=1,Fight=1,Learn=0,Field=\Récurrences,Type=\Exercices,Origin=\Capaces] cu. 
Pour $n\ge0$, montrer que $\ds\sum_{k=0}^n(-1)^k{(2k+1)^3\F(2k+1)^4+4}=(-1)^n{n+1\F 4(n+1)^2+1}$. 

\exo [Level=1,Fight=1,Learn=0,Field=\Récurrences,Type=\Exercices,Origin=\Capaces] cv. 
Pour $n\ge1$, montrer que $\ds u_n={1\F1.2.3}+\cdots+{1\F n(n+1)(n+2)}={n(n+3)\F4(n+1)(n+2)}$. 

\exo [Level=1,Fight=3,Learn=3,Field=\Fonctions,Type=\Problèmes,Origin=\Capaces] cw. 
Soit $p$ un nombre premier vérifiant $p\equiv1\ \,[4]$, soit $k$ l'unique entier vérifiant $p=4k+1$ et soit $E$ l'ensemble 
$$
E:=\b\{(x,y,z)\in\ob N^3:x^2+4yz=p\b\}.
$$
a) Démontrer que le nombre d'éléments de $E$ est fini. \pn
b) Prouver que l'on définit une application $\phi$ de $E$ dans $\ob N^3$ en posant  
$$
\forall (x,y,z)\in E, \qquad 
\phi(x,y,z):=\Q\{\eqalign{
&(x+2z,z,y-x-z)\quad\hbox{ si }x<y-z\cr
&(2y-x,y,x-y+z)\quad\hbox{ si }y-z<x<2y\cr
&(x-2y,x-y+z,y)\quad\hbox{ si }x>2y\cr}
\W.
$$
c) De plus, démontrer que $\phi$ est une application de $E$ dans $E$. \pn
d) Enfin, prouver que $\phi$ est une involution de $E$, c'est-à-dire que 
$$
\forall(x,y,z)\in E, \qquad\phi\circ\phi(x,y,z)=(x,y,z).
$$  
e) Démontrer que $(1,1,k)$ est l'unique point fixe de l'application $\phi$, c'est-à-dire l'unique triplet $(x,y,z)\in E$ vérifiant 
$$
\phi(x,y,z)=(x,y,z).
$$
f) Prouver que le nombre d'éléments de l'ensemble $E$ est impair. \pn
g) Montrer que $\psi:(x,y,z)\mapsto (x,z,y)$ est une application de $E$ dans $E$ admettant au moins un point fixe. \pn
h) Quel théorème avez-vous démontré ?

\exo [Level=1,Fight=3,Learn=2,Field=\Fonctions,Type=\Others,Origin=\Lakedaemon] cx. 
Soit $f:\ob N^2\to\ob N$ la fonction définie par   
$$
\forall (i,j)\in\ob N^2, \qquad f(i,j):={(i+j)(i+j+1)\F2}+j.
$$
a) Démontrer que $f$ est une bijection. \pn 
b) En déduire que l'ensemble $\ob Q$ est dénombrable (c'est-à-dire qu'il existe une surjection de $\ob N$ dans $\ob Q$). 

\exo [Level=1,Fight=2,Learn=1,Field=\NombresComplexes,Type=\Exercices,Origin=] cy. 
Soient $(a,b,c,d)$ des nombres complexes distincts deux à deux. On pose 
$$
x:={a-d\F b-c},\qquad y:={b-d\F c-a}\qquad\hbox{et}\qquad z={c-d\F a-b}.
$$
Démontrer que si $x$ et $y$ sont imaginaires purs, alors $z$ l'est aussi. 


\exo  [Level=1,Fight=3,Learn=3,Field=\NombresComplexes,Type=\Problèmes,Origin=\Capaces]  cz. 
Pour chaque entier $j\ge0$ et pour chaque entier $n\ge0$, on pose 
$$
S_j:=S_j(n):=\sum_{k=0}^nk^j.
$$
Le but de ce problème est d'établir l'identité de Jacobi $2S_3^2=S_5+S_7$ et de prouver plus généralement que  
$$
\forall m\ge1, \qquad 2^{2m-1}S_3^m=\sum_{p=1}^m{2m\choose 2p-1}S_{2m+2p-1}. \leqno{(*)}
$$ 
a) Démontrer que l'on a  
$$
\forall m\in\ob N, \qquad \forall n\ge m, \qquad \sum_{k=m}^n{k\choose m}={n+1\choose m+1}. \leqno{(**)}
$$
b) En appliquant la formule $(**)$ pour $2\le m\le 4$, calculer $S_j$ pour $2\le j\le 4$. \pn
c) Trouver une relation entre $S_1$ et $S_3$.  \pn
d) Etablir que 
$$
\forall m\ge1,\qquad \forall n\ge0, \qquad  2^{2m}S_3^m=\sum_{k=0}^n\big((k^2+k)^{2m}-(k^2-k)^{2m}\big).
$$
e) Démontrer que 
$$
\forall k\ge0, \qquad \forall m\ge1, \qquad (k+1)^{2m}-(k-1)^{2m}=2\sum_{0\le 2p-1\le 2m}{2m\choose 2p-1}k^{2p-1}.
$$
f) En déduire l'identité $(*)$ puis l'identité de Jacobi. 

\exo  [Level=1,Fight=2,Learn=2,Field=\Fonctions,Type=\Cours,Origin=\Mercier] da. 
Soit $f:E\to F$ une application. Montrer que : \pn
a) $f\hbox{ est injective } \Leftrightarrow\forall A\subset E, A=f^{-1}\b(f(A)\b)$. \pn
b) $f\hbox{ est surjecive } \Leftrightarrow\forall B\subset F, B=f\b(f^{-1}(B)\b)$. 

\exo  [Level=1,Fight=2,Learn=2,Field=\NombresComplexes,Type=\Problèmes,Origin=] db. 
Soient $j:=\e^{2\pi i\F3}$ et $n\in\ob N$. On pose 
$$
\eqalign{
&S:=\sum_{0\le 3k\le n}{n\choose 3k}={n\choose 0}+{n\choose 3}+\cdots{n\choose 3k}+\cdots
\cr
&T:=\sum_{0\le 3k+1\le n}{n\choose 3k+1}={n\choose 1}+{n\choose 4}+\cdots+{n\choose 3k+1}+\cdots
\cr
&U:=\sum_{0\le 3k+2\le n}{n\choose 3k+2}={n\choose 2}+{n\choose 5}+\cdots+{n\choose 3k+2}+\cdots
}
$$
a) Calculer $S+T+U$. \pn
b) Vérifier que $(1+j)^n=S+jT+j^2U$. \pn 
c) Calculer de même $(1+j^2)^n$ en fonction de $S$, $T$ et $U$. \pn
d) En déduire l'expression de $S$, $T$ et $U$ en fonction de $n$ (sans signe somme). 


\exo [Level=1,Fight=1,Learn=1,Field=\Fonctions,Type=\Exercices,Origin=] dc. 
Soit $\sc D:=\sc D(0,1)$ le disque ouvert de centre $0$ et de rayon $1$ et soit $\sc P$ le demi-plan de Poincaré, c'est-à-dire l'ensemble $\sc P:=\{z\in\ob C:\im(z)>0\}$. \pn
a) Démontrer que la fonction $\psi:z\mapsto\ds {z-i\F z+i}$ est une bijection de $\sc P$ dans $\sc D$. \pn
b) Quelle est l'image par $\psi$ du cercle de centre $0$ et de rayon $1$ privé de $-i$ ?



\exo [Level=1,Fight=3,Learn=3,Field=\NombresEntiers,Type=\Problèmes,Origin=\Capaces] de. 
Dans tout ce problème, on note $E_n$ l'ensemble des entiers $k\in[1,n]$
$$
E_n:=\{1,2,\cdots, n\}
$$
et on note $S_{n,p}$ le nombre des applications surjectives de $E_n$ dans $E_p$. \pn
I.1. Calculer  $S_{n,1}$ et $S_{n,2}$ pour $n\ge1$. \pn
I.2. Calculer $S_{n+1,n}$, $S_{n,n}$ et $S_{n,n+1}$ pour $n\ge1$. \pn
II.1. Pour toute la suite, $n$ et $p$ sont des entiers vérifiant $1\le p\le n$. Prouver que 
$$
{p\choose q}{q\choose k}={p\choose k}{p-k\choose q-k}\qquad (k\le q\le p)
$$
et en déduire que 
$$
\sum_{q=k}^p(-1)^q{p\choose q}{q\choose k}=0\qquad (0\le k<p).
$$
II.2. Démontrer que 
$$
p^n=\sum_{q=0}^p{p\choose q}S_{n,q}
$$
II.3. En déduire la formule 
$$
S_{n,p}=(-1)^p\sum_{k=0}^p(-1)^k{p\choose k}k^n.
$$
II.4. En déduire que 
$$
S_{n,p}=p\B(S_{n-1,p}+S_{n-1,p-1}\B)\quad \hbox{si}\quad p\ge2
$$
II.5. Retrouver $S_{n+1,n}$ puis démontrer que 
$$
S_{n+2,n}={n(3n+1)\F24}(n+2)!
$$
II.6 En s'inspirant du triangle de pascal, montrer que l'on peut construire la table des $S_{n,p}$. La construire pour $1\le p\le n\le 7$. 

\exo [Level=1,Fight=0,Learn=0,Field=\EquationsDifférentiellesLinéairesDuPremierOrdre,Type=\TravauxDirigés,Origin=] df. 
Résoudre l'équation différentielle $2y'-3y=x$. 

\exo [Level=1,Fight=0,Learn=0,Field=\EquationsDifférentiellesLinéairesDuPremierOrdre,Type=\TravauxDirigés,Origin=] dg. 
Résoudre l'équation différentielle $iy'+y=\sin(x)$ avec la condition initiale $y'(0)=1$. 

\exo [Level=1,Fight=1,Learn=1,Field=\EquationsDifférentiellesLinéairesDuPremierOrdre,Type=\TravauxDirigés,Origin=] dh. 
Résoudre $\cos(x)y'+\sin(x)y=\tan(x)$. 

\exo [Level=1,Fight=1,Learn=1,Field=\EquationsDifférentiellesLinéairesDuPremierOrdre,Type=\TravauxDirigés,Origin=] di. 
Résoudre $(x^2-1)y'+xy=x^3-x$. 

\exo [Level=1,Fight=1,Learn=1,Field=\EquationsDifférentiellesLinéairesDuPremierOrdre,Type=\TravauxDirigés,Origin=] dj. 
Résoudre $(1-x^2)y'-2xy=x^2$. 

\exo [Level=1,Fight=0,Learn=0,Field=\EquationsDifférentiellesLinéairesDuPremierOrdre,Type=\Exercices,Origin=] dk. 
Résoudre $y'-y\tan(x)=-\cos^2(x)$. 

\exo [Level=1,Fight=0,Learn=0,Field=\EquationsDifférentiellesLinéairesDuPremierOrdre,Type=\Exercices,Origin=] dl. 
Résoudre $y'\sin(x)=2y\cos(x)$. 

\exo [Level=1,Fight=1,Learn=1,Field=\EquationsDifférentiellesLinéairesDuPremierOrdre,Type=\Exercices,Origin=] dm. 
Résoudre $2x(1+\sqrt x)y''+(1+2\sqrt x)y'=0$. 

\exo [Level=1,Fight=1,Learn=0,Field=\EquationsDifférentiellesLinéairesDuPremierOrdre,Type=\Exercices,Origin=] dn. 
Résoudre $(3x^2-2x)y'=(6x-2)y$. 

\exo [Level=1,Fight=,Learn=,Type=\Cours,Field=\EquationsDifférentiellesLinéairesDuPremierOrdre,Origin=] do. 
Résoudre $y'+y\cos x=\cos x$. 

\exo [Level=1,Fight=0,Learn=0,Field=\EquationsDifférentiellesLinéairesDuPremierOrdre,Type=\Exercices,Origin=] dp. 
Résoudre $\ds y'-y\tan x={1\F\cos^3x}$. 

\exo  [Level=1,Fight=1,Learn=0,Field=\EquationsDifférentiellesLinéairesDuPremierOrdre,Type=\Exercices,Origin=] dq. 
Résoudre $3x^3\ln^2x=y'x\ln x-y$. 

% Redondant % dr %

% Redondant % ds %

% Redondant % dt %

% Redondant % du %

% Redondant % dv %

% Redondant % dw %

% Redondant % dx %

% Redondant % dy %

\exo [Level=1,Fight=0,Learn=0,Field=\NombresComplexes,Type=\Exercices,Origin=] dz. 
Montrer que 
$$
\forall (a,b)\in\ob C^2,\qquad |a-b|^2\le \b(1+|a|^2\b)\b(1+|b|^2\b).
$$

\exo [Level=1,Fight=3,Learn=2,Field=\NombresComplexes,Type=\Exercices,Origin=] ea. 
a) Pour $(u,v)\in\ob C^2$ vérifiant $u\neq0\neq0$, prouver que 
$$
\Q|{u\F|u|^2}-{v\F|v|^2}\W|={|u-v|\F|uv|}. 
$$
b) Pour $(x,y,z)\in\ob C^3$, prouver que 
$\ds |x|.|y-z|\le |y|.|x-z|+|z|.|x-y|$. 
\pn
c) En déduire l'inégalité de Ptolémé : 
$$
\forall (u,v,w,z)\in\ob C^4, \qquad |u-v|.|w-z|\le |u-w|.|v-z|+|u-z|.|v-w|.
$$

\exo [Level=1,Fight=1,Learn=1,Field=\Trigonométrie,Type=\Exercices,Origin=] eb. 
Prouver que 
$$
\forall x\in\ob R, \qquad |\sin x|\le|x|. 
$$
En déduire que 
$$
\forall x\in\ob R,\qquad |\e^{ix}-1|\le |x|. 
$$
Quand l'égalité est-elle réalisée ?

\exo [Level=1,Fight=2,Learn=1,Field=\NombresComplexes|\Trigonométrie,Type=\Exercices,Origin=] ec. 
a) Donner la forme algébrique de $z=(5-i)^4(1+i)$. \pn
b) Calculer les arguments respectifs de $5-i$ et $z$. \pn
c) En déduire la formule de Machin : 
$$
4\arctan{1\F5}-\arctan{1\F239}={\pi\F4}.
$$

\exo [Level=1,Fight=2,Learn=2,Field=\Trigonométrie,Type=\Exercices,Origin=] ed. 
Résoudre dans $\ob R$ les équations 
$$
\eqalignno{
\cos^3(x)+\sin^3(x)&=1&(a)
\cr
\cos^4(x)+\sin^4(x)&=1&(b)
\cr
\sqrt{\cos(x)}+\sqrt{\sin(x)}&=1&(c)
\cr
\cos^n(x)+\sin^n(x)&=1&(d)
}
$$

\exo [Level=1,Fight=3,Learn=1,Field=\NombresComplexes,Type=\Colles,Origin=] ee. 
Pour $n\ge1$ et $(z_1,\cdots,z_n)\in\ob C^n$, prouver que 
$$
{\ds\Q|\sum_{k=1}^nz_k\W|\F\ds 1+\Q|\sum_{k=1}^nz_k\W|}\le \sum_{k=1}^n{|z_k|\F1+|z_k|}.
$$

\exo [Level=1,Fight=1,Learn=1,Field=\Applications,Type=\Cours,Origin=] ef. 
Soient $A$, $B$, $C$, $D$ des ensembles et $f:A\to B$, $g:B\to C$ et $h:C\to D$ des applications. Montrer que : \pn
a) $g\circ f$ injective $\Longrightarrow$ $f$ injective. \pn
b) $g\circ f$ surjective $\Longrightarrow$ $g$ surjective. \pn
c) $h\circ g$ et $g\circ fi$ bijectives $\Longleftrightarrow$ $h$, $g$ et $f$ bijectives. 

\exo [Level=1,Fight=1,Learn=1,Field=\Applications,Type=\Colles,Origin=] eg. 
Soient $E$, $F$ deux ensembles et $f:E\to F$ une application. 
On note respectivement $\sc P(E)$ et $\sc P(F)$ l'ensemble des parties de $E$ et $F$ et on considère les deux applications 
$$
g:\App\sc P(E)\to\sc P(F), A\mapsto f(A)\ppA\qquad \hbox{et}\qquad h:\App\sc P(F)\to\sc P(E), A\mapsto f^{-1}(A)\ppA. 
$$
1) Prouver que $f$ injective $\Longleftrightarrow$ $g$ injective. \pn
2) Prouver que $f$ surjective $\Longleftrightarrow$ $h$ injective. 

\exo [Level=1,Fight=0,Learn=0,Field=\Applications,Type=\Cours,Origin=] eh. 
a) L'application $\ds
f:\App \ob R^2\to\ob R^2,(x,y)\mapsto (x+y,x-y)\ppA
$
est elle injective, surjective ?\medskip\noindent
b) Même question pour l'application $\ds
g:\App \ob R^2\to\ob R^2,(x,y)\mapsto(x+y,xy)\ppA
$

\exo [Level=1,Fight=1,Learn=2,Field=\Applications,Type=\Cours,Origin=] ei. 
Soient $E$, $F$ deux ensembles, soit $f:E\to F$ une application. \pn
a) Pour $A\subset B\subset F$, prouver que $f^{-1}(A)\subset f^{-1}(B)$. \pn 
b) Pour $A\subset F$ et $B\subset F$, prouver que 
$$
f^{-1}(A)\cup f^{-1}(B)=f^{-1}(A\cup B)\qquad \hbox{et que }\qquad f^{-1}(A\cap B)=f^{-1}(A)\cap f^{-1}(B).
$$
c) Pour $A\subset F$, prouver que $f^{-1}(F\ssm A)=E\ssm f^{-1}(A)$. 

\exo [Level=1,Fight=1,Learn=1,Field=\Applications,Type=\Exercices,Origin=] ej. 
Démontrer que la fonction $\ds x\mapsto {1+x^2\F 1-x^2}$ est une bijection de $\Q]-1,1\W[$ dans $\Q[1,+\infty\W[$. 

\exo [Level=1,Fight=1,Learn=1,Field=\Applications,Type=\Exercices,Origin=] ek. 
Démontrer que l'application $\eqalign{
f:\Q]0,+\infty\W[^2&\to\Q]0,+\infty\W[^2\cr
(u,v)&\mapsto\Q(uv,{u\F v}\W)\cr
}$ est une bijection. 

\exo [Level=1,Fight=1,Learn=1,Field=\Applications,Type=\Cours,Origin=] el. 
Prouver que $\ds\eqalign{f: \{(x,y)\in\ob R^2:x>0\}&\to\Q]0,+\infty\W[\times\Q]-{\pi\F 2},{\pi\F 2}\W[\cr(x,y)&\mapsto\Q(\sqrt{x^2+y^2},\arctan\Q({y\F x}\W)\W)\cr}$
est bijective. 

\exo [Level=1,Fight=0,Learn=0,Field=\EquationsDifférentiellesLinéairesDuSecondOrdre,Type=\Maple,Origin=] em. 
Soit $f:\ob R\to\ob R$ la fonction définie par 
$$
f(x):=\Q\{\eqalign{{\sh\Q(\sqrt{-x}\W)\F\sqrt{-x}}&\quad\hbox{si} x<0,\cr
1&\quad\hbox{si }x=0,\cr
{\sin\Q(\sqrt x\W)\F\sqrt x}&\quad\hbox{si }x>0}\W.
$$
a) Montrer que $f$ est continue sur $\ob R$ en calculant ses limites en $0^-$ et en $0^+$. \pn
b) De même, montrer que $f$ est deux fois dérivable, de dérivée continue sur $\ob R$. \pn
c) Dessiner le graphe de la fonction $f$ sur $[-2,2]$. \pn
d) Trouver une équation linéaire du type $ay''+by'+cy=d$ dont $f$ est solution. 

\exo [Level=1,Fight=1,Learn=1,Field=\DéveloppementsLimités,Type=\Cours,Origin=] en. 
Calculer la limite lorsque $x$ tends vers $+\infty$ de 
$$
u(x)=\Q((x^3+ax^2+2)^{1/3}-(x^3+1)^{1/3}\W)^x
$$
lorsque $a>0$ et lorsque $a=0$. 
 
\exo [Level=1,Fight=1,Learn=1,Field=\CourbesParamétréesCartésiennes,Type=\Exercices,Origin=] eo. 
Soit $\lambda\in \ob R$ et soit $f_\lambda$ la fonction définie par  
$$
\forall x\in\ob R^*,\qquad f_\lambda(x):=(x-\lambda)\e^{1/x}.
$$
a) Étudier $f_\lambda$ au voisinage de $0$.\pn
b) Étudier les branches infinies de la fonction $f_\lambda$.  \pn
c) Calculer la dérivée de $f_\lambda$ et étudier son signe. \pn
d) Si $\lambda<\mu$, que peut on dire des graphes des fonctions $f_\lambda$ et $f_\mu$. \pn
e) Dessiner sur un même graphique les graphiques de plusieurs fonction $f_\lambda$. 


\exo  [Level=1,Fight=0,Learn=0,Field=\CourbesParamétréesCartésiennes,Type=\Exercices,Origin=] ep. 
Étudier la fonction définie par 
$$ 
h:x\mapsto\arcsin\Q({x+1\F\sqrt2\sqrt{x^2+1}}\W)
$$ 
puis tracer son graphe. 


\exo  [Level=1,Fight=0,Learn=0,Field=\CourbesParamétréesCartésiennes,Type=\Exercices,Origin=] eq. 
Étudier et tracer la courbe $\ds t\mapsto \B(\cos t,\sin t(1+\cos t)\B)$. 

\exo [Level=1,Fight=0,Learn=0,Field=\CourbesParamétréesCartésiennes,Type=\Exercices,Origin=] er.  
Étudier et tracer la courbe $\ds x=\cos(4t)+4\cos t$ et $y=\sin(3t)$. 

\exo  [Level=1,Fight=0,Learn=0,Field=\CourbesParamétréesCartésiennes,Type=\Exercices,Origin=] es.  
Étudier et tracer la courbe $\ds \vec{OM(t)}=(t-\sin t)\vec i+(1-\cos t)\vec j$. 

\exo  [Level=1,Fight=0,Learn=0,Field=\CourbesParamétréesCartésiennes,Type=\Exercices,Origin=] et.  
Étudier et tracer la courbe $\ds x=2\cos t+\cos (2t)$ et $y=2\sin t-\sin(2t)$. 

\exo  [Level=1,Fight=0,Learn=0,Field=\CourbesParamétréesCartésiennes,Type=\Exercices,Origin=] eu.  
Étudier et tracer la courbe $\ds x=\sin t$ et $\ds y={\cos^2t\F2-\cos t}$.

\exo  [Level=1,Fight=0,Learn=0,Field=\CourbesParamétréesCartésiennes,Type=\Exercices,Origin=] ev.  
Étudier et tracer la courbe $\ds t\mapsto\Q({t-\sin t\F t^2},{1-\cos t\F t^2}\W)$. 
 
\exo  [Level=1,Fight=0,Learn=0,Field=\CourbesParamétréesCartésiennes,Type=\Exercices,Origin=] ew.  
Étudier et tracer la courbe $\ds x=t^2+{2\F t}$ et $y=t^2+{1\F t^2}$. Etude des points doubles. 

\exo  [Level=1,Fight=0,Learn=0,Field=\CourbesParamétréesCartésiennes,Type=\Exercices,Origin=] ex.  
Étudier et tracer la courbe $$
\Q\{\eqalign{
&x=3\cos t+3\cos (2t)+\cos(3t)
\cr
&y=3\sin t+3\sin(2t)+\sin(3t)}\W.
$$ 

\exo  [Level=1,Fight=0,Learn=0,Field=\CourbesParamétréesCartésiennes,Type=\Exercices,Origin=] ey.  
Étudier et tracer la courbe $t\mapsto(\cos^3t,\sin^3t)$. 

\exo  [Level=1,Fight=0,Learn=0,Field=\CourbesParamétréesPolaires,Type=\Exercices,Origin=] ez.  
Étudier et tracer la courbe polaire $\ds\rho=1+\tan{\theta\F2}$. 

\exo [Level=1,Fight=0,Learn=0,Field=\CourbesParamétréesPolaires,Type=\Exercices,Origin=] fa. 
Étudier et tracer la courbe polaire $\ds\rho=\cos\theta\cos(2\theta)$. 

\exo [Level=1,Fight=0,Learn=0,Field=\CourbesParamétréesPolaires,Type=\Exercices,Origin=] fb. 
Étudier et tracer la courbe polaire $\ds\rho=1-\sin{\theta\F2}$. 

\exo [Level=1,Fight=1,Learn=0,Field=\CourbesParamétréesPolaires,Type=\Exercices,Origin=] fc. 
Étudier et tracer la courbe polaire $\ds\rho={\sin(3\theta/2)\F1-2\cos\theta}$. 

\exo [Level=1,Fight=2,Learn=1,Field=\CourbesParamétréesPolaires,Type=\Exercices,Origin=] fd. 
Étudier et tracer la courbe polaire $\ds\rho={1\F \sqrt{1+\sin(2\theta)}+\sqrt{1-\sin(2\theta)}}$. Quelle figure obtient-on ?

\exo [Level=1,Fight=0,Learn=0,Field=\CourbesParamétréesPolaires,Type=\Exercices,Origin=] fe. 
Comportement au voisinage de $0$ de $\ds\rho={\sin^3\theta\F1+\cos^2\theta}$.  

\exo [Level=1,Fight=0,Learn=0,Field=\GéométriePlane,Type=\Exercices,Origin=] ff. 
Etude des Ovales de Cassini : Lieu des points dont le produit des distances à deux points fixes (appelés) foyers est constant. 

\exo [Level=1,Fight=0,Learn=0,Field=\GéométriePlane,Type=\Exercices,Origin=] fg.  
Soient deux points fixes $F$ et $F'$ et $a>0$. Etude des Ovales de Descartes : les lieux des points $M$ tels que $|MF\pm MF'|=2a$. 

\exo [Level=1,Fight=0,Learn=0,Field=\CourbesParamétréesPolaires,Type=\Exercices,Origin=] fh. 
Etude de la cochléoïde $\ds \rho={\sin\theta\F\theta}$. 

\exo [Level=1,Fight=2,Learn=1,Field=\CourbesParamétréesPolaires,Type=\Exercices,Origin=] fi. 
On considère un lima\c con de Pascal d'équation polaire $\rho=a+b\cos\theta$. Une droite coupe le lima\c con en quatres points de rayons $\rho_1$, $\rho_2$, $\rho_3$ et $\rho_4$. \pn
a) Calculer $\rho_1+\rho_2+\rho_3+\rho_4$. \pn
b) Démontrer que  le produit $\rho_1\rho_2\rho_3\rho_4$ est constant lorsque la droite est tangente à un cercle fixe de centre~$O$. 


\exo [Level=2,Fight=2,Learn=1,Field=\EquationsDifférentiellesLinéairesDuSecondOrdre,Type=\Exercices,Origin=] fj. 
Soit $\ds\theta\in\Q]0, {\pi\F2}\W[$. Résoudre l'équation différentielle
$$
y''\cos^2(\theta)-y'\sin(2\theta)+y=x\cos^2(\theta)\e^{x\tan\theta}.$$Puis, déterminer la solution $\varphi$ qui s'annule ainsi que sa dérivée en $x=0$. Enfin, calculer les primitives de $\phi$. 

\exo [Level=2,Fight=2,Learn=1,Field=\EquationsDifférentiellesLinéairesDuSecondOrdre,Type=\Exercices,Origin=,Indication={On pourra remarquer que la fonction $g=y''-y$ satisfait une équation différentielle plutôt sympathique.}] fk. 
Résoudre l'équation différentielle
$$
y'''+y''-y'-y=x\e^{-x}
$$

\exo [Level=1,Fight=1,Learn=1,Field=\EquationsDifférentiellesLinéairesDuPremierOrdre,Type=\Exercices,Origin=,Indication={On procédera au changement de fonction inconnue $f(x)=y(x)^{-3}$.}] fl. 
Trouver les solutions ne s'annulant pas de l'équation différentielle de Bernouilli
$$
5y'(x)-y(x)\sin(x)+y(x)^4\sin(x)=0.
$$

% Redondant % fm %

\exo [Level=2,Fight=1,Learn=2,Field=\EquationsDifférentiellesLinéairesDuSecondOrdre,Type=\TravauxDirigés,Origin=] fn. Dans cet exercice, nous cherchons à résoudre l'équation d'Euler
$$
x^2f''(x)+3xf'(x)+f(x)={\ln x\F x}\qquad(x>0).\leqno{(*)}
$$
en procédant au changement de variable (bijectif) $$
x=\e^t,
$$ 
dont nous détaillons les étapes, car il y a quelques subtilités théoriques. 
En bref, nous cherchons les solutions $f$ de l'équation différentielle~$(*)$ sous la forme $$
f(x)=g(t).
$$
\medskip
\noindent
a) Soit $f:\Q]0,+\infty\W[\to\ob R$ une fonction de classe $\sc C^2$. On pose 
$$
\forall t\in\ob R, \qquad g(t):=f(\e^t).
$$
Justifier que l'application $g$ est définie et de classe $\sc C^2$ sur $\ob R$. \pn
b) Exprimer $f(x)$ en fonction de $g$ et de $x>0$. \pn
c) En déduire l'expression de $f'(x)$ et de $f''(x)$ en fonction de $g$, $g'$, $g''$ et $x>0$. \pn
d) En reportant ces expressions dans $(*)$ puis en substituant $\e^t$ à $x$, prouver que l'équation différentielle $(*)$ est équivalente à
$$
\forall t\in\ob R, \qquad g''(t)+2g'(t)+g(t)=t\e^{-t}.\leqno{(**)}
$$
e) Résoudre $(**)$ et en déduire toutes les solutions $f$ de $(*)$. 

\exo [Level=2,Fight=2,Learn=2,Field=\EquationsDifférentiellesLinéairesDuSecondOrdre,Type=\TravauxDirigés,Origin=] fo. Soit $m\in\ob R$. En procédant au changement de variable $x=\tan(t)$, résoudre l'équation différentielle
$$
\forall x\in\ob R, \qquad (1+x^2)^2f''(x)+2x(1+x^2)f'(x)+mf(x)=0.$$


\exo [Level=1,Fight=1,Learn=2,Field=\EquationsDifférentiellesLinéairesDuPremierOrdre,Type=\TravauxDirigés,Origin=] fp. On cherche à résoudre l'équation différentielle
$$
\forall x>1, \qquad 2f'(x)+{x\F x^2-1}f(x)-xf(x)^3=0.\leqno{(\dag)}
$$
a) Vérifier que la fonction nulle est solution de $(\dag)$. \pn 
b) En cherchant les solutions de $(\dag)$ sous la forme $f(x)={1\F\sqrt{g(x)}}$, quelles solutions nous échapent  ? 
\pn
c) En procédant au changement de fonction inconnue $f(x)={1\F\sqrt{g(x)}}$, montrer que $g$ satisfait l'équation 
$$
-g'(x)+{x\F x^2-1}g(x)-x=0. \leqno{(\ddag)}
$$
d) Résoudre l'équation différentielle $(\ddag)$ et en déduire {\bf des} solutions $f$ de $(\dag)$. 


\exo [Level=1,Fight=2,Learn=2,Field=\EquationsDifférentiellesAVariablesSéparables,Type=\TravauxDirigés,Origin=] fq. 
Résoudre l'équation différentielle à variables séparables
$$
\forall x\in\ob R, \qquad y'\sqrt{1+x^2}=\sqrt{1+y^2}
$$

\exo [Level=1,Fight=1,Learn=1,Field=\EquationsDifférentiellesLinéairesDuPremierOrdre,Type=\TravauxDirigés,Origin=] fr. 
Trouver toutes les solutions $f$ de classe $\sc C^1$ de l'équation différentielle
$$
x-y+xy'=0
$$
a) sur l'intervalle $I=\Q]0,+\infty\W[$. \pn
b) sur l'intervalle $I=\ob R$. 

\exo [Level=2,Fight=1,Learn=1,Field=\EquationsDifférentiellesLinéairesDuSecondOrdre,Type=\Exercices,Origin=] fs. 
La fonction $\ds x\mapsto {x\F x-1}$ est solution de l'équation différentielle 
$$
x^2(1-x)y''-x(1+x)y'+y=0.
$$
Trouver toutes solutions de cette équation différentielle. 

\exo [Level=1,Fight=0,Learn=0,Field=\CourbesParamétréesCartésiennes,Type=\Exercices,Origin=] ft. 
a) Tracer la courbe paramétrée
$$
\Q\{\eqalign{
x(t)=t^3-4t,\cr
y(t)=2t^2-3}
\W.
$$
b) Calculer l'angle des tangentes au point double. 

\exo [Level=1,Fight=2,Learn=0,Field=\Polynômes,Type=\Exercices,Origin=]  fu. 
a) Déterminer les zéros du polynôme $P=X^4-X^3-3X^2+X+1$. \pn
b) Déterminer les zéros de $Q=X^4-X^3-2X^2-2X-1$ et le décomposer en facteurs irréductibles. \pn
c) Donner les valeurs de $a$ et $b$ pour que $2+2i$ soit racine de $P=X^4+aX^3+\sqrt3X^2+X+b$. Factoriser $P$. 

\exo [Level=1,Fight=1,Learn=1,Field=\Polynômes,Type=\Maple,Origin=] fv. 
Ecrire une procédure  ``Poly'' (récursive et comportant un test) permettant de calculer $P_{11}$, les polynomes $P_n$ étant définis par
$$
T_0:=1, \quad T_1:=X\quad\hbox{et}\quad T_n:=2XT_{n-1}-T_{n-2}\qquad (n\ge2).
$$

\exo [Level=1,Fight=0,Learn=0,Field=\CourbesParamétréesPolaires,Type=\Exercices,Origin=] fw. 
Etude et nature de la courbe paramétrée 
$$
\rho:={\sin(\theta)\F\cos(2\theta)}.
$$


\exo  [Level=1,Fight=0,Learn=0,Field=\CourbesParamétréesCartésiennes,Type=\Exercices,Origin=] fx. 
Branche infinies, points doubles de la couirbe paramétrée 
$$
\Q\{\eqalign{
x(t)={t-1\F t^2-4},\cr
y(t)={t^2-3\F t+2}}
\W.
$$

\exo [Level=1,Fight=1,Learn=1,Field=\EquationsDifférentiellesLinéairesDuPremierOrdre,Type=\Exercices,Origin=] fy. 
Résoudre l'équation différentielle 
$$
(1+x^2)y'-2xy=x^3.
$$
Trouver la solution $y(x)$ telle que la courbe $(x,y(x))$ passe par le point $(0,n)$ ou $0\le n\le 5$. 

\exo [Level=1,Fight=1,Learn=1,Field=\EquationsDifférentiellesLinéairesDuPremierOrdre,Type=\Exercices,Origin=] fz. 
Résoudre l'équation différentielle 
$$
xy'+(1-x)y={x\e^x\F x^4+1}
$$
Tracer quelques courbes intégrales $x\mapsto (x,y(x))$. Solutions continues sur $\ob R$ ?

\exo [Level=1,Fight=0,Learn=0,Field=\Coniques,Type=\Exercices,Origin=] ga. 
Soit $\sc P$ un plan affine euclidien muni d'un repère orthonormé $(O,\vec i,\vec j)$. \pn
a) Déterminer l'ensemble $\sc E$ des points $F$ tels que $F$ soit le foyer d'une conique $\sc C_F$ de directrice l'axe $(Ox)$ passant par les points $A(-1,2)$ et $B(1,1)$. \pn
b) Pour chaque point $F$ de $\sc E$, préciser la nature de la conique $\sc C-F$. 

\exo  [Level=1,Fight=0,Learn=0,Field=\GéométriePlane,Type=\Exercices,Origin=] gb. 
Soient $A$ et $B$ deux points distincts du plan et $I$ le milieu du segment $[AB]$. \pn 
Déterminer l'ensemble des points $M$ du plan vérifiant $MI^2=MA\times MB$. 

\exo  [Level=1,Fight=1,Learn=0,Field=\Coniques,Type=\Exercices,Origin=] gc. 
Lorsque $\lambda$ décrit $\Q]0,+\infty\W[$, trouver le lieu des sommets et des foyers de l'ellipse d'équation $\lambda x^2+y^2-2x=0$. 

\exo  [Level=1,Fight=1,Learn=0,Field=\GéométriePlane,Type=\Exercices,Origin=] gd. 
Dans un plan affine euclidien $\sc P$ muni d'un repère orthonormé $(O,\vec i,\vec j)$, on  note $\sc H$ l'hyperbole équilatère d'équation cartésienne $xy=1$. 
Soient $A$, $B$, $C$ trois points de $\sc H$, d'abscisse respective $a$, $b$ et $c$. 
Montrer que l'orthocentre (intersection des hauteurs) du triangle $ABC$ est sur $\sc H$ et donner ses coodonnées. 

\exo  [Level=1,Fight=0,Learn=0,Field=\Coniques,Type=\Exercices,Origin=] ge. 
Déterminer l'ensemble des projections orthogonales d'un foyer sur les tangentes d'une ellipse $\sc E$. 


\exo [Level=1,Fight=1,Learn=0,Field=\CourbesParamétréesPolaires,Type=\Exercices,Origin=] gf. 
Déterminer la nature des courbes d'équation polaire
$$
a:\qquad \rho(\theta)={2\F 2+\cos\theta+\sin\theta}\qquad \qquad b:\qquad \rho(\theta)={2\F 2+\cos\theta+\sin\theta}
$$

\exo  [Level=1,Fight=0,Learn=0,Field=\Coniques,Type=\Exercices,Origin=] gg. 
Déterminer la nature, les éléments caractéristiques puis tracer la conique d'équation $y^2+3x-4y=2$. 

\exo  [Level=1,Fight=0,Learn=0,Field=\Coniques,Type=\Exercices,Origin=] gh. 
Déterminer la nature, les éléments caractéristiques puis tracer la conique d'équation $x^2-2y^2+x-2y=0$. 

\exo  [Level=1,Fight=0,Learn=0,Field=\Coniques,Type=\Exercices,Origin=] gi.  
Déterminer les éléments caractéristiques puis tracer la conique d'équation $x^2+xy+y^2-4x-5y+2=0$. 

\exo  [Level=1,Fight=0,Learn=0,Field=\Coniques,Type=\Exercices,Origin=] gj. 
Soit $\sc P$ un plan muni d'un repère orthonormé $(O,\vec i,\vec j)$. Déterminer l'équation dans ce repère de la parabole de foyer $(5,2)$ et de directrice $\sc D:x-1=0$. 

\exo   [Level=1,Fight=0,Learn=0,Field=\Coniques,Type=\Exercices,Origin=] gk. 
Soit $\sc P$ un plan muni d'un repère orthonormé $(O,\vec i,\vec j)$. Déterminer l'équation dans ce repère de l'ellipse de foyers $(\pm1,0)$ et d'exentricité ${\ds1\F2}$. 

\exo   [Level=1,Fight=0,Learn=0,Field=\Coniques,Type=\Exercices,Origin=] gl. 
Soit $\sc P$ un plan muni d'un repère orthonormé $(O,\vec i,\vec j)$. Déterminer l'équation dans ce repère de l'hyperbole de directrice d'équations $x=\pm3$ et d'exentricité $2$. 

\exo  [Level=1,Fight=1,Learn=0,Field=\Coniques,Type=\Exercices,Origin=] gm. 
La normale et la tangente en un point $M$  à une parabole $\sc P$ coupent l'axe focal respectivement en $N$ et en $T$. Déterminer le lieu du point $P_M$ tel que $NMTP_M$ soit un rectangle lorsque $M$ décrit $\sc P$. 
Que peut on dire des abscisses des points $M$ et $N$ ?

\exo  [Level=1,Fight=0,Learn=0,Field=\Coniques,Type=\Exercices,Origin=] gn. 
A chaque point $M$ d'une ellipse $\sc E$, on associe la projection orthogonale $I$ de $M$ sur l'axe focal, on associe l'un des deux points $P\in\sc E$ en lequel la tangente à l'ellipse $\sc E$ est parallèle à la droite $(OM)$ et on associe la projection orthogonale $J$ du point $P$ sur l'axe focal. \pn
a) Calculer l'aire du triangle $MOP$ (premier théorème d'Apollonius). \pn
b) Calculer $OM^2+OP^2$ (second théorème d'Apollonius) ainsi que $OI^2+OJ^2$. 

\exo  [Level=1,Fight=0,Learn=0,Field=\Coniques,Type=\Exercices,Origin=] go. 
Soit $\sc E$ une ellipse de foyer $F$ et soit $\Delta$ une droite passant par $F$ et coupant $\sc E$ en deux points $M$ et $N$. \pn
a) Prouver que $\ds {1\F FM}+{1\F FN}$ est indépendant de la droite variable $\Delta$. \pn
b) Déterminer le minimum de $\ds {1\F FM^2}+{1\F FN^2}$ lorsque $\Delta$ varie. 

\exo  [Level=1,Fight=0,Learn=0,Field=\Coniques,Type=\Exercices,Origin=,Indication={On pourra travailler dans le repère de centre $M$ d'axes la tangente et la normale en $M$ à la parabole.}] gp. 
Soit $\sc P$ un miroir parabolique de foyer $F$, d'axe focal $\Delta$ et de paramètre $p$.
Prouver que le reflet d'un rayon lumineux parallèle à $\Delta$ en un point $M$ de la parabole passe par le foyer $F$. 

\exo  [Level=1,Fight=0,Learn=0,Field=\Coniques,Type=\Exercices,Origin=,Indication={On pourra travailler dans le repère de centre $M$ d'axes la tangente et la normale en $M$ à l'ellipse.}] gq. 
Soit $\sc E$ un billard elliptique de foyer $F$ et $F'$ , d'axe focal $\Delta$ et de paramètre $p$.
Prouver qu'une boule partant du foyer $F$ et faisant un rebond  en un point $M$ de l'ellispe $\sc E$ passe nécéssairement par le foyer $F'$ (si elle est lancée assez fort, pour les esprits chagrins). 

\exo  [Level=1,Fight=0,Learn=0,Field=\Coniques,Type=\Exercices,Origin=] gr. 
Dans un plan affine euclidien $\sc P$ muni d'un repère orthonormé $(O,\vec i,\vec j)$, on  note $\sc H$ l'hyperbole équilatère d'équation cartésienne $xy=1$. \pn
Soient $A$, $B$, $C$ trois points de l'hyperbole $\sc H$, d'abscisse respective $a$, $b$ et $c$. \pn
a) Montrer que l'orthocentre (intersection des hauteurs) du triangle $ABC$ est sur l'hyperbole $\sc H$ et déterminer ses coodonnées. \pn
b) Soit $\alpha$ la perpendiculaire à $(BC)$ passant par  l'intersection de $(BC)$ et de $(Ox)$,  soit $\beta$ la perpendiculaire à $(AC)$ passant par  l'intersection de $(AC)$ et de $(Ox)$ et soit 
$\gamma$ la perpendiculaire à $(AB)$ passant par  l'intersection de $(AB)$ et de $(Ox)$. Prouver que les droites $\alpha$, $\beta$ et $\gamma$ sont concourantes en un point $I$. \pn
c) Donner les coordonnés du point $I$ dans le repère $(O,\vec i,\vec j)$. \pn
d) Mêmes questions que b) et c) pour les droites $\alpha'$, $\beta'$, $\gamma'$ définies avec $(Oy)$ à la place de $(Ox)$ et avec $I'$ à la place de $I$. \pn
e) Calculer $\vec{AI}.\vec{AI'}$, $\vec{BI}.\vec{BI'}$ et $\vec{CI}.\vec{CI'}$. \pn
f) Ecrire l'équation du cercle circonscrit $\sc C$ au triangle $ABC$ et démontrer que $\sc C$ coupe l'hyperbole $\sc H$ en un point $H'$ qui est le symétrique de $H$ par rapport à l'origine $O$. 

\exo [Level=1,Fight=2,Learn=2,Field=\EquationsDifférentiellesLinéairesDuPremierOrdre,Type=\Problèmes,Origin=\Lakedaemon] gs. 
Soit $f:\ob R\to \ob R$ une fonction dérivable vérifiant $f(0)=0$ et 
$$
\forall x\in\ob R, \qquad \ch(x)f'(x)+\arctan(x)f(x)=\cos(x). \leqno{(E)}
$$
a) Prouver que $g:x\mapsto -f(-x)$ est aussi une solution de $(E)$. \pn
b) En déduire que l'application $f$ est impaire. \pn
c) Soit $h:\ob R\to\ob R$ une fonction dérivable telle que l'application~$\phi$ définie~par  
$$
\forall x\in\ob R, \qquad \phi(x)=2xh'(x)-h(x). 
$$ 
soit continue et paire.  Montrer que l'application $k:x\mapsto h(-x)$ vérifie la relation 
$$
\forall x\in\ob R, \qquad  \phi(x)=2xk'(x)-k(x). 
$$
d) Exprimer $k(x)$ en fonction de $h(x)$ pour $x<0$ et pour $x>0$. \pn
e) En déduire que l'application $h$ est paire. 

\exo  [Level=1,Fight=2,Learn=2,Field=\EquationsDifférentiellesAVariablesSéparables,Type=\Problèmes,Origin=\Lakedaemon] gt. 
a) Pour $\alpha\in\ob R$, déterminer les solutions sur $I=\Q]-\infty,\alpha\W[$ de l'équation différentielle
$$
\forall x\in I, \qquad y'=\e^{x+y}.\leqno{(*)}
$$
b)  Prouver que l'équation $(*)$ n'admet aucune solution sur $I=\ob R$. 

\exo [Level=1,Fight=0,Learn=0,Field=\CourbesParamétréesCartésiennes,Type=\Exercices,Origin=] gu. 
Étudier et tracer la courbe paramétrée donnée par 
$$
\Q\{\eqalign{
x&=\cos(t),\cr
y&=\sin(2t).}\W. 
$$


\exo [Level=1,Fight=1,Learn=1,Field=\EquationsDifférentiellesAVariablesSéparables,Type=\Exercices,Origin=] gv. 
Résoudre l'équation différentielle $2x^2y'+y^2=1$. 

\exo [Level=1,Fight=0,Learn=0,Field=\CourbesParamétréesPolaires,Type=\Exercices,Origin=] gw.  
Étudier et tracer la courbe polaire $\ds\rho=1+2\cos(3\theta)$. 

\exo [Level=1,Fight=0,Learn=0,Field=\CourbesParamétréesPolaires,Type=\Exercices,Origin=] gx.  
Étudier et tracer la courbe polaire $\ds\rho=\ln\b(|\cos\theta|\b)$. 

\exo [Level=1,Fight=0,Learn=0,Field=\CourbesParamétréesPolaires,Type=\Exercices,Origin=] gy.  
Étudier et tracer la courbe polaire $\ds\rho={\cos\theta\F\cos\theta+\sin\theta}$. 

\exo [Level=1,Fight=0,Learn=0,Field=\CourbesParamétréesPolaires,Type=\Exercices,Origin=] gz.  
Étudier et tracer la courbe polaire $\ds\rho=\sin^3\Q({\theta\F3}\W)$. 

\exo [Level=1,Fight=0,Learn=0,Field=\CourbesParamétréesPolaires,Type=\Exercices,Origin=] ha.  
Étudier et tracer la courbe polaire $\ds\rho=1+2\cos(3\theta)$. 

\exo [Level=1,Fight=0,Learn=0,Field=\CourbesParamétréesPolaires,Type=\Exercices,Origin=] hb.  
Étudier et tracer la courbe polaire $\ds\rho={1\F\cos(\theta/2)}$. 

\exo [Level=1,Fight=0,Learn=0,Field=\CourbesParamétréesCartésiennes,Type=\Exercices,Origin=] hc.  
Étudier et tracer la courbe paramétrée 
$$
\Q\{
\eqalign{
\ds x(t)=t+{1\F t},\cr
\ds y(t)=\ln(t).
}
\W.
\qquad(t>0).
$$ 
On commencera par chercher une symétrie... 

\exo  [Level=1,Fight=1,Learn=0,Field=\GéométriePlane,Type=\Exercices,Origin=] hd. 
Déterminer et construire le lieu géométrique $\Gamma$ des points d'ou l'on peut mener deux tangentes perpendiculaires à la courbe
$$
\Q\{
\eqalign{
x(t)=3t^2,\cr
y(t)=2t^3}
\W.\qquad(t\in\ob R)\leqno{(\sc C)}
$$
L'ensemble $\Gamma$ est appelé la courbe orthoptique de $\sc C$. La construire. 

\exo Level=1,Fight=2,Learn=1,Field=\EquationsDifférentiellesLinéairesDuSecondOrdre,Type=\Exercices,Origin=] he. 
Trouver les toutes les applications dérivables $f:\ob R\to\ob R$ telles que 
$$
\forall x\in\ob R, \qquad f'(x)+f(-x)=x\e^x.
$$ 

\exo [Level=1,Fight=1,Learn=1,Field=\EquationsDifférentiellesLinéairesDuPremierOrdre,Type=\Exercices,Origin=] hf. 
Résoudre sur $\ob R$ l'équation différentielle $y'+y=|x|$. 

\exo [Level=1,Fight=1,Learn=1,Field=\EquationsDifférentielles,Type=\Exercices,Origin=] hg. Trouver toutes les applications dérivables sur $\ob R$ vérifiant
$$
\forall x\in\ob R,\qquad f'(x)f(-x)=1.
$$
On pourra considérer l'application $g:x\mapsto f(x)f(-x)$.

\exo [Level=1,Fight=0,Learn=0,Field=\CourbesParamétréesPolaires,Type=\Exercices,Origin=] hh. 
On considère le {\it Folium de Descartes} $\sc F$ d'équation polaire
$$
\rho:={\sin(\theta)\cos(\theta)\F \sin(\theta)-\cos(\theta)}.\leqno{(\sc F)}
$$
a) Déterminer $\sc D\rho$ et réduire l'ensemble d'étude. 
\bigskip
\noindent
b) Étudier le signe et les variations de $\rho$. Faire figurer les tangentes dans le tableau de variation.  
\bigskip
\noindent
c) Branches infinies et position de $\sc F$ par rapport à ses asymptotes ? 
\bigskip
\noindent
d) Tracer la courbe. 


\exo [Level=1,Fight=3,Learn=3,Field=\Suites,Type=\Problèmes,Origin=,Indication={d) On pourra calculer $b_{n+1}+b_n$. \pn
e) Faire intervenir $a_{2n+1}$ et $a_{2n}$. \pn
j) On pourra considérer le nombre d'éléments de $A_n\cup B_n$.}]  hi.  
Le nombre d'or étant noté $\varphi:=\ds{1+\sqrt5\F2}$, pour $x\in\ob R$, on rappelle que le symbole~$[x]$ désigne la partie entière de $x$, i.e. l'unique entier $n$ vérifiant 
$$
n\le x<n+1.
$$ 
L'objectif de ce problème est d'établir que les ensembles 
$$
A:=\B\{[n\varphi]:n\in\ob N^*\B\}\qquad \hbox{et}\qquad B:=\B\{[n\varphi^2]:n\in\ob N^*\B\}
$$ 
forment une partition de l'ensemble $\ob N^*$, autrement dit que  
$$
A\neq\emptyset, \qquad B\neq\emptyset,  \qquad A\cap B=\emptyset\quad\hbox{et}\quad A\cup B=\ob N^*,
$$
ce qui est une propriété remarquable du nombre d'or. 
\medskip
\noindent
On rappelle que la suite (des entiers) de Fibonacci est définie par $F_0=0$,~$F_1=1$~et  
$$
\forall n\in\ob N, \qquad F_{n+2}:=F_{n+1}+F_n.
$$
a) {\it (2 points)} Pour $n\in\ob N$, montrer que $F_n\ge n-1$. \pn
b) {\it (1 point)} Exprimer $\varphi^2$ et $\ds{1\F\varphi}$ en fonction de $\varphi$. \pn
c) {\it (2 points)} Pour $n\in\ob N^*$, montrer que $a_n:=\ds{\varphi F_{n+1}+F_n\F\varphi F_n+F_{n-1}}$ est égal à $\varphi$. \smallskip\noindent
d) {\it (2 points)} Pour $n\in\ob N^*$, exprimer en fonction de $n$ le nombre  
$$
b_n:=F_n^2-F_{n+1}F_{n-1}
$$ 
e) {\it (3 points)} Pour $n\in\ob N^*$, prouver que 
$$
{F_{2n+2}\F F_{2n+1}}<\varphi<{F_{2n+1}\F F_{2n}}
$$
f) {\it (2 points)} Pour $n\in\ob N$, montrer que 
$$
0<\varphi F_{2n+1}-F_{2n+2}<{1\F F_{2n}}.
$$
et en déduire une expression de la partie entière $[\varphi F_{2n+1}]$ en fonction d'un terme de la suite de Fibonacci. \pn
g) {\it (2 points)} En procédant comme en f), calculer $[\varphi F_{2n}]$ et en déduire $[\varphi^2F_{2n}]$. \pn
h) {\it (5 points)} En utilisant que $\ds {1\F \varphi}+{1\F\varphi^2}=1$ et que $\varphi$ est irrationnel, prouver que 
$$
\forall p\in\ob N^*, \qquad \forall q\in\ob N^*, \qquad [p\varphi]\neq[q\varphi^2]
$$
Que peut-on en déduire pour les ensembles $A$ et $B$ ?\pn
i)  {\it (2 points)} Pour $p\in\ob N^*$ et $q\in\ob N^*$,  montrer que 
$$
p<q\Longrightarrow [p\varphi]<[q\varphi].
$$ 
j)  {\it (2 points)} Pour $n\in\ob N^*$, déterminer la réunion des ensembles 
$$
A_n:=\B\{[k\varphi]:1\le k\le F_{2n+1}\B\}\qquad\hbox{et}\qquad B_n:=\B\{[k\varphi^2]:1\le k\le F_{2n}\B\}. 
$$
k)  {\it (2 points)} En déduire que $A$ et $B$ forment une partition de l'ensemble $\ob N^*$. 

\exo [Level=1,Fight=0,Learn=0,Field=\CourbesParamétréesPolaires,Type=\Exercices,Origin=] hj.  
Étudier et tracer la courbe polaire $\ds\rho=1+2\cos\theta-4\cos^2\theta$. 

\exo [Level=1,Fight=0,Learn=0,Field=\CourbesParamétréesPolaires,Type=\Exercices,Origin=] hk. 
Étudier et tracer la courbe polaire $\ds\rho={\sin(3\theta)\F \sin(\theta)}$. 

\exo [Level=1,Fight=0,Learn=0,Field=\CourbesParamétréesCartésiennes,Type=\Exercices,Origin=] hl. 
Déterminer les points multiples de la courbe paramétrée par 
$$
\Q\{\eqalign{
x(t)={t\F t^2-1},
\cr
y(t)={t^2\F t^2-1}
}\W.
$$

\exo [Level=1,Fight=0,Learn=0,Field=\CourbesParamétréesCartésiennes,Type=\Exercices,Origin=] hm. 
Déterminer les points multiples de la courbe paramétrée par 
$$
\Q\{\eqalign{
x(t)=t^2-2t,
\cr
y(t)=t^2+{1\F t^2}
}\W.
$$

\exo [Level=1,Fight=0,Learn=0,Field=\CourbesParamétréesCartésiennes,Type=\Exercices,Origin=] hn. 
Déterminer les points multiples de la courbe paramétrée par 
$$
\Q\{\eqalign{
x(t)={(1+t)^3\F t},
\cr
y(t)={t^2-1\F t}
}\W.
$$

\exo [Level=1,Fight=0,Learn=0,Field=\CourbesParamétréesCartésiennes,Type=\Exercices,Origin=] ho. 
Déterminer les points multiples de la courbe paramétrée par 
$$
\Q\{\eqalign{
x(t)=t^2+t-2,
\cr
y(t)={t^3-3t\F t-1}
}\W.
$$

\exo  [Level=1,Fight=1,Learn=0,Field=\GéométriePlane,Type=\Exercices,Origin=] hp. 
Soit $(O,\vec i,\vec j)$ un repère orthonormé direct du plan $\sc P$ et soit $\vec w:=2\vec i+\vec j$. \pn
a) Trouver une base orthonormée directe $\{\vec u,\vec v\}$ vérifiant la relation $\vec W=\|\vec w\|\vec u$. \pn
b) Soit $\Omega$ le point vérifiant $\vec{O\Omega}=3\vec i+2\vec j$. Déterminer les relations liant les coordonnées $(x,y)$ du point~$M$ dans le repère $(O,\vec i,\vec j)$ avec ses coordonnées $(X,Y)$ dans le repère $(\Omega,\vec u,\vec v)$. 

\exo [Level=1,Fight=1,Learn=1,Field=\GéométriePlane,Type=\Exercices,Origin=]  hq. Soient $A,B,C,D, E, F, G$ sept points du plan vérifiant 
$$
\vec{EB}=4\vec{EA}, \qquad 3\vec{CF}=\vec{CD}\quad\hbox{et}\quad6\vec{AG}=\vec{BD}+2\vec{AC}.
$$
Montrer que les points $E$, $F$ et $G$ sont alignés. Trouver un nombre réel $\lambda$ tel que $\vec{EG}=\lambda\vec{EF}$. 


\exo  [Level=1,Fight=1,Learn=0,Field=\GéométriePlane,Type=\Exercices,Origin=] hr. 
Dans un triangle non-plat $ABC$, on note $E$ le milieu du coté $[BC]$, on note $F$ le milieu du segment $[AE]$ et on note $G$ l'intersection des droites $(BF)$ et $(AC)$. \pn
a) Calculer les coordonnées des points $E$, $F$ et $G$ dans le repère $(A,\vec{AB}, \vec{AC})$. \pn
b) En déduire le nombre réel $t$ tel que $\vec{AG}=t\vec{AC}$. 

\exo  [Level=1,Fight=2,Learn=2,Field=\GéométriePlane,Type=\Exercices,Origin=] hs. 
Soient $A$, $B$ et $C$ trois points non alignés du plan. A chaque nombre réel $m$, on associe le barycentre $G_m$ au système $\b\{(A,2);(B,m-1);(C,m+1)\b\}$. \pn
a) Déterminer l'ensemble des points $G_m$ lorsque $m$ décrit $\ob R$. \pn
b) Même question avec le système $\b\{(A,2);(B,1-m);(C,m+1)\b\}$.\pn
c) Même question avec le système $\b\{(A,m+1);(B,-2);(C,3)\b\}$.

\exo  [Level=1,Fight=0,Learn=0,Field=\GéométriePlane,Type=\Exercices,Origin=] ht. 
Déterminer le centre du cercle circonscrit au triangle $ABC$, avec $A(0,1)$, $B(2,3)$ et $C(-4,-12)$. 
Donner une équation cartésienne de ce cercle. 

\exo  [Level=1,Fight=0,Learn=0,Field=\GéométriePlane,Type=\Exercices,Origin=] hu. 
Soient $A(1,0)$ et $B(3,2)$. Donner une équation polaire de la droite $(AB)$. 

\exo [Level=1,Fight=0,Learn=0,Field=\CourbesParamétréesPolaires,Type=\Exercices,Origin=] hv. 
Reconnaitre les courbes d'équations polaires $\ds r={1\F\cos\theta+3\sin\theta}$ et $r=3\cos\theta-4\sin\theta$. 


\exo  [Level=1,Fight=1,Learn=1,Field=\GéométriePlane,Type=\Exercices,Origin=] hw. 
Soient deux points $A$ et $B$ tels que $AB=3$. Résoudre le système 
$\ds
\Q\{\eqalign{AM=4,\cr
\vec{AB}.\vec{AM}=6.
}\W.
$

\exo  [Level=1,Fight=0,Learn=0,Field=\GéométriePlane,Type=\Exercices,Origin=] hx. 
Dans le triangle $ABC$ avec $A(4,1)$, $B(2,3)$ et $C(-5,-3)$, déterminer des équations de la hauteur issue de $C$ et de la médiatrice du segment $[BC]$. 

\exo  [Level=1,Fight=0,Learn=0,Field=\GéométriePlane,Type=\Exercices,Origin=] hy. 
Prouver que les diagonales d'un losange ABCD sont perpendiculaires. 

\exo  [Level=1,Fight=1,Learn=0,Field=\GéométriePlane,Type=\Exercices,Origin=] hz. 
Soient $ABC$ un triangle non-plat du plan $\sc P$. \pn
a) Pour $M\in\sc P$, prouver que $\vec{MA}.\vec{BC}+\vec{MB}.\vec{CA}+\vec{MC}.\vec{AB}=0$. \pn
b) Soit $H$ le point d'intersection des hauteurs issues de $B$ et de $C$. Montrer que $\vec {HA}.\vec{BC}=0$. Conclusion ?

\exo  [Level=1,Fight=1,Learn=0,Field=\GéométriePlane,Type=\Exercices,Origin=] ia. 
Etant donné un parallèlogramme $ABCD$ du plan $\sc P$, 
montrer que $AD^2+BC^2=2AB^2+2AC^2$. \pn Qu'en conclure pour les diagonales d'un rectangle ?

\exo  [Level=1,Fight=2,Learn=2,Field=\GéométriePlane,Type=\Exercices,Origin=] ib. 
Soit $ABC$ un triangle. On note $a=BC$, $b=AC$, $c=AB$, $p$ le demi-périmètre et $S$ la surface de~$ABC$. \pn
a) Montrer que $a^2=b^2+c^2-2bc\cos\hat A$. \pn
b) Démontrer que $\ds C=bc\sin{\hat A\F 2}$. Trouver des formules analogues. \pn
c) Montrer que $\ds{a\F\sin\hat A}={b\F\sin\hat B}={c\F\sin\hat C}={abc\F2S}$. \pn
d) Montrer que $ABC$ est rectangle en $A$ si et seulement si $\sin^2\hat A=\sin^2\hat B+\sin^2\hat C$. \pn
e) Montrer que 
$$
\sin^2\hat A={(a+b+c)(a+b-c)(a-b+c)(-a+b+c)\F4b^2c^2}
$$
et en déduire que $S=\sqrt{p(p-a)(p-b)(p-c)}$. 

\exo  [Level=1,Fight=1,Learn=1,Field=\GéométriePlane,Type=\Exercices,Origin=] ic. 
Soient $A$ et  $B$ deux points de $\sc P$ et $(\alpha,\beta)\in\ob R^2$ tels que $\alpha+\beta\neq0$. Déterminer les lignes de niveau de 
$$
\varphi:M\mapsto \alpha MA^2+\beta MB^2.
$$

\exo  [Level=1,Fight=1,Learn=0,Field=\GéométriePlane,Type=\Exercices,Origin=] id. 
Soient $A$ et $B$ des points distinctsdu plan $\sc P$. Déterminer les lignes de niveau de $\ds{MA\FMB}$.

\exo  [Level=1,Fight=2,Learn=2,Field=\GéométriePlane,Type=\Exercices,Origin=] id. 
Soit $ABC$ un triangle du plan $\sc P$ et soit $k\in\ob R$. Déterminer l'ensemble des points $M\in\sc P$ tels que 
$$
\eqalignno{
&\|\vec{MA}+\vec{MB}\|=\|\vec{MA}+2\vec{MC}\|&(a)
\cr
&2\|\vec{MA}+\vec{MB}+\vec{MC}\|=3\|\vec{MB}+\vec{MC}\|&(b)
\cr
&\|\vec{MA}+\vec{MB}\|=\|\vec{MA}+k\vec{MC}\|&(c)
}
$$

\exo  [Level=1,Fight=1,Learn=1,Field=\GéométriePlane,Type=\Exercices,Origin=] ie. 
Soit $ABC$ un triangle. Déterminer les lignes de niveau de l'application du plan à valeurs réelles
$$
\varphi:M\mapsto(2\vec{MA}-3\vec{MB}+4\vec{MC}).(\vec{MA}-3\vec{MB}+2\vec{MC}).
$$

\exo  [Level=1,Fight=1,Learn=1,Field=\GéométriePlane,Type=\Exercices,Origin=] if. 
Etant donnés $A(1,2)$, $B(2,3)$ et $C(3,0)$, calculer l'aire du triangle $ABC$ et la distance de chaque sommet au coté opposé. 

\exo  [Level=1,Fight=0,Learn=0,Field=\GéométriePlane,Type=\Exercices,Origin=] ig. 
Déterminer l'aire du quadrilatère $ABCD$ avec $A(-1,2)$, $B(4,3)$, $C(3,5)$ et $D(2,6)$. 

\exo [Level=1,Fight=1,Learn=1,Field=\NombresComplexes,Type=\Exercices,Origin=] ih. 
Soient $(a,b,c)\in\ob R^3$. Résoudre le système $\ds\Q\{\eqalign{
x\sin a+y\sin b=\sin c\cr x\cos a+y\cos b=\cos c}\W.$

\exo  [Level=1,Fight=0,Learn=0,Field=\GéométriePlane,Type=\Exercices,Origin=] ii. 
Déterminer une équation de la droite passant par $A(2,3)$ et $B(-4,7)$. Donner un vecteur normal de cette droite. 

\exo  [Level=1,Fight=0,Learn=0,Field=\GéométriePlane,Type=\Exercices,Origin=] ij. 
Soit $\sc D$ la droite d'équation $3x-\sqrt3y-6=0$ calculer la distance de $\sc D$ à l'origine. 

\exo  [Level=1,Fight=0,Learn=0,Field=\GéométriePlane,Type=\Exercices,Origin=] ik. 
Soit $ABC$ un triangle non plat. Déterminer dans le repère $(A,\vec{AB},\vec{AC})$ des équations cartésiennes des médianes du triangle et retrouver qu'elles sont concourantes. 

\exo  [Level=1,Fight=1,Learn=0,Field=\GéométriePlane,Type=\Exercices,Origin=] 
il. Pour $m\in\ob R$, on note $\sc D_m$ la droite d'équation $(1-m^2)x+2my-2(m+1)=0$. 
Montrer qu'il existe un point fixe $A$ dont la distance aux droites $\sc D_m$ est constante. 
Qu'en déduire pour les droites $(\sc D_ml)_{m\in\ob N}$ ?

\exo  [Level=1,Fight=0,Learn=0,Field=\GéométriePlane,Type=\Exercices,Origin=] im. 
Soient $\sc D$ et $\sc D'$ les droites d'équations respectives $3x+4y+3=0$ et $12x-5y+4=0$. 
Sont elles concourantes ? si oui, déterminer des équations des deux bissectrices. 

\exo  [Level=1,Fight=0,Learn=0,Field=\GéométriePlane,Type=\Exercices,Origin=] in. 
Soient $A$, $B$ et $C$ des points d'affixes respectives $a$, $b$ et $c$. Déterminer l'affixe de l'orthocentre de $ABC$. 

\exo  [Level=1,Fight=0,Learn=0,Field=\GéométriePlane,Type=\Exercices,Origin=] io. 
Déterminer une équation de la médiatrice de $[AB]$ avec $A(3,2)$ et $B(-5,1)$. 

\exo  [Level=1,Fight=1,Learn=0,Field=\GéométriePlane,Type=\Exercices,Origin=] ip. 
Soient $A$ et  $B$ deux points de $\sc P$. Déterminer les lignes de niveau de 
$$
\varphi:M\mapsto MA^2-MB^2.
$$
Application : montrer que les hauteurs d'un triangle sont concourantes.  

\exo  [Level=1,Fight=0,Learn=0,Field=\Coniques,Type=\Exercices,Origin=] iq. 
Réduire et trouver les éléments caractéristiques de la conique 
$$
x^2+11y^2-10\sqrt3xy+4y+4\sqrt3x=20.
$$

\exo  [Level=1,Fight=1,Learn=0,Field=\Coniques,Type=\Exercices,Origin=] ir. 
Soient $p>0$, $q>0$ et $\sc P$ et $\sc C$ les paraboles d'équations respectives $y^2=2px$ et $y^2=2qx$. 
Déterminer les tangentes communes à $\sc P$ et $\sc D$. 
 
\exo  [Level=1,Fight=1,Learn=0,Field=\Coniques,Type=\Exercices,Origin=] is. 
Soient deux paraboles distinctes de même axe focal et de même foyer $F$ ayant un point en commun. 
Déterminer une mesure de l'angle des deux tangentes en ce point. 

\exo  [Level=1,Fight=1,Learn=1,Field=\Coniques,Type=\Exercices,Origin=] it. 
Un point $M$ d'une hyperbole se projette orthogonalement en $H$ et en $H'$ sur ses asymptotes. 
Montrer que le produit $MH.MH'$ reste constant lorsque $M$ varie sur l'hyperbole. 

\exo  [Level=1,Fight=0,Learn=0,Field=\Coniques,Type=\Exercices,Origin=] iu. 
Déterminer le lieu des points $M$ d'ou l'on peut mener deux tangentes orthogonales à une ellipse $\sc E$. 

\exo  [Level=1,Fight=0,Learn=0,Field=\Coniques,Type=\Exercices,Origin=] iv. 
Réduire et trouver les éléments caractéristiques de la conique 
$$
x^2-xy+y^2+x+y+1=0.
$$

\exo  [Level=1,Fight=1,Learn=0,Field=\Coniques,Type=\Exercices,Origin=] iw. Déterminer le lieu des points $M$ d'ou l'on peut mener deux tangentes orthogonales à une parabole $\sc P$. 

\exo [Level=1,Fight=1,Learn=0,Field=\Coniques,Type=\Exercices,Origin=]  ix. Déterminer le lieu des points $M$ d'ou l'on peut mener deux tangentes orthogonales à une hyperbole $\sc H$. 

\exo [Level=1,Fight=0,Learn=0,Field=\CourbesParamétréesCartésiennes,Type=\Exercices,Origin=] iy. 
Déterminer le support de la courbe paramétré
$$
\Q\{\eqalign{x(t)=2-3t^2,\cr
y(t)=-1+t^2}\W.\leqno{(\sc C)}
$$

\exo [Level=1,Fight=0,Learn=0,Field=\CourbesParamétréesCartésiennes,Type=\Exercices,Origin=] iz. 
Déterminer le support de la courbe paramétré
$$
\Q\{\eqalign{x(t)={2t\F1-t},\cr
y(t)={2-3t\F 1-t}}\W.\leqno{(\sc C)}
$$

\exo [Level=1,Fight=0,Learn=0,Field=\CourbesParamétréesCartésiennes,Type=\Exercices,Origin=] ja. 
Déterminer le support de la courbe paramétrée
$$
\Q\{
\eqalign{
x(t)=2\sqrt{1-t^2},
\cr
y(t)=2+\sqrt{1-t^2}
}\W.
\leqno{(\sc C)}
$$

\exo [Level=1,Fight=2,Learn=0,Field=\GéométrieSpatiale,Type=\Exercices,Origin=] jb. 
Soient $A$, $B$, $C$ et $D$ quatre points de l'espace tels que $\widehat{ABC}=\widehat{BCD}=\widehat{CDA}=\widehat{DAB}={\pi\F2}$. 
Montrer que $A$, $B$, $C$ et $D$ sont coplanaires. 

\exo [Level=1,Fight=2,Learn=0,Field=\GéométrieSpatiale,Type=\Exercices,Origin=] jc. 
Soient $A$, $B$, $C$ et $D$ tels que $\vec{BA}.\vec{BD}\le 0$ et $\vec{DB}.\vec{DC}\le 0$. Montrer que $AC\ge BD$. 

\exo [Level=1,Fight=2,Learn=0,Field=\GéométrieSpatiale,Type=\Exercices,Origin=] jd. 
Soient $A$, $B$, $C$ et $D$ quatres points de l'espace tels que $(BD)\perp(AC)$. Montrer que 
$$
BC^2+DA^2=AB^2+CD^2.
$$

\exo [Level=1,Fight=0,Learn=0,Field=\GéométrieSpatiale,Type=\Exercices,Origin=] je. 
En introduisant des barycentres bien choisis, déterminer pour $\lambda\notin\{-2,-{3\F2},-1\}$ l'ensemble 
$$
\Q\{M\in\sc E:\|\vec{MA}+\vec{MC}+\lambda\vec{MC}\|=\|\vec{MD}+\lambda\vec{ME}\|\W\}
$$

\exo  [Level=1,Fight=2,Learn=0,Field=\GéométrieSpatiale,Type=\Exercices,Origin=] jf. 
Soient $A$, $B$, $C$ et $D$ quatre points de l'espace tels que $\widehat{ABC}=\widehat{BCD}=\widehat{CDA}
=\widehat{DAC}={\pi\F 2}$. Montrer que $A$, $B$, $C$ et $D$ sont coplanaires. 

\exo  [Level=1,Fight=2,Learn=0,Field=\GéométrieSpatiale,Type=\Exercices,Origin=] jg. 
Soient $A$, $B$, $C$ et $D$ tels que $\vec{BA}.\vec{BD}\le 0$ et $\vec{DB}.\vec{DC}\le0$. Montrer que $AC\ge BD$. 

\exo  [Level=1,Fight=2,Learn=0,Field=\GéométrieSpatiale,Type=\Exercices,Origin=] jh. 
Soient $ABD$ et $ACD$ deux triangles équilatéraux. On note $\theta\in[0,{\pi\F2}]$ l'angle des plans $(ABD)$ et $(ACD)$. Calculer en fonction de $\theta$ l'angle $\phi\in[0,{\pi\F2}]$ des droites $(AB)$ et $(AC)$. 

\exo  [Level=1,Fight=1,Learn=0,Field=\GéométrieSpatiale,Type=\Exercices,Origin=] ji. 
Soient $A$, $B$, $C$, $D$ et $E$ des points de l'espace et $\lambda\neq-1$. Déterminer l'ensemble 
$$
E:=\b\{M\in\sc E:MA^2+MB^2+2\lambda MC^2=MD^2+\lambda ME^2\b\}
$$
en introduisant des barycentres bien choisis. 

\exo  [Level=1,Fight=1,Learn=0,Field=\GéométrieSpatiale,Type=\Exercices,Origin=] jj.  
Soient $A$, $B$, $C$, $D$ et $E$ des points de l'espace et $\lambda\in\ob R\ssm\{-2,-3/2,-1\}$. Déterminer l'ensemble 
$$
E:=\b\{M\in\sc E:\|\vec{MA}+\vec{MB}+\lambda\vec{MC}\|=\|\vec{MD}+\lambda\vec{ME}\|\b\}
$$
en introduisant des barycentres bien choisis. 

\exo  [Level=1,Fight=0,Learn=0,Field=\GéométrieSpatiale,Type=\Exercices,Origin=] jk. 
Soient $\vec u$, $\vec v$ et $\vec w$ trois vecteurs de l'espace. Montrer que 
$$
\vec u\wedge(\vec v\wedge\vec w)+\vec w\wedge(\vec u\wedge\vec v)+\vec v\wedge(\vec w\wedge\vec u)=\vec 0. 
$$

\exo  [Level=1,Fight=1,Learn=0,Field=\GéométrieSpatiale,Type=\Exercices,Origin=] jl. 
Pour $A$, $B$, $C$ et $D$dans $E$, prouver que 
$$
\vec{AB}\wedge\vec{AC}-\vec{BC}\wedge\vec{BD}+\vec{CD}\wedge\vec{CA}-\vec{DA}\wedge{DB}=\vec 0.
$$

\exo  [Level=1,Fight=1,Learn=0,Field=\GéométrieSpatiale,Type=\Exercices,Origin=] jm. 
Soient $A$, $B$ et $C$ dans $\sc E$. Montrer que 
$$
\forall M\in\sc E, \qquad \vec{MA}\wedge\vec{MB}+\vec{MB}\wedge\vec{MC}+\vec{MC}\wedge\vec{MA}=\vec{AB}\wedge\vec{AC}
$$

\exo  [Level=1,Fight=0,Learn=0,Field=\GéométrieSpatiale,Type=\Exercices,Origin=] jn. 
Soient $A$, $B$ dans $\sc E$ et $\vec u$ un vecteur. Déterminer l'ensemble
$$
\b\{M\in\sc E:\vec{MA}\wedge\vec{MB}=\vec u\b\}
$$


\exo  [Level=1,Fight=1,Learn=0,Field=\GéométrieSpatiale,Type=\Exercices,Origin=] jo. 
Soient $A$, $B$ et $C\neq A$ dans $\sc E$. Déterminer l'ensemble 
$$
\b\{
M\in\sc E:\vec{MA}\wedge\vec{MB}+\vec{MB}\wedge\vec{MC}=2\vec{MC}\wedge\vec{MA}
\b\}
$$ 

\exo  [Level=1,Fight=0,Learn=0,Field=\GéométrieSpatiale,Type=\Exercices,Origin=] jp. 
On considère les points $A(1,2,-1)$, $B(3,2,0)$, $C(2,1,-1)$, $D(1,0,4)$ et $E(-1,1,1)$. Déterminer un vecteur directeur de la droite $(ABC)\cap(ADE)$. 

\exo  [Level=1,Fight=0,Learn=0,Field=\GéométrieSpatiale,Type=\Exercices,Origin=] jq. 
Calculer l'aire du triangle défini par les 3 points $A(1,2,-1)$, $B(-1,1,0)$ et $C(0,1,2)$. 

\exo  [Level=1,Fight=0,Learn=0,Field=\GéométrieSpatiale,Type=\Exercices,Origin=] jr. 
Calculer la distance du point $A(1,1,1)$ au plan $\sc P$ de représentation paramétrique
$$
\Q\{\eqalign{
x=2+\lambda-\mu,
\cr
y=3-\lambda+2\mu, 
\cr
z=1+2\lambda+\mu
}\W.\qquad(\lambda,\mu)\in\ob R^2
$$

\exo  [Level=1,Fight=0,Learn=0,Field=\GéométrieSpatiale,Type=\Exercices,Origin=] js. 
Calculer la distance du point $A(3,-1,2)$ au plan $\sc P:2x+6y-z=7$.  

\exo  [Level=1,Fight=0,Learn=0,Field=\GéométrieSpatiale,Type=\Exercices,Origin=] jt. 
Calculer la distance du point $A(1,2,-3)$ au plan $\sc P$ passant par $B(-2,1,0)$ et dirigé par $\vec v(1,-6,2)$ et $\vec w(3,-1,1)$. 

\exo  [Level=1,Fight=0,Learn=0,Field=\GéométrieSpatiale,Type=\Exercices,Origin=] ju. 
Déterminer les plans bissecteurs de $\sc P:7x-4y+4z=8$ et $\sc P':4x+8y+z=11$. 

\exo  [Level=1,Fight=0,Learn=0,Field=\GéométrieSpatiale,Type=\Exercices,Origin=] jv. 
Former les équations cartésiennes des plans $\sc P$ situés à la distance $1$ de $A(1,-1,0)$ et contenant la droite 
$$
(D)\qquad \Q\{
\eqalign{
x=3z+2,
\cr
y=-5z+1
}
\W.
$$   

\exo  [Level=1,Fight=0,Learn=0,Field=\GéométrieSpatiale,Type=\Exercices,Origin=] jw. 
Calculer l'angle des plans $\sc P:x+y+z=3$ et $\sc P':2x+y-z=4$. 

\exo  [Level=1,Fight=0,Learn=0,Field=\GéométrieSpatiale,Type=\Exercices,Origin=] jx. 
Calculer l'angle des droites 
$$
(\sc D)\qquad\Q\{\eqalign{
x-2y+z=0\cr
2x+y-z=0
}\W.\qquad\hbox{et}\qquad
(\sc D')\qquad 
\Q\{
\eqalign{x+y+z=0,\cr
2x+3y-z=0}
\W.
$$


\exo  [Level=1,Fight=0,Learn=0,Field=\GéométrieSpatiale,Type=\Exercices,Origin=] jy. 
Calculer l'angle du plan $\sc P:x+2z=3$ et de la droite 
$$
(\sc D)\qquad\Q\{\eqalign{
x+2y+z=4\cr
2x-y+z=1
}\W.
$$

\exo  [Level=1,Fight=0,Learn=0,Field=\GéométrieSpatiale,Type=\Exercices,Origin=] jz. 
Déterminer l'équation du plan parallèle à $(O,\vec j)$ et passant par les points $A(0,-1,2)$ et $B(-1,2,3)$. 

\exo  [Level=1,Fight=0,Learn=0,Field=\GéométrieSpatiale,Type=\Exercices,Origin=] ka. 
Trouver une condition nécéssaire et suffisante sur $\lambda\in\ob R$ pour que les trois plans d'équations $\sc P_1:x+\lambda y-z+1=0$, $\sc P_2:(\lambda+1)x+3y+4z-2=0$ et $\sc P_3:y+(2\lambda+4)z-(2\lambda+2)=0$ contiennent une même droite $\sc D$. 

\exo  [Level=1,Fight=0,Learn=0,Field=\GéométrieSpatiale,Type=\Exercices,Origin=] kb. 
Calculer la distance du point $A(4,-3,2)$ à la droite $\sc D$ passant par $B(1,0,-1)$ et dirigée par $\vec v(2,-1,3)$. 

\exo  [Level=1,Fight=0,Learn=0,Field=\GéométrieSpatiale,Type=\Exercices,Origin=] kc. 
Calculer la distance du point $A(2,-1,1)$ et la droite qui a pour équation le système 
$$
(\sc D)\qquad\Q\{\eqalign{
x=2z-1\cr
y=3z+1
}\W.
$$

\exo  [Level=1,Fight=0,Learn=0,Field=\GéométrieSpatiale,Type=\Exercices,Origin=] kd. 
Donner un système d'équations de la droite $\sc D$ dirigée par $\vec u(1,2,3)$ et re,ncontrant les deux droites admettant les systèmes 
$$
(\sc D_1)\qquad\Q\{\eqalign{
x=0\cr
z=1
}\W.\qquad\hbox{et}\qquad
(\sc D_2)\qquad 
\Q\{
\eqalign{x=1,\cr
y=0}
\W.
$$



\exo  [Level=1,Fight=0,Learn=0,Field=\GéométrieSpatiale,Type=\Exercices,Origin=] ke. 
Donner un système d'équations de la droite $\sc D$ passant par $A(1,1,2)$ et rencontrant les deux droites 
$$
(\sc D)\qquad\Q\{\eqalign{
z=0\cr
2x+3y-1=0
}\W.\qquad\hbox{et}\qquad
(\sc D')\qquad 
\Q\{
\eqalign{y=0,\cr
x-2z+3=0}
\W.
$$

\exo  [Level=1,Fight=0,Learn=0,Field=\GéométrieSpatiale,Type=\Exercices,Origin=] kf. 
Montrer qu'il existe un unique plan parallèle et équidistant des deux droites
$$
(\sc D)\qquad\Q\{\eqalign{
y=3x+1\cr
z=2
}\W.\qquad\hbox{et}\qquad
(\sc D')\qquad 
\Q\{
\eqalign{x=z-4,\cr
y=2z+1}
\W.
$$
En donner une équation cartésienne.

\exo  [Level=1,Fight=0,Learn=0,Field=\GéométrieSpatiale,Type=\Exercices,Origin=] kg. 
Montrer que les droites 
$$
(\sc D)\qquad\Q\{\eqalign{
x=2z+1\cr
y=z-1
}\W.\qquad\hbox{et}\qquad
(\sc D')\qquad 
\Q\{
\eqalign{x=-z-2,\cr
y=3z+1}
\W.
$$
sont sécantes et former un système d'équations cartésiennes de leur bissectrices. 

\exo  [Level=1,Fight=0,Learn=0,Field=\GéométrieSpatiale,Type=\Exercices,Origin=] kh. 
Soient $A(1,2,-1)$ et $B(0,1,1)$. Former une équation cartésienne de l'ensemble
$$
\B\{
M\in\sc E:d\b(A,(OM)\b)=d\b(B,(OM)\b)
\B\}
$$

\exo  [Level=1,Fight=0,Learn=0,Field=\GéométrieSpatiale,Type=\Exercices,Origin=] ki. 
Montrer que les $5$ points $A(4,7,1)$, $B(3,-3,6)$, $C(-5,1,4)$, $D(5,6,-1)$ et $E(-4,3,-3)$ sont cospériques. Déterminer le rayon et le centre de la sphère les contenant. 

\exo  [Level=1,Fight=0,Learn=0,Field=\GéométrieSpatiale,Type=\Exercices,Origin=] kj. 
Former une équation de la sphère circonscrite au tétraèdre $ABCD$ pour $A(0,2,4)$, $B(1,3,2)$, $C(2,1,3)$ et $D(-2,-3,-1)$. 

\exo  [Level=1,Fight=0,Learn=0,Field=\GéométrieSpatiale,Type=\Exercices,Origin=] kk. 
Soient $\sc C_1$ et $\sc C_2$ deux cercles de $\sc E$ non coplanaires et admettant deux points communs $A$ et $B$. 
Montrer qu'il existe une unique sphère $\sc S$ contenant $\sc C_1$ et $\sc C_2$. 

\exo  [Level=1,Fight=0,Learn=0,Field=\GéométrieSpatiale,Type=\Exercices,Origin=] kl. 
Déterminer le centre et le rayon de la sphère circonscrite au tétraèdre dont les faces ont 
pour équations $x+y+z=0$, $x+y-z=2$, $x-y+z=4$ et $-x+y+z=6$. 

\exo  [Level=1,Fight=0,Learn=0,Field=\GéométrieSpatiale,Type=\Exercices,Origin=] km. 
Former une équation cartésienne de la sphère tangente en $O$ à $(O,\vec k)$ et tangente à la droite 
$$
(\sc D)\qquad 
\Q\{
\eqalign{y=2,\cr
x=3z}
\W.
$$ 

\exo  [Level=1,Fight=0,Learn=1,Field=\EspacesVectoriels,Type=\Cours,Origin=] kn. 
Parmi les ensembles suivants de $\sc F(\ob R,\ob R)$ dire lesquels sont des sous-espaces vectoriels : 
\pn
Les fonctions bornées sur $\ob R$, les fonctions continues sur $\ob R$, les fonctions impaires, les fonctions paires, les fonctions périodiques, les fonctions périodiques de période $1$, Les fonctions monotones, Les fonctions affines par morceaux (avec un  nombre fini de morceaux), les fonctions discontinues en $0$, les fonctions convexes, les fonctions polynômes de degré $n$, les fonctions polynômes. 

\exo [Level=1,Fight=1,Learn=1,Field=\EspacesVectoriels,Type=\Exercices,Origin=] ko. 
Soient $\alpha$ et $\beta$ deux nombres complexes. Trouver une relation de dépendance linéaire entre les vecteurs de $\ob C^4$ suivant : $u:=(\alpha,-\alpha,\beta,-\beta)$, $v:=(\alpha,-\alpha,\beta,\beta)$, $w:=(\beta,\beta,\alpha,\alpha)$, $x:=(\beta,-\beta,\alpha,\alpha)$ et $y=(1,1,1,1)$. 

\exo [Level=1,Fight=1,Learn=2,Field=\EspacesVectoriels,Type=\Exercices,Origin=]  kp. 
Soient $p$ et $q$ deux projecteurs d'un espace $e$. On suppose que $p+q$ est un projecteur. Démontrer que $p\circ q+q\circ p=0$ puis en déduire que $p\circ q=0$ et que $q\circ p=0$. 

\exo [Origin=,Level=1,Fight=3,Learn=2,Field=\EspacesVectoriels,Type=\Exercices] kq. 
Soit $E$ un espace vectoriel et $u\in\sc L(E)$ un endomorphisme vérifiant : 
$$
\sbox{Pour chaque $x\in E$, il existe une relation de dépendance linéaire entre $x$ et $u(x)$.}
$$ 
Prouver que $u$ est une homothétie de $E$. 

\exo [Origin=,Level=1,Fight=1,Learn=2,Field=\EspacesVectoriels,Type=\Exercices] kr. 
Soit $p$ un projecteur de $E$ et soit $q=\hbox{Id}_E-p$. On pose $F=\hbox{Im}(p)$ et $G=\hbox{Im}(q)$. Montrer que $u\in\sc L(E)$ commute avec $p$ si, et seulement si,  $u(F)\subset F$ et $u(G)\subset G$. 

% Duplicate % ks. %

\exo [Level=1,Fight=1,Learn=0,Field=\EspacesVectoriels,Type=\Exercices,Origin=\Capaces] kt. 
La famille d'applications $\{x\mapsto|x-a|\}_{a\in\ob R}$ est-elle libre dans $\sc F(\ob R,\ob R)$ ?

\exo [Level=1,Fight=0,Learn=0,Field=\EspacesVectoriels,Type=\Exercices,Origin=\Capaces] ku.  
La famille d'applications $\{x\mapsto|x-a|\}_{a\in\ob R}$ est-elle libre dans $\sc F([0,1],\ob R)$ ?

\exo [Level=1,Fight=1,Learn=1,Field=\EspacesVectoriels,Type=\Exercices,Origin=\Capaces] kv.  
La famille d'applications $\{x\mapsto\e^{inx}\}_{n\in\ob Z}$ est-elle libre dans $\sc F(\ob R,\ob R)$ ?

\exo [Level=1,Fight=1,Learn=1,Field=\EspacesVectoriels,Type=\Exercices,Origin=\Capaces] kw.  
La famille d'applications $\{x\mapsto\cos(nx)\}_{n\in\ob N}$ est-elle libre dans $\sc F(\ob R,\ob R)$ ?

\exo [Level=1,Fight=1,Learn=1,Field=\EspacesVectoriels,Type=\Exercices,Origin=\Capaces] kx.  
La famille d'applications $\{x\mapsto\cos(nx)\}_{n\in\ob N}\cup\{x\mapsto\sin(nx)\}_{n\in\ob N^*}$ est-elle libre dans $\sc F(\ob R,\ob R)$ ?

\exo [Level=1,Fight=1,Learn=0,Field=\EspacesVectoriels,Type=\Exercices,Origin=\Capaces] ky.  
La famille d'applications $\{x\mapsto x^\alpha\}_{\alpha\ge0}$ est-elle libre dans $\sc F\Q(\Q]0,+\infty\W[,\ob R\W)$ ?

\exo [Level=1,Fight=1,Learn=0,Field=\EspacesVectoriels,Type=\Exercices,Origin=\Capaces] kz. 
La famille d'applications $\{x\mapsto x^n\}_{n\in\ob N}$ est-elle libre dans $\sc F\Q([0,1],\ob R\W)$ ?

\exo [Level=1,Fight=1,Learn=0,Field=\EspacesVectoriels,Type=\Exercices,Origin=\Capaces] la. 
La famille d'applications $\{x\mapsto\sin(x)^n\}_{n\in\ob N}$ est-elle libre dans $\sc F\b(\ob R,\ob R\b)$ ?

\exo [Level=1,Fight=1,Learn=0,Field=\EspacesVectoriels,Type=\Exercices,Origin=\Capaces] lb. 
La famille d'applications $\{x\mapsto\e^{\alpha x}\}_{\alpha\in \ob R}$ est-elle libre dans $\sc F\b(\ob R,\ob R\b)$ ?

\exo [Level=1,Fight=2,Learn=2,Field=\EspacesVectoriels,Type=\Exercices,Origin=\Capaces] lc. 
Soit $E:=\sc F(\ob N,\ob R)$ et soit $u\in E$. Montrer que l'application $\Phi_u$ définie par 
$$
\Phi_u:\App E\to E,v\mapsto u\times v\ppA, 
$$
où $\times$ designe la multiplication des applications, est linéaire. Quel est son noyau ? son image ? sont-ils en somme directe ou supplémentaires ?
\pn
Montrer que l'application 
$$
\Phi: \App E\to\sc L(E),u\mapsto \Phi_u\ppA
$$
est linéaire. Quel est son noyau ? Est-il vrai que $\hbox{Im}(\Phi)=\sc L(E)$ ? 

\exo [Origin=,Level=1,Fight=1,Learn=2,Field=\EspacesVectoriels,Type=\Exercices]  ld. 
Soient $F$ et $G$ deux sous-espaces vectoriels d'un $\ob K$-espace vectoriel $E$. Montrer que 
$$
F\cup G\hbox{ est un sous-espace vectoriel de }E\Longleftrightarrow F\subset G\hbox{ ou }G\subset F. 
$$

\exo  [Origin=,Level=1,Fight=1,Learn=2,Field=\EspacesVectoriels,Type=\Exercices] le. 
Soit $E$ un $\ob K$-espace vectoriel et $F$ et $G$ deux sous-espaces vectoriels de $E$ tels que $E=F+G$. Soit $H$ un supplémentaire de $F\cap G$ dans $G$. Montrer que $E=F\oplus H$. 

\exo[Origin=,Level=1,Fight=1,Learn=2,Field=\DimensionFinie,Type=\Exercices] lf. 
Trouver trois nombres réels $\alpha$, $\beta$ et $\gamma$ tels que pour tout polynôme de degré $n\le 2$, on ait
$$
\int_2^4P(t)\d t=\alpha P(0)+\beta P(1)+\gamma P(2).
$$

\exo [Level=1,Fight=0,Learn=1,Field=\EspacesVectoriels,Type=\Exercices,Origin=] lg. 
Soit $f$ l'application donnée sur $\ob R^3$ par $f(x,y,z):=(x+y-z,2x-y)$. Vérifier que $f$ est une application linéaire et déterminer son noyau et son image. 

\exo [Level=1,Fight=0,Learn=1,Field=\EspacesVectoriels,Type=\Exercices,Origin=] lh. 
Les applications suivantes sont-elles linéaires ?
$$
f:\App 
\ob R^3\to\ob R^3,(x,y,z)\mapsto(x-y,y-z)
\ppA
\qquad 
g:\App \ob R^2\to\ob R,(x,y)\mapsto xy
\ppA
\qquad 
h:\App \ob C\to\ob R,z\mapsto\re(z)
\ppA
\qquad 
i:\App\sc F(\ob N,\ob R)\to\ob R^3, u\mapsto \b(u(0),u(1),u(2)\b)
\ppA
$$

\exo [Level=1,Fight=1,Learn=1,Field=\EspacesVectoriels,Type=\Exercices,Origin=] li. 
Soit $E$ un $\ob K$-espace vectoriel et $f\in\sc L(E)$ tel que $f^3=\hbox{Id}_E$. 
\pn
a) Montrer que $\hbox{Ker}(f-\hbox{Id}_E)\oplus\hbox{Im}(f-\hbox{Id}_E)=E$. 
\pn
b) Montrer que $\hbox{Ker}(f-\hbox{Id})=\hbox{Im}(f^2+f+\hbox{Id}_E)$ et $\hbox{Im}(f-\hbox{Id}_E)=\hbox{Ker}(f^2+f+\hbox{Id}_E)$. 

\exo [Level=1,Fight=0,Learn=0,Field=\EspacesVectoriels,Type=\Exercices,Origin=] lj. 
Soient $(a,b,c)$ trois nombres réels. Montrer que les trois applications $x\mapsto\sin(x+a)$, $x\mapsto\sin(x+b)$ et $x\mapsto\sin(x+c)$ forment une famille liée. 

\exo  [Origin=,Level=1,Fight=1,Learn=0,Field=\EspacesVectoriels,Type=\Exercices] lk. 
Dans $\ob R^4$, montrer que l'ensemble des vecteurs $u=(x,y,z,t)$ tels que 
$$
\Q\{\eqalign{
x+3y-2z-5t=0\cr
x+2y+z-t=0,\cr
x+3t=0
}\W.
$$
est un espace vectoriel. 

\exo [Level=1,Fight=0,Learn=1,Field=\EspacesVectoriels,Type=\Exercices,Origin=] ll. 
Soit $E$ un $\ob K$-espace vectoriel. \pn
a) Montrer que si $(x_1,x_2,x_3)$ est une famille libre de $E$, alors la famille $(x_1+x_2,x_2+x_3,x_3+x_1)$ est une famille libre de $E$. En est-il de même pour la famille $(x_1-x_2,x_2-x_3,x_3-x_1)$ ? \pn
b) Montrer que si $(x_1,x_2,x_3)$ est une famille génératrice de $E$ alors la famille $(x_1+x_2,x_2+x_3,x_3+x_1)$ est une famille génératrice de $E$. 

\exo [Level=2,Fight=1,Learn=1,Field=\EquationsDifférentiellesLinéairesDuSecondOrdre,Type=\Exercices,Origin=] lm. 
On considère l'équation différentielle $y''-2y'+y=x\e^{-x}\cos x$. \pn
Résoudre l'équation avec les conditions initiales $y(0)=0$ et $y'(0)=1$, on notera $f$ cette solution. \pn
b) Vérifier que $f$ est bien solution de l'équation. \pn
c) Tracer le graphe de $f$ sur $[-3,2]$. \pn
d) Résoudre l'équation $f(x)=-{1\F4}$. 

\exo [Level=1,Fight=0,Learn=1,Field=\NombresComplexes,Type=\Exercices,Origin=] ln. 
Soient $(a,b)\in\ob C^2$ tels que $|a|=|b|=1$ et $a\neq b$ et $a\neq-b$. \pn
a) Montrer que $\ds {1+ab\F a+b}$ est un réel. \pn
b) Montrer que, quel que soit $z\in\ob C$, le nombre 
$$
{z+ab\overline z-(a+b)\F a-b}
$$
est un imaginaire pur. 

\exo [Level=1,Fight=0,Learn=1,Field=\NombresComplexes,Type=\Exercices,Origin=] lo. 
Soit $f:\ob C^*\to\ob C^*$ la fonction qui à $z$ associe
$$
f(z)={1\F2}\Q(z+{1\F z}\W).
$$
a) Donner l'image par $f$ du cercle de centre $O$ et de rayon $R$. \pn
b) Donner l'image par $f$ de la demi-droite partant de $O$ et d'angle polaire $\theta$. 

\exo [Level=1,Fight=0,Learn=1,Field=\NombresComplexes,Type=\Maple,Origin=] lp. 
Centre et rayon de la sphère passant par les points de coordonnées $(1,2,3)$, $(-3,4,5)$, $(3,-4,5)$ et $(-3,4,-5)$. Trouver tous les points à coordonnées entières de cette sphère (on pourra utiliser le booléen {\bf type(d, square)}). 


\exo [Level=1,Fight=0,Learn=1,Field=\NombresComplexes,Type=\Exercices,Origin=] lq. 
On pose $P:=\{z\in\ob C:\IM(z)>{1\F 2}\}$,  $D:=\{z\in\ob C:|z|<1\}$ et 
$$
\forall z\in\ob C\ssm\{-i\}, \qquad f(z):={z-2i\F z+i}
$$
a) En notant $z=x+iy$ et $z'=f(z)=u+iv$, exprimer $z$ en fonction de $z'$ et inversement, puis exprimer $x$ et $y$ en fonction de $u$ et $v$ et inversement. 
\medskip\noindent
b)  Montrer que l'application $f:\ob C\ssm\{-i\}\to\ob C\ssm\{1\}$ et que sa restriction $f:P\to D$ sont des applications bijectives. 

\exo [Level=1,Fight=0,Learn=1,Field=\Suites,Type=\Exercices,Origin=\Maple] lr. 
Soit une suite $(u_n)_{n\in\ob N}$ définie par $u_0\in\ob N^*$ et 
$$
u_{n+1}=\Q\{
\eqalign{
1\hbox{ si }u_n=1\cr
{u_n\F2}\hbox{ si }u_n \hbox{est pair}\cr
3u_n+1\hbox{ si }u_n \hbox{ est impair et différent de $1$}
}
\W.
$$
a) Définir une fonction $f$ ({\bf avec piecewise}) telle que $u_{n+1}=f(u_n)$. \pn
b) Ecrire une procédure {\bf CalculUn} prenant en entrée $n$ et $u_0$ et renvoyant en sortie $u_n$

\exo [Level=1,Fight=0,Learn=0,Field=\DimensionFinie,Type=\Exercices,Origin=] ls. 
Montrer que l'espace $E=\{(x,y,z,t)\in\ob R^4:x+y=z+2t=0\}$ est un espace vectoriel (sur quel corps ?), puis en donner une base et la dimension. 

\exo  [Level=1,Fight=0,Learn=0,Field=\DimensionFinie,Type=\Exercices,Origin=] lt. 
Montrer que l'espace $F=\{(x,y,z)\in\ob C^3:x+y+z=x+iy-z=0\}$ est un espace vectoriel (sur quel corps ?), puis en donner une base et la dimension. 

\exo  [Level=1,Fight=0,Learn=0,Field=\DimensionFinie,Type=\Exercices,Origin=] lu. 
Montrer que l'espace $G=\{(x_1,\cdots,x_n)\in\ob R^n:x_1+x_2+\cdots+x_n=0\}$ est un espace vectoriel (sur quel corps ?), puis en donner une base et la dimension. 

\exo  [Level=1,Fight=0,Learn=0,Field=\DimensionFinie,Type=\Exercices,Origin=] lv. 
Montrer que l'espace $H=\{(x_1,\cdots,x_n)\in\ob R^n:x_1=x_2=\cdots=x_n\}$ est un espace vectoriel (sur quel corps ?), puis en donner une base et la dimension. 

\exo  [Level=1,Fight=0,Learn=1,Field=\DimensionFinie,Type=\Exercices,Origin=] lw. 
Dans $E:=\{f:\ob R\to\ob R\hbox{ qui à }x\mapsto a+bx+cx^2:(a,b,c)\in\ob R^3\}$, on considère
$$
F=\Vect(x\mapsto 1-x-x^2,x\mapsto 1+x+x^2)\qquad\hbox{et}\qquad G=\Vect(x\mapsto 1-x+x^2).
$$
Montrer que $F$ et $G$ sont supplémentaires. Soit $p$ le projecteur dur $F$ de direction $G$. Déterminer les images par $p$ des vecteurs de la base canonique de $E$. 

\exo  [Level=1,Fight=0,Learn=1,Field=\DimensionFinie,Type=\Exercices,Origin=] lx. 
Soient $a\in\ob R$ et soit $E:=\{f:\ob R\to\ob R\hbox{ qui à }x\mapsto a+bx+cx^2:(a,b,c)\in\ob R^3\}$. \pn
a) déterminer une base de $E$ et sa dimension. \pn
b) Montrer que l'application $f:\App E\to\ob R^3,P\mapsto\b(P(a),P'(a),P''(a)\b)\ppA$ est un isomorphisme. \pn
c) Trouver un polynôme $P$ de degré inférieur à $2$ tel que $\b(P(a),P'(a),P''(a)\b)=(0,1,2)$. 

\exo  [Level=1,Fight=0,Learn=1,Field=\DimensionFinie,Type=\Exercices,Origin=] ly. 
Soit $E$ un $\ob K$-espace vectoriel de dimension finie et soit $f\in\sc L(E)$. \pn
a) Prouver que $\Ker(f)\subset\Ker(f^2)$ et que $\IM(f^2)\subset\Im(f)$. \pn
b) Montrer que $\Ker(f)=\Ker(f^2)\Longleftrightarrow E=\Ker(f)\oplus\IM(f)\Longleftrightarrow \Im(f)=\Im(f^2)$. 

\exo  [Level=1,Fight=2,Learn=2,Field=\DimensionFinie,Type=\Exercices,Origin=] lz. 
Soit $E$ un $\ob K$-espace vectoriel et $f$ un endomorphisme de $E$ tel que $f^p=0$ et $f^{p-1}\neq0$ (on dit alors que $f$ est nilpotent d'ordre $p$). \pn
a) montrer qu'il existe un vecteur $x$ de $E$ tel que la famille $\Q(x,f(x),f^2(x),\cdots,f^{p-1}(x)\W)$ soit libre.  \pn
b) Si $E$ est de dimension finie $d\ge1$ et si $f^m=0$ pour un entier $md$, prouver que $f^d=0$. 

\exo  [Level=1,Fight=0,Learn=1,Field=\DimensionFinie,Type=\Exercices,Origin=] ma. 
Dans $\ob R^4$, on pose $u:=(1,2,3,4)$, $v:=(1,1,1,3)$, $w:=(2,1,1,1)$, $x:=(-1,0,-1,2)$ et $y:=(2,3,0,1)$. 
Dimensions (à justifier) des espaces $U:=\Vect(u,v,w)$, $V:=\Vect(x,y)$, $U+V$ et $U\cap V$ ? 

\exo  [Level=1,Fight=0,Learn=1,Field=\DimensionFinie,Type=\Exercices,Origin=] mb. 
Soit $u$ l'application définie par 
$$
\eqalign{
u:\ob R^3&\to\ob R^3\cr
(x,y,z)&\mapsto (y+z,z+x,x+y)
}
$$
a) Montrer que $u\in\sc Gl(\ob R^3)$. \pn
b) On considère $E:=\{(x,y,z)\in\ob R^3:x+y+z=0\}$ et $F:=\{(x,y,z):x=y=z\}$. 
Montrer que $E$ et $F$ sont des sous-espaces vectoriels de $\ob R^3$ et déterminer $u(E)$ et $u(F)$. 

\exo  [Level=1,Fight=0,Learn=1,Field=\Rang,Type=\Exercices,Origin=] mc. 
Soient $E$ et $F$ deux $\ob K$-espaces vectoriels, $E$ étant de dimension finie, et soient $(f,g)\in\sc L(E,F)^2$. \pn
a) Comparer $\IM(f+g)$ et $\IM(f)+\IM(g)$. En déduire que $\hbox{rg}(f+g)\le \hbox{rg}(f)+\hbox{rg}(g)$. \pn
montrer que 
$$
\hbox{rg}(f+g)=\hbox{rg}(f)+\hbox{rg}(g)\Longleftrightarrow\Q\{
\eqalign{
\IM(f)\cap\IM(g)=\{0\},\cr
E=\Ker(f)+\Ker(g).
}
\W.
$$

\exo [Origin=,Level=2,Fight=1,Learn=1,Field=\RécurrencesLinéaires,Type=\Exercices,Indication={Faire apparaitre une récurence linéaire et en déduire une expression de $u$.}] md. 
Soient $(a,b)\in\Q]0,+\infty\W[^2$ et soit $u$ la suite définie par $u_0:=a$, $u_1:=b$ et 
$$
\forall n\in\ob N, \qquad u_{n+1}=\sqrt{u_{n+1}u_n}. 
$$
Montrer que $u$ est une suite convergente et calculer sa limite. 

\exo [Origin=,Level=2,Fight=2,Learn=1,Field=\RécurrencesLinéaires,Type=\Exercices] me. 
Calculer la suite $u$ définie par $u_0=u_1=1$ et 
$$
\forall n\in\ob N^*, \qquad u_{n+1}-u_n-u_{n-1}={n\F2^n}
$$
puis en déduire la limite de $u$. 

\exo [Level=1,Fight=1,Learn=1,Field=\Suites,Type=\Exercices,Origin=] mf. 
Étudier la suite de terme général $\ds u_{n+1}={4u_n+2\F u_n+5}$ et $u_0\in\ob R$. 
 
\exo [Level=1,Fight=1,Learn=1,Field=\Suites,Type=\Exercices,Origin=] mg. 
Étudier la suite de terme général $\ds u_{n+1}={7u_n-12\F 3u_n-5}$ et $u_0\in\ob R$. 

\exo [Level=1,Fight=1,Learn=1,Field=\Suites,Type=\Exercices,Origin=] mh. 
Calculer $\ds\lim_{n\to+\infty}\Q(\cos \Q({n\pi\F 3n+1}\W)+\sin\Q({n\pi\F 6n+1}\W)\W)^n$. 

\exo [Level=1,Fight=2,Learn=2,Field=\Suites,Type=\Exercices,Origin=] mi. 
Soit $u\in\ob K^{\ob N^*}$ une suite convergeant vers $\ell\in\ob K$. Prouver que la suite de terme général 
$$
v_n={1\F n}\sum_{1\le k\le n}u_k
$$
converge et déterminer sa limite. 

\exo [Level=1,Fight=2,Learn=1,Field=\Suites,Type=\Exercices,Origin=] mj. 
Démontrer que les suites $\ds u_n=\Q(1+{1\F n}\W)^n$ et $\ds v_n=\Q(1+{1\F n}\W)^{n+1}$ sont adjacentes

\exo [Level=1,Fight=1,Learn=1,Field=\Suites,Type=\Exercices,Origin=] mk. 
Prouver que les suites $\ds u_n=\prod_{1\le k\le n}\Q(1+{1\F k^2}\W)$ et $\ds v_n=u_n+{u_n\F n}$ sont~adjacentes

\exo [Level=1,Fight=1,Learn=1,Field=\Suites,Type=\Exercices,Origin=] ml. 
Soient $a>0$, $b>0$ et soient $u$ et $v$ les suites de terme général $u_n={1\F n}\sum_{1\le k\le n}(a+bk)$ et $v_n=\prod_{1\le k\le n}(a+bp)^{1/n}$. 
Calculer $\lim_{n\to+\infty}{u_n\F v_n}$. 

\exo [Level=2,Fight=2,Learn=2,Field=\Suites,Type=\Exercices,Origin=] mm. 
Soit $\alpha\in\ob R$. On pose $v_1:=\alpha$, 
$$
\forall n\in\ob N^*, \qquad v_n=\sqrt{1+v_{n-1}^2}\qquad\hbox{et}\qquad s_n=\sum_{1\le k\le n}{1\F v_k}.
$$
Étudier les suites de terme général $u_n=s_n-2v_n$, $v_n$, $s_n$ et $\ds{s_n\F v_n}$. 

\exo [Level=1,Fight=1,Learn=1,Field=\Suites,Type=\Exercices,Origin=] mn. 
Montrer que $\sum_{0\le k\le n}\e^{k^2}\sim\e^{n^2}$. 

\exo [Level=1,Fight=2,Learn=2,Field=\Suites|\DéveloppementsLimités,Type=\Exercices,Origin=] mo. 
Démontrer que la suite $u_n=\prod_{1\le k\le n}\cos\Q({k\F n^2}\W)$ converge vers une limite $L$ et donner un équivalent simple de $u_n-L$. 

\exo [Level=1,Fight=1,Learn=1,Field=\Suites,Type=\Exercices,Origin=] mp. 
Étudier l'existence d'un plus grand et d'un plus petit élément, d'une borne inférieure et d'une borne supérieure pour l'ensemble 
$$
A:=\Q\{(-1)^n(1-{1\F n}):n\in\ob N^*\W\}
$$
Les déterminer en cas d'existence. 

\exo [Level=1,Fight=1,Learn=1,Field=\NombresEntiers,Type=\Exercices,Origin=] mq. 
a) Pour $x\in\ob R$, on pose $f(x):=[x]+[-x]$. Calculer $f(x)$ pour $x\in\ob R$. \pn
b) Pour $x\in\ob R$ et $n\in\ob N^*$ montrer que 
$$
0\le [nx]-n[x]\le n-1
$$
et en déduire $\Q[{[nx]\F n}\W]$. 

\exo [Level=1,Fight=0,Learn=0,Field=\Suites,Type=\Cours,Origin=] mr. 
Soit $u$ une suite convergeante d'entiers relatifs. Montrer qu'elle est stationnaie (i.e. constante à partir d'un certain rang). 

\exo [Level=1,Fight=1,Learn=1,Field=\Suites,Type=\Cours,Origin=] ms. 
Soit $u_n$ une suite réelle et $\ell\in\ob R$. 
Traduire en fran\c cais les phrases mathématiques suivantes : 
$$
\eqalignno{
&\exists \epsilon> 0, \quad \exists N\in\ob N, \quad \forall n\ge N, \qquad |u_n-\ell|\le\epsilon,&a:
\cr
&\exists N\in\ob N, \quad \forall\epsilon> 0, \quad \forall n\ge N, \qquad |u_n-\ell|\le\epsilon&b:
\cr
&\forall\epsilon> 0,\quad \forall n\in\ob N, \quad \exists N\in\ob N, \qquad (n\ge N\Longrightarrow |u_n-\ell|\le\epsilon)& c:
}
$$

\exo [Level=1,Fight=1,Learn=1,Field=\Suites,Type=\Exercices,Origin=] mt. 
On définit la suite $u$ par $u_0=1$ et par 
$$
\forall n\in\ob N, \qquad u_{n+1}={u_n\F u_n^2+1}
$$
Montrer que la suite $u$ est convergeante et déterminer sa limite. 

\exo [Level=1,Fight=1,Learn=1,Field=\Suites,Type=\Exercices,Origin=] mu. 
Etablir que 
$$
\forall x> 0, \qquad x-{x^2\F2}<\ln(1+x)<x.
$$
Soit $a\in\ob R_+^*$. On considère la suite $u=(u_n)_{n\in\ob N^*}$ définie par 
$$
\forall n\in\ob N^*, \qquad u_n:=\prod_{k=1}^n\Q(1+{ka\F n^2}\W)
$$
Montrer que $u$ est convergente et calculer sa limite. 

\exo [Level=1,Fight=1,Learn=1,Field=\Suites,Type=\Exercices,Origin=] mv. 
Soit $u$ la suite définie par 
$$
\forall n\ge1, \qquad  u_n:=\sum_{n\le k\le 2n}{1\F k}.
$$ 
Montrer que la suite $u$ est convergente et prouver que sa limite $\ell$ vérifie ${1\F 2}\le\ell\le 1$. 

\exo [Level=1,Fight=2,Learn=2,Field=\Suites,Type=\Exercices,Origin=] mw. 
Déterminer la nature et la limite éventuelle des suites de terme général 
$$
u_n={1\F n^2}\sum_{k=1}^n[kx],\qquad v_n=\sum_{1\le k\le n}{n\F n^2+k}, \qquad w_n={1!+2!+\cdots+n!\F (n+1)!}
$$

\exo [Level=1,Fight=2,Learn=2,Field=\Suites,Type=\Exercices,Origin=] mx. 
a) Etablir que 
$$
\forall x> 0, \qquad {1\F x+1}\le \ln(1+x)-\ln(x)\le {1\F x}
$$
b) Soient $u$ et $v$ les suites de terme général 
$$
u_n=\sum_{k=1}^n{1\F k}-\ln(n)\qquad\hbox{ et }\qquad v(n)=\sum_{k=1}^n{1\F k}-\ln(n+1)
$$
Montrer que $u$ et $v$ admettent la même limite, que l'on ne calculera pas, c'est la constante d'Euler $\gamma$. 

\exo [Level=1,Fight=1,Learn=1,Field=\Suites,Type=\Exercices,Origin=] my. 
Soient $u$ et $v$ les deux suites définies par la donnée de $(u_0,v_0)\in\ob R^2$ et par 
$$
\forall n\in\ob N, \qquad \Q\{\eqalign{u_{n+1}={u_n+v_n\F 2}\cr
v_{n+1}={2u_nv_n\F u_n+v_n}
}
\W. 
$$ 
Montrer que $u$ et $v$ convergent vers la même limite, que l'on determinera. 

\exo [Level=1,Fight=1,Learn=1,Field=\Suites,Type=\Exercices,Origin=] mz. 
Soit $u$ la suite de terme général $\ds u_n=\sum_{k=1}^n{(-1)^{k-1}\F k}$. Montrer que les suites de termes général $v_n=u_{2n}$ et $w_n=u_{2n+1}$ sont adjacentes. En déduire la nature de la suite $u$. 

\exo [Level=1,Fight=1,Learn=1,Field=\Suites,Type=\Exercices,Origin=] na. 
soit $u$ la suite définie par $u_0=1$ et 
$$
\forall n\in\ob N^*, \qquad u_n={1\F 3+u_{n-1}}
$$
Soient $\alpha$ et $\beta$ les deux solutions de l'équation $x={1\F 3+x}$ avec $\alpha> \beta$. Montrer que la suite $z$ de terme général $\ds z_n={u_n-\alpha\F u_n-\beta}$ est géométrique, puis conclure que la suite $u$ converge vers $\alpha$. 


\exo [Level=1,Fight=1,Learn=1,Field=\Suites,Type=\Maple,Origin=,Indication={On pourra utiliser une boucle pour les calculer...}] nb. 
a) Pour différentes valeurs de $u_0> 0$ et $v_0> 0$, calculer les $10$ premiers termes des suites définies par 
$$
\forall n\in\ob N,\qquad   u_n+1:={u_n+v_n\F 2}\qquad \hbox{et}\qquad v_{n+1}:=\sqrt{u_nv_n}
$$
b) Que peut on conjecturer ? 

\exo [Level=1,Fight=1,Learn=1,Field=\Suites,Type=\Maple,Origin=] nc. 
Soient $u$ et $v$ les deux suites définies par 
$$
\forall n\in\ob N^*, \qquad u_n:=2\sqrt n-\sum_{k=1}^n{1\F \sqrt k}\qquad \hbox{et}
\qquad v_n:=2\sqrt{n+1}-\sum_{k=1}^n{1\F\sqrt k}
$$
a) Tracer sur un graphique les deux suites pour $n$ variant de $1$ à $100$. \pn
{\it Pour tracer ces points, on pourra utiliser PLOT(POINTS([x1,y1],[x2,y2],...,[xn,yn]))}\pn
b) Montrer que $\lim_{n\to+\infty}(v_n-u_n)=0$, que $u_n$ est croissante et que $v_n$ est décroissante.
\pn
c) Calculer $\lim_{n\to+\infty}u_n$. 

\exo [Level=1,Fight=2,Learn=1,Field=\DimensionFinie,Type=\Exercices,Origin=] nd. 
Soit $f:\ob R^3\to\ob R^3$ l'application définie par 
$$
\forall (x,y,z)\in \ob R^3, \qquad f(x,y,z):={1\F 3}(5x-y-z,-x+5y-z,-x-y+5z)
$$ 
a) Définir correctement $f$ puis montrer que $f$ est linéaire. \pn
b) Montrer que $f$ est un automorphisme de $\ob R^3$. \pn
c) Montrer que $a_1:=(1,1,1)$, $a_2:=(1,-1,0)$ et $a_3=(1,1,-2)$ est une base de $\ob R^3$. \pn
d) Exprimer $f(a_1)$, $f(a_2)$ et $f(a_3)$ comme une combinaison linéaire des vecteurs $a_1$, $a_2$ et $a_3$. \pn
e) Calculer $f\circ f(x,y,z)$ puis $f\circ f\circ f(x,y,z)$ et essayez d'en déduire des réels $(\alpha, \beta,\gamma)$ tels que 
$$
\forall (x,y,z)\in\ob R^3, \qquad \alpha f^3(x,y,z)+\beta f^2(x,y,z)+\gamma f(x,y,z)=(x,y,z)
$$

\exo [Level=1,Fight=1,Learn=1,Field=\GéométrieSpatiale,Type=\Exercices,Origin=] ne. 
Trouver $m$ pour que les droites suivantes soient coplanaires  
$$
\Delta:\qquad \Q\{\eqalign{
x={z\F m}+m\cr
y=-2z+3-m
}\W.\qquad\qquad D:\quad \Q\{\eqalign{
x={-z\F2m}-m+{1\F 2m},\cr
y=-{z\F 2}-m+3
}\W.
$$

\exo [Level=1,Fight=1,Learn=1,Field=\Suites,Type=\Exercices,Origin=] nf. 
Etude graphique de la suite $u$ définie par $u_0=1$ et 
$$
\forall n\in\ob N, \qquad u_{n+1}={u_n-\ln(1+u_n)\F u_n^2}
$$

\exo [Level=1,Fight=1,Learn=1,Field=\Rang,Type=\Exercices,Origin=,Indication={Utiliser la méthode diabolique}] ng. 
Soit $E$ un $\ob K$-espace vectoriel de dimension finie et $u\in\sc L(E)$. Prouver que 
$$
\Ker(u)=\IM(u)\Longleftrightarrow\Q\{\eqalign{
u^2=0, 
\cr
\dim(E)=2\hbox{rang(u)}
}\W.
$$

\exo [Level=1,Fight=1,Learn=1,Field=\Limites,Type=\Exercices,Origin=] nh. 
Calculer les limites $\ds \lim_{x\to{\ss\pi\F\ss4}}{\sin x-\cos x\F x-{\pi\F 4}}$ et $\ds \lim_{x\to{\ss\pi\F\ss2}}{\ln(\sin^2x)\F(x-{\pi\F 2})^2}$. 


\exo [Level=1,Fight=1,Learn=1,Type=\Exercices,Field=\Continuité,Origin=] ni. 
a) Soit $I$ un intervalle réel et  soit $k\ge0$. Prouver qu'une fonction $k$-lipshitzienne $f:I\to\ob C$ est continue. \pn
b) Montrer que $x\mapsto x^2$ est $2$-lipschitzienne sur $[0,1]$ mais pas lipschitzienne sur $\ob R$. 

\exo [Level=1,Fight=1,Learn=1,Type=\Exercices,Field=\Continuité,Origin=] nj. 
Peut-on prolonger par continuité en $-1$, en $0$, en $1$ la fonction $f$ définie par 
$$
\forall x\in\ob R\ssm\{-1,0,1\}, \qquad f(x):={\b(\ln|x|\b)^n\F x^2-1}.
$$

\exo [Level=1,Fight=1,Learn=1,Type=\Exercices,Field=\Continuité,Origin=] nk. 
Soit $f:\ob R\to\ob R$ la fonction qui à $x\in\ob R$ associe $f(x):=x-[x]-{1\F 2}$. Montrer que $f$ est périodique et bornée sur $\ob R$. Étudier la continuité de $f$ sur $\ob R$. 

\exo [Level=1,Fight=1,Learn=1,Type=\Exercices,Field=\Continuité,Origin=] nl. 
a) Soit $f:\Q[a,+\infty\W[\to\ob R$ une fonction continue admettant une limite finie en $+\infty$. \pn Montrer que $f$ est bornée sur $\Q[a,+\infty\W[$. \pn
b) Montrer qu'une fonction périodique et continue $f:\ob r\to\ob R$ est nécéssairement bornée. \pn
c) Montrer qu'une fonction polynomiale de degré impair admet au moins une racine réelle. 

\exo [Level=1,Fight=1,Learn=2,Type=\Exercices,Field=\Continuité,Origin=,Indication={On pourra utiliser la fonction $\varphi$ définie sur $[a,b]$ par $\varphi(x):=f(x)-x$.}] nm. 
Soit $f:[a,b]\to[a,b]$ une fonction continue. Montrer qu'il existe $c\in[a,b]$ telle que $f(c)=c$. 


\exo [Level=1,Fight=2,Learn=2,Type=\Exercices,Field=\Continuité,Origin=] nn. 
Soient $f$ une fonction continue sur $[0,1]$. On suppose que 
$$
\forall x\in[0,1], \qquad 0<f(x)<1. 
$$  
Soit $x=(x_n)_{n\in\ob N}$ une suite d'éléments de $[0,1]$. Pour tout $n\in\ob N$, on pose $y_n:=f(x_n)^n$. \pn 
a) Montrer qu'il existe $m\in\Q]0,1\W[$ tel que 
$$
\forall x\in[0,1], \qquad 0<f(x) \le m.  
$$
En déduire que la suite $y=(y_n)_{n\in\ob N}$ converge et déterminer sa limite. 

\exo [Level=1,Fight=1,Learn=2,Type=\Colles,Field=\Continuité,Origin=] no. 
Déterminer toutes les fonctions continues $f:\ob R\to\ob R$ telles que 
$$
\forall x\in\ob R, \qquad f(x)^2=f(x). 
$$

\exo [Level=1,Fight=2,Learn=2,Type=\Colles,Field=\Continuité,Origin=] np. 
Montrer que les fonctions constantes sont les seules fonctions continues $f:\ob R\to\ob R$ vérifiant 
$$
\forall x\in\ob R, \qquad f(2x)=f(x). 
$$

\exo [Level=1,Fight=2,Learn=3,Type=\Exercices,Field=\TravauxDirigés,Origin=] nq. 
Soit $f:\ob R\to\ob R$ une fonction continue vérifiant 
$$
\forall (x,y)\in\ob R^2, \qquad f(x+y)=f(x)+f(y).\eqdef{Norrah}
$$
a) Montrer qu'il existe une unique constante réelle, notée $c$, telle que $\forall n\in\ob N, \ f(n)=cn$.  \pn
b) En déduire que $\forall n\in\ob Z, \ f(n)=cn$.  \pn
c) En déduire que $\forall x\in\ob Q, \ f(x)=cx$.  \pn
d) En déduire que $\forall x\in\ob R, \ f(x)=cx$. \pn
e) Déterminer toutes les fonctions continues $f:\ob R\to\ob R$ vérifiant \eqref{Norrah} 

\exo [Level=1,Fight=1,Learn=1,Type=\Exercices,Field=\TravauxDirigés,Origin=,Indication={Le nombre $f(t)$ representera la distance parcourue au bout de $t$ heures. On admettra que la fonction $f$ est continue, on considérera la fonction $g:t\mapsto f(t+1)-f(t)$ et on pourra envisager trois cas, où $25$ represente ici la fonction $x\mapsto 25$,  
cas 1 : $g<25$, cas 2 : $g> 25$ et cas 3 : ? }] nr. 
En cycliste parcours $100$ km en quatre heures. Montrer qu'il existe une intervalle d'une heure pendant lequel il a parcopuru exactement $25$ km. 

\exo [Level=1,Fight=1,Learn=1,Field=\Suites,Type=\Exercices,Origin=] ns. 
Que dire de deux suites $(u_n)$ et $(v_n)$ de $[0,1]$ vérifiant $\lim_{n\to+\infty}u_nv_n=1$ ?
 
\exo [Level=1,Fight=2,Learn=2,Field=\Suites,Type=\Exercices,Origin=] nt. 
Soit $(u_n)$ une suite bornée vérifiant  
$$
\forall n\ge1,\qquad  2u_n\le u_{n-1}+u_{n+1}.
$$ 
Montrer la suite $u_n$ est convergente. 

\exo [Level=1,Fight=0,Learn=0,Field=\Suites,Type=\Exercices,Origin=] nu. 
Soient $a$, $b$ et $c$ trois réels distincts, $a$ étant non nul.
On suppose que $a,b,c$ sont en progression arithmétique et que $3a,b,c$
sont en progression géométrique. Que dire de la raison de cette
progression géométrique?

\exo [Level=1,Fight=2,Learn=2,Field=\Suites,Type=\Exercices,Origin=] nv. 
On se donne une suite réelle $(u_n)$. On suppose que les suites de terme général respectif $a_n=u_{2n}$, 
$b_n=u_{2n+1}$ et $c_n=u_{3n}$ sont convergentes. Montrer
que la suite $(u_n)$ est convergente. 

\exo [Level=1,Fight=1,Learn=1,Field=\Suites,Type=\Exercices,Origin=] nw. 
On considère la suite de terme général $u_n=\sqrt{1+\sqrt{2+\sqrt{\ldots+\sqrt n}}}$. 
Montrer que 
$$
\forall n\in\ob N, \qquad , u_{n+1}^2\le1+\sqrt2u_n.
$$ 
La suite $(u_n)$ est-elle convergente?

\exo [Level=1,Fight=2,Learn=2,Type=\Exercices,Field=\TravauxDirigés,Origin=] nx. 
Déterminer les fonctions continues $f:\ob R\to\ob R$ telles que 
$$
\forall x\in\ob R, \quad f(2x+1)=f(x).
$$

\exo [Level=1,Fight=2,Learn=2,Type=\Exercices,Field=\Exercices,Origin=] ny. 
Détermioner les applications continues $f:\ob C^*\to\ob C$ telles que 
$$
\forall (a,b)\in\ob C^*\times\ob C^*,\qquad f(ab)=f(a)+f(b).
$$

\exo [Level=1,Fight=0,Learn=0,Field=\Matrices,Type=\Exercices,Origin=] nz. 
Soit $a=\pmatrix{1&-1\cr2&0\cr3&1}$.  Expliciter l'application linéaire $f$ associé à $A$ dans les bases canoniques.

\exo [Level=1,Fight=0,Learn=0,Field=\Matrices,Type=\Exercices,Origin=] oa. 
Soit $f:\App\ob R^2\to\ob R^2,(x,y)\mapsto(ax+cy,bx+dy)\ppA$. Déterminez la matrice canoniquement associée à $f$. 

\exo [Level=1,Fight=2,Learn=2,Field=\Matrices,Type=\Exercices,Origin=] ob. 
Soit $A:=\pmatrix{1&2\cr3&4\cr-1&1}$. Déterminez toutes les matrices $B$ telles que $BA=\pmatrix{0&2\cr1&0}$. 

\exo [Level=1,Fight=0,Learn=0,Field=\Matrices,Type=\Exercices,Origin=] o
c. On pose $A=\pmatrix{1&-1\cr3&1}$, $B=\pmatrix{1&0\cr0&-1}$, $C=\pmatrix{2&0\cr1&0}$, $D:=\pmatrix{-1&0\cr0&-1}$ et $E=\pmatrix{-1&-2\cr7&3}$. Les familles $\sc F_1:=(A,B,C)$,  $\sc F_2:=(A,B,C,D)$ et $\sc F_3:=(A,B,C,D,E)$ 
sont-elles libres dans $\sc M_2(\ob R)$ ? 


\exo [Origin=,Level=1,Fight=0,Learn=0,Field=\Matrices,Type=\Exercices] od. 
Calculer l'inverse de la matrice $A=\pmatrix{8&4&2\cr2&8&4\cr4&2&8}$. 

\exo [Level=1,Fight=2,Learn=2,Field=\Matrices,Type=\Exercices,Origin=] oe. 
On pose $A:=\pmatrix{-1&4&6\cr0&-1&3\cr0&0&-1}$. Exprimer $A$ sous la forme $A=-I_3+N$. Calculer $N^2$ puis $N^3$. En déduire $A^n$ pour $n\in\ob N$, puis montrer que la formule est vraie également pour $n\in\ob Z$. 


\exo [Level=1,Fight=1,Learn=1,Field=\Matrices,Type=\Exercices,Origin=] of. 
Soit $J$ la matrice de $\sc M_n(\ob K)$ dont tous les coeffficients sont égaux à $1$. Calculer $J^2$ puis $J^p$ pour $p\in\ob N^*$. 

\exo [Level=1,Fight=1,Learn=1,Field=\Matrices,Type=\Exercices,Origin=] og. 
Soit $n\ge2$ et $A=(a_{i,j})\in\sc M_n(\ob R)$ telle que 
$\ds
a_i,j=\Q\{\eqalign{
&1\hbox{ si $i\neq j$}\cr
&0\hbox{ si $i=j$}}
\W.
$
Calculer $(A+I_n)^2$, en déduire que $A$ est inversible et calculer $A^{-1}$. 

\exo [Level=1,Fight=0,Learn=0,Field=\Matrices,Type=\Exercices,Origin=] oh. 
Déterminer toutes les matrices diagnales $M\in\sc M_3(\ob C)$ telles que $M^2-5M+6I_3=0$. 

\exo [Origin=,Level=1,Fight=3,Learn=3,Field=\EspacesVectoriels|\Matrices|\RécurrencesLinéaires,Type=\Exercices] oi. 
On pose $M=\pmatrix{2&-2&1\cr2&-3&2\cr-1&2&0}$. \pn
a) Calculer $(M-I_3)(M+3I_3)$. En déduire $M^2$ en fonction de $M$ et $I_3$. \pn
b) Justifier sans calculs que $M$ est inversible et déterminer $M^{-1}$. \pn
c) Montrer que : pour chaque  $n\in\ob N$, $\exists!(a_nb_n)\in\ob R^2$ tels que $M^n=a_nM+b_nI_3$ et déterminer une relation de récurence donnant $a_{n+1}$ et $b_{n+1}$ en fonction de $a_n$ et $b_n$. \pn
d) En déduire que la suite $a_n$ satisfait  une récurence linéaire homogène du second ordre. \pn
e) En remarquant que $a_n+b_n$ reste constant, montrer que la suite $a_n$ satisfait une récurence linéaire du premier ordre, avec second membre. \pn
f) Déterminer $a_n$ et $b_n$ en fonction de $n$. 

\exo [Level=1,Fight=0,Learn=0,Field=\Matrices,Type=\Exercices,Origin=] oj. 
Soit $A=\pmatrix{a&c\cr b&d}\in\sc M_2(\ob K)$. \pn
a) Montrer que $A^2-(a+d)A+(ad-bc)I_2=0$. En déduire une condition nécéssaire et suffisante pour que $A\in\sc Gl_2(\ob K)$ et l'expression de $A^{-1}$ dans ce cas. \pn
b) Montrer que $A^n\in\hbox{Vect}(A,I_2)$ pour $n\in\ob N$ (et pour $n\in\ob Z$ si $A$ est inversible). 

\exo [Level=1,Fight=1,Learn=2,Field=\Matrices,Type=\Exercices,Origin=] ok. 
Soit $(e_1,e_2)$ la base canonique de $E=\ob R^2$, soit $(f_1,f_2,f_3)$ celle de $F=\ob R^3$ et $(g_1,g_2,g_3,g_4)$ celle de $G=R^4$. On considère les matrices
$$
A=\pmatrix{1&1&1&2\cr1&2&0&1\cr1&3&-1&0}\qquad\hbox{et}\qquad B=\pmatrix{1&b&c\cr3&3b&3c}
$$
a) On note $u$ l'application linéaire associée à $A$. Déterminer des bases de $\ker(u)$ et $\IM(u)$. \pn
b) Déterminer de même des bases de $\ker(v)$ et $\IM(v)$, où $v$ est l'application linéaire associée à $B$. 
\pn
c) Déterminer $b$ et $c$ pour que $\IM(u)\subset\ker(v)$ et montrer qu'alors $\IM(u)=\ker(v)$. \pn
d) Soit $w=v\circ u$. (pour $b$ et $c$ quelconques). Préciser la matrice de $w$, puis $\ker(w)$ et $\IM(w)$. 

\exo [Origin=,Level=1,Fight=1,Learn=1,Field=\EspacesVectoriels|\Matrices,Type=\Exercices] ol. 
Soit $\sc B:=(e_1,e_2,e_3)$ une base d'un espace $E$ et soit $u\in\sc L(E)$ tel que
$$
u(e_1)=u(e_3)=e_1+2e_2-e_3\qquad\sbox{ et } u(e_2)=0.
$$
a) Déterminer la matrice $A$ de $u$ dans la base $\sc B$. \pn
b) En déduire le noyau $\ker(u)$ puis l'image $\IM(u)$ de l'endomorphisme $u$\pn
c) Montrer que $\IM(u)\subset\ker(u)$. \pn
d) Que vaut alors $u^2$ ? Vérifier ce résultat en calculant $A^2$. 


\exo [Level=1,Fight=2,Learn=2,Field=\Matrices,Type=\Exercices,Origin=] om. 
Matrices nilpotentes. On dit que $A\in\sc M_n(\ob K)$ est nilpotente s'il existe $p\in\ob N^*$ tel que $A^p=0$ et qu'elle est nilpotente d'ordre $p$ si $A^{p-1}\neq0$ et $A^p=0$. \pn
a) En utilisant l'endomorphisme de $\ob K^n$ associé, calculer les puissances successives de la matrice 
$$
J_n:=\pmatrix{
0&1&0&\ldots&\ldots&0\cr
\vdots&0&1&0&&\vdots\cr
\vdots&&\ddots&\ddots&\ddots&\vdots\cr
\vdots&&&0&1&0\cr
\vdots&&&&0&1\cr
0&\ldots&\ldots&\ldots&\ldots&0\cr
}
$$
b) Soient $A\in\sc M_n(\ob K)$ nilpotente d'ordre $p$ et $B\in\sc M_n(\ob K)$ nilpotente d'ordre $q$. Si $A$ et $B$ commuttent montrer que $A+B$ est nilpotente d'ordre au plus $m=p+q-1$. Autrement dit que $(A+B)^m=0$. \pn
c) Soit $A$ nilpotente d'ordre $p$. Montrer que $C:=I_n-A$ est inversible et calculer son inverse à l'aide des puissances de $A$. 
\pn
d) Soient $A\in\sc Gl_n(\ob K)$ et $N\in\sc M_n(\ob K)$. On suppose que $A$ et $B$ commuttent et que $B^4=0$. Montrer que $A-B$ est inversible et calculer son inverse en fonction de $A$, $B$ et $A^{-1}$. 

\exo  [Level=1,Fight=1,Learn=1,Field=\Dérivation,Type=\Exercices,Origin=] oo. 
Soit $f$ la fonction définie sur $\Q]-1,0\W[\cup\Q]0,1\W[$ par $\ds f(x)={|x|\F x}\ln\b(1-|x|\b)$. Montrer que $f$ admet un prolongement par continuité en $0$, noté $\tilde f$. La fonction $\tilde f$ est elle dérivable en $0$ ? 

\exo [Level=1,Fight=0,Learn=0,Field=\Limites,Type=\Exercices,Origin=] op. 
Calculer la limite $\ds\lim_{x\to{\pi\F3}}{\sin(2x+{\pi\F3})\F x-{\pi\F3}}$. 

\exo [Level=1,Fight=0,Learn=0,Field=\Limites,Type=\Exercices,Origin=] oq. 
Pour $a\in\ob R^*$, calculer la limite $\ds \lim_{x\to a}{\cos(ax)-\cos(a^2)\F x^2-a^2}$. 

\exo  [Level=1,Fight=1,Learn=1,Field=\Dérivation,Type=\Exercices,Origin=] or. 
Étudier la dérivabilité et calculer la dérivée de la fonction $\ds f:x\mapsto\arcsin\Q({1+x\F 1-x}\W)$. 

\exo [Level=1,Fight=1,Learn=1,Field=\Dérivation,Type=\Exercices,Origin=] os. 
Étudier la dérivabilité et calculer la dérivée de la fonction $\ds f:x\mapsto\arctan\Q({1-\sin(x)\F 1+\sin(x)}\W)$.

\exo [Level=1,Fight=1,Learn=1,Field=\Dérivation,Type=\Exercices,Origin=] ot. 
Pour $n\in\ob N$ et $x\in\ob R$, calculer la dérivée $n^\ieme$  suivante : $\ds {\d^n\F \d x^n}\B(x^2(1+x)^n\B)$.

\exo [Level=1,Fight=1,Learn=1,Field=\Dérivation,Type=\Exercices,Origin=] ou. 
Pour $n\in\ob N$, établir que 
$$
\forall x\in\ob R^*, \qquad {\d^n\F\d x^n}\B(x^n\e^{1/x}\B)={(-1)^{n+1}\F x^{n+2}}\e^{1/x}
$$

\exo [Level=1,Fight=1,Learn=1,Field=\Dérivation,Type=\Exercices,Origin=] ov. 
i) Pour $n\in\ob N$ et $(a,b)\in\ob C^2$, calculer $\ds {\d^n\F\d x^n}\B((x-a)^n(x-b)^n\B)$. \pn 
ii) En utilisant ce qui précède pour $a=b$, calculer 
$$ 
S_n:=\sum_{k=0}^n{n\choose k}^2.
$$ 

\exo [Level=1,Fight=1,Learn=1,Field=\Fonctions,Type=\Exercices,Origin=] ow. Pour $x\in[-1,1]$, on pose $f(x)=\arcsin(x)$. Trouver une équation différentielle  faisant intervenir $x$, $f$, $f$, $f''$ afin d'en déduire $f^{(n)}(0)$ pour $n\in\ob N^*$. 

\exo [Level=1,Fight=1,Learn=1,Field=\Dérivation,Type=\Exercices,Origin=] ox. 
Soit $f$ une fonction dérivable sur $\ob R$. On pose 
$$
g(x):=\Q\{
\eqalign{
{f(x)-f(0)\F x}\qquad\hbox{si $x\neq 0$},\cr
f'(0)\qquad\hbox{si $x=0$}.
}
\W.
$$
Montrer que $g$ est continue sur $\ob R$. Si $f$ est de classe $\sc C^2$, prouver que $g$ est de classe $\sc C^1$. 

\exo [Level=1,Fight=1,Learn=1,Field=\Intégration,Type=\Exercices,Origin=] oy. 
Soit $f$ une fonction deux fois dérivable sur $\ob R$, non constante, telle que $f$, $f'$ et $f''$ soient positives sur $\ob R$. Montrer que $\lim_{x\to+\infty}f(x)=+\infty$. 

\exo [Level=1,Fight=1,Learn=1,Field=\ThéorèmeDeRolle,Type=\Exercices,Origin=,Indication=
{On pourra utiliser la fonction $g$ définie par $\ds g(x)={f(x)-f(a)\F x-a}$.}] oz. 
Soit $f$ une fonction dérivable sur $[a,b]$ telle que $f(a)=f'(a)=f(b)=0$. Montrer qu'il existe $c\in\Q]a,b\W[$ tel que 
$$
f'(c)={f(c)-f(a)\F c-a}.
$$

\exo [Level=1,Fight=3,Learn=3,Field=\ThéorèmeDeRolle,Type=\Exercices,Origin=] pa. 
a) Soit $f$ une fonction dérivable sur un intervalle $I$. \pn Montrer que si $I$ s'annule en $n$ points de $I$, alors $f'$ s'annule en $n-1$ points de $I$. \pn
b) Soit $f$ une fonction trois fois dérivable qur $[a,b]$ telle que $f$ s'annule $(n+1)$ fois sur $[a,b]$. \pn 
Montrer qu'il existe $c\in\Q]a,b\W[$ tel que $$
f^{(n)}(c)=0.
$$ 
c) Pour $p\ge 3$ eyt $(a,b)\in\ob R^2$, quel est le nombre maximal de racines réelles de l'équation $x^p+ax+b=0$


\exo [Level=1,Fight=2,Learn=2,Field=\ThéorèmeDeRolle,Type=\Exercices,Origin=] pb. 
Soient $f$ et $g$ de classe $\sc C^2$ sur $[a,b]$ telles que $f(a)=g(a)$, $f(b)=g(b)$ et 
$$
\forall x\in[a,b],\qquad f''(x)\le g''(x). 
$$
Montrer alors que 
$$
\forall x\in[a,b], \qquad g(x)\le f(x).
$$ 

\exo [Level=1,Fight=1,Learn=1,Field=\ThéorèmeDeRolle,Type=\Exercices,Origin=,Indication={On pourra raisonner par l'absurde.}] pc. 
Soit $f$ une fonction dérivable sur $[0,1]$ telle que $f(0)=0$ et $f'$ ne s'annule pas sur $[0,1]$. \pn
Montrer que $f$ garde un signe constant sur $[0,1]$.  

\exo [Level=1,Fight=1,Learn=1,Field=\Intégration,Type=\Exercices,Origin=] pd. 
a) Montrer que 
$$
\forall x> 0, \qquad {1\F x+1}< \ln\Q(1+{1\F x}\W)< {1\F x}
$$
b) En utilisant la formule de Taylor, montrer que 
$$
\forall x< 0, \qquad 1+x+{x^2\F 2}+{x^3\F 6}< \e^x< 1+x+{x^2\F 2}.
$$

\exo [Level=1,Fight=3,Learn=3,Field=\Dérivation,Type=\Problèmes,Origin=] pe. 
a) Pour chaque nombre réel $t$, établir que 
$$
-1\le {2t\F1+t^2}\le 1.
$$
On désigne par $f$ une fonction dérivable sur $\ob R$ telle que 
$$
\forall (x,y)\in\ob R^2, \qquad f(x+y)={f(x)+f(y)\F1+f(x)f(y)}\leqno{(*)}
$$
b) Etablir que 
$$
\forall x\in\ob R, \qquad -1\le f(x)\le 1.
$$
c) Trouver toutes les fonctions constantes vérifiant $(*)$. \pn
d) Dans la suite on suppose que $f$ n'est pas constante. Etablir que 
$$
\forall x\in\ob R, \qquad -1< f(x)< 1
$$
e) Calculer $f(0)$. \pn
f) On pose $a=f'(0)$. En utilisant la définition de la dérivée, exprimer $f'(x)$ en fonction de $f(x)$ et de $a$. \pn
g) Montrer que $f$ définit une bijection de $\ob R$ sur l(intervalle $\Q]-1,1\W[$ et que la fonction réciproque $f^{-1}$ est dérivable sur cet intervalle. \pn
h) Calculer la dérivée de $f^{-1}$. \pn
i) Expliciter $f^{-1}(y)$ en fonction de $y$ et de $a$. \pn
j) Déterminer toutes les fonctions dérivables sur $\ob R$ et non constantes vérifiant la relation $(*)$. 

\exo [Level=1,Fight=1,Learn=0,Field=\Intégration,Type=\Exercices,Origin=] pf. 
Pour $n\ge0$, on pose 
$$
I_n:=\int_0^1t^n\ln(1+t^2)\d t.
$$ 
Après avoir justifié l'existence de $I_n$, déterminer la limite de la suite $(I_n)_{n\in\ob N}$. 

\exo [Level=1,Fight=2,Learn=2,Field=\Intégration,Type=\Exercices,Origin=,Indication={On utilisera chasles pour écrire que $u_n=\int_0^a+\int_a^1$ en utilisant un nombre $a<1$ proche de $1$.}] pg. 
Soit $f:[0,1]\to\ob R$ une fonction continue. Montrer que $u_n=\int_0^1f(t^n)\d t$ converge vers $f(0)$ quand $n\to+\infty$. \pn
 

\exo [Level=1,Fight=1,Learn=1,Field=\Intégration,Type=\Exercices,Origin=,Indication={Faire une intégration par partie.}] ph. 
Chercher un équivalent de la suite $\ds I_n=\int_0^1{t^n\d t\F 1+t^n}$. 


\exo [Level=1,Fight=0,Learn=0,Field=\Intégration,Type=\Exercices,Origin=] pi. 
Pour chaque entier $n\ge0$, on pose $\ds u_n:=\sum_{k=0}^n{(-1)^n\F k+1}$. En applicant la formule de Taylor à la fonction $x\mapsto\ln(1+x)$, montrer que la suite $u_n$ converge et déterminer sa limite. 

\exo [Level=1,Fight=1,Learn=1,Field=\Intégration,Type=\Exercices,Origin=] pj. 
Pour $k\in\ob N^*$, prouver que $\ds{1\F k+1}\le \int_k^{k+1}{\d t\F t}\le {1\F k}$. 
Encadrer la suite $\ds u_n=\sum_{k=1}^n{1\F k}$ puis montrer que $u_n\sim \ln n$. 

\exo [Level=1,Fight=2,Learn=2,Field=\Intégration,Type=\Exercices,Origin=,Indication={Commencer par encadrer l'intégrale $\ds\int_k^{k+1}\ln(x)\d x$.} ]  pk. 
Pour $n\in\ob N^*$, montrer que $\ds \int_1^n\ln (x)\d x\le\ln(n!)\le \int_2^{n+1}\ln(x)\d x$ puis en déduire que $\ln(n!)\sim n\ln n$. \pn


\exo [Level=1,Fight=3,Learn=3,Field=\Intégration,Type=\Exercices,Origin=]  pl. 
Pour $n\in\ob N$, on pose $\ds I_n=\int_0^{\pi/2}\cos(t)^n\d t$. \pn
a) Justifier l'existence de $I_n$ pour $n\in_ob N$. Comparer $I_n$ et $\ds \int_0^{\pi/2}\sin(t)^n\d t$. \pn
b) Utiliser Chasles pour intégrer que $[0,\alpha]$ et $[\alpha,{\pi\F2}]$ plutot que sur $[0,{\pi\F 2}]$ et 
démontrer que $I_n$ converge vers $0$ lorsque $n\to+\infty$. \pn
c) Chercher une relation de récurence entre $I_n$ et $I_{n+2}$. En déduire $I_{2k}$ et $I_{2k+1}$ en fonction de $k$. \pn
d) Démontrer que $\ds nI_nI_{n+1}={\pi\F 2}$ pour $n\ge0$. \pn
e) Montrer que $I_n\sim I_{n+1}$ et en déduire un équivalent simple de $I_n$ puis de ${2n \choose n}$ quand $n\to+\infty$. 

\exo [Level=1,Fight=2,Learn=2,Field=\Intégration,Type=\Exercices,Origin=] pm. 
Pour $n\in\ob N$, on pose $\ds I_n:=\int_0^1{t^n\F n!}\e^{1-t}\d t$. \pn
a) Étudier la limite de $I_n$ lorsque $n$ tends vers $+\infty$. \pn
b) Trouver une relation de récurence entre $I_n$ et $I_{n+1}$. En déduire $I_n$ en fonction de $n$. \pn
c) En déduire la valeur de $\ds\lim_{n\to+\infty}\sum_{k=0}^n{1\F k!}$, c'est à dire de la série $\ds\sum_{k=0}^\infty{1\F k!}$. 

\exo [Level=1,Fight=1,Learn=1,Field=\Suites,Type=\Exercices,Origin=] pn. 
Soit $k\ge2$ un entier. Déterminer la limite de la suite $(u_n)_{n\in\ob N^*}$ définie par 
$$
\forall n\ge1, \qquad u_n={1\F n+1}+{1\F n+2}+\cdots+{1\F kn}
$$

\exo [Level=1,Fight=1,Learn=1,Field=\Suites,Type=\Exercices,Origin=] po. 
Déterminer la limite de la suite $(v_n)_{n\in\ob N^*}$ définie par 
$$
\forall n\ge1, \qquad v_n:={1\F n^2}\Q(\sqrt{1(n-1)}+\sqrt{2(n-2)}+\cdots+\sqrt{(n-1)1}\W)
$$

\exo [Level=1,Fight=2,Learn=2,Field=\FonctionsDéfiniesParUneIntégrale,Type=\Exercices,Origin=] pp. 
Étudier la fonction $f$ définie par $\ds f(x)=\int_x^{2x}{\d t\F\root 3\of{1+t^3}}$. \pn
Préciser le domaine de définition, les variations et la limite en $+\infty$. \pn

\exo [Level=1,Fight=2,Learn=2,Field=\FonctionsDéfiniesParUneIntégrale,Type=\Exercices,Origin=] pq. 
Déterminer la limite lorsque $x\to0^+$ de $\ds\int_x^{3x}{t\d t\F \tan(t)^2}$. 

\exo [Level=1,Fight=2,Learn=2,Field=\FonctionsDéfiniesParUneIntégrale,Type=\Exercices,Origin=] pr. 
Déterminer la limite lorsque $x\to0^+$ de $\ds {1\F x^3}\int_0^x{t^2\d t\F t+\e^{3t}}$. 

\exo [Level=1,Fight=2,Learn=2,Field=\FonctionsDéfiniesParUneIntégrale,Type=\Exercices,Origin=] ps. 
soit $f$ la fonction $f$ définie sur $\Q[0,+\infty\W[\ssm\{1\}$ par 
$$
f(x)={1\F x-1}\int_1^x{t^2\d t\F\sqrt{1+t^8}}
$$
Montrer que $f$ est prolongeable par continuité en $1$ et étudier les variations de $f$ sur $\Q[0,+\infty\W[$. 

\exo [Level=1,Fight=2,Learn=2,Field=\Intégration,Type=\Exercices,Origin=,Indication={On pourras scinder l'intégrale en deux et utiliser la formule de la moyenne.}] pt. 
Étudier la limite $\ds\lim_{x\to+\infty}{1\F x^2}\int_3^{x^2+x}{\sin t\d t\F 3+\ln(\ln t)}$. \pn



\exo [Level=1,Fight=2,Learn=2,Field=\Intégration,Type=\Exercices,Origin=] pu. 
Lemme de Riemann-Lebesgue : \pn
a) Montrer que pour chaque fonction en escalier $\varphi$ sur $[a,b]$, on a 
$$
\lim_{n\to+\infty}\int_a^b\varphi(t)\sin(nt)\d t=0.
$$
b) En déduire que pour toute fonction $f$ continue par morceaux sur $[a,b]$, on a 
$$
\lim_{n\to+\infty}\int_a^bf(t)\sin(nt)\d t=0.
$$

\exo [Level=1,Fight=2,Learn=2,Field=\FonctionsDéfiniesParUneIntégrale,Type=\Exercices,Origin=] pv. 
Pour $x\in\ob R$, on note $I(x):=\int_0^{2\pi}\ln(x^2-2x\cos t+1)\d t$. \pn
a) Justifier l'existence de $I(x)$ pour $x\in\ob R\ssm\{-1,1\}$. \pn
b) En approchant l'intégrale à l'aide de sommes de Riemann, calculer $I(x)$ pour $|x|<1$ et pour $|x|>1$.  

\exo [Level=1,Fight=2,Learn=2,Field=\FonctionsDéfiniesParUneIntégrale,Type=\Exercices,Origin=,Indication={2) On pourra introduire la fonction $\ds t\mapsto {t\ln t\F 1-t^2}$.}] pw. 
Soit $g$ la fonction définie sur $\ob R^+$ par $g(t):=t\ln(t)$ si $t>0$ et $g(0):=0$. On considère la suite $(u_n)_{n\in\ob N}$ définie par 
$$
u_n=\int_0^1t^ng(t)\d t\qquad(n\ge0)
$$
et la suite $(v_n)$ définie par 
$$
v_n:=\int_0^1{t^n\ln(t)\F1-t^2}\d t\qquad(n\ge1).
$$
1) Justifier l'existence de $u_n$ puis calculer ensuite sa valeur pour chaque entier $n$. \pn
2) Pour $n\in\ob N^*$, justifier l'existence de $v_n$ et déterminer la limite de la suite $(v_n)$. \pn 
3) Démontrer la relation 
$$
\forall t\in\Q[0,1\W[,\qquad {1\F1-t^2}=1+t^2+t^4+\cdots+t^{2n}+{t^{2n+2}\F 1-t^2}
$$
puis en déduire que 
$$
\lim_{n\to+\infty}\Q[1+{1\F 2^2}+{1\F 3^2}+\cdots+{1\F (n+1)^2}\W]=-4\int_0^1{t\ln t\F 1-t^2}\d t
$$

\exo [Level=1,Fight=1,Learn=1,Field=\Suites,Type=\Exercices,Origin=] px. Soit $u=\{u_n\}_{n\in\ob N}$ une suite vérifiant la relation 
$$
\forall n\ge1, \qquad u_{n+1}=5u_n-6u_{n-1}
$$
a) si $u_0:=1$ et $u_1:=2$, prouver que  
$$
\forall n\ge0, \qquad u_n=2^n.
$$
b) si $u_0:=0$ et $u_1:=1$, prouver que 
$$
\forall n\ge0, \qquad u_n=3^n-2^n.
$$

\exo [Level=1,Fight=1,Learn=1,Field=\Suites,Type=\Exercices,Origin=] py. 
Trouver une formule pour le terme $u_n$ de la suite définie par récurence par $u_0=-\pi$ et 
$$
\forall n\ge0, \qquad u_{n+1}=(2n+1)u_n. 
$$

\exo [Level=1,Fight=1,Learn=1,Field=\NombresComplexes,Type=\Problèmes,Origin=] pz. 
Dans le plan complexe orienté $\sc P$, on considère les points $O$, $A$, $B$ et $C$ d'affixes respectives $0$, $i$, $-i$ 
et $1+i$. Soit $\varphi:\sc P\ssm\{B\}\to\sc P$ l'application qui à un point $M$ d'affixe $z$ associe le point $M'$ d'affixe 
$$
z'={1+iz\F z+i}.
$$
a) Déterminer $\varphi(O)$ et $\varphi^{-1}(C)$, l'antécédent de $C$ par $\varphi$. Plus généralement, exprimer l'affixe $z$ 
du point $M$ en fonction de l'affixe $z'$ de son image $M'$. \pn
b) Trouver les points fixes de l'application $\varphi$ en résolvant l'équation 
$$
z={1+iz\F z+i}.
$$
c) Soient $M\in\sc P$ et $M':=\varphi(M)$. Montrer que $\ds OM'={AM\F BM}$ et que 
$$
\Q(\widehat{\vec u,\vec{OM}}\W)\equiv \Q(\widehat{\vec{MB},\vec{MA}}\W)+{\pi\F2}\quad[2\pi]
$$
d) Déterminer l'image de l'axe des absicces par $\varphi$. \pn
e) Soit $\sc C$ le cercle de diamètre $[AB]$. Déterminer l'image de $\sc C\ssm\{A,B\}$. 


\exo  [Level=1,Fight=1,Learn=1,Field=\NombresComplexes,Type=\Exercices,Origin=] qa. 
Soit $A$ le point d'affixe $1/2$ du plan complexe orienté $\sc P$ et soit $f:\sc P\to\sc P$ 
qui a un point $M$ d'affixe $z$ associe le point $M'$ d'affixe 
$$
z'=z-z^2. 
$$
Montrer que le disque fermé $\sc D$ de centre $A$ et de rayon $1/2$ est stable par $f$, c'est à dire que $f(\sc D)=\sc D$. 


\exo  [Level=1,Fight=1,Learn=1,Field=\NombresComplexes,Type=\Problèmes,Origin=] qb. 
Soient $K$, $L$ et $M$ les points d'affixes  $z_K=1+i$, $Z_L:=1-i$ et $z_M:=-i\sqrt3$. \pn 
a) Calculer l'affixe $z_N$ du symétrique $N$ du point $M$ par rapport au point $L$. \pn
b) Déterminer les affixes $z_A$ et $z_C$ des images $A$ et $C$ 
des points $M$ et $N$ par la rotation de centre $O$ et d'angle $\pi/2$.  \pn
c) Préciser l'affixe $z_B$ et $z_D$ des images $B$ et $D$ des points $M$ et $N$ par la translation de vecteur $\vec u$ d'affixe $2i$. 
\pn
d) Montrer que $K$ est le milieu des segments $[AC]$ et $[BD]$. \pn
e) Calculer le rapport $\ds{z_C-z_K\F z_B-z_K}$ et en déduire la nature du quadrilatère $ABCD$. 


\exo [Level=1,Fight=1,Learn=1,Field=\GéométrieSpatiale,Type=\Problèmes,Origin=] qc. 
Trouver une perpendiculaire commune aux deux droites 
$$
(D)\qquad\Q\{\eqalign{x-y=z+1\cr 3x+4=z-y}\W.\qquad \qquad(\Delta)\qquad\Q\{\eqalign{2x-4y+z=1\cr 3x-2y+z-2=0}\W.
$$
c'est à dire une droite perpendiculaire et sécante avec  $D$ et $\Delta$. En déduire la plus courte distance entre ces deux droites.  

\exo  [Level=1,Fight=1,Learn=1,Field=\GéométriePlane,Type=\Problèmes,Origin=] qd. 
Soient $A$, $B$, $C$ trois points distincts deux à deux d'affixes respectives $a$, $b$ et $c$. \medskip
\noindent
a) Calculer l'affixe $a'$, $b'$ et $c'$ des points $A'$, $B'$, $C'$ tels que les triangles $CBA'$, $ACB'$ et $BAC'$ soient équilatéraux et directs. 
\medskip
\noindent
b) Calculer l'affixe $a''$, $b''$ et $c''$ des centres de gravités $A''$, $B''$ et $C''$ des triangles $CBA'$, $ACB'$ et $BAC'$. 
\medskip
\noindent
c) En déduire que le triangle $A''B''C''$ est équilatéral (théorème de Napoléon). 
\medskip
\noindent
d) Montrer que $\vec{AA'}+\vec{BB'}+\vec{CC'}=\vec 0$. 
\medskip
\noindent
e) Calculer les affixes des centres de gravité des triangles $ABC$, $A'B'C'$ et $A''B''C''$. Remarque ?  
\medskip
\noindent
f) Montrer que $AA'=BB'=CC'$. 

\exo  [Level=1,Fight=1,Learn=1,Field=\NombresComplexes,Type=\Problèmes,Origin=] qe. 
Trouver l'équation des plans bissecteurs du dièdre formé par les plans d'équation 
$$
P:\quad x+5y-2z-1=0\quad\hbox{et}\quad P':\quad 2x-y-z+7=0.
$$ 
{\it Le plan bissecteur est l'ensemble des points situés à égale distance des plans~du~dièdre}


\exo [Level=1,Fight=1,Learn=1,Field=\NombresComplexes,Type=\Exercices,Origin=] qf. 
Résoudre dans $\ob C$ l'équation $\ds\Q({z^2+1\F z^2-1}\W)^8=1$
et représenter dans le plan complexe les points admettant ces solutions pour affixes. 

\exo [Level=1,Fight=1,Learn=1,Field=\Suites,Type=\Exercices,Origin=] qg. 
a) Simplifier quand elle a un sens l'expression $\ds {\sin(2\alpha)\F\sin(\alpha)}$. \medskip\noindent
b)  Calculer $\ds P_n:= \prod_{k=0}^n{c_k}$ où $\ds c_n:= \cos{\pi\F2^{n+2}}$ (relation de Viète).
\medskip\noindent
c) En déduire la valeur de l'expression
$$
{1\F2}\sqrt{{1\F 2}+{1\F2}\sqrt{1\F2}}\sqrt{{1\F2}+{1\F2}\sqrt{{1\F2}+{1\F2}\sqrt{1\F2}}}\sqrt{{1\F2}+{1\F2}\sqrt{{1\F2}+{1\F2}\sqrt{{1\F2}+{1\F2}\sqrt{1\F2}}}}.
$$
d) Simplifier (lorsqu'elle a un sens) l'expression $\ds{1\F \tan(\alpha)}-{2\F\tan(2\alpha)}$.
\medskip\noindent
e) En déduire une expression simple (sans le signe somme) de $S_n =\sum_{k=0}^nt_k$ en fonction de $n$ et de $\ds t_n:={1\F2^n}\tan\Q({\pi\F2^{n+2}}\W)$ (relation de Descartes). 
\medskip\noindent
f) En admettant que $\ds \lim_{u\to 0}{\sin(u)\F u}=1$ et que $\ds\lim_{u\to0}{\tan(u)\F u}=1$, que peut-on en déduire quant aux limites éventuelles de $S_n$ et de $P_n$ ?

\exo [Level=1,Fight=1,Learn=1,Field=\NombresEntiers,Type=\Exercices,Origin=,
Indication={Au besoin, on pourra procéder à un changement d'indices.}] qh. 
Exprimer en fonction de $n\ge2$ et de $x\in\ob R$ les sommes 
$$
\eqalign{
&A:= \sum_{k=0}^nk{n\choose k}x^k(1-x)^{n-k},\qquad\qquad B:=\sum_{k=0}^nk(k-1){n\choose k}x^k(1-x)^{n-k}, \cr
&C:=\sum_{k=0}^nk^2{n\choose k}x^k(1-x)^{n-k},\qquad\qquad D:=\sum_{k=0}^n(k-nx)^2{n\choose k}x^k(1-x)^{n-k},
}
$$

\exo [Level=1,Fight=1,Learn=1,Field=\Applications,Type=\Exercices,Origin=] qi. 
Soient $E$ et $F$ des ensembles non-vides et soit $f:E\to F$ une application. \pn
a) S'il existe $g:F\to E$ telle que $g\circ f=\Id_E$, montrer que $f$ est injective. \pn
b) S'il existe $h:F\to E$ telle que $f\circ h=\Id_F$, montrer que $f$ est surjective. \pn
c) Si $E=F$, montrer que 
$$
f\circ f=\Id_E\longleftrightarrow f\hbox{ est bijective et }f^{-1}=f.
$$


\exo [Level=1,Fight=1,Learn=1,Field=\Applications,Type=\Exercices,Origin=] qj. 
Soient $f:E\to F$ et $g:F\to H$ deux applications. \pn
a) Si $f$ et $g$ sont injectives, montrer que $g\circ f$ est injective. \pn
b) Si $f$ et $g$ sont surjectives, montrer que $g\circ f$ est surjective. \pn
c) Si $f$ et $g$ sont bijectives, montrer que $g\circ f$ est bijective et que 
$$
(g\circ f)^{-1}=f^{-1}\circ g^{-1}. 
$$
d) Si $h=g\circ f$ et si $2$ des trois applications $f:E\to F$, $g:F\to G$, $h:E\to G$ sont bijectives, montrer que la troisième l'est aussi. 

\exo [Level=1,Fight=1,Learn=1,Field=\Applications,Type=\Exercices,Origin=] qk. 
Trouver des bijections $f_0:\Q]2,5\W[\to\Q]1,8\W[$, $f_1:\Q[0,1\W[\to\Q]0,1\W]$, $f_2:\ob R_+^*\to\ob R$, $f_3:\Q]0,1\W[\to\ob R$ et  $f_4:\Q]0,\infty\W[\to\Q]0,1\W[$. 


\exo [Level=1,Fight=3,Learn=3,Field=\NombresEntiers,Type=\Exercices,Origin=,Indication={a) On pourra chercher une bijection entre $\ob N$ et $\ob Z$.\pn b) Au besoin, représenter $f(m,n)$ pour $0\le m\le 4$ et $0\le n\le 4$.\pn d) Etant donnée une surjection $f:\ob N\to\ob R$, on pourra poser 
$$
\forall n\in\ob N, \qquad c_n:=\Q\{\eqalign{& \hbox{ si le $n+1^\ieme$ chiffre après la virgule de $f(n)$ est $6$}\cr &6\hbox{ sinon}}\W.
$$  
et obtenir une contradiction en considérant le nombre réel.}] ql. 
On dit qu'un ensemble $E$ est dénombrable s'il existe une surjection $f:N\to E$. \pn 
a) Montrer que $\ob Z$ est dénombrable. \pn
b) Prouver que l'application $f:\ob N^2\to\ob N$ définie par
$$
\forall (m,n)\in\ob N^2, \qquad f(m,n):={m+n+1\F 2}(m+n)+n
$$
est une bijection. 
\pn
c) En déduire que $\ob N^2$ est dénombrable, puis que $\ob Z^2$ et $\ob Q$ sont dénombrables.  \pn
d) Montrer par l'absurde que $\ob R$ est dénombrable. 
$$
y=c_0,c_1c_2c_3c_4\cdots c_n\cdots
$$

\exo [Level=1,Fight=1,Learn=1,Field=\RécurrencesLinéaires,Type=\Maple,Origin=\Lakedaemon] qm.
 A l'aide de la commande {\it rsolve}, déterminer la suite $u$ définie par $u_0:=0$, $u_1:=1$ et 
$$
\forall n\in\ob N, \qquad u_{n+2}=2u_{n+1}+u_n.
$$

\exo [Level=1,Fight=0,Learn=0,Field=\CourbesParamétréesCartésiennes,Type=\Exercices,Origin=] qn. 
Déterminer les points doubles, les asymptotes et la position par rapport aux asymptotes de la courbe paramétrée
$$
\Q\{
\eqalign{&x(t)={3\F t^2-2t}\cr
&y(t)={t^2-3\F t}}
\W. 
$$

\exo [Level=1,Fight=0,Learn=0,Field=\CourbesParamétréesCartésiennes,Type=\Exercices,Origin=] qo. 
Déterminer les points doubles, les asymptotes et la position par rapport aux asymptotes de la courbe paramétrée
$$
\Q\{
\eqalign{&x(t)={t+1\F t^3}\cr
&y(t)={t-1\F t^2}}
\W. 
$$

\exo [Level=1,Fight=0,Learn=0,Field=\CourbesParamétréesCartésiennes,Type=\Exercices,Origin=] qp. 
Étudier les branches infinies, les points doubles et représenter la courbe paramétrée par 
$$
\Q\{
\eqalign{x={t-1\F t^2-4}\cr y={t^2-3\F t+2}}
\W. 
$$

\exo [Level=1,Fight=0,Learn=0,Field=\CourbesParamétréesCartésiennes,Type=\Exercices,Origin=] qq. 
Étudier les branches infinies, les points doubles et représenter la courbe paramétrée par 
$$
\Q\{
\eqalign{x={u^3\F u^2-9}\cr y={u(u-2)\F u-3}}
\W. 
$$

\exo [Level=1,Fight=0,Learn=0,Field=\Coniques,Type=\Exercices,Origin=] qr. 
Reconnaitre et tracer la courbe d'équation
$$
x^2+xy+y^2-2=0
$$

\exo [Level=1,Fight=0,Learn=0,Field=\Coniques,Type=\Exercices,Origin=] qs. 
Reconnaitre et tracer la courbe d'équation
$$
31x^2+10\sqrt3xy+21y^2-144=0
$$


\exo [Level=1,Fight=0,Learn=0,Field=\Coniques,Type=\Exercices,Origin=] qs. 
Reconnaitre et tracer la courbe d'équation
$$
x^2-xy+y^2+x+y+1=0
$$

\exo [Level=1,Fight=1,Learn=0,Field=\Coniques,Type=\Exercices,Origin=] qt. 
Soient deux paraboles de même axe focal et de même foyer $F$ ayant un point commun $M$. 
Déterminer une mesure de l'angle des deux tangentes en ce point. 

\exo [Level=1,Fight=1,Learn=0,Field=\Coniques,Type=\Exercices,Origin=] qu. 
Un point $M$ d'une hyperbole se projette orthogonalement en $H$ et $H'$ sur ses asymptotes. 
Montrer que le produit $MH.MH'$ est constant (i.e. qu'il ne dépend pas du choix du point $M$). 

\exo [Level=1,Fight=1,Learn=0,Field=\Coniques,Type=\Exercices,Origin=] qv. 
Soit $M$ un point d'une ellipse. On note $I$ sa projection sur l'axe focal, $P$ l'un des deux points de l'ellipse où la tangente est parallèle à $(OM)$ et $J$ la projection de $P$ sur l'axe focal. 
Calculer l'aire du triangle $MOP$ ainsi que les réels $OM^2+OP^2$ et $OI^2+OJ^2$. 

\exo [Level=1,Fight=1,Learn=0,Field=\Coniques,Type=\Exercices,Origin=] qw. 
Montrer que le produit des distances des deux foyers d'une ellipse à l'une de ses tangentes est une constante indépendante de la tangente choisie (utiliser des mesures algébriques, en orientant la normale à la tangente). 

\exo  [Level=1,Fight=1,Learn=0,Field=\Coniques,Type=\Exercices,Origin=] qx. 
a) Déterminer les points du plan par lesquels passent deux tangentes à une ellipse $\sc E$ fixée. \smallskip\noindent
b) même question avec la condition supplémentaire que les tangentes sont perpendiculaires. 

\exo  [Level=1,Fight=1,Learn=0,Field=\Coniques,Type=\Exercices,Origin=] qy. 
Déterminer les points d'où l'on peut mener deux tangentes orthogonales à une hyperbole. 

\exo  [Level=1,Fight=1,Learn=0,Field=\Coniques,Type=\Exercices,Origin=] qz. 
Déterminer le lieu des points d'où l'on peut mener deux tangentes orthogonales à une parabole. 

\exo  [Level=1,Fight=1,Learn=0,Field=\Coniques,Type=\Exercices,Origin=] ra. 
Déterminer en fonction de $(a,b)\in\ob R^2$ la nature de la conique 
$$
x^2+2axy+y^2+2bx-a^2=0.
$$

\exo  [Level=1,Fight=1,Learn=1,Field=\Coniques,Type=\Exercices,Origin=] rb. 
Déterminer le lieu du milieu $I$ du segment $[MM']$ lorsque $M$ et $M'$ sont les intersections d'une ellipse fixe $\sc E$ avec une droite $D$ variable de direction fixe. 

\exo [Level=1,Fight=3,Learn=3,Field=\Coniques,Type=\Problèmes,Origin=] rc. 
On considère dans le plan affine euclidien $\ob R^2$ la courbe $\Gamma$ définie paramétriquement dans le repère orthonormé $(O,\vec i,\vec j)$ par 
$$
\Q\{\eqalign{
x(t)&={3\F t^2+t+1}
\cr
y(t)&={3t\F t^2+t+1}
}\W.
$$ 
1) Étudier les variations de $t\mapsto x(t)$ et $t\mapsto y(t)$. \pn
2) En déduire que la courbe $\Gamma$ est contenue dans un carré que l'on determinera. \pn
3) Dans cette question, on étudie la courbe $\Gamma$ au voisinage du point $O=(0,0)$. \pn
a) Montrer que l'on peut prolonger $\Gamma$ par continuité en lui ajoutant le point $O$. \pn
b) Soit $M(t)$ le point de coordonnées $\b(x(t),y(t)\b)$. Calculer les limites du vecteur~$t\vec{OM}(t)$ lorsque $t$ tends vers $-\infty$ et lorsque $t$ tends vers $+\infty$. \pn
c) Montrer que la droite passant par $O$ de vecteur directeur $t\vec{OM}(t)$ admet une position limite lorsque  $t$ tends vers $-\infty$ et lorsque $t$ tends vers $+\infty$ ; en déduire que la courbe $\Gamma$ admet une tangente verticale en ce point. \pn
Désormais, on note $\tilde \Gamma$ la courbe $\Gamma$ à laquelle on a ajouté le point $O=(0,0)$. \pn
4) Etablir une équation cartésienne de $\tilde \Gamma$ du type $ax^2+bxy+cy^2+dx+ey+f=0$. \pn
5) En déduire que $\tilde \Gamma$ est une ellipse. \pn
6) Montrer que $\tilde \Gamma$ admet une tangente verticale en un unique point $A$ autre que le point $O$. 
Montrer que le milieu du segment $[OA]$ est le centre $G$ de $\tilde \Gamma$. \pn
7) Déterminer les sommets de l'ellipse $\tilde\Gamma$. \pn
8) Représenter l'ellipse $\tilde \Gamma$. 

\exo  [Level=1,Fight=3,Learn=3,Field=\CourbesParamétréesCartésiennes,Type=\Problèmes,Origin=] rd. 
Le plan euclidien $\ob R^2$ étant rapporté au repère orthonormé $(O,\vec i,\vec j)$, on considère la courbe $\sc C$ de représentation paramétrique 
$$
\Q\{\eqalign{
x&=a(1+\cos\theta)
\cr
y&=a\sin\theta
}\W.\qquad(-\pi<\theta\le\pi)
$$ 
où $a$ est un nombre réel strictement positif. \pn
1) Soit $\sc L$ l'ensemble des projections orthogonales de $O$ sur les tangentes à la courbe~$\sc C$. \pn
a) Préciser la nature géométrique de la courbe $\sc C$. \pn
b) Déterminer une représentation paramétrique de l'ensemble $\sc L$. \pn
c) Donner une équation polaire de l'ensemble $\sc L$ (de la forme $\rho(\theta)=...$). \pn
2) Soit $M$ un point de $\sc L$ d'angle polaire $\theta$, autrement dit tel que $\widehat{(\vec i,\vec{OM})}=\theta$,  et soit $\vec T$ un vecteur tangent en $M$ à $\sc L$ orienté dans le sens des $\theta$ croissants. \pn
a) Exprimer les angles $\widehat{(\vec{OM},\vec T)}$ et $\widehat{(\vec i,\vec T)}$ à l'aide de $\theta$. \pn
b) En déduire les points de $\sc L$ où la tangente est parallèle aux axes $Ox$ ou $Oy$. \pn
3) Soit $D_\alpha$ la droite passant par $O$ d'angle polaire $\alpha\in\Q[-{\pi\F2},{\pi\F2}\W]$. \pn
a) Montrer que $D_\alpha$ recoupe $\sc L$ en deux points $M_\alpha$ et $M_\alpha'$. \pn
b) Calculer la longueur du segment $M_\alpha M_\alpha'$. \pn
c) Déterminer le lieu $H$ du milieu $I_\alpha$ du segment $M_\alpha M_\alpha'$. \pn
d) En déduire une construction géométrique des points $M_\alpha$ et $M_\alpha'$. \pn
4) Représenter la courbe $\sc L$. 

\exo  [Level=1,Fight=3,Learn=3,Field=\GéométrieSpatiale,Type=\Problèmes,Origin=] re. 
On considère un carré $ABCD$ dont la longueur des cotés est $a>0$. 
On note $(\sc P)$ le plan du carré. Sur une demi-droite $\Delta_A$ perpendiculaire à $(\sc P)$ en~$A$,  
on considère un point $M$ et l'on note $d$ la distance de $A$ à $M$. \pn
1) La perpendiculaire en $M$ au plan du triangle $MBC$ rencontre $(\sc P)$ en un point~$R$ et 
la perpendiculaire en $M$ au plan du triangle $MCD$ rencontre $(\sc P)$ en un point~$S$. \pn
a) Faire une figure. \pn
b) Montrer que $R$ appartient à la droite $(AB)$. \pn
De même, on admet que $S$ appartient à la droite $(AD)$. \pn
2) Montrer que les longueurs $AR$ et $AS$ sont égales. \smallskip\noindent
Quelle est la nature du triangle $ARS$ ? \pn
3) Quel est le lieu géométrique décrit par le milieu $K$ du segment $[RS]$ lorsque le point $M$ 
décrit la demi-droite $\Delta_A$ ? \pn
4) Démontrer que la droite $(MC)$ est perpendiculaire au plan $(MRS)$. \pn
5) La hauteur issue de $A$ du triangle $MAK$ rencontre le coté $[MK]$ en un point~$H$. 
Montrer que $(AH)$ est la hauteur issue de $A$ du tétraèdre $ARMS$ et que $H$ est l'orthocentre du triangle $MRS$. 

\exo  [Level=1,Fight=1,Learn=1,Field=\Polynômes,Type=\Exercices,Origin=] rf. 
Décomposer les polynômes $X^8+1$ et $X^8+X^4+1$ en produits 
de facteurs irréductibles de $\ob R[X]$. 

\exo  [Level=1,Fight=1,Learn=1,Field=\Polynômes,Type=\Exercices,Origin=] rg. 
Déterminer $a$ et $b$ pour que le polynôme $aX^{n+1}+bX^n+1$ soit divisible par $(X-1)^2$. 

\exo  [Level=1,Fight=1,Learn=1,Field=\Polynômes,Type=\Exercices,Origin=] rh. 
Quel est le reste de la division euclidienne d'un polynôme $P$ par $X-a$ ? par $(X-a)^2$ ? par $(X-a)^n$ ? 

\exo  [Level=1,Fight=1,Learn=1,Field=\Polynômes,Type=\Exercices,Origin=] ri. 
Quel est le reste de la division euclidienne de $\b(\cos(\alpha)+\sin(\alpha)X\b)^n$ 
par $X^2+1$ ? 

\exo  [Level=1,Fight=1,Learn=1,Field=\Polynômes,Type=\Exercices,Origin=] rj. 
Trouver un polynôme à coefficients entiers ayant $1+\sqrt3$ comme racine. Calculer la valeur numérique du polynôme $P=2X^4-4X^3-7X-14$. 

\exo  [Level=1,Fight=1,Learn=1,Field=\Polynômes,Type=\Exercices,Origin=] rk. 
Déterminer $\underline{mhbox{un}}$ polynôme $P$ dont le reste par la division euclidienne par $X-1$ 
vaut $2$ et dont le reste par la division euclidienne par $X-2$ vaut $1$. Quel est le reste par la division euclidienne de $P$ par $(X-1)(X-2)$ ?  
Trouver $\underline{\hbox{tous}}$ les polynômes $P$ possédant les propriétés précédentes. . 

\exo  [Level=1,Fight=0,Learn=0,Field=\Polynômes,Type=\Exercices,Origin=] rl. 
Dans quel cas le polynôme $X^n+a^n$ est-il divisible par $X^2+X+1$ ? Même question avec $P=X^{2n}+X^n+1$. 

\exo  [Level=1,Fight=1,Learn=1,Field=\Polynômes,Type=\Exercices,Origin=] rm. 
Décomposer dans $\ob C[X]$ le polynôme $2X^3-(5+6i)X^2+9iX+1-3i$ sachant qu'il admet une racine réelle. 

\exo  [Level=1,Fight=0,Learn=0,Field=\Polynômes,Type=\Exercices,Origin=] rn. 
Factoriser le polynôme $X^3+(2-11i)X^2-(39+12i)X-18+45i$ sachant qu'il admet une racine double. 

\exo  [Level=1,Fight=0,Learn=0,Field=\Polynômes,Type=\Exercices,Origin=] ro. 
Déterminer $a$, $b$ et $c$ réels pour que le polynôme $P=X^6-5X^4-aX^2+bX+c$ admette une racine d'ordre $4$ au moins. 

\exo  [Level=1,Fight=1,Learn=1,Field=\Polynômes,Type=\Exercices,Origin=] rp. 
a) Résoudre l'équation $1+z+z^2+\cdots+z^{n-1}=0$  dans le corps $\ob C$ des nombres complexes 
puis factoriser le polynôme $P=1+X+X^2+\cdots+X^{n-1}$. \pn
b) En déduire la relation
$$
{n\F2^{n-1}}=\sin\Q({\pi\F n}\W)\sin\Q({2\pi\F n}\W)\cdots\sin\Q({(n-1)\pi\F n}\W).
$$

\exo  [Level=1,Fight=1,Learn=1,Field=\Polynômes,Type=\Exercices,Origin=] rq. 
 Montrer que le polynôme $\ds1+X+{X^2\F 2!}+\cdots+{X^n\F n!}$ n'a que des racines simples. 

\exo  [Level=1,Fight=1,Learn=1,Field=\Polynômes,Type=\Exercices,Origin=] rr. 
Soit $P$ un polynôme scindé de degré $p\ge2$ n'ayant que des racines simples. Montrer que $P'$ est scindé et ne possède que des racines simples.  

\exo  [Level=1,Fight=1,Learn=1,Field=\Polynômes,Type=\Exercices,Origin=] rs. 
Calculer $\ds n+(n-1){n \choose1}+(n-2){n\choose 2}+(n-3){n\choose 3}+\cdots+2{n\choose n-2}+{n\choose n-1}$. 

\exo  [Level=1,Fight=2,Learn=2,Field=\Polynômes,Type=\Exercices,Origin=] rt. 
Soient $a\neq1$ et $P:=(1+X)(1+aX)(1+a^2X)\cdots(1+a^{n-1}X)$. Etablir une relation entre $P(aX)$ et $P(X)$. 
En déduire la valeur des coefficients de $P$. 

\exo  [Level=1,Fight=2,Learn=1,Field=\Polynômes,Type=\Exercices,Origin=] ru. 
Calculer les coefficients de $(1+X+X^2+\cdots+X^n)^2$ et de $(1-X+X^2-\cdots+(-1)^nX^n)^2$. 

\exo  [Level=1,Fight=2,Learn=2,Field=\Polynômes,Type=\Colles,Origin=] rv. 
Déterminer les polynômes $P$ de $\ob C[X]$ vérifiant $(X+3)P(X)=XP(X+1)$. 

\exo  [Level=1,Fight=2,Learn=1,Field=\Polynômes,Type=\Exercices,Origin=] rw. 
Soient $P$, $Q$ et $R$ trois polynômes de $\ob R[X]$ vérifiant $P^2-XQ^2=XR^2$. Montrer que l'on a $P=Q=R=0$. 

\exo  [Level=1,Fight=1,Learn=1,Field=\Polynômes,Type=\Exercices,Origin=] rx. 
Prouver que $\mbox{PGCD}(X^n-1,X^m-1)=X^{\mbox{\sevenrm pgcd}}(n,m)-1$. 

\exo  [Level=1,Fight=2,Learn=1,Field=\Polynômes,Type=\Exercices,Origin=,Indication={On commencera par établir cette propriété pour $P=X$.}] ry. 
Montrer que si $a$ et $b$ sont deux entiers premiers entre eux alors
$$
\forall P\in\ob KL[X], \qquad (P^a-1)(P^b-1)\hbox{ divise }(P-1)(P^{ab}-1). 
$$


\exo  [Level=1,Fight=1,Learn=1,Field=\Polynômes,Type=\Exercices,Origin=] rz. 
Pour $m\in\ob N^*$ et $\theta\in\ob R$, on pose $F_{m,\theta}:=X^{2m}-2X^m\cos(m\theta)+1$. \pn
Vérifier que $F_{m,\theta}$ est divisible par $F_{1,\theta}$ et calculer le quotient. 

\exo  [Level=1,Fight=1,Learn=1,Field=\Polynômes,Type=\Exercices,Origin=] sa. 
Factoriser sur $\ob C[X]$ et sur $\ob R[X]$ le polynôme $X^{2n}-2X^m\cos(\alpha)+1$ pour $\alpha\in\ob R$ puis écrire l'égalité obtenue en substituant $1$ à $X$. 

\exo  [Level=1,Fight=1,Learn=1,Field=\Polynômes,Type=\Exercices,Origin=] sb. 
Vérifier que $X^n\sin(\theta)-X\sin(n\theta)+\sin((n-1)\theta)$ est divisible dans $\ob C[X]$ par $X^2-2X\cos(\theta)+1$ et calculer le quotient. 

\exo  [Level=1,Fight=1,Learn=1,Field=\Polynômes,Type=\Exercices,Origin=] sc. 
Vérifier que $X^{n+1}\cos\b((n-1)\theta\b)-X^n\cos(n\theta)-X\cos(\theta)+1$ est divisible dans $\ob C[X]$ par $X^2-2X\cos(\theta)+1$ et calculer le quotient. 

\exo  [Level=1,Fight=1,Learn=1,Field=\Polynômes,Type=\Exercices,Origin=] sd. 
Démontrer que $(X-1)^{n+2}+X^{2n+1}$ est divisible par $X^2-X+1$. On appelle $Q_n$ le quotient. Trouver une relation de récurence entre les $Q_n$. 

\exo  [Level=1,Fight=1,Learn=1,Field=\Polynômes,Type=\Others,Origin=] se. 
Dans $\ob Q[X]$, on donne $A(X):=X^5+5X^4+9X^3+7X^2+5X+3$ et $B(X):=X^4+2X^3+X+1$. Chercher $\mbox{PGCD}(A,B)$ et l'écrire sous la forme $AU+BV$. 

\exo  [Level=1,Fight=2,Learn=1,Field=\Polynômes,Type=\Others,Origin=] sf. 
Bezouter les polynômes $(X-a)^{2n}$ et $(X-b)^{2n}$ pour $a\neq b$. 

\exo  [Level=1,Fight=1,Learn=1,Field=\Polynômes,Type=\Exercices,Origin=] sg. 
Soient $A$ et $B$ deux polynômes de $\ob R[X]$ avec $B\neq0$. \pn
Montrer que $B$ divise $A$ dans $\ob R[X]\quad\Longleftrightarrow\quad B$ divise $A$ dans $\ob C[X]$. 

\exo  [Level=1,Fight=1,Learn=1,Field=\Polynômes,Type=\Exercices,Origin=] sh. 
Démontrer que si $P\in\ob Q[X]$ admet $a+b\sqrt7$ comme racine, avec $(a,b)\in_ob Q^2$, alors $P$ admet $a-b\sqrt7$ comme racine avec même ordre de multiplicité. 

\exo  [Level=1,Fight=1,Learn=1,Field=\Polynômes,Type=\Exercices,Origin=] si. 
Trouver tous les polynômes $P$ de $\ob C[X]$ vérifiant 
$$
\forall (x,y)\in\ob C^2, \qquad P(x+y)=P(x)P(y). 
$$

\exo  [Level=1,Fight=1,Learn=1,Field=\Polynômes,Type=\Exercices,Origin=] sj. 
Trouver tous les polynômes $P$ de $\ob C[X]$ vérifiant 
$$
\forall (x,y)\in\ob C^2, \qquad P(xy)=P(x)+P(y). 
$$

\exo  [Level=1,Fight=1,Learn=1,Field=\Polynômes,Type=\Exercices,Origin=] sk. 
Trouver tous les polynômes $P$ de $\ob C[X]$ vérifiant 
$$
\forall (x,y)\in\ob C^2, \qquad P(x+y)=P(x)+P(y). 
$$

\exo [Level=1,Fight=1,Learn=1,Field=\Polynômes,Type=\Exercices,Origin=] sl. 
Résoudre dans $\ob C[X]$ l'équation $P(X^2)=P(X)P(X+1)$. 

\exo [Level=1,Fight=1,Learn=1,Field=\Polynômes,Type=\Exercices,Origin=]  sm. 
Résoudre dans $\ob C[X]$ l'équation $P(X^2)=P(X-1)P(X+1)$. 


\exo [Level=1,Fight=1,Learn=1,Field=\Polynômes,Type=\Exercices,Origin=]  sn. 
Trouver toutes les racines du polynômes $2X^3-(5+6i)X^2+9iX+1-3i$, sachant qu'il admet au moins une racine réelle. 

\exo [Level=1,Fight=1,Learn=1,Field=\Polynômes,Type=\Maple,Origin=] so. 
Le but de cet exercice est de trouver tous les polynômes $P\in\ob R[X]$ tels que 
$$
\lim_{n\to\infty} n^2\Q((n^6+3n^4)^{1/6}-P(n)^{1/3}\W)=0 
$$
a) Determiner à la main le degré $p$ des polynômes solutions $P$. \pn
b) A l'aide de la fonction  $P(n):=a_0+a_1n+\cdots+a_pn^p$ et de la commande {\it series}, obtenir une expression du type
$$
 n^2\Q((n^6+3n^4)^{1/6}-P(n)^{1/3}\W)=\alpha n^3+ \beta n^2+\gamma n+\delta+... 
$$
et exprimer $\alpha$, $\beta$, $\gamma$ et $\delta$ en fonction de $a_0, \cdots a_p$. \pn
c) Conclure

\exo  [Level=1,Fight=0,Learn=0,Field=\Coniques,Type=\Exercices,Origin=] sp. 
Réduire les coniques d'équation 
$$
\eqalignno{
&13x^2-32xy+37y^2-2x+14y-5=0
& (\sc C_1)
\cr
&x^2+\sqrt xy+3x+5y-4=0& (\sc C_2)
\cr
&(2x+3y)^2+4x+5y-5=0&(\sc C_3)
}
$$
Donner leur équation réduite ainsi que le repère orthonormé associé. 

\exo  [Level=1,Fight=0,Learn=0,Field=\Coniques,Type=\Exercices,Origin=] sq. 
a) Determiner une expression (dependant de $\theta$) des sommets et des foyers de la conique 
$$
x^2\sin^2(\theta)-xy\sin(2\theta)+y^2(1+\cos^2(\theta))=9\sin^2(\theta)
$$
b) graphe du lieu des sommets et des foyers. 


\exo [Level=1,Fight=2,Learn=1,Field=\Polynômes,Type=\Maple,Origin=] sr. 
a) On rappelle que le pgcd de deux polynomes est le dernier reste non nul dans l'algorithme d'Euclide. 
A l'aide des commandes {\it while} (boucle), {\it degree} (degré) et {\it rem} (reste de division euclidienne), 
écrire une procédure qui prend deux polynômes $P$ et $Q$ en entrée 
et qui renvoie le pgcd des polynômes $P$ et $Q$. \pn
b) Application calculer le pgcd des polynômes 
$$
P:=(1+X+2X^2+3X^3+4X^4+5X^5+6X^6+7X^7)(X+1)^6\quad\hbox{ et }\quad P'.
$$ 


\exo [Level=1,Fight=3,Learn=3,Field=\Polynômes,Type=\Cours,Origin=] st. 
Soit $P=a_0+a_1X+\cdots+a_nX^n$ un polynôme de degré $n$ dont tous les coefficients sont entiers (dans $\ob Z$). \pn
a) A quelle condition simple portant sur $a_0, \cdots a_n$ les nombres $0$, $-1$ et $1$ sont-ils racines de $P$ ? \pn
b) Soit $\alpha$ une racine rationnelle de $P$ et soient $p$ et $q$ des entiers premiers entre eux tels que 
$$
\alpha={p\F q}.
$$ 
En calculant $q^nP(\alpha)$, prouver que $p$ divise $a_n$ et que $q$ divise $a_0$. \pn
c) En calculant $q^nP(\alpha)-q^n P(1)$, prouver que $p-q$ divise $a_0+a_1+\cdots+a_n$. \pn
d) En calculant $q^nP(\alpha)-q^n P(-1)$, prouver que $p+q$ divise $a_0-a_1+\cdots+(-1)^na_n$. \pn
e) Comment faire pour trouver toutes les racines rationnelles d'un polynôme à coefficients entiers? \pn
f) Trouver les racines rationnelles du polynôme $36X^3-7X+1$. \pn
g) Factoriser le polynôme $6X^3+5X^2+4X+1$. \pn
h) On suppose maintenant que $P$ est à coefficients rationnels. 
Comment faire pour trouver toutes les racines rationnelles de $P$ ? \pn
i) Factoriser le polynôme 
$$
2X^4-{56\F 15}X^3+{5\F2}X^2-{7\F 10}X+{1\F 15}
$$
j) Prouver que les racines doubles du polynôme $P:=X^6-5X^5+X^3+X^2-X+1$ sont forcément rationnelles 
puis que toutes les racines du polynôme $P$ sont simples. 

\exo [Level=2,Fight=1,Learn=1,Field=\EquationsDifférentiellesLinéairesDuSecondOrdre,Type=\Exercices,Origin=] su. 
Déterminer la solution $f$ de l'équation différentielle $y''+2y'-4y=\sin(x)\e^{2x}$\pn
a) qui vérifie les conditions initiales $f(0)=f'(0)=0$. \pn
b) qui vérifie $f(0)=0$ et $f(\pi)=0$. 

\exo [Level=1,Fight=1,Learn=1,Field=\RécurrencesLinéaires,Type=\Maple,Origin=] sv. 
On considère la suite récurente définie par $u_0:=\sqrt5-1$, $u_1:=\sqrt5-3$ et par la relation 
$$
\forall n\ge2, \qquad u_n=u_{n-1}+u_{n-2}.
$$
a) Ecrire une procédure récursive permettant de calculer la valeur de $u_n$ et utiliser cette procédure pour calculer $u_{66}$. \pn
b) Grace à la commande ``rsolve'', exprimer $u_n$ en fonction de $n$ puis calculer $u_{666}$ à l'aide de la formule obtenue. \pn
c) Retrouver le résultat précédent en composant  l'application $f:(x,y)\mapsto(x+y,x)$ plusieurs fois. 

\exo [Level=1,Fight=1,Learn=1,Field=\Polynômes,Type=\Exercices,Origin=] sw. 
Calculer le reste de la division euclidienne de $X^n+2X^m+1$ par le polynôme 
$$
(X-1)(X-2)(X-3)(X-4), 
$$
où $n$ et $m$ sont des entiers quelconques. 
Vérifier le résultat pour $n=43$ et $m=101$

\exo [Level=1,Fight=1,Learn=1,Field=\Intégration,Type=\Maple,Origin=] sx. 
A l'aide de la fonction {\it int} calculer les intégrales
$$
\int_0^\pi\sin(x)^8\d x, \qquad \int_{-1}^1\sqrt{(1+t)(1-t)}\d t\qquad\hbox{et}\qquad \int_0^1{\d u\F(u^2+1)^4}
$$
On pourra présenter le calcul à l'aide de la commande {\it Int} qui écrit le symbole $\int$. 

\exo [Level=1,Fight=1,Learn=1,Field=\Polynômes,Type=\Maple,Origin=] sy. 
A l'aide de la commande {\it sum}, calculer les sommes
$$
\sum_{k=0}^nk, \qquad \sum_{k=0}^nk^2, \qquad \sum_{k=0}^nk^3, \qquad \sum_{k=0}^nk^4
$$
On pourra les factoriser à l'aide de la commande {\it combine}. De même, pour $x\neq1$, calculer 
$$
\sum_{k=0}^nx^k, \qquad \sum_{k=0}^nkx^k\qquad \qquad\hbox{et}\qquad \sum_{k=0}^nk^2x^k
$$
On pourra présdenter les calculs à l'aide de la commande {\it Sum} qui écrit le symbole $\sum$.  

\exo [Level=1,Fight=1,Learn=1,Field=\Polynômes,Type=\Maple,Origin=]  sz. 
A l'aide de la commande {\it evalf}, obtenir une valeur approchée à $10^{-13}$ près de l'intégrale 
$$
\int_0^\pi{\d x\F\sqrt{3-\cos(x)^2}}.
$$

\exo [Level=1,Fight=0,Learn=0,Field=\Intégration,Type=\Maple,Origin=] ta. 
Faire {\it with(student):} puis utiliser la commande {\it changevar} pour effectuer le changement de variable $x=\sin(u)$ dans l'intégrale (que l'on calculera)  
$$
\int_0^1{x^2\F\sqrt{1-x^2}}\d x. 
$$
On utilisera eventuellement la commande {\it simplify(...,symbolic)} pour simplifier ainsi que  {\it value}. 

\exo  [Level=1,Fight=1,Learn=1,Field=\Intégration,Type=\Maple,Origin=] tb. 
Faire {\it with(student):} puis utiliser la commande {\it intparts} pour effectuer une intégration par partie de l'intégrale 
$$
\int_0^1x\arctan\Q(\sqrt{1-x\F1+x}\W)\d x. 
$$
On pourra ensuite faire {\it assume(x, RealRange(0,1));} pour s'assurer que $0\le x\le 1$ puis simplifier à l'aide des commandes  {\it simplify} et {\it combine(..., radical)}. Enfin, on calculera la valeur de l'intégrale en utilisant au besoin la commande {\it value}. 

\exo [Level=1,Fight=0,Learn=0,Field=\Primitives,Type=\Exercices,Origin=] tc. 
Décomposer en éléments simples sur $\ob R$ puis donner une primitive (préciser sur quel ensemble) 
de la fraction rationnelle suivantes : 
$$
F(x)={X+2\F (X+1)(X^2+1)}
$$

\exo  [Level=1,Fight=2,Learn=2,Field=\Primitives,Type=\TravauxDirigés,Origin=] td. 
Décomposer en éléments simples, sur $\ob C$,  sur $\ob R$ puis donner une primitive réelle 
(préciser sur quel ensemble) 
des fractions rationnelles suivantes : 
$$
\eqalign{
&A(x):={1\F1-x^2}, \qquad B(x):={x+2\F(x+1)(x^2+1)},\qquad C(x):={1\F(x-1)^2(x-2)}\cr
&D(x)={1\F(x-1)(x-2)\cdots(x-n)},\qquad E(x):={1\F(x^2-1)^2},\cr 
& F(x):={x\F x^4+x^2+1},\qquad G(x):={x^5\F (x^4-1)^2}, \qquad H(x):={x^7-x^6-x+1\F(x-1)^5}, \cr
&I(x):={x^2\F x^4-2x^2\cos(\alpha)+1}\quad(\alpha\in\ob R).
}
$$

\exo [Level=1,Fight=1,Learn=1,Field=\Intégration,Type=\Exercices,Origin=] te. 
En procédant au changement de variable $u=\sin(x)$, calculer 
$$
\ds\int_0^{\pi\F 3}{\cos(x)^3+3\cos(x)\F \sin(x)^4+4\sin(x)^2+4}\d x.$$


\exo  [Level=1,Fight=1,Learn=1,Field=\Intégration,Type=\Exercices,Origin=] tf. 
a) Pour $x\not\equiv\pi\ [2\pi]$, on pose $\ds u=\tan\Q({x\F 2}\W)$. Etablir que 
$$
\sin(x)={2u\F1+u^2}, \qquad \cos(x)={1-u^2\F1+u^2} \quad\hbox{et}\quad  \tan(x)={2u\F 1-u^2}
$$
b) En procédant au changement de variable $\ds u=\tan\Q({x\F 2}\W)$, calculer l'intégrale 
$$
\ds\int_0^{\pi\F 2}{1+\sin(x)+\cos^2(x)\F (1+\sin x)(1+\cos x)}\d x.
$$


\exo  [Level=1,Fight=1,Learn=1,Field=\Intégration,Type=\Exercices,Origin=] tg. 
En procédant au changement de variable $u=\sqrt x$, calculer 
$$
\ds\int_0^1{2x^2\sqrt x+x\F(\sqrt x+2)(6+5\sqrt x+x)^2}\d x.
$$

\exo  [Level=1,Fight=1,Learn=1,Field=\Intégration,Type=\Exercices,Origin=] th. 
a) Pour $x\in\ob R$, on pose $\ds u=\th\Q({x\F 2}\W)$. Etablir que 
$$
\sh(x)={2u\F1-u^2}, \qquad \ch(x)={1+u^2\F1-u^2} \quad\hbox{et}\quad  \th(x)={2u\F 1+u^2}
$$
b) En procédant au changement de variable $\ds u=\th\Q({x\F 2}\W)$, calculer l'intégrale 
$$
\ds\int_0^\e{1+\sh(x)+\ch^2(x)\F (1+\sh x)(1+\ch x)}\d x.
$$

\exo [Level=1,Fight=1,Learn=1,Field=\Primitives,Type=\Exercices,Origin=] ti. 
Calculer les primitives  suivantes
$$
a:\quad\int{x\d x\F\sqrt{x+1}},\qquad b:\quad\int\e^x\cos(2x)\d x,\qquad c:\quad \int{\arctan x\F x^2}\d x.
$$

\exo  [Level=1,Fight=1,Learn=1,Field=\Intégration,Type=\Exercices,Origin=] tj. 
Soit $a>0$ et soit $f$ une fonction continue par morceaux sur $[-a,a]$. \pn
a) Si $f$ est paire, prouver  que  
$$
\int_{-a}^af(x)\d x=\int_0^af(x)\d x\qquad\hbox{puis que}\qquad \int_{-a}^af(x)\d x=2\int_0^af(x)\d x.
$$
b) Si $f$ est impaire, prouver que  
$$
\int_{-a}^af(x)\d x=-\int_0^af(x)\d x\qquad\hbox{puis que}\qquad \int_{-a}^af(x)\d x=0.
$$
c) on suppose desormais que la fonction continue par morceaux $f:\ob R\to\ob R$ admet la période  $T>0$. 
Pour $a\in\ob R$, prouver que 
$$
\int_0^af(x)\d x=\int_T^{a+T}f(x)\d x\qquad\hbox{puis que}\qquad 
\int_a^{a+T}f(x)\d x=\int_0^Tf(x)\d x.
$$
{\it En bref, l'intégrale d'une fonction $T$-périodique sur un intervalle de longueur $T$ ne dépend pas de l'intervalle choisi.}
\null 


\exo [Level=1,Fight=2,Learn=2,Field=\FonctionsDéfiniesParUneIntégrale,Type=\Exercices,Origin=] tk. 
Soit $f:[0,1]\to\ob C$ une fonction continue. Calculer la limite de la suite
$$
u_n:=\int_0^1{f(x)\F 1+nx}\d x.
$$ 

\exo [Level=1,Fight=2,Learn=2,Field=\Intégration,Type=\Exercices,Origin=,Indication={On pourra poser $\alpha_n:={\pi\F 2}-{1\F \root 3\of n}$ 
et majorer séparément les intégrales $\int_0^{\alpha_n}$ et $\int_{\alpha_n}^{\pi\F 2}$.}] tl. 
Soit $f:[0,1]\to\ob C$ une fonction continue. Calculer la limite de la suite
$$
u_n:=\int_0^{\pi\F 2}f(x)\sin(x)^n\d x.
$$ 

\exo [Level=1,Fight=2,Learn=2,Field=\FonctionsDéfiniesParUneIntégrale,Type=\Exercices,Origin=,Indication={1) Commencer par fixer $x\in\ob R$, puis étudier l'existence de $I(x)$.}] tm. 
Pour $x\in\ob R$, on pose $\ds I(x)=\int_0^\pi\ln(1-2x\cos t+x^2)\d t$. \pn
1) Déterminer l'ensemble de définition $E$ de la fonction $I$. \pn
2) Pour $x\in E$, montrer que $I(-x)=I(x)$. \pn
3) Pour $x\in E\ssm\{0\}$, calculer $I(1/x)$ en fonction de $I(x)$. \pn
4) Pour $x\in E$, calculer $I(x^2)$ en fonction de $I(x)$. \pn
5) On admet que la fonction $I$ est continue. En déduire $I(x)$ pour~$x\in E$. 

\exo [Level=1,Fight=0,Learn=0,Field=\Matrices,Type=\Exercices,Origin=] tn. 
Soient $\ds M:=\pmatrix{1&0&0\cr0&-2&-9\cr0&1&4}$ et 
$\ds I_n:=\pmatrix{1&0&0\cr0&1&0\cr0&0&1}$. \pn
a) Calculer $(M-I_3)^2$. \pn
b) En admettant que l'on peut appliquer le binôme de Newton aux matrices $A:=M-I_3$ et $B:=I_3$ qui commuttent, en déduire $M^n$ pour $n\ge2$. 

\exo [Level=1,Fight=0,Learn=0,Field=\Matrices,Type=\Maple,Origin=] to. 
A l'aide de la commande {\it det}, calculer le déterminant $D:=\Q|\matrix{1& a & a^2 & a^4\cr 1& b& b^2&b^4\cr1&c&c^2&c^4\cr1&d&d^2&d^4}\W|$.

\exo [Level=1,Fight=0,Learn=0,Field=\Matrices,Type=\Maple,Origin=] tp. 
A l'aide de la commande {\it inverse}, calculer l'inverse de la matrice carrée de 9 lignes, 
dont les coefficients valent $1$ sauf ceux sur la diagonale principale qui valent $0$. 

\exo [Level=1,Fight=0,Learn=0,Field=\Groupes,Type=\Exercices,Origin=] tq. 
Soit $G$ un groupe noté miltiplicativement et soient deux éléments $a$ et $b$ de $G$ 
vérifiant $a^2b=ba$ et $b^2a=ab$. Montrer que $a=b=1$. 

\exo [Level=1,Fight=1,Learn=1,Field=\Anneaux,Type=\Exercices,Origin=] tr. 
Soit $E$ un ensemble. On rappelle que la différence symétrique $\Delta$ est définie sur 
l'ensemble $\sc P(E)$ des parties de $E$ par 
$$
\forall (A,B)\in\sc P(E)^2, \qquad A\Delta B:=(A\cup B)\ssm(A\cap B).
$$
a) Montrer que $(\sc P(E),\Delta,\cap)$ est un anneau commutatif {\it (avec les 10 axiomes..)}\pn
b) l'anneau précédent est il intégre ? Quels éléments sont inversibles ? \pn
c) Soit $C$ un sous ensemble de $E$. Montrer que $\sc A:=\{\emptyset,E, C, E\ssm C\}$ est un sous anneau de $(\sc P(E),\cup,\Delta)$. 

\exo [Level=1,Fight=2,Learn=2,Field=\Anneaux,Type=\Exercices,Origin=] ts. 
On pose $\ob Q[i]:=\{a+ib:(a,b)\in\ob Q^2\}$. \pn
a) Montrer que $(\ob Q[i],+,\times)$ est un corps commutatif. \pn
b) Montrer que l'application $\varphi:z\to\overline z$ est un automorphisme de $\ob Q[i]$. \pn
c) Pour $(a,b)\in \ob Q[i]^2$, on pose $N(a+ib):=a^2+b^2$. Prouver que l'on a 
$$
\forall (x,y)\in\ob Q[i]^2, \qquad N(xy)=N(x)N(y). 
$$ 
d) Determiner les éléments $x$ de $\ob Z[i]$ possédant un inverse dans $\ob Z[i]$ ? 

\exo [Level=1,Fight=0,Learn=0,Field=\Anneaux,Type=\Exercices,Origin=] tt. 
Montrer que le centre $Z(A):=\{x\in A:\forall a\in A, xa=ax\}$ d'un anneau $(A,+,\times)$ est un anneau pour les mêmes lois. 

\exo [Level=1,Fight=2,Learn=2,Field=\Anneaux,Type=\Exercices,Origin=] tu. 
Soit $(A,+,\times)$ un anneau tel que $\forall x\in A, \ x^2=x$. \pn
a) Montrer que $\forall x\in a, x+x=0$. \pn
b) En déduire que l'anneau $(A,+,\times)$ est commutatif. 

\exo [Level=1,Fight=2,Learn=2,Field=\Anneaux,Type=\Exercices,Origin=] tv. 
Soit $(A,+,\times)$ un anneau. On dit qu'un élément $x$ de $A$ est nilpotent si et seulement s'il existe $n\in\ob N$ tel que $x^n=0$. \pn
a) Soient $x$ et $y$ dans $A$ tels que $xy$ soit nilpotent. Prouver que $yx$ est nilpotent.  
b) Soient $x$ et $y$ deux éléments nilpotents de $A$, qui commuttent.  
Prouver que $xy$ et $x+y$ sont nilpotents\pn
c) Soit $x$ un élément nilpotent de $A$. Prouver que $1-x$ est inversible dans $A$ et calculer son inverse. 


\exo [Level=1,Fight=1,Learn=1,Field=\EspacesVectoriels,Type=\Exercices,Origin=] tw. Soient $E$ un $\ob C$-espace vectoriel et soit $f\in\sc L(E)$ tel que $f^2=-\Id_E$. On pose
$$
F:=\Ker(f-i\Id_E)\qquad \hbox{et}\qquad G:=\Ker(f+i\Id_E). 
$$
a) Prouvez que $F$ et $G$ sont supplémentaires dans $E$ (i.e. que $E=F\oplus G$). \pn
b) Exprimer $f$ en fonction de la projection $p\in\sc L(E)$ sur $F$ parallèlement à $G$ et de la projection $p'\in\sc L(E)$ sur $G$ parallèlement à $F$. 

\exo [Level=1,Fight=0,Learn=0,Field=\EspacesVectoriels,Type=\Exercices,Origin=] tx. 
Soient $u$ et $v$ deux endomorphismes d'un espace vectoriel $E$ tels que $u\circ v=v\circ u$. \pn
a) Démontrez que $\Ker\  u$ est stable par $v$ (i.e. que $v(\Ker\  u)\subset \Ker\  u$). \pn
b) Démontrez que $\IM\  u$ est stable par $v$ (i.e. que $v(\IM\  u)\subset \IM\  u$). \pn
c) Remarquez que $\Ker\  v$ et $\IM\  v$ sont stables par $u$.  

\exo [Level=1,Fight=0,Learn=0,Field=\EspacesVectoriels,Type=\Exercices,Origin=] ty. 
Soit $E$ un espace vectoriel et $f\in\sc L(E)$ tel que $f^2-4f+\Id_E=0$. \pn
Montrer que $f$ est bijective et déterminer $f^{-1}$ en fonction de $f$. 

\exo [Level=1,Fight=1,Learn=1,Field=\FonctionsDéfiniesParUneIntégrale,Type=\Exercices,Origin=] tz. 
Soient $A$, $B$, $C$ et $D$ quatres sous-espaces vectoriels de $E$ tels que $E=A+B$, $A\subset C$ et $B\subset D$. \pn
a)  On suppose que $C\oplus D$. Prouver que $A=C$ et $B=D$. \pn
b) pourquoi cela ne marche t-il pas si on a seulement $E=C+D$ ? 

\exo [Level=1,Fight=1,Learn=1,Field=\EspacesVectoriels,Type=\Exercices,Origin=] ua. 
Soit $E$ un espace vectoriel et soit $f:E\to E$ telle que $f^3=f$. \pn Prouver que l'on a 
$$
E=\Ker(f-\Id_E)\oplus\Ker(f+\Id_E)\oplus\Ker f.
$$

\exo [Level=1,Fight=1,Learn=1,Field=\EspacesVectoriels,Type=\Exercices,Origin=] ub. 
Soient $E$ un espace vectoriel et $f\in\sc L(E)$. \pn
a) Montrer que 
$$
E=\IM f+\Ker f\ssi \IM f=\IM f^2
$$
b) Montrer que 
$$
\IM f\cap\Ker f=\{0\}\ssi \Ker f=\Ker f^2.
$$

\exo [Level=1,Fight=0,Learn=0,Field=\Matrices,Type=\Exercices,Origin=] uc. 
Soit $f:\ob R^2\to\ob R^3$ l'unique application linéaire vérifiant $$
f(1,0)=(1,2,3)\qquad \hbox{et}\qquad f(0,1)=(2,1,3).
$$
a) Pour $(x,y)\in \ob R^2$, calculer $f(x,y)$. \pn
b) Ecrire la matrice de $f$ dans les bases canoniques de $\ob R^2$ et $\ob R^3$. \pn
c) mêmes questions pour $g:\ob R^2\to\ob R^2$ telle que $$
g(1,1)=(2,1)\qquad\hbox{et}\qquad g(-1,1)=(3,0).
$$ 


\exo [Level=1,Fight=0,Learn=0,Field=\Matrices,Type=\Exercices,Origin=] ud. 
Soit $\sc B:=(e_1,e_2,e_3)$ une base d'un $\ob C$-espace vectoriel $E$ de dimension $3$. Et soit $u$ l'unique endomorphisme de $E$ vérifiant 
$$
u(e_1)=e_2, \qquad u(e_2)=e_3\quad\hbox{et}\quad u(e_3)=e_1. 
$$
a) Calculer $u(x)$ pour $x=ae_1+be_2+ce_3$. \pn
b) Ecrire la matrice de l'application linéaire $u$ dans la base $\sc B$. \pn
c) l'endomorphisme $u$ est il bijectif ? \pn
d) Pour quels nombres $a$ l'équation $u(x)=ax$ possède-t'elle une solution $x\neq0$ ? 

\exo [Level=1,Fight=0,Learn=0,Field=\DimensionFinie,Type=\Exercices,Origin=] ue. 
Montrer que la famille $\{e,f,g\}$ avec 
$$
e=(1+i,1,i), \qquad g=(i-1,1,-i) \quad \hbox{et}\quad h=(-2+i,0,-i) 
$$
est une base du $\ob C$-espace vectoriel  $\ob C^3$. 
Determiner dans la base $\{e,f,g\}$ les coordonnées du vecteur $(a,b,c)$. 

\exo [Level=1,Fight=0,Learn=0,Field=\Matrices,Type=\Exercices,Origin=] uf. 
Les vecteurs $u=(0,1,1)$, $v=(1,0,1)$ et $w=(1,1,0)$ forment ils une base de $\ob R^3$ ? 
Si oui, quelles sont les coordonnées du vecteur $(2,3,1)$ dans cette base (on ecrira la matrice correspondante) ?  

\exo [Level=1,Fight=1,Learn=1,Field=\Matrices,Type=\Exercices,Origin=] ug. 
Montrer que la famille $\{X(X-1), (X-1)(X-2), (X-1)(X-2)\}$ est une base de $\ob C_2[X]$. \pn
a) Déterminer les coordonnées dans cette base d'un polynôme quelconque $P=a+bX+cX^2$ et écrire la matrice correspondante. \pn
b) Déterminer l'unique polynôme $P\in\ob C_2[X]$ vérifiant $$
P(0)=\pi, \qquad P(1)=\sqrt2 \quad et \quad P(2)=i.
$$  

\exo [Level=1,Fight=0,Learn=1,Field=\EspacesPréHilbertiens,Type=\Cours,Origin=] uh. 
Soit $E$ un espace pré-hilbertein. Prouvez le théorème de Pythagore, c'est à dire que pour $(x,y)\in E^2$, on a 
$$
x\perp y\ssi\|x+y\|^2=\|x\|^2+\|y\|^2.
$$


\exo  [Level=1,Fight=1,Learn=2,Field=\EspacesPréHilbertiens,Type=\Cours,Origin=] ui. 
Soit $E$ un espace pré-hilbertein et $A\neq\emptyset$ un sous-ensemble de $E$. \pn
a) Prouver que $A^\perp$ est un espace-vectoriel. \pn
b) Prouver que $A\subset(A^\perp)^\perp$. \pn
On suppose desormais que $E$ est de dimension finie (euclidien) et on fixe un sous-espace vectoriel $F$ de $E$. \pn 
c) Prouver que $E=F\oplus F^\perp$. \pn
d) En déduire que $F=(F^\perp)^\perp$. 


\exo [Level=1,Fight=2,Learn=1,Field=\EspacesPréHilbertiens,Type=\Exercices,Origin=\Lakedaemon,Indication={2) On pourra écrire $\|p(x_1+\lambda x_2)\|\le\|x_1+\lambda x_2\|$ avec $x_1\in F$ et  $x_2\in G$.}] uj. 
Soient $F$ et $G$ deux sous-espaces vectoriels d'un espace pré-hilbertien $E$ et soit $p\in\sc L(E)$ une projection sur $F$ parallèlement à $G$. 
\pn
a) Si $F$ et $G$ sont orthogonaux, prouver que 
$$
\forall x\in E, \qquad \|p(x)\|\le \|x\|. \leqno{(*)}
$$
b) Si $(*)$ est vérifiée, prouver que $F$ et $G$ sont orthogonaux.


\exo [Level=1,Fight=2,Learn=1,Field=\EspacesPréHilbertiens,Type=\Exercices,Origin=\Lakedaemon] uk. 
Pour chaque $X=(x_1,\cdots, x_n)$ de l'espace $E=\ob R^n$, on pose 
$$
N_1(X):=|x_1|+\cdots+|x_n|\qquad \hbox{et}\qquad N_\infty(X):=\max\{|x_1|,\cdots,|x_n|\}.
$$
a) Prouver que $N_1$ et $N_2$ sont deux normes de $E=\ob R^n$. \pn
b) Si $N_1$ et $N_2$ étaient des normes euclidiennes de $E$, quelles seraient leurs formes polaires $\phi_1(X,Y)$ et $\phi_\infty(X,Y)$ ?  \pn
c) Montrer que ces formes polaires ne sont pas des produits scalaires de $E$ et en déduire que $N_1$ et $N_2$ ne sont pas des normes euclidiennes. 

\exo [Origin=,Level=1,Fight=0,Learn=0,Type=\TravauxDirigés,Field=\Déterminant,Notion=Opérations élémentaires,Solution={$D=0$.}]  ul. 
Calculer le déterminant $D:=\Q|\matrix{1&1&1\cr a&b&c\cr b+c&c+a&a+b}\W|$.

\exo [Origin=,Level=1,Fight=0,Learn=0,Type=\TravauxDirigés,Field=\Déterminant,Notion=Opérations élémentaires,Solution={$D=(a+b+c)^3$.}] um. 
Calculer le déterminant $D:=\Q|\matrix{a-b-c&2a&2a\cr2b&b-c-a&2b\cr2c&2c&c-a-b}\W|$.

\exo [Origin=,Level=1,Fight=1,Learn=1,Type=\Exercices,Field=\Déterminant,Notion=Opérations élémentaires,Indication={Faire $L_2-L_1\to L_2$, $L_3-L_1\to L_3$, $C_1-C_2-C_3\to C_1$, $L_1-(b/2)L_2-(a/2)L_3\to L_1$ puis développer},Solution={$D=2abc(a+b+c)^3$.}] un. 
Calculer le déterminant $D:=\Q|\matrix{(b+c)^2&b^2&c^2\cr a^2&(c+a)^2&c^2\cr a^2&b^2&(a+b)^2}\W|$.

\exo [Origin=,Level=1,Fight=1,Learn=1,Type=\Exercices,Field=\Déterminant,Notion=Operations elementaires,Indication={Utiliser les formules de trigo pour $\cos p-\cos q$ et $\sin p-\sin q$},Notion=Operations elementaires|Trigonometrie,Solution={$D=4\sin\Q({b-a\F2}\W)\sin\Q({c-a\F2}\W)\sin\Q({b-c\F2}\W)$}] uo. 
Calculer le déterminant $D:=\Q|\matrix{1&1&1\cr\sin a&\sin b&\sin c\cr\cos a&\cos b&\cos c}\W|$.

\exo [Origin=,Level=1,Fight=1,Learn=1,Type=\Exercices,Field=\Déterminant,Notion=Operations elementaires,Solution={$D=(b-a)(c-a)(d-a)(c-b)(d-b)(d-c)(a+b+c+d)$}]  up. 
Calculer le déterminant $D:=\Q|\matrix{1&a&a^2&a^4\cr1&b&b^2&b^4\cr1&c&c^2&c^4\cr1&d&d^2&d^4}\W|$.

\exo [Origin=,Level=1,Fight=0,Learn=0,Type=\TravauxDirigés,Field=\Déterminant,Notion=Opérations élémentaires,Solution={$D=(b-a)^2(a+b+2c)(a+b-2c)$.}]  uq. 
Calculer le déterminant $\det\pmatrix{a&c&c&b\cr c&a&b&c\cr c&b&a&c\cr b&c&c&a}$.

\exo [Origin=,Level=1,Fight=0,Learn=0,Type=\Exercices,Field=\Déterminant,Notion=Opérations élémentaires,Solution={$D:=2abc(b-a)(c-a)(c-b)$}] ur. 
Calculer le déterminant $\det\pmatrix{b+c&c+a&a+b\cr b^2+c^2&c^2+a^2&a^2+b^2\cr b^3+c^3&c^3+a^3&a^3+b^3}$.

\exo [Origin=,Level=1,Fight=1,Learn=1,Type=\Exercices,Field=\Déterminant,Indication={Faire $L_1-\sum_{2\le k\le n}L_k\to L_1$},Solution={$D=2-n$.}]  us. 
Calculer le déterminant $D:=\det\pmatrix{1&1&1&\cdots&1\cr1&1&0&\cdots&0\cr1&0&1&\ddots&0\cr\vdots&\vdots&\ddots&\ddots&0\cr1&0&\cdots&0&1}$.

\exo [Origin=,Level=1,Fight=3,Learn=1,Type=\Exercices,Field=\Déterminant,Indication={Faire $L_k-L_1\to L_k$ pour $2\le k\le n$ puis $L_1-\sum_{k=2}^{n+1}{L_k\F k+1}\to L_1$},Solution={$\Q(3+\sum_{k=1}^{n+1}{2\F k}\W){(n+1)!\F2}$}]  ut. 
Calculer le déterminant $\det\pmatrix{3&1&1&\cdots&1\cr1&4&\ddots&1&\vdots\cr\vdots&\ddots&5&\ddots&1\cr\vdots&1&\ddots&\ddots&1\cr1&\cdots&\cdots&1&(n+2)}$.

\exo [Origin=,Level=2,Fight=4,Learn=2,Type=\Others,Field=\Déterminant] uu. 
Calculer le déterminant $\det\pmatrix{0&1&\cdots&1\cr-1&0&\ddots&\vdots\cr\vdots&\ddots&\ddots&1\cr-1&\cdots&-1&0 }$.

\exo [Origin=,Level=1,Fight=0,Learn=0,Type=\Exercices,Field=\Déterminant,Solution={%
	S'il existait un tel endomorphisme, on aurait 
	$$
		\det(f)^2=\det(f^2)=\det(-\mbox{Id}_E)=-1{\mbox{dim}(E)}=-1,	
	$$
	ce qui est impossible dans $\ob R$.%
}] uv. 
Soit $E$ un $\ob R$-espace vectoriel de dimension finie impaire. Prouver qu'il n'existe aucun $f\in\sc L(E)$ vérifiant $f^2=-\Id_E$. 

% redondant uw.

\exo [Origin=,Level=2,Fight=0,Learn=0,Type=\TravauxDirigés,Field=\Déterminant,Notion=Operations elementaires,Indication=Transformer en matrice diagonale,Solution={$D=(-1)^{[n/2]}\prod_{i=1}^n\lambda_i$}] ux. 
Pour $(\lambda_1,\cdots,\lambda_n)\in\ob C^n$, calculer le déterminant $D_n(\lambda_1,\cdots,\lambda_n):=\Q|\matrix{
	\ob O&&\lambda_n\cr
	&\LD@addots& \cr
	\lambda_1&&\ob O\cr
}\W|$. 

\exo  [Level=1,Fight=1,Learn=1,Field=\Topologie,Type=\Cours,Origin=] uy.  
Pour chaque $x=(x_1,\cdots, x_n)$ de l'espace $E=\ob R^n$, on pose 
$$
\eqalign{
N_1(x):=|x_1|+\cdots+|x_n|\qquad \hbox{et}\qquad N_\infty(X):=\max\{|x_1|,\cdots,|x_n|\}}.
$$
a) Dans cette question $n=2$.  Représenter sur le même dessin, la boule (ouverte) unité de centre $0$ pour les normes $N_1$, $N_2$, $N_\infty$ de $\ob R^2$. \pn
b) Dans cette question, $n\ge1$. Montrer qu'il existe des constantes strictement positives $a$, $b$, $\alpha$ et $\beta$ telles que 
$$
\forall x\in E, \qquad a N_2(x)\le N_\infty(x)\le b N_2(x)\qquad \hbox{et}\qquad \alpha N_2(x)\le N_1(x)\le \beta N_2(x).
$$
c) en déduire que les trois boules précédentes sont des ouverts. 

\exo  [Origin=,Level=1,Fight=1,Learn=1,Type=\Cours,Field=\Topologie] uz. 
Montrer que l'ensemble $A:=\{(x,y)\in\ob R^2:x+y>1\}$ est ouvert. 

\exo [Origin=,Level=1,Fight=1,Learn=1,Type=\Cours,Field=\Topologie] va. 
Prouver que l'ensemble $B:=\{(x,y)\in\ob R^2:xy=2\}$ est fermé. 

\exo [Origin=,Level=1,Fight=1,Learn=1,Type=\Exercices,Field=\Topologie] vb. 
a) Prouver que $f:(x,y)\mapsto \sin(xy)+\cos(x)$ est continue sur $\ob R^2$. \pn
b) En déduire que l'ensemble $\{(x,y)\in\ob R^2:-1<\sin(xy)+\cos(x)<5\}$ est ouvert. 

\exo [Origin=,Level=1,Fight=1,Learn=1,Type=\Exercices,Field=\Topologie] vc. 
Dans $E=\ob R$ muni de sa norme euclidienne classique. \pn
a) Que vaut $\|x\|$ pour $x\in\ob R$ ? \pn
b) Pour $a\in\ob R$ et $r>0$, quel objet géomètrique est $B(a,r)$ ? \pn
c) Parmi les ensembles suivants, lesquels sont ouverts/fermés ? 
$$
\eqalign{
&a: \quad \Q]-\infty,0\W[ \quad\qquad b:\quad \Q]-\infty, 0\W]\quad\qquad  c:\quad \Q]0,1\W[\quad \qquad d:\quad \Q[0, 1\W[\cr
&e:\quad  [0,1] \quad\qquad f: \quad \Q]1,\infty\W[\ssm[2,3]\quad \qquad g:\quad \ob R\quad\qquad h:\quad \Q]0,1\W[\cup\Q]2,3\W[
}
$$

\exo [Origin=,Level=1,Fight=1,Learn=1,Type=\Exercices,Field=\Continuité] vd. 
Prouver que l'on définit une fonction continue en posant $f(0,0)=0$~et 
$$
\forall (x,y)\neq(0,0)\qquad f(x,y):={xy\F |x|+|y|}.
$$


\exo [Origin=,Level=1,Fight=1,Learn=1,Type=\Exercices,Field=\Continuité] ve. 
Prouver que l'application définie par $f(0,0)=0$ et 
$$
\forall (x,y)\neq(0,0)\qquad f(x,y):={xy\F x^2+y^2}
$$
n'est pas continue sur $\ob R^2$. 


\exo [Origin=,Level=1,Fight=1,Learn=1,Type=\Exercices,Field=\Continuité] vf. 
L'application définie par $f(0,0):=0$ et 
$$
\forall (x,y)\neq(0,0)\qquad f(x,y):={xy^2\F x^4+y^2}
$$
est elle continue sur $\ob R^2$ ? 

\exo [Level=2,Fight=1,Learn=1,Field=\Orthonormalisation,Type=\TravauxDirigés,Origin=\Lakedaemon] vg. 
Le but de cet exercice est de calculer la borne inférieure
$$
\inf_{(a,b,c)\in\ob R^3}\int_{-1}^1\Q(ax^2+bx+c-x^3\W)^2\d x
$$
1) Orthonormaliser la famille $\{1, X, X^2\}$ pour le produit scalaire $\langle f, g\rangle:=\int_{-1}^1f(t)g(t)\d t$ \pn
2) Déterminer la projection orthogonale $P$ du polynôme $X^3$ sur l'espace $\ob R_2[X]$. \pn
3) Calculer la distance de $X^3$ à $P$  et conclure. 


\exo [Level=2,Fight=1,Learn=1,Field=\Orthonormalisation,Type=\TravauxDirigés,Origin=\Lakedaemon] vh. 
Le but de cet exercice est de calculer la borne inférieure
$$
\inf_{(x,y)\in\ob R^2}\int_0^\pi\Q(x\cos t+y\sin t-t\W)^2\d t
$$
1) Orthonormaliser la famille $\{\cos,\sin\}$ pour le produit scalaire $\langle f, g\rangle:=\int_0^\pi f(t)g(t)\d t$ \pn
2) Déterminer la projection orthogonale $f$ de l'application $\hbox{Id}:t\mapsto t$ sur l'espace $\hbox{Vect}(\cos,\sin)$. \pn
3) Calculer la distance de $f$ à l'application $\hbox{Id}:t\mapsto t$ et conclure. 

\exo [Level=1,Fight=0,Learn=0,Field=\EspacesVectoriels,Type=\Exercices,Origin=] vi. 
Donner une base du sous-espace de $\ob R^5$ défini par 
$$
\eqalign{
&x_1+2x_2-x_3+3x_4+x_5=0\cr
&x_2+x_3-2x_4+2x_5=0\cr
&2x_1+x_2-5x_3-4x_5=0
}
$$

\exo  [Level=1,Fight=0,Learn=1,Field=\IntégralesMultiples,Type=\Exercices,Origin=] vj. 
Soit $D$ le triangle de sommets $(0,0)$, $(1,1)$ et $(2,-1)$. Calculer l'intégrale $\ds\int_D(x+2y)^2\d x\d y$. 

\exo [Level=1,Fight=0,Learn=1,Field=\IntégralesMultiples,Type=\Exercices,Origin=] vk. 
Soit $D:=\{(x,y)\in\ob R^2:x+y\ge1,x^2+y^2\le 1\}$. Calculer l'intégrale $\ds I:=\int_Dxy^2\d x\d y$. 

\exo [Level=1,Fight=0,Learn=1,Field=\IntégralesMultiples,Type=\Exercices,Origin=] vl. 
Soit $D:=\{(x,y)\in\ob R^2:y\ge0, x^2+y^2-x\ge0,x^2+y^2-2x\le 0\}$. Calculer $\ds I=\int_D{x-y\F x^2+y^2}\d x\d y$

\exo [Level=1,Fight=0,Learn=1,Field=\IntégralesMultiples,Type=\Exercices,Origin=] vm. 
Soient $R>0$ et $$
D:=\Q\{(x,y)\in\Q[0,+\infty\W[^2:x^2+y^2-2Rx\le 0,x^2+y^2-2Ry\le 0\W\}.
$$
Calculer l'intégrale $\int_D(x^2-y^2)\d x\d y$.


\exo [Level=1,Fight=0,Learn=1,Field=\IntégralesMultiples,Type=\Exercices,Origin=] vn. 
Soit $D:=\{(x,y)\in\ob R^2:x^2+y^2-y\le 0,x^2+y^2-x\le 0\}$.
Calculer l'intégrale 
$$
\int_D(x+y)^2\d x\d y.
$$


\exo [Level=1,Fight=1,Learn=2,Field=\IntégralesMultiples,Type=\Exercices,Origin=] vo. 
Soient $a>0$ et $^b>0$. Calculer $\int_D\sqrt{1-{x^2\F a^2}-{y^2\F b^2}}\d x\d y$ pour 
$$
D:=\{(x,)\in\ob R^2:{x^2\F a^2}+{y^2\F b^2}\le 1\}. 
$$

\exo [Level=1,Fight=1,Learn=1,Field=\IntégralesMultiples,Type=\Exercices,Origin=] vp. 
Calculer $\int_D{\d x\d y\F \root 4\of x\sqrt y}\d x\d y$ pour 
$$
D:=\{(x,)\in\ob R^2:0<x<1,0<y<{1\F x}\}. 
$$
en utilisant le changement de variable $x=u$ et $y={v\F u}$. 

\exo [Level=1,Fight=1,Learn=1,Field=\IntégralesMultiples,Type=\Exercices,Origin=] vq. 
Calculer $\int_D{\d x\d ydz\F (x+y+z+1)^3}$ pour 
$$
D:=\{(x,y,z)\in\Q[0,+\infty\W[^3:x+y+z\le1\}. 
$$

\exo [Level=1,Fight=0,Learn=1,Field=\IntégralesMultiples,Type=\Exercices,Origin=] vr. 
Calculer $\int_D\cos(x+y-z)\d x\d ydz$ pour 
$$
D:=\Q[0,{\pi\F2}\W]^3. 
$$

\exo [Level=1,Fight=1,Learn=1,Field=\IntégralesMultiples,Type=\Exercices,Origin=] vs.  
Calculer $I=\int_Dxyz\d x\d ydz$ pour 
$$
D:=\{(x,y,z)\in\Q[0,+\infty\W[^3:x+y+z\le 1\} 
$$
en utilisant le changement de variables $x+y+z=u$, $y+z=uv$ et $z=uvw$. 

\exo [Level=1,Fight=0,Learn=1,Field=\Volumes,Type=\Exercices,Origin=] vt. 
Calculer le volume $D$ limité par la sphère de centre $(0,0,0)$ de rayon $2$ et le cylindre d'équation $x^2+y^2=y$. 

\exo [Origin=,Level=1,Fight=1,Learn=1,Field=\EspacesVectoriels,Type=\Exercices] vw. 
a) Prouver que l'on définit un endomorphisme en posant 
$$
\eqalign{u:\sc M_2(\ob R)&\to\sc M_2(\ob R)\cr M&\mapsto{}^tM-M+\mbox{Tr}\ (M)I_2}
$$
b) Déterminer sa matrice dans la base canonique de $\sc M_2(\ob R)$. \pn
c) Est-ce que $u$ est un automorphisme ? 


\exo [Origin=,Level=1,Fight=2,Learn=1,Field=\EspacesVectoriels,Type=\Exercices] vx. 
Etant donné  $A:=\pmatrix{1&1\cr1&0}$, Calculez les puissances $A^n$ pour $n\ge1$, puis pour $n\in\ob Z$.

\exo [Level=1,Fight=2,Learn=2,Field=\Rang,Type=\Exercices,Origin=] vy. 
Soient $f$ et $g$ deux endomorphismes d'un espace vectoriel $E$ de dimension $n$. \pn
a) Montrer que $g\circ f=0\ssi\IM(f)\subset\ker(g)$. \pn
b) Montrer que $\IM(f+g)\subset\IM(f)+\IM(g)$ et que $\mbox{rg}(f+g)\le\mbox{rg}(f)+\mbox{rg}(g)$. \pn
c) On suppose que $g\circ f=0$ et que $f+g=\mbox{Id}_E$. Montrer que $\IM(f)=\ker(g)$. \pn
Calculer $\mbox{rg}(f)+\mbox{rg}(g)$ en fonction de $n$. 

\exo [Level=1,Fight=1,Learn=0,Field=\Matrices,Type=\Exercices,Origin=,Solution={$D=0$ car  $C_1+C_3=2\cos(a)C_3$ pour tout $a$, d'après la relation $\cos p+\cos q=2\cos\Q({p+q\F2}\W)\cos\Q({p-q\F2}\W)$.},Notion={Opérations élémentaires|$\cos p+\cos q=2\cos\Q({p+q\F2}\W)\cos\Q({p-q\F2}\W)$}] vz. 
Calculer et factoriser le déterminant de la matrice 
$$
\pmatrix{1&\cos(a)&\cos(2a)\cr \cos(a)&\cos(2a)&\cos(3a)\cr \cos(2a)&\cos(3a)&\cos(4a)}
$$ 
Pour quelle valeur de $a$ est il nul ? 

\exo [Level=1,Fight=0,Learn=0,Field=\Intégration,Type=\Exercices,Origin=] wa. 
Calculer $\int_{-3}^0\Q|x^2-x-2\W|\d x$. 

\exo [Level=1,Fight=0,Learn=0,Field=\Intégration,Type=\Exercices,Origin=] wb. 
Montrer qu'il existe des nombres réels $a$ et $b$ tels que 
$$
\forall x\in\ob R\ssm\Q\{-{\pi\F 4}+k\pi:k\in\ob Z\W\},\qquad{\sin(x)\F \sin(x)+\cos(x)}=a{\cos(x)-\sin(x)\F\cos(x)+\sin(x)}+b 
$$
Pour $|\alpha|<{\pi\F 4}$, en déduire l'intégrale 
$$
I(\alpha):=\int_\alpha^{{\pi\F 2}-\alpha}{\sin(x)\F \sin(x)+\cos(x)}\d x.
$$

\exo [Level=1,Fight=1,Learn=1,Field=\Intégration,Type=\Exercices,Origin=] wc. 
Étudier la fonction (tableau de variation complet) définie par 
$$
f(x)={1\F x-1}\int_1^x{t^2\F\sqrt{1+t^8}}\d t.
$$

\exo [Level=1,Fight=1,Learn=1,Field=\Intégration,Type=\Exercices,Origin=] wd. 
Étudier la fonction (tableau de variation complet) définie par 
$$
f(x)=\int_x^{2x}\e^{-t^2}\d t.
$$

\exo [Level=1,Fight=2,Learn=1,Field=\Intégration,Type=\Exercices,Origin=] we. 
Lorsque $x\not\equiv0\ [\pi]$, calculer l'intégrale $$
J(x)=\int_{-\pi}^\pi{\d \theta\F1+\cos(x)\cos(\theta)}.
$$ 

\exo [Level=1,Fight=1,Learn=1,Field=\Intégration,Type=\Exercices,Origin=] wf. 
Calculer $\ds\int_0^{\pi\F 4}{\d t\F 1+\cos(t)^2}$ en posant $u={1\F\sqrt2}\tan(t)$. 


\exo [Level=1,Fight=1,Learn=1,Field=\Intégration,Type=\Exercices,Origin=] wg. 
Calculer $\ds \int_{\ln(5)}^{\ln(13)}{\e^x\d x\F (3+\e^x)\sqrt{\e^x-1}}$ en posant $t={1\F 2}\sqrt{\e^x-1}$. 


\exo [Level=1,Fight=1,Learn=1,Field=\Intégration,Type=\Exercices,Origin=] wh. 
Calculer $\ds\int_0^{\pi\F 4}{\sin(2x)\F 1+\cos(x)}\d x$ et $\ds\int_0^{\pi\F 2}{\sin(2x)+\sin(x)\F 3+\cos(2x)}\d x$. 


\exo [Level=1,Fight=3,Learn=2,Field=\Intégration,Type=\Exercices,Origin=] 
wi. {\bf Pour les mathématiciens-padawan du coté obscur uniquement}. Pour $x>1$, montrer qu'il existe un unique nombre $f(x)\in\Q]1,+\infty\W[$ tel que 
$$
\int_x^{f(x)}{\d t\F \ln(t)}=1. 
$$
Prouver que la fonction $f$ est de classe $\sc C^\infty$ sur $\Q]1,+\infty\W[$, étudier $f$ sur son ensemble de définition et  trouver une équation différentielle (non-linéaire) satisfaite par $f$. 

\exo [Level=1,Fight=1,Learn=1,Field=\Intégration,Type=\Exercices,Origin=] wj. 
Calculer $\ds \int_0^{1\F\sqrt 2}{x\arcsin(x)\F\sqrt{1-x^2}}\d x$. 

\exo [Level=1,Fight=1,Learn=1,Field=\Intégration,Type=\Exercices,Origin=] wk. 
Calculer $\ds \int_{\pi\F 3}^{\pi\F 2}\cos(x)\ln\Q(1-\cos(x)\W)\d x$. 

\exo [Level=1,Fight=1,Learn=1,Field=\Intégration,Type=\Exercices,Origin=] wl. 
Calculer $\ds \int_0^1x^2\arctan(x)\d x$. 

\exo [Level=1,Fight=1,Learn=1,Field=\Intégration,Type=\Exercices,Origin=] wm. 
Pour $n\in_ob N^*$, calculer $\ds \int_e^{e^2}{\d x\F x\ln(x)^n}$. 

\exo [Level=1,Fight=0,Learn=0,Field=\CourbesParamétréesCartésiennes,Type=\Exercices,Origin=] wn. 
Tracer la courbe paramétrée
$$
\Q\{\eqalign{
x(t)=\sin(2t),\cr
y(t)=\sin(3t)}
\W.
$$

\exo [Level=1,Fight=0,Learn=0,Field=\CourbesParamétréesPolaires,Type=\Exercices,Origin=] wo. 
Tracer la courbe paramétrée polaire
$$
r=\cos(\theta)-\cos(2\theta)
$$

\exo [Level=1,Fight=0,Learn=0,Field=\CourbesParamétréesCartésiennes,Type=\Exercices,Origin=] wp. 
Tracer la courbe paramétrée
$$
\Q\{\eqalign{
x(t)={1-t^2\F1+t^2},\cr
y(t)=t{1-t^2\F 1+t^2}}
\W.
$$

\exo [Level=1,Fight=0,Learn=0,Field=\CourbesParamétréesPolaires,Type=\Exercices,Origin=] wq. 
Tracer la courbe paramétrée polaire
$$
r={\cos(\theta)\F 1+\sin(\theta)}.
$$

\exo [Level=1,Fight=0,Learn=0,Field=\CourbesParamétréesPolaires,Type=\Exercices,Origin=] wr. 
Tracer la courbe paramétrée polaire
$$
r={1\F \sqrt{\sin(2t)}}.
$$

\exo [Level=1,Fight=0,Learn=0,Field=\CourbesParamétréesPolaires,Type=\Exercices,Origin=] ws. 
Tracer la courbe paramétrée polaire
$$
r={\tan(\theta)\F \cos(\theta)}.
$$

\exo [Origin=Fac,Level=1,Fight=2,Learn=1,Type=\TravauxDirigés,Field=\EspacesVectoriels] wt. 
Montrer que l'ensemble $E:=\Q\{f:x\mapsto a\cos (x-\varphi):(a,\varphi)\in\ob R^2\W\}$ forme un sous-espace vectoriel de $\sc F(\ob R,\ob R)$. 

\exo  [Origin=,Level=1,Fight=0,Learn=1,Type=\TravauxDirigés,Field=\EspacesVectoriels] wu. 
On pose $M=\pmatrix{4&-2&1\cr2&-1&2\cr-1&2&2}$. \pn
a) Calculer $(M-3I_3)(M+I_3)$. En déduire une relation poynômiale du type $P(M)=0$.  \pn
b) Justifier sans calculs que $M$ est inversible et déterminer $M^{-1}$. 

\exo [Origin=\Quercia,Level=1,Fight=0,Learn=0,Type=\Exercices,Field=\Rang,Solution={$\mbox{rang}(A)=3$.}] wv. 
Calculer le rang de la matrice $A:=\pmatrix{1&2&-4&-2&-1\cr0&-2&4&2&0\cr1&1&-2&-1&1}$. 

\exo [Origin=\Quercia,Level=1,Fight=0,Learn=0,Type=\Exercices,Field=\Rang,Solution={$\mbox{rang}(A)=4$.}] ww. 
Calculer le rang de la matrice $A:=\pmatrix{
	0  &-1 &2  &-2 \cr
	-7 &-7 &2  &-8 \cr
	0  &4  &-6 &6  \cr
	2  &-2 &0  &-2 \cr }$.

\exo [Origin=\Quercia,Level=1,Fight=0,Learn=0,Type=\Exercices,Field=\Rang,Solution={$\mbox{rang}(A)=2$.}] wx. 
Calculer le rang de la matrice $A:=\pmatrix{
	1  &7  &2  &5  \cr
	-2 &1  &1  &5  \cr
	-1 &2  &1  &4  \cr
	1  &4  &1  &2  \cr }$.

\exo [Origin=\Quercia,Level=1,Fight=0,Learn=0,Type=\Exercices,Field=\Rang,Solution={$\mbox{rang}(A)=3$.}] wy. 
Calculer le rang de la matrice $A:=\pmatrix{
	1  &4  &-1 &2  &4  \cr
	2  &0  &-3 &-1 &7  \cr
	-2 &3  &2  &1  &4  \cr }$.

\exo [Origin=\Quercia,Level=1,Fight=1,Learn=1,Type=\Colles,Field=\Rang,Solution={$\mbox{rg}(A)=\Q\{\eqalign{%
	&3\mbox{ si }\lambda\notin\{-20,-3\}\cr
	&2\mbox{ si }\lambda\in\{-20,-3\}\cr
	}\W.$ et $\Q\{\eqalign{%
	5C_2+C_3&=0\mbox{ si }\lambda=-20\cr
	L_1+L_3&=0\mbox{ si }\lambda=-3\cr
	}\W.$}] wz. 
Étudier le rang de la matrice $A:=\pmatrix{
	\lambda &1 &-5\cr
	-1 &4 &\lambda\cr
	3 & -1 &5
}$ en fonction de $\lambda$. \pn
Lorsque $A$ n'est pas inversible, donner une relation de dépendance linéaire entre ses lignes.

\exo [Origin=\Fac,Level=1,Fight=0,Learn=0,Type=\Exercices,Field=\Déterminant] xa.  
Calculer le déterminant $\displaystyle D:= \Q\vert\matrix{
	1&2&3&4\cr
	2&3&4&1\cr 
	3&4&1&2\cr
	4&1&2&3}\W\vert$ 

\exo [Origin=\Quercia,Level=1,Fight=0,Learn=0,Type=\Colles,Field=\Rang] xb.  
Montrer que les vecteurs $\vec i:= (1,0,1)$,
$\vec j:= (-1,-1,2)$ et $\vec k:= (-2,1,-2)$ forment une base de $\ob R^3$ et calculer
les coordonnées dans cette base d'un vecteur quelconque $\vec v := (x,y,z)$.

\exo [Origin=,Level=1,Fight=0,Learn=0,Type=\Exercices,Field=\Matrices] xc. 
Inverser la matrice $A:=\pmatrix{
	-1  &2  &5   \cr
	1  &2  &3  \cr
	-2 &8  &10  \cr }$.

%% Redondant xd.

\exo [Origin=,Level=1,Fight=0,Learn=0,Type=\TravauxDirigés,Field=\Rang,Solution={$\mbox{rang}(\{a,b,c,d\})=2$.}] xe. 
Calculer le rang de la famille de vecteurs 
$$
a=\pmatrix{1\cr -2\cr -1\cr 1}, \qquad b=\pmatrix{7\cr 1\cr 2\cr 4}, \qquad c=\pmatrix{2\cr 1\cr 1\cr 1}, \qquad d=\pmatrix{5\cr5\cr4\cr2}. 
$$

\exo [Origin=,Level=1,Fight=3,Learn=0,Type=\TravauxDirigés,Field=\Rang] xf. 
Calculer en fonction de $m$ le rang de la matrice $\pmatrix{1&-1&0&1\cr m&1&-1&-1\cr1&-m&1&0\cr1&-1&m&2}$. 

\exo [Origin=\Lakedaemon,Level=1,Fight=0,Learn=0,Type=\Cours,Field=\SystèmesLinéaires] xg. 
Résoudre le système 
$$
\Q\{\eqalign{
x_2+x_3+\cdots+x_n&=1\cr
x_1+x_3+\cdots+x_n&=2\cr
\vdots&=\vdots\cr
x_1+\cdots+x_{n-1}&=n\cr}\W.
$$

\exo [Origin=\Lakedaemon,Level=1,Fight=2,Learn=1,Type=\Exercices,Field=\Matrices|\Récurrences] xh. 
Exprimer la matrice $A^n$ en fonction de $A=\pmatrix{1&1\cr1&0\cr}$,  de $\mbox I_2$ et de $n\ge1$.

\exo [Origin=\Lakedaemon,Level=1,Fight=2,Learn=1,Type=\Cours,Field=\EquationsDifférentiellesLinéairesDuPremierOrdre] xi. 
Résoudre l'équation différentielle $xy'=2y+1$ sur $\ob R$.

\exo [Origin=\Lakedaemon,Level=1,Fight=1,Learn=1,Type=\TravauxDirigés,Field=\EquationsDifférentiellesAVariablesSéparables] xj. 
Résoudre l'équation différentielle $y'-1-ty^2=t+y^2$. 


\exo [Origin=Deug06,Level=1,Fight=1,Learn=2,Type=\Problèmes,Field=\Suites|\Matrices|EquationsDifférentielles,Indication={3b) On pourra simplifier $\sum_{1\le k\le n}v_k$.}] xk. 
Les trois exercices suivants sont indépendants. \pn
$\underline{\hbox{\bf Exercice 1}}${\bf. Etude d'une suite récurrente. }\pn
 Soit $I$ l'intervalle $\Q]0,{1\F\sqrt6}\W[$ et soit $u=(u_n)_{n\ge1}$ la suite définie par $u_1:={1\F 10}$ et
$$
u_{n+1}:=u_n-2u_n^3\qquad(n\ge1).
$$ 
Soit $f:I\to\ob R$ la fonction définie pour $x\in I$ par $f(x):=x-2x^3$. \pn
1. {\it Etude de la convergence de $u$. }\pn
a. Déterminer les variations de $f$ sur $I$ puis comparer $f(I)$ et $I$. \pn
b. Déterminer la monotonie de la suite $u$. \pn
c. Montrer que la suite $u$ est convergente et déterminer sa limite. 
\medskip\noindent
2. {\it Théorème de Cesàro. }\pn
Soit $v=(v_n)_{n\ge1}$ une suite convergeant vers un nombre réel $\ell$. \pn
On définit alors la suite $M$ en posant $M_n:=(v_1,\cdots,v_n)/n$ pour $n\ge1$. 
Le nombre $M_n$ est la moyenne arithmétique des $n$ premiers termes de la suite $v$. \pn
a. Traduire à l'aide de quantificateurs le fait que la suite $v$ converge vers $\ell$. \pn
b. Soit $n$ un entier naturel non nul et $p$ un entier tel que $1\le p\le n$. Montrer que 
$$
\Q|M_n-\ell\W|\le{1\F n}\sum_{1\le k\le p}|v_k-\ell|+\max_{p< k\le n}|v_k-\ell|\qquad (n\ge1). 
$$
c. Conclure avec soin que si la suite $v$ vonverge vers $\ell$, alors la suite $M$ converge aussi vers $\ell$. 
{\it Ce résultat porte le nom de théorème de Cesàro. }\medskip\noindent
3. {\it Application à la recherche d'un équivalent de $u$. }\pn
a. Déterminer la limite de ${1\F(x-2x^3)^2}-{1\F x^2}$ lorsque $x$ tend vers $0$. \pn
En déduire la limite de la suite $v$ définie par $v_n:={1\F u_{n+1}^2}-{1\F u_n^2}$ pour $n\ge1$. \pn
b. Utiliser tous les résultats précédents pour donner un équivalent de la suite $u$. \medskip\noindent
$\underline{\hbox{\bf Exercice 2}}${\bf .  Etude d'un endomorphisme sur l'espace des polynômes. }\pn
Soit $n\ge0$. On définit une application $f:\ob R_n[X]\to\ob R[X]$ en posant 
$$
\forall P\in\ob R_n[X], \qquad f(P):=X\big(P(X)-P(X-1)\big).
$$
4. {\it Résultats préliminaires. }\pn
a. Calculer $f(1)$, $f(X)$, $f(X^2)$. \pn
b. Si $P=a_nX^n+a_{n-1}X^{n-1}+\cdots+a_0$ avec $a_n\neq0$, quel est le terme de plus haut degré du polynôme $P(X-1)$ ?\pn
c. Soit $P\in\ob R_n[X]$ tel que $P(X)=P(X-1)$. On pose $Q:=P(X)-P(0)$. Montrer que $Q$ est un polynôme constant que l'on précisera. \medskip\noindent
5. Montrer que $f$ est un endomorphisme de $\ob R_n[X]$. \medskip\noindent
6. Déterminer le noyau de $f$ et en déduire la dimension de l'image de $f$. \medskip\noindent
7. Dans cette question {\bf uniquement}, on suppose que $n=2$. \pn
a. Quelle est la base canonique de $\ob R_2[X]$ ? Ecrire la matrice de l'endomorphisme $f$ dans cette base. \pn
b. L'endomorphisme $f$ est-il diagonalisable ?\medskip\noindent
8. {\it Etude de la diagonalisation dans le cas général. }\pn
On pose $P_0:=1$, $P_1:=X$ et 
$$
\forall k\ge2, \qquad P_k:=X(1-X)(2-X)\cdots(k-1-X).
$$
a. Montrer que la famille $(P_0,P_1, P_2, \cdots, P_n)$ est une base de $\ob R_n[X]$. \pn
b. Soit $k\in\ob N$. Déterminer un nombre réel $c_k$ telk que $f(P_k)=c_kP_k$. \pn
c. Déterminer les valeurs propres de $f$. L'endomorphisme $f$ est-il diagonalisable ?\medskip\noindent
$\underline{\hbox{\bf Exercice 3}}$. {\bf Résolution d'une équation différentielle. }\medskip\noindent
9. Soit $(E)$ l'équation différentielle $|x|y'+(x-1)y=x^2$. \medskip\noindent
a. Résoudre $(E)$ sur l'intervalle $\Q]0,+\infty\W[$. \pn
b. Sachant que les solutions de $(E)$ sur $\Q]-\infty,0\W[$ sont les fonctions $x\mapsto x+2+{2\F x}+B{\e^x\F x}$ où $B\in\ob R$, existe-t-il des solutions de $(E)$ définies sur $\ob R$ ? Si oui, les expliciter. 

\exo [Level=1,Fight=-1,Learn=0,Field=\Logique,Type=\Exercices,Origin=\Isabelle,Solution={\item{a. } Faux : prendre $x=0$.
	\item{b. } Vrai.
	\item{c. } Faux : prendre $x=1$ ; alors $x^2-4<0$ ne peut pas être égal 
				à une racine carrée.
				\item{d. } Vrai.
				\item{e. } Vrai.
				\item{f. } Vrai.
				\item{g. } Vrai : prendre $x={1\F 4}$.
				\item{h. } Vrai : prendre $x=0$.
				\item{i. } Vrai : utiliser les variations de la fonction 
				$x \mapsto{1\F x}-e^x$.
				\item{j. } Faux : car $\forall x>0, \ln x-\ln (2x) 
				= -\ln 2 \neq 3$.
				\item{k. } Faux : car $\forall x\ge0, 3x+5 \ge 5$.
\item{l. } Vrai : prendre $x = -\sqrt{\e^6+1}$.}] xl. 
Déterminer la valeur logique des propositions suivantes : 
$$
\eqalign{
	&a.\quad\forall x\in\ob R : x^2>0\cr
	&b.\quad\forall x\in\ob R : x^2 \ge -5,\cr
	&c.\quad \forall x\ge0 : x^2-4=\sqrt{(x^2-4)^2},\cr
	&d.\quad \forall x \in \ob R : x=\ln (e^x),\cr
	&e.\quad \forall x>0 : x=e^{\ln x},\cr
	&f.\quad \forall x\le0 : \sqrt{x^2}=-x,\cr
}\qquad\qquad
\eqalign{
	&g.\quad\exists x\in\ob R : \sqrt x > x,\cr
	&h.\quad \exists x\in\ob R : \sqrt x = -x,\cr
	&i.\quad \exists x \in \ob R : {1\F x}=e^x,\cr
	&j.\quad \exists x \in \ob R : \ln x -\ln (2x)=3,\cr
	&k.\quad \exists x\ge0 : 3x+5 < 0,\cr
	&l.\quad \exists x<0 : \ln \Q(\sqrt{x^2-1}\W) = 3.\cr
}
$$

\exo [Level=1,Fight=-1,Learn=0,Field=\Logique,Type=\Exercices,Origin=\Isabelle,Solution={
	\item{a. } $\Longrightarrow$ directement, et $\Longleftarrow$ par la contraposée.
	\item{b. } Montrer que $n^2$ est pair $\Longleftrightarrow n$ est pair, et que $n$ est pair $\Longleftrightarrow n^3$ est pair, 
	puis transitivité de l'équivalence.
	\item{c. } Implication sans problème, réciproque fausse en prenant $2\times 3^2+1 = 10$ impair, alors que 3 est impair. 
	\item{d. } Aligner les implications. Réciproque fausse : il suffit de prendre $x = -{3\F 2}$.
\item{e. } Aligner les équivalences.}] xm.
\item{a. } Montrer que $\forall n \in \ob N, n$ est pair $\Longleftrightarrow 3n+1$ est impair.
\item{b. } Montrer que $\forall n \in \ob N, n^2$ est pair $\Longleftrightarrow n^3$ est pair.
\item{c. } Montrer que $\forall n \in \ob N, n$ est pair $\Longrightarrow 2n^2+1$ est impair.
\item{} La réciproque est-elle vraie ? Justifier la réponse
\item{d. } Soit $x\in\ob R$. Montrer que $x\in\Q]-1;2\W[ \Longrightarrow {1\F\sqrt{x^2+1}}\in\Q]{\sqrt 5\F 5},1\W]$.
\item{} La réciproque est-elle vraie ? Justifier la réponse.
\item{e. } Soit $x\in\ob R$. Montrer que $x\ge 0\Longleftrightarrow \sqrt{\ln (3x+1)+4} \ge 2$.

\exo [Level=1,Fight=1,Learn=0,Field=\Logique,Type=\Exercices,Origin=\Isabelle,Solution={
$$
\eqalign{
	S_1&= \sum_{k=1}^n 3k+5 = 3 \sum_{k=1}^n k + 5 \sum_{k=1}^n 1 = 5n + 3 \sum_{k=1}^n k=5n+3{n(n+1)\F 2},\cr 
	S_2&= \sum_{i=1}^{2n} 3^{i+n} = 3^n \sum_{i=1}^{2n} 3^i=3^n{3^{2n+1}-3^1\F 2},\cr
	S_3&= \sum_{n=0}^{50} \Q({5\F n+3}-{5\F n+4}\W)= 5 \sum_{n=0}^{50}{1\F n+3} - 5 \sum_{n=0}^{50} {1\F n+4}\cr
	&= 5 \sum_{n=3}^{53}{1\F n}-5 \sum_{n=4}^{54} {1\F n}= {5\F 3}-{5\F 54}={85\F 54},\cr
	S_4 &= \sum_{0 \le j \le n}\Q(3(-1)^n+{4\F n+1}\W)= (n+1) \Q(3(-1)^n+{4\F n+1}\W) = 4+3(n+1)(-1)^n,\cr
	S_5&= \sum_{3\le k\le n+3} \Q((k-3)^2+{5\F k-2}-(k-3) \W)= \sum_{j=0}^n \Q( j^2+{5\F j+1}-j \W)\cr
&= \sum_{j=0}^n j^2-\sum_{j=0}^n j + 5 \sum_{j=0}^n {1\F j+1}.}
$$
}] xn.
Simplifier les sommes suivantes, puis les calculer lorsque c'est possible.
$$
\eqalign{
	S_1 &:= \sum_{k=1}^n (3k+5),\qquad S_2 := \sum_{i=1}^{2n} 3^{i+n},\qquad 
	S_3:= \sum_{n=0}^{50}\Q( {5\F n+3}-{5\F n+4}\W),\cr
	S_4&:= \sum_{0 \le j \le n}\Q( 3(-1)^n+{4\F n+1}\W),\qquad 
	S_5: = \sum_{3\le k\le n+3}\Q((k-3)^2+{5\F k-2}-(k-3) \W).
}
$$

\exo [Level=1,Fight=-1,Learn=0,Field=\Logique,Type=\Exercices,Origin=\Isabelle,Solution={
$S_1 = \sum\limits_{k=0}^{49} (2k+1)$, $S_2= \sum_{i=0}^n x^i$, $S_3 = \sum\limits_{k=1}^{n+2}{k\F 3^k}$, 
$S_4 = \sum_{i=0}^{n+2} 3^{2i}$, $S_5 = \sum_{j=1}^p{1\F 2j \sqrt{2j}}$. 
}] xo.	
Ecrire les formules suivantes en utilisant le symbole $\sum$.
$$
\eqalign{
S_1&:= 1+3+5+ \ldots + 99, \qquad 
S_2:= 1+x+x^2+x^3+ \ldots + x^n, \qquad 
S_3:={1\F 3} + {2\F 9} + {3\F 27} + \ldots + {n+2\F 3^{n+2}}, \cr
S_4&:= 1+3^2+3^4+3^6+ \ldots + 3^{2n+4}, \qquad
S_5:={1\F 2\sqrt 2} + {1\F 4\sqrt 4}+ {1\F 6\sqrt 6} + \ldots {1\F 2p\sqrt {2p}}.
}$$

\exo [Level=1,Fight=0,Learn=2,Field=\Récurrences,Type=\Exercices,Origin=\Isabelle] xp. 	
\item{a. } Montrer que $\forall n \in \ob N, \sum_{k=1}^n k = {n(n+1)\F 2}$ et que $\sum_{k=1}^n k^2 ={n(n+1)(2n+1)\F 6}$.
\item{b. } Montrer que $\forall n \in \ob N, 3^{2n}-2^n$ est divisible par 7.
\item{c. } Montrer que $\forall n \in \ob N^*,{1\F n!} \le {1\F 2^{n-1}}$.
\item{d. } Montrer que $\forall p \in \ob N,\forall n \in \N,n \ge p,\sum_{k=p}^n {k\choose p} = {n+1\choose p+1}$.
\item{e. } Soit $u$ la suite définie par $u_0:=3, u_1:=5$, et $\forall n \in \ob N, u_{n+2}:=3 u_{n+1}-2 u_n$. 
\item{} Montrer que $\forall n \in \ob N,  u_n=2^{n+1}+1$.
\item{f. } Montrer que $\forall n \in \ob N^*,\ \sum_{k=0}^{n-1} (2k+1)=n^2$.
\item{g. } Montrer que $\forall n \in \ob N,\ \sum_{k=0}^n a^k = {1-a^{n+1}\F 1-a}$ pour $a \ne 1$.
\item{h. } Montrer que $\forall n \ge 3,\ \sum_{k=3}^n 4k(k-1)(k-2) = n(n+1)(n-1)(n-2)$.
\item{k. } On considère une suite $(u_n)$ telle que $\forall n \in \ob N,\ u_{n+1} \le a u_n$, où $a$ est un  réel positif.
\item{} Montrer que $\forall n \in \ob N,  u_n \le a^n u_0$.
	
\exo [Level=1,Fight=0,Learn=0,Field=\Programmation,Type=\Maple,Origin=,Solution=] xq. 
Ecrire une procedure $Max$ qui prend en entrée deux nombres réels $a$ et $b$ et qui renvoie en sortie le  maximum des deux nombres $\ds Max(a,b)=\cases{a& si $a>b$\cr b& sinon}$. 

\exo [Level=1,Fight=0,Learn=0,Field=\Programmation,Type=\Maple,Origin=,Solution=] xr. 
Ecrire un  petit programme Maple pour calculer la somme suivante : 
$$
\sum_{i=1}^{100}{1\F i}.
$$
Ecrire un  petit programme Maple pour calculer la somme suivante : 
$$
\sum_{i=1}^{100}\sum_{j=1}^i{1\F i+j}.
$$

\exo [Level=1,Fight=0,Learn=0,Field=\Programmation,Type=\Maple,Origin=,Solution=] xs. 
Ecrire une procedure récursive $Fact$ qui prend en entrée un entier $n\ge0$ et qui renvoie en sortie $Fact(n)=n!$. 

\exo [Level=1,Fight=0,Learn=0,Field=\Programmation,Type=\Maple,Origin=,Solution=,Indication={Indication : pour la définition récursive, on utilisera que 
$$
\ds {n-1\choose k-1}+{n-1\choose k}={n\choose k}\qquad(1\le k\le n).
$$ }] xt. 
Ecrire une procedure récursive $c$ qui prend en entrée deux entiers $n\ge k\ge0$ et qui renvoie en sortie $c(n,k)={n\choose k}$. 

\exo [Level=1,Fight=0,Learn=0,Field=\Programmation,Type=\Maple,Origin=,Solution=] xu. 
Un apprenti-trader avec $n$ euros essaie de faire fortune en jouant à pile ou face. Il joue jusquà la ruine complete. \medskip\noindent
Ecrire une procédure qui trace le graphe de l'évolution de son argent au fil des parties. \pn
Les commandes recommandées  pour traiter cette question sont : \Maple{proc, for, by, while, do, od, listplot, seq, rand, end}. \pn
Comme le générateur de nombres aléatoires de Maple est ``moisi'', on prendra des valeurs de $n$ raisonnables (genre $n=10$). 

\exo [Level=1,Fight=0,Learn=0,Field=\Programmation,Type=\Maple,Origin=,Solution=] xv. 
En attendant sa petite amie, un étudiant joue à un jeu : \pn 
Il part d'un nombre entier $n\ge1$ puis a chaque tour de jeu, il remplace $n$ par ${n\F 2}$ si $n$ est pair et par $3n+1$ si $n$ est impair. 
Le jeu s'arrète lorsqu'il obtient $n=1$ (sa petite amie arrive). \pn
Ecrire une procédure qui prend un nombre entier $n$ en entrée et qui renvoie le nombre d'e tours de jeu effectuées pour arriver à $n=1$. 

\exo [Level=1,Fight=0,Learn=0,Field=\Programmation,Type=\Maple,Origin=,Solution=] xw. 
Soient 2 nombres réels $b>a>0$ et les suites $u$ et $v$ definies par 
$$
\eqalign{
	u_0&:=a\cr 
	v_0&:=b\cr
}\qquad\mbox{et}\qquad
\eqalign{
	u_{n+1}&:=\sqrt{u_nv_n}\cr
	v_{n+1}&:={u_n+v_n\F2}\cr 
}
$$
Ces suites sont adjacentes, donc admettent une limite commune $\ell$ vérifiant  
$$
u_n\le \ell\le v_n\qquad(n\ge0).
$$
Ecrire une procédure $Approx$ qui prend en entrée a, b et un nombre réel $\epsilon>0$ et qui retourne une approximation de $\ell$ à ${\epsilon\F 2}$ près. 

\exo [Level=1,Fight=0,Learn=0,Field=\Programmation,Type=\Maple,Origin=,Solution=,Indication={ On pourra utiliser une procédure récursive.}] xx. 
La methode de Viète du calcul de $\pi$ affirme que 
$$
\pi\sim 2\times{2\F\sqrt2}\times{2\F\sqrt{2+\sqrt 2}}\times{2\F\sqrt{2+sqrt{2+\sqrt 2}}}\times\cdots
$$
Réalisant un petit programme Maple permettant d'effectuer ce calcul.

\exo [Level=1,Fight=0,Learn=0,Field=\Programmation,Type=\Maple,Origin=,Solution=] xy. 
En 1973, Eugène Salamin et Richard Brent ont publié un algorithme fondé sur la moyenne arithmetico-géométrique vérifiant $\lim_{k\to\infty}p_k=\pi$ et dont la convergence est très rapide :
$$
\eqalign{
&a_0:=1, \qquad b_0:={\sqrt2\F 2},\qquad s_0:={1\F2},\cr
& a_k:={a_{k-1}+b_{k-1}\F 2},\qquad b_k:=\sqrt{a_{k-1}b_{k-1}},\qquad c_k:=a_k^2-b_k^2\cr
&s_k:=s_{k-1}-2^kc_k,\qquad p_k:=2{a_k^2\F s_k}
}
$$
Ecrire une procédure qui prend en entrée $k$ et qui renvoie en sortie $p_k$. 

\endinput
