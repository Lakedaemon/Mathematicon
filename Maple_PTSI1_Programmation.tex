\newif\ifbook\booktrue
\ifbook\magnification 1100\else\magnification 1400
\fi
\input color.txs
\input pstricks
\input pst-plot
\input eplaingt
\input Macrols
\input epsf
\SecLabelEqtrue             % le numero de section prefixe le label des equations
\NumReftrue                    % les references bibliographiques sont numerotees automatiquement
\titreun=0



\long\def\Touche#1{\psshadowbox[shadowcolor=gray,shadowsize=2pt,framearc=0]{\hbox{\eightpts #1}}}
\long\def\touche#1{\psframebox[framearc=0]{\hbox{\sevenpts #1}}}
\def\UnderBrace#1_#2{{\black\underbrace{#1}_{\hbox{\sevenpts\black #2}}}}


\book
\hautspages{O. L. Bindus}{Structures de Maple}
\book
\pagetitretrue


\CSect Maple1, Structures de Maple. 
\bigskip

En math\'ematiques, les nombres peuvent \^etre de diff\'erents types : entiers, d\'ecimaux, rationnels, r\'eels ou complexes et les techniques de calcul varient en cons\'equence. 
De m\^eme, Maple permet l'utilisation d'objets de diff\'erents types. 
Pour faire face \`a la majorit\'e des probl\`emes pos\'es en situation de concours, il est n\'ecessaire d'apprendre \`a manipuler ces objets et \`a les convertir d'un type vers un autre. 
\bigskip


\Secti Mapple2, Types. 

\Section Reconnaitre le type d'un objet

Pour connaitre le type d'un objet Maple, on utilise la commande {\it whattype}. 
$$
\eqalign{
&\hskip-16em\red >whattype(1/3);
\cr
&\blue fraction
\cr
&\hskip-16em\red >whattype([1,2,3]);
\cr
&\blue list
\cr
&\hskip-16em\red >whattype(4,2,1);
\cr
&\blue exprseq
}
$$

\Section Les types issus des math\'ematiques

$$
\vbox{
\offinterlineskip
\halign{
\vrule#&\quad#\hfil\strut\quad&\vrule#&\quad\hfil#\hfil\quad&\vrule#&\quad\hfil#\hfil\quad&\vrule#\tabskip=0pt\cr
\noalign{\hrule}\cr
&Type Maple&&Math\'ematiques&&Exemples&\cr
\noalign{\hrule}\cr
&integer&&Nombre entier relatif&&-13&\cr
\noalign{\hrule}\cr
&fraction&&Nombre rationnel&&$2/3$&\cr
\noalign{\hrule}\cr
&float&&Nombre \`a virgule flottante&& $2.71828$\hbox{ ou }$2.3E-3$&\cr
\noalign{\hrule}\cr
&set&&Ensemble d'objets&&$\{-1,0,1,exp(1),\hbox{Pi},\hbox{metal}\}$&\cr
\noalign{\hrule}\cr
}}
$$

\Remarque : l'expression Maple $2.3E-3$ d\'esigne le nombre $2.3\times10^{-3}=0.0023$. 
\smallskip 
\noindent
Les calculs sur les entiers et les fractions sont exacts contrairement aux calculs sur les nombres \`a virgules flottantes (approximations). 
\bigskip


\Section Les types issus de l'informatique

$$\vbox{
\offinterlineskip
\halign{
\vrule#&\quad#\hfil\strut\quad&\vrule#&\quad\hfil#\hfil\quad&\vrule#&\quad\hfil#\hfil\quad&\vrule#\tabskip=0pt\cr
\noalign{\hrule}\cr
&Type&&Informatique&&Exemples&\cr
\noalign{\hrule}\cr
&string&&Chaine de caract\`eres&&"Hekathomb", "r0cks", u&\cr
\noalign{\hrule}\cr
&symbol&&Nom Maple&& $'\hbox{x1}'$, \hbox{x1}, $'\hbox{a}'$, a, U\_r\_my\_Sl4v3s&\cr
\noalign{\hrule}\cr
&list&&liste d'objets&&$[\ 9,10,\hbox{valet}, \hbox{dame}, \hbox{roi}\ ]$&\cr
\noalign{\hrule}\cr
&exprseq&&s\'equence&&$1,\hbox{Death}, 2, \hbox{Emperor},  ..., 9999, \hbox{Britney}$&\cr
\noalign{\hrule}\cr
}}
$$

\Remarques : une chaine de caract\`eres est constitu\'ee de caract\`eres accol\'es les uns aux autres entre deux guillemets \touche{"} et \touche{"}.
\smallskip\noindent
Une liste est une s\'equence entre crochets \touche{[} et \touche{]}. 
\smallskip\noindent
Un ensemble est une s\'equence entre accolades \touche{$\{$} et \touche{$\}$}, dont les \'el\'ements sont distincts deux \`a deux et ne sont pas ordonn\'es (comme en Maths).  
\smallskip\noindent
Un nom Maple commence par une lettre et peut contennir des lettres, des chiffres, des underscores.  Il peut \'eventuellement \^etre entre deux apostrophes \touche{$'$} et \touche{$'$}. 

\Secti Maple2, Conversion. 

\Section Conversion de s\'equence

$$\vbox{
\offinterlineskip
\halign{
\vrule#&\quad\hfil#\hfil\quad&\vrule#&\quad\hfil$\vcenter to1.3em{}$#\hfil\quad&\vrule#\tabskip=0pt\cr
\noalign{\hrule}\cr
&Conversion&&Commande&\cr
\noalign{\hrule}\cr
&S\'equence$\Rightarrow$Liste&& $[$Sequence$]$&\cr
\noalign{\hrule}\cr
&S\'equence$\rightarrow$Ensemble&& $\{$S\'equence$\}$&\cr
\noalign{\hrule}\cr
}}
$$

\Remarque : j'emploie le symbole $\Rightarrow$ pour d\'ecrire les conversions sans perte d'information et le symbole $\rightarrow$ pour les autres : 
lors d'une conversion vers le type ensemble, il y a eventuellement perte d'information : ordre et occurences...
\bigskip

\Section Conversion en s\'equence

$$
\vbox{
\offinterlineskip
\halign{
\vrule#&\quad\hfil#\hfil\quad&\vrule#&\quad\hfil$\vcenter to1.3em{}$#\hfil\quad&\vrule#\tabskip=0pt\cr
\noalign{\hrule}\cr
&Conversion&&Commande&\cr
\noalign{\hrule}\cr
&Liste$\Rightarrow$S\'equence&& op({\it Liste})&\cr
\noalign{\hrule}\cr
&Ensemble$\Rightarrow$S\'equence&& op({\it Ensemble})&\cr
\noalign{\hrule}\cr
}}
$$


Les autres conversions s'en d\'eduisent naturellement par composition, en faisant un d\'etour par les s\'equences. 
\bigskip

$$\vbox{
\offinterlineskip
\halign{
\vrule#&\quad\hfil#\hfil\quad&\vrule#&\quad\hfil$\vcenter to1.3em{}$#\hfil\quad&\vrule#\tabskip=0pt\cr
\noalign{\hrule}\cr
&Conversion&&Commande&\cr
\noalign{\hrule}\cr
&Liste$\rightarrow$Ensemble&& $\{$op({\it Liste})$\}$&\cr
\noalign{\hrule}\cr
&Ensemble$\Rightarrow$Liste&& [op({\it Liste})]&\cr
\noalign{\hrule}\cr
}}
$$


Consid\'erons la s\'equence $S$, la liste $L$ et l'ensemble $E$ suivants
$$
\eqalign{
&\hskip-14em
\red S:=1,3,3,7\ ;\ \red L:= [6,6,1,6,1]\ ;\  \red E:=\{7,8,9\}\ ;
\cr
&\blue S:=1,3,3,7
\cr
&\blue L:=[6,6,1,6,1]
\cr&\blue E:=\{7,8,9\}\cr
&\hskip-14em\red op(E);
\cr
&\blue 7,8,9
\cr
&\hskip-14em
\red [S];
\cr
&\blue [1,3,3,7]
\cr
&\hskip-14em\red \{S\};
\cr
&\blue \{1,3,7\}
\cr
&\hskip-14em\red \{op(L)\};
\cr
&\blue \{1,6\}
}
$$

\Secti Maple2, Op\'erations. 

\Section Op\'erations sur les ensembles

Comme en math\'ematiques, les op\'erations suivantes sont peuvent \^etre \'eefectu\'ees sur des objets de type ensemble. 

$$\vbox{
\offinterlineskip
\halign{
\vrule#&\quad\hfil#\hfil\strut\quad&\vrule#&\quad\hfil#\hfil\quad&\vrule#&\quad\hfil#\hfil\quad&\vrule#\tabskip=0pt\cr
\noalign{\hrule}\cr
&Math\'ematiques&&Maple&&Description&\cr
\noalign{\hrule}\cr
&$A\cup B$&&A union B&&la r\'eunion des ensembles $A$ et $B$&\cr
\noalign{\hrule}\cr
&$A\cap B$&&A intersect B&&l'intersection des ensembles $A$ et $B$&\cr
\noalign{\hrule}\cr
&$A\ssm B$&&A minus B&&L'ensemble $A$ priv\'e de $B$&\cr
\noalign{\hrule}\cr}}
$$

$$
\eqalign{
&\hskip-14em
\red A:=\{1,3,7\}\ ;\ \red B:= \{1,6\}\ ;\ 
\cr
&\blue A:=\{1,3,7\}
\cr
&\blue B:=\{1,6\}
\cr
&\hskip-14em\red A\hbox{ union }B;
\cr
&\blue \{1,3,6,7\}
\cr
&\hskip-14em
\red A\hbox{ intersect }B;
\cr
&\blue \{1\}
\cr
&\hskip-14em\red A\hbox{ minus }B;
\cr
&\blue \{3,7\}
}
$$

\Section Op\'erations sur les s\'equences

La s\'equence vide se note {\it NULL}. 
$$
\eqalign{
&\hskip-14em
\red S:=NULL\ ; 
\cr
&\blue S:=
}
$$

$$\vbox{
\offinterlineskip
\halign{
\vrule#&\quad\hfil#\hfil\strut\quad&\vrule#&\quad\hfil#\hfil\quad&\vrule#&\quad\hfil#\hfil\quad&\vrule#\tabskip=0pt\cr
\noalign{\hrule}\cr
&Math\'ematiques&&Maple&&Description&\cr
\noalign{\hrule}\cr
&$f(1),f(2),\cdots,f(n)$&& seq(f(k),k=1..n)&&la s\'equence des valeurs de $f(k)$ pour $1\le k\le n$&\cr
\noalign{\hrule}\cr
&$\underbrace{x,x,x,x,x...,x}_n$&& x\$n&&la s\'equence consitu\'ee de n occurences de $x$. &\cr
\noalign{\hrule}\cr
&$\underbrace{1,2,3}_S,\underbrace{4,5}_T$&& S,T&&La concat\'enation des s\'equence $S$ et $T$&\cr
\noalign{\hrule}\cr}}
$$


$$
\eqalign{
&\hskip-6em
\red f:=x-> x\hat{\quad}2\ ;  
\cr
&\hskip8em\blue f:=x\rightarrow x^2
\cr
&\hskip-6em\red S:=seq(f(k),k=2..4)\ ;
\cr
&\hskip8em\blue S:=4,9,16
\cr
&\hskip-6em\red 666, S, Warf\$3, 667\ ;
\cr
&\hskip8em\blue 666,4,9,16,Warf, Warf, Warf, 667
}
$$

\Section Extraction d'\'el\'ements


Les op\'erations suivantes marchent pour listes, s\'equences et ensembles. La syntaxe est donn\'ee pour les s\'equences. 

$$\vbox{
\offinterlineskip
\halign{
\vrule#&\quad\hfil#\hfil\strut\quad&\vrule#&\quad\hfil#\hfil\quad&\vrule#&\quad\hfil#\hfil\quad&\vrule#\tabskip=0pt\cr
\noalign{\hrule}\cr
&Math\'ematiques&&Maple&&Description&\cr
\noalign{\hrule}\cr
&$L:=x_1,x_2,\cdots,\underbrace{x_i}_{\llap{\sevenrm\'el\'ement retourn\'e}},\cdots,x_n$&& L[i]&&
donne le $i^\ieme$ \'el\'ement&\cr
\noalign{\hrule}\cr
&$L:=x_{-n},\cdots,\underbrace{x_{-i}}_{\llap{\sevenrm\'el\'ement retourn\'e}},\cdots,x_{-1}$&&L[-i]&&
donne le $i^\ieme$ \'el\'ement en partant de la fin. &\cr
\noalign{\hrule}\cr
&$L:=x_1,\cdots,\underbrace{x_a,\cdots,x_b}_{\llap{\sevenrm\'el\'ements retourn\'es}},\cdots,x_n$&& L[a..b]&&donne les \'el\'ements des rangs $a$ \`a $b$. &\cr
\noalign{\hrule}\cr
}}
$$

$$
\eqalign{
&\hskip-6em
\red f:=x-> x\hat{\quad}2\ ;  
\cr
&\hskip8em\blue f:=x\rightarrow x^2
\cr
&\hskip-6em\red S:=seq(f(k),k=1..10)\ ;
\cr
&\hskip8em\blue S:=1,4,9,16,25,36,49,64,81,100
\cr
&\hskip-6em\red S[3]\ ;\ S[-2]\ ; S[4..-3]\ ; 
\cr
&\hskip8em\blue 9
\cr
&\hskip8em\blue 81
\cr
&\hskip8em\blue 16,25,36,49,64
}
$$

On peut \'egalement extraire en cascade des \'el\'ements de listes/s\'equences/ensembles imbriqu\'es en extrayant de fa\c con successive \`a l'aide de la syntaxe $L[i][j]$ ou $L[i,j]$, etc... 
$$
\eqalign{
&\hskip-2em
\red L:=[1,2,[31,32,33,[341,342,343],35,36],4,5,6,[71,72]]\ ;  
\cr
&\hskip8em\blue L:=[1,2,[31,32,33,[341,342,343],35,36],4,5,6,[71,72]]\ ;  
\cr
&\hskip-2em\red L[-1]\ ;
\cr
&\hskip8em\blue [71,72]
\cr
&\hskip-2em\red L[-1][2]\ ; 
\cr
&\hskip8em\blue 72
\cr
&\hskip-2em\red L[3][4][2]\ ; L[3,4,2]\ \;
\cr
&\hskip8em\blue 342
\cr
&\hskip8em\blue 342
}
$$


\Secti Mapple2, Programmation. 


\Section boucle {\it for}

Pour effectuer des taches/calculs r\'ep\'etitifs, on peut utilisier la syntaxe suivante
$$
\eqalign{
&\touche{for}\ \overbrace{\hbox{ \it i }}^{\llap{compteur}}\ \touche{ from }\ \hbox{ \it d\'ebut }\ \touche{ to }\ \hbox{ \it fin }\ \overbrace{\touche{by}\ \hbox{ \it incr\'ement }}^{\llap{facultatif}}\ \touche{do}
\cr
&\hbox{\it commandes}
\cr
&\touche{od :}
\cr
}
$$

$$
\eqalign{
&\hskip-12em
\red > s:=0; \cr
&\hskip-12em\red
> \hbox{for }a\hbox{ from }1\hbox{ to }3\hbox{ do}\cr
&\hskip-12em
\red \quad s:=s+a\ ;
\cr
&\hskip-12em\red\quad \hbox{od\ ;}
\cr
&\hskip8em\blue s:=0
\cr
&\hskip8em\blue s:=1
\cr
&\hskip8em\blue s:=3
\cr
&\hskip8em\blue s:=6
}
$$

Une syntaxe alternative est 
$$
\eqalign{
&\touche{for}\ \overbrace{\hbox{ \it i }}^{\llap{compteur}}\ \touche{ in }\ \hbox{ \it liste }\ \touche{do}
\cr
&\hbox{\it commandes}
\cr
&\touche{od :}
\cr
}
$$
$$
\eqalign{
&\hskip-12em
\red > s:=0; L:=[2,4,9]\ ;\cr
&\hskip-12em\red
> \hbox{for }a\hbox{ in }L\hbox{ do}\cr
&\hskip-12em
\red \quad s:=s+a\ ;
\cr
&\hskip-12em\red\quad \hbox{od\ ;}
\cr
&\hskip8em\blue s:=0
\cr
&\hskip8em\blue L:=[2,4,9]
\cr
&\hskip8em\blue s:=2
\cr
&\hskip8em\blue s:=4
\cr
&\hskip8em\blue s:=9
}
$$


\Section boucle {\it for}



\vfill\null
\eject
\centerline{\fourteenbf TD Maple n$^\circ 2$}
\bigskip


\noindent{\bf Exercice 1. }Le but de cet exercice est d'apprendre � manipuler liste, ensemble et  s\'equences pour tapper le moins de caract\`eres possibles 
en entrant les donn\'ees dans Maple.   \pn
a) d\'efinir la s\'equence 
$$
S_0:=1,1,2,2,2,2,3,3,3,3,3,3,4,\cdots,8, 9,9,9,9,9,9,9,9,9,9,9,9,9,9,9,9,9,9
$$ 
a) Quel est le $59^\ieme$ \'el\'ement de cette liste ? \pn
b) Convertir la s\'equence $S_0$ en ensemble $E_0$. Observation ? \pn
c) D\'efinir la liste 
$$
L_1:=[1,3,5,7,9,11,\cdots, 255].
$$ 
d) Convertir la liste $L_1$ en ensemble $E_1$. \pn
e) Extraire de $L_1$ la liste $L_2$ allant de $15$ \`a $251$ et convertir $L_2$ en ensemble $E_2$. \pn
f) Calculer $E_0\cap E_1$ et $E_0\cup(E_1\ssm E_2)$. \pn
g) Fabriquer la liste $L_2:=[1,9,25,\cdots,255^2]$ constitu\'ee du carr\'e des \'el\'ements de $L_1$. On pourra utiliser {\it map}. 
\medskip

\noindent{\bf Exercice 2. }a) Soit $n$ un entier positif quelconque. Ecrire une boucle {\it for} qui calcule $n!$. \pn
b) En modifiant l\'eg\'erement le programme pr\'ec\'edent, fabriquer une fonction/proc\'edure ``fact'' qui mange un nombre entier positif $n$ et retourne $n!$. 
\medskip

\noindent{\bf Exercice 3. }a) Ecrire une boucle  qui, \'etant donn\'ee une liste quelconque $L$ de nombres calcule leur somme (respectivement leur produit). 
Au besoin, on pourra calculer le nombre d'\'el\'ements de la liste avec {\it nops}. \pn
b) En modifiant l\'eg\`erement le programme pr\'ec\'edent, fabriquer une fonction/procedure quiu calcule la somme (resp. le produit) des nombres d'une liste quelconque $L$. 
\medskip

\noindent{\bf Exercice 4. }Calculer la double somme suivante 
$$
\sum_{k=3}^{15}\sum_{\ell=-1}^7{k\F k+\ell}
$$
d'abord en utilisant deux sommes  {\it sum} puis en utilisant deux boucles {\it for} imbriqu\'ees. 
\medskip
 
\noindent{\bf Exercice 5. }On consid\`ere la suite de Fibonnacci, qui est d\'efinie par les conditions initiales $F_0:=0$, $F(1):=1$ et par la relation de r\'ecurence
$$
\forall n\ge2, \qquad F(n):=F(n-1)+F(n-2)
$$
a) Ecrire une procedure r\'ecursive ``F'' qui mange un entier $n\ge0$ et qui retourne $F(n)$. \pn
b) Calculer $F(25)$. Votre programme est lent, pourquoi ? 
\bigskip

\noindent
{\bf Exercice 6. }Le crible d'\'Eratosth\`ene. C'est une m\'ethode permettant de d\'eterminer quels sont les nombres premiers parmi une liste d'entiers $L$ allant de $2$ \`a $N$. \medskip
\noindent
Etape 1. On initialise la liste $L$ des entiers de $2$ \`a  $N$ en leur octroyant la valeur ``{\it true}''. \pn
Etape 2. On raye de la liste les multiples de $2$ en leur octroyant la valeur ``{\it false}''. \pn
Etape 3. On raye de la liste les multiples de $3$ en leur octroyant la valeur ``{\it false}''. \pn\qquad 
\vdots\pn
Etape $P$. On raye de la liste les multiples de $P$ en leur octroyant la valeur ``{\it false}''. \pn
\bigskip
\noindent
a) Pourquoi la derni\`ere \'etape utile $P$ v\'erifie-t-elle $2\le P\le n/2$ ? \pn
b) Ecrire le programme complet. \pn
c) En d\'eduire la liste des nombres premiers inf\'erieurs \`a $1000$ (avec {\it select}). 
 \bigskip

\bye