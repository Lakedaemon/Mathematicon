\catcode`@=11\relax
\def\Api{Mathematicon@Api}%
\input LD@Header.tex
\input LD@Library.tex
\input LD@Typesetting.tex
%\input LD.tex

\DefineRGBcolor F0F9E3=VLGreen.
\DefineRGBcolor E5F9D1=LGreen.
\DefineRGBcolor DAF9BE=TGreen.
\DefineRGBcolor 5DA93B=Green.
\DefineRGBcolor F6DCCA=VLRed.
\DefineRGBcolor F6D4BD=LRed.
\DefineRGBcolor DAF9BE=TRed.
\DefineRGBcolor B5F9A1=TTRed.
\DefineRGBcolor F6B080=Red.
\DefineRGBcolor F9F5E3=VLOrange.
\DefineRGBcolor F9F5D0=LOrange.
\DefineRGBcolor DAF9BE=TOrange.
\DefineRGBcolor B5F9A1=TTOrange.
\DefineRGBcolor D7A93B=Orange.
\DefineRGBcolor EEEEEE=VLBlack.
\DefineRGBcolor DDDDDD=LBlack.
\DefineRGBcolor CCCCCC=TBlack.
\DefineRGBcolor B5F9A1=TTBlack.
\DefineRGBcolor 000000=Black.

%\DefineRGBcolor 000000=Green.
%\definecolor{ColorVLGreen}{rgb}{1,1,1}%
%\definecolor{ColorLGreen}{rgb}{1,1,1}%
%\definecolor{ColorTGreen}{rgb}{1,1,1}%
%\expandafter\definecolor\temp
%\DefineRGBcolor 000000=Red.
%\definecolor{ColorVLRed}{rgb}{1,1,1}%
%\definecolor{ColorLRed}{rgb}{1,1,1}%
%\definecolor{ColorTRed}{rgb}{1,1,1}%
\def\Students{%
	\DefineRGBcolor FFFFFF=VLGreen.
	\DefineRGBcolor FFFFFF=LGreen.
	\DefineRGBcolor FFFFFF=TGreen.
	\DefineRGBcolor 000000=Green.
	\DefineRGBcolor FFFFFF=VLRed.
	\DefineRGBcolor FFFFFF=LRed.
	\DefineRGBcolor FFFFFF=TRed.
	\DefineRGBcolor FFFFFF=TTRed.
	\DefineRGBcolor 000000=Red.
	\DefineRGBcolor FFFFFF=VLOrange.
	\DefineRGBcolor FFFFFF=LOrange.
	\DefineRGBcolor FFFFFF=TOrange.
	\DefineRGBcolor FFFFFF=TTOrange.
	\DefineRGBcolor 000000=Orange.
	\DefineRGBcolor FFFFFFF=VLBlack.
	\DefineRGBcolor FFFFFF=LBlack.
	\DefineRGBcolor FFFFFF=TBlack.
	\DefineRGBcolor FFFFFF=TTBlack.
	\DefineRGBcolor 000000=Black.
}
\Students

\catcode`@=11\relax
\input LD@Exercices

%%%%%%%%%%%%%%%%%%%%%%%%%%%%%%%%%%%%%%%%%%%%%%%%%%%%%%%%%%%%%%%%%%
%															%
%						Revisions : 1er jour de spé						%
%															%
%%%%%%%%%%%%%%%%%%%%%%%%%%%%%%%%%%%%%%%%%%%%%%%%%%%%%%%%%%%%%%%%%%

\vglue-10mm\rightline{PT\hfill R\'evision 1 :  Alg\`ebre lin\'eaire\hfill}
\bigskip
\bigskip
Le but de ces exercices est de faire r\'eviser les bases et les techniques sp\'ecifiques de l'alg\`ebre lin\'eaire, avant d'aborder d\'eterminant, \'el\'ements propres et r\'eduction.
\bigskip
\hrule
\centerline{Rang\strut}
\hrule\medskip
%\Exercice{PTSIwv}%

%\Exercice{PTSIww}%

%\Exercice{PTSIwx}%

\Exercice{PTSIwy}%

%\Exercice{PTSIxf}%

\Exercice{PTSIwz}%

\Methode [Pour calculer le rang d'une matrice $A$]
Utiliser $\underline{\hbox{\bf sur les lignes et les colonnes}}$ les op\'erations \'el\'ementaires, $\underline{\hbox{\bf qui ne changent pas~le~rang}}$ :\pn
a)  $L_i\leftrightarrow L_j$ : \'echanger la ligne $i$ avec la ligne $j$,\pn
b)  $L_i\leftarrow \lambda L_i$ : multiplier la ligne $i$ par un scalaire non nul $\lambda$,\pn
c)  $L_i\leftarrow Li+\sum_{j\neq i}\lambda_j L_j$ : ajouter \` a  la ligne $i$ une combinaison lin\'eaire des autres lignes, \pn
jusqu'\`a ce que tous les coefficients non-nuls soient isol\'es sur leur ligne et leur~colonne. \pn
Le rang de la matrice de d\'epart est alors le nombre de coefficients non-nuls restant \`a la fin. 

\Conseil : Une bonne strat\'egie consiste \`a utiliser le pivot de Gauss pour faire appara\^\i tre des z\'eros, en cherchant \`a rendre la matrice triangulaire (cela \'evite de tourner en rond)...

\Exercice{PTSIxe}%

\Rappel : Le rang d'une famille $\{a,b,c,d\}$ de vecteurs est le rang de l'espace $\hbox{Vect}(a,b,c,d)$ qu'ils engendrent. C'est aussi le rang de la matrice $\pmatrix{a|b|c|d}$ des vecteurs mis c\^ote \`a c\^ote. 

\hrule
\centerline{D\'eterminant\strut}
\hrule\medskip
\Exercice{PTjn}%

\Exercice{PTSIxa}%

\Methode[Pour calculer le determinant d'une matrice $A$]
Se ramener aux cas simples suivants : \medskip\noindent
1) Le d\'eterminant d'une matrice triangulaire est le produit des termes diagonaux, \pn 
2) Le det. d'une matrice triangulaire par blocs est le produit des det. des blocs diagonaux. 

\Methode[Pour simplifier le determinant d'une matrice $A$]
Utiliser sur les lignes et/ou sur les colonnes les op\'erations \'el\'ementaires a,b et c, qui ont les effets suivants : \medskip\noindent
a)  $L_i\leftrightarrow L_j$ : multiplie le determinant par $-1$,\pn
b)  $L_i\leftarrow \lambda L_i$ : multiplie le determinant par $\lambda$. Plus simplement, si vous pouvez factoriser un nombre $\lambda$ sur une ligne, vous pouvez le sortir du determinant. \pn
c)  $L_i\leftarrow Li+\sum_{j\neq i}\lambda_j L_j$ : ne change pas le d\'eterminant. 

\hrule
\centerline{Inverse\strut}
\hrule\medskip

\Exercice{PTSIod}%

\Exercice{PTSIxc}%


\Methode[Pour inverser une matrice $A${,} via l'algorithme $A|I_n\to I_n|A^{-1}$]
Ecrire $A|I_n$ puis effectuer {\bf sur les  lignes} des deux matrices les op\'erations \'el\'ementaires a,b et c. 
Le but est d'obtenir la matrice $I_n$ \`a gauche pour r\'ecup\'erer la matrice $A^{-1}$ \`a droite.

\Exercice{PTSIwu}%
\medskip

\Conseil : Retenez la technique pr\'ec\'edente. Cela reviendra dans les exercices et c'est une fa\c con agr\'eable et peu calculatoire de savoir si une matrice est inversible et, le cas \'ech\'eant, d'en trouver un inverse. 
\medskip



\hrule
\centerline{Sous-espace vectoriel\strut}
\hrule\medskip

\Exercice{PTaim}%
\medskip

\Exercice{PTSIwt}%


\Methode [Pour prouver que $E$ est un $\ob K$-espace vectoriel]
Montrer que c'est un sous-espace vectoriel d'un $\ob K$-espace vectoriel du cours : \pn
1) Prouver que $E\neq\emptyset$, \pn
2) Prouver que $E$ est inclus dans un $\ob K$-espace vectoriel de r\'ef\'erence, \`a d\'eterminer. \pn
3) Prouver que $\lambda x+\mu y\in E$ pour $(\lambda,\mu)\in \ob K^2$ et $(x,y)\in E^2$. 


\hrule
\centerline{Familles libres, g\'en\'eratrices, bases\strut}
\hrule\medskip

\Exercice{PTSIxb}%

\Conseil :  Lorsque vous \'etudiez un espace vectoriel, essayez de trouver sa dimension le plus t\^ot possible. Souvent, cela vous aidera pour la suite (utilisation du th\'eor\`eme du rang, etc...). 

\Exercice{PTaip}%
\medskip

\Exercice{PTSIkq}%

\Exercice{PTps}%


\Methode [Pour prouver que deux ensembles $E$ et $F$ sont \'egaux]
Proc\`eder par double-inclusion : \pn
1) Prouver que $E\subset F$, \qquad\qquad
2) Prouver que $F\subset E$.

\Methode [Pour prouver que deux $\ob K$-espaces vectoriels de dim finie sont \'egaux]
Etablir une inclusion ainsi que l'\'egalit\'e~des~dimensions : \pn
1) Prouver que $E\subset F$ (par exemple)\qquad\qquad 2) Prouver que $\dim E=\dim F$. 

\hrule
\centerline{Technique diabolique (de r\'esolution des exos durs sans refl\'echir)\strut}
\hrule\medskip

\Exercice{PTais}%
\medskip
\Exercice{PTaio}%

\Methode [Pour prouver une \'equivalence {$A\Longleftrightarrow B$}]
Proc\`eder par double-implication : \pn
1) Prouver que $A\Longrightarrow B$, \qquad\qquad
2) Prouver que $A\Longleftarrow B$.

\Methode [Pour prouver une implication {$A\Longrightarrow B$}]
Supposer que $A$ est vrai et \'etablir que~$B$ est vrai. 


\Exercice{PTSIol}%


\Methode [Pour \'ecrire la matrice  de $u$ dans les bases {$\sc E=\{e_1, \cdots, e_n\}$ et $\sc F=\{f_1, \cdots, f_p\}$}]
$$
\vtop{\hsize 9cm\noindent 1) Dessiner la matrice vide avec ses donn\'ees comme ceci : \pn 2) Completer en pla\c cant dans la $j^{\hbox{\sevenrm i\`eme}}$ colonne, les~coefficients de la d\'e\-com\-po\-si\-tion de 
l'image $u(e_j)$ du vecteur~$e_j$ par $u$ sur la base d'arriv\'ee $\{f_1, \cdots, f_p\}$ }\quad\raise-0.3cm\hbox{$\sc Mat_{\sc E, \sc F}(u)=$}{e_1\ \cdots\ e_n\quad\atop \pmatrix{&&&&&\cr&&&&&\cr&&&&&\cr&&&&&}\matrix{f_1\cr \vdots\cr f_p}}
$$

\Exercice{PTSIvw}%


\Methode [Pour prouver que l'application $u:E\to F$ est $\ob K$-lin\'eaire]
Prouver que l'image d'une combinaison lin\'eaire est la combinaison lin\'eaire des images : \pn
1) Prouver que $E$ et $F$ sont des $\ob K$-espaces vectoriels,\pn
2) Prouver que $u(\lambda x+\mu y)=\lambda u(x)+\mu u(y)$ pour $(\lambda,\mu)\in \ob K^2$ et $(x,y)\in E^2$. 



\Exercice{PTain}%


\Methode [Pour savoir si une application lin\'eaire $f:E\to F$ est  injective]
Regarder son noyau
$$
f\sbox{ injective}\ssi \Ker f=\{0\}.
$$

\Exercice{PTaiq}%


\Methode [Pour savoir si une application {\bf lin\'eaire} $f:E\to F$ est  surjective]
Si $E$ et $F$ sont de m\^eme dimension {\bf finie}, se ramener \`a une \'etude d'injectivit\'e  : \pn
$$
f\sbox{  injective }\ssi f\sbox{ surjective }\ssi f \sbox{ bijective.}
$$ 
Si $E$ est de dimension finie diff\'erente de celle de $F$, utiliser le th\'eor\`eme du rang et calculer la dimension du noyau de $f$
$$
\dim E=\dim\Ima f+\dim\Ker f.
$$
Si $\dim E=+\infty$, essayer de se ramener \`a la dimension finie ou utiliser  une autre m\'ethode. 

\Methode [Pour savoir si une application $f:E\to F$ est  injective (resp. surjective{,} bijective)] 
Fixer $y$ dans $F$ et regarder si l'\'equation $y=f(x)$ a au plus (resp au moins, resp. exactement) une solution $x$ dans $E$. 




\hrule
\centerline{R\'ecurrences lin\'eaires\strut}%
\hrule
\medskip
%\Exercice{PTSIoi}%

\Exercice{PTaib}%

\Exercice{PTSImd}%

\Exercice{PTSIme}%

\hrule
\centerline{Calcul matriciel et R\'ecurrence\strut}%
\hrule
\medskip

\Exercice{PTSIvx}%

\bye











