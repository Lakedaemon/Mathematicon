%\def\Variables{MathsVariables}%
\catcode`@=11\relax
\def\Api{Mathematicon@Api}%
%%%% Newif
%\def\@firstofone#1{#1}

\newif\ifexonumber
%%%% Switches
\exonumberfalse
\catcode`@=11\relax
\input LD@Header.tex
\input LD@Library.tex
\input LD@Typesetting.tex
\input LD@Exercices.tex

\def\LD@Exercice@Display@Code{\eightpts}%
% debug tikz
\let\@firstofone\pgfutil@firstofone
\let\@ifnextchar\pgfutil@ifnextchar

\olspept
\DefineRGBcolor F0F9E3=VLGreen.
\DefineRGBcolor E5F9D1=LGreen.
\DefineRGBcolor DAF9BE=TGreen.
\DefineRGBcolor 5DA93B=Green.
\DefineRGBcolor F6DCCA=VLRed.
\DefineRGBcolor F6D4BD=LRed.
\DefineRGBcolor DAF9BE=TRed.
\DefineRGBcolor B5F9A1=TTRed.
\DefineRGBcolor F6B080=Red.
\DefineRGBcolor F9F5E3=VLOrange.
\DefineRGBcolor F9F5D0=LOrange.
\DefineRGBcolor DAF9BE=TOrange.
\DefineRGBcolor B5F9A1=TTOrange.
\DefineRGBcolor D7A93B=Orange.
\DefineRGBcolor EEEEEE=VLBlack.
\DefineRGBcolor DDDDDD=LBlack.
\DefineRGBcolor CCCCCC=TBlack.
\DefineRGBcolor B5F9A1=TTBlack.
\DefineRGBcolor 000000=Black.

\def\transparent{%
	\CS long\EC\def\Demonstration##1\CQFD{}%
}%
%\transparent
\def\Students{%
	\DefineRGBcolor FFFFFF=VLGreen.
	\DefineRGBcolor FFFFFF=LGreen.
	\DefineRGBcolor FFFFFF=TGreen.
	\DefineRGBcolor 000000=Green.
	\DefineRGBcolor FFFFFF=VLRed.
	\DefineRGBcolor FFFFFF=LRed.
	\DefineRGBcolor FFFFFF=TRed.
	\DefineRGBcolor FFFFFF=TTRed.
	\DefineRGBcolor 000000=Red.
	\DefineRGBcolor FFFFFF=VLOrange.
	\DefineRGBcolor FFFFFF=LOrange.
	\DefineRGBcolor FFFFFF=TOrange.
	\DefineRGBcolor FFFFFF=TTOrange.
	\DefineRGBcolor 000000=Orange.
	\DefineRGBcolor FFFFFFF=VLBlack.
	\DefineRGBcolor FFFFFF=LBlack.
	\DefineRGBcolor FFFFFF=TBlack.
	\DefineRGBcolor FFFFFF=TTBlack.
	\DefineRGBcolor 000000=Black.
}
\Students
\def\red{}
\def\blue{}
\def\Red#1{#1}%%%% Fix this !
\def\Blue#1{#1}%
\def\Font #1@#2pt{\font\olbi=cmr10\olbi}
\font\SvgText=cmr10\relax
%
%
%\catcode`@=11\relax
%\def\Api{Mathematicon@Api}%
%
%\input LD@Header.tex
%\input LD.tex
%\input LD@Exercices.tex
%\input LD@Typesetting.tex
%
%\catcode`@=11\relax
\font\LD@Font@Arial="Arial" at 10pt
%%%%%%%%%%%%%%%%%%%%%%%%%%%%%%%%%%%%%%%%%%%%%%%%%%%%%%%%%%%%%%%%%%
%															%
%						Métrique des courbes							%
%															%
%%%%%%%%%%%%%%%%%%%%%%%%%%%%%%%%%%%%%%%%%%%%%%%%%%%%%%%%%%%%%%%%%%
\newcount\LD@Count@Temp
\def\LD@Exercice@Display@Code{}%%\LD@Option@@Label\qquad\eightpts}%
\def\LD@Exercice@Display@Code@Post{%
	\ifcsname LD@Exo@@Solution\endcsname
		\unless\ifx\LD@Exo@@Solution\LD@Empty
			\pn{\eightpts Solution : \eightpts \LD@Exo@@Solution}%
		\fi
	\fi
}%
\def\LD@Display#1{%
	\LD@Count@Temp=#1\relax
	\ifcase\LD@Count@Temp
	\or
	Math. Sup.
	\or
	Math. Sp\'e
	\else
	\fi
}%
\newcount\LD@Exo@Total\LD@Exo@Total=0\relax

%%% TD 9.
\vglue-10mm\rightline{Sp\'e PT\hfill TD 9 : M\'etriques des courbes\hfill}%\date}
\bigskip


\centerline{\fourteenbf Abscisse curviligne}
\bigskip


\Definition [$f(t)=\vec{OM}(t)$ arc param\'etr\'e de classe $\sc C^1$ sur un intervalle $I$]
L'abscisse curviligne du point $M(t)$ est l'int\'egrale  
$$
s(t)=\int_{t_0}^t\Q|\!\Q|f'(u)\W|\!\W|\d u=\int_{t_0}^t\Q|\!\Q|{\d\vec{OM}(u)\F \d u}\W|\!\W|\d u\qquad (t\in I),  
$$
l'origine de l'abscisse curviligne \'etant le point $M(t_0)$. 

\Exercice{PTaai}%
\bigskip

\centerline{\fourteenbf Rep\`ere de Frenet et  courbure dans le plan}
\bigskip

\Definition [arc param\'etr\'e $f(t)=\vec{OM}(t)$]\noindent
1) calculer $f'(t)=\vec v={\d\vec{OM}\F\d t}$ et en d\'eduire le vecteur unitaire tangent
$$
\vec T:={f'(t)\F\Norme{f'(t)}}={\vec v\F\Norme{\vec v}}.
$$
2) Le rep\`ere de Frenet en $M$ est le rep\`ere orthonorm\'e direct $(M,\vec T,\vec N)$. On obtient le vecteur unitaire normal $\vec N$ en faisant pivoter $\vec T$ de $+90^\circ$ de la façon suivante :
$$
\hbox{si }\quad\vec T=\pmatrix{a\cr b}a\vec i+b\vec j\quad\hbox{ alors }\quad\vec N=\pmatrix{-b\cr a}=-b\vec i+a\vec j.
$$
3) calculer $f''(t)=\vec a={\d^2\vec{OM}\F\d t^2}$ et en d\'eduire la courbure
$$
\gamma={f''(t)\F\Norme{f'(t)}^2}.\vec N={\vec a\F\Norme{\vec v}^2}.\vec N
$$
4) Si $\gamma\neq0$, en d\'eduire le rayon de courbure 
$$
r_c={1\F\gamma}.
$$
5) D\'eduire le centre de courbure $C$ de la relation $\vec{MC}=r_c\vec N$ en \'ecrivant 
$$
\vec{OC}=\vec{OM}+r_c\vec N.
$$

\Exercice{PTaag}%
\bigskip

%\Exercice{PTach}%
%\bigskip

\centerline{\fourteenbf Rep\`ere de Frenet, courbure et torsion dans l'espace}
\bigskip

\Definition [arc param\'etr\'e $f(t)=\vec{OM}(t)$]\noindent
1) calculer $f'(t)=\vec v={\d\vec{OM}\F\d t}$ et en d\'eduire le vecteur unitaire tangent
$$
\vec T:={f'(t)\F\Norme{f'(t)}}={\vec v\F\Norme{\vec v}}.
$$
2) calculer $f''(t)=\vec a={\d^2\vec{OM}\F\d t^2}$ et en d\'eduire le vecteur unitaire binormal 
$$
\vec B={f'(t)\wedge f''(t)\F\|f'(t)\wedge f''(t)\|}={\vec v\wedge \vec a\F\Norme{\vec v\wedge\vec a}}
$$
puis le vecteur unitaire normal (principal) $\vec N=\vec B\wedge\vec T$. Le rep\`ere de frenet est le rep\`ere orthonorm\'e direct $(M,\vec T,\vec N,\vec B)$. \pn
3) En d\'eduire la courbure, le rayon de courbure et le centre courbure avec 
$$
\gamma={f''(t)\F\Norme{f'(t)}^2}.\vec N={\vec a\F\Norme{\vec v}^2}.\vec N, \qquad r_c={1\F\gamma}\quad\hbox{et}\quad \vec{MC}=r_c\vec N
$$
On pourra aussi utiliser la formule
$$
\gamma={\|f'(t)\wedge f''(t)\|\F \|f'(t)\|^3}={\vec v\wedge\vec a\F \Norme{\vec v\Norme}^3}. 
$$
4) calculer  $f'''(t)=\vec a'={\d^3\vec{OM}\F\d t^3}$ et en d\'eduire la torsion et  le rayon de torsion 
$$
\tau={\det\b(f'(t),f''(t),f'''(t)\b)\F\|f'(t)\wedge f''(t)\|^2}\qquad r_t={1\F \tau}.
$$

\Exercice{PTaam}%
\bigskip

%\Exercice{PTacd}%
%\bigskip

\centerline{\fourteenbf Rep\`ere de Frenet dans le plan en polaire}
\bigskip

\Exercice{PTaaj}%
\bigskip


\Exercice{PTaak}%
\bigskip

\bye