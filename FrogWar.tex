%\magnification 1200
%\hsize210truemm\vsize 297truemm\hoffset=50truemm\voffset=0truemm
%\pretolerance=500\tolerance=1000\brokenpenalty=5000
%\parindent3mm

%\input color.txs
%\input pstricks
%\input pst-plot
%\input eplaingt
%\input Macrols
%\input ExosPTSI

\catcode`@=11\relax

\input LD@Header.tex
\input LD@Library.tex
\input LD@Typesetting.tex
\catcode`@=11\relax
\font\LD@Font@Arial="Arial" at 10pt
\font\LD@Font@Arial@Italic="Arial Italic" at 10pt
\font\LD@Font@Big@Arial="Arial Bold" at 16pt
\font\LD@Font@Big@Arial@Italic="Arial Italic" at 13pt

\hsize180truemm\vsize 270truemm\hoffset=-10truemm\voffset=-10truemm
\pretolerance=500\tolerance=1000\brokenpenalty=5000
\parindent3mm
\overfullrule=0pt

\font\fourbb=msbm10 at 4pt
\font\fivemsa=msam5
\font\fourex=cmex10 at 4pt

\null
\vfill\bigskip\bigskip\bigskip
\twocolumns
\centerline{\LD@Font@Big@Arial La guerre des crapauds IV}
\centerline{\LD@Font@Big@Arial@Italic un nouvel espoir...}
\bigskip

\noindent\LD@Font@Arial
Il était une fois un vilain crapaud,\pn
Très vieux, fort laid et bardé de défauts,\pn
Mais au coeur d'or et l'âme emplie d'amour.
\bigskip

\noindent
OL était son nom et en bord de Seine,\pn
Sa demeure, auprès d'une allée de chênes,\pn
Jouxtait Lutèce et ses proches fauxbourgs.
\bigskip

\noindent
Triste et las de n'être qu'un batracien,\pn
Ce crapaud insensé constamment rêvait,\pn
Par profond respect pour le genre humain,\pn
D'en faire partie, un jour, s'il pouvait.
\bigskip

\noindent
Ayant appris d'une amie sorcière,\pn
Que seule une femme au regard aimant\pn
Peut transformer un crapaud solitaire\pn
En homme et un OL en prince charmant,
\bigskip

\noindent
Il venait dans ce fabuleux jardin,\pn
Que chérissait l'Aphrodite mythique,\pn
Et, par ses coassements pathétiques,\pn
Ses cabrioles ou son ton badin,
\bigskip

\noindent
En se montrant sous son meilleur profil,\pn
Il tentait de susciter l'intérêt\pn
Des promeneuses au pas assuré\pn
Qui passaient , sinon, sans battre d'un cil.
\bigskip

\noindent
O gente dame ! O fière demoiselle !\pn
Si d'aventure un crapaud vous croisez,\pn
Ne l'embrassez pas ! Celle qui essaient\pn
N'obtiennent qu'urticaire et varicelle.
\bigskip

\noindent
Accordez lui plutôt un beau regard,\pn
Un sourire, quelque encouragement\pn
Devenez son amie, en attendant\pn
De rencontrer, tout deux, la perle rare...
\bigskip
\centerline{\LD@Font@Big@Arial La guerre des crapauds V}
\centerline{\LD@Font@Big@Arial@Italic les princesses contre-attaquent.}
\bigskip
\noindent
Un jour, OL, notre vieux crapaud solitaire\pn
Vint à passer au pied d'une sombre tour,\pn
Dans laquelle habitait, dans ses beaux atours\pn
Une jeune princesse, noble et fière.
\bigskip

\noindent
"Par decret du roi, mon père, tout mâle\pn
Se doit de m'éblouir avec esprit, tendresse\pn
Ou connaître honte, opprobe et subir le pal !\pn
Mais s'il brille, il aura mon corps, mes largesses\pn
Et tout ce que peux livrer femme timide...\pn
Donc, vite ! Execute toi et sois viril !"\pn
Tonna son altesse à l'être vert et vil,
\bigskip

\noindent
Que le charnier, au pied du fort, intimide,\pn
Car moult apprenti-prince y git trépassé.
\bigskip

\noindent
"Coaaaa ? mais c'est indu : je ne fais que passer...\pn
Répondit le batracien avec emphase,\pn
Avant d'obtemperer sous la menace :
\bigskip

\noindent
"Ma mie, vous êtes la fleur, je suis la vase,\pn
Si belle que pour vous d'amour je coasse...\pn
Soyez mon nénuphar, mon étang, ma mare !\pn
Vivons d'eau, d'amour et d'air pur, tout de suite !\pn
\hbox{Craignons deux cents brochets, la passion des truites}\pn
Et repeuplons l'univers de nos tétârds !"
\bigskip

\noindent
Voyant que la belle soudain s'empourprait,\pn
Et assimilant rougeur à déplaisir,\pn
Vers le crapaud, le bourreau s'avance outré\pn
Afin de l'estoquer, de le voir gésir.
\bigskip

\noindent
Se croyant perdu et sautant dans le vide\pn
Notre gentil héro ne dut son salut\pn
Qu'aux douves, bien remplies car il avait plu.
\bigskip

\noindent
"Quelle morale à cela ?" dirait Ovide.
\bigskip

\noindent
La première est que parmi les princesses,\pn
La plupart cherchent un prince charmant honnête,\pn
Mais bien peu, pour un amour, se font reinettes.
\bigskip

\noindent
La seconde est que la séduction cesse\pn
D'être utile quand elle n'est mutuelle.
\bigskip

\noindent
Pourtant, combien de dames, de demoiselles,\pn
Exigent éloge, ôdes, talents littéraires\pn
Et refusent aux crapauds, la saveur des vers !
\onecolumn\vfill\null
\bye