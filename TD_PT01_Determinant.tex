\catcode`@=11\relax
\def\Api{Mathematicon@Api}%
\input LD@Header.tex
\input LD@Library.tex
\input LD@Exercices.tex

\DefineRGBcolor F0F9E3=VLGreen.
\DefineRGBcolor E5F9D1=LGreen.
\DefineRGBcolor DAF9BE=TGreen.
\DefineRGBcolor 5DA93B=Green.
\DefineRGBcolor F6DCCA=VLRed.
\DefineRGBcolor F6D4BD=LRed.
\DefineRGBcolor DAF9BE=TRed.
\DefineRGBcolor B5F9A1=TTRed.
\DefineRGBcolor F6B080=Red.
\DefineRGBcolor F9F5E3=VLOrange.
\DefineRGBcolor F9F5D0=LOrange.
\DefineRGBcolor DAF9BE=TOrange.
\DefineRGBcolor B5F9A1=TTOrange.
\DefineRGBcolor D7A93B=Orange.
\DefineRGBcolor EEEEEE=VLBlack.
\DefineRGBcolor DDDDDD=LBlack.
\DefineRGBcolor CCCCCC=TBlack.
\DefineRGBcolor B5F9A1=TTBlack.
\DefineRGBcolor 000000=Black.

%\DefineRGBcolor 000000=Green.
%\definecolor{ColorVLGreen}{rgb}{1,1,1}%
%\definecolor{ColorLGreen}{rgb}{1,1,1}%
%\definecolor{ColorTGreen}{rgb}{1,1,1}%
%\expandafter\definecolor\temp
%\DefineRGBcolor 000000=Red.
%\definecolor{ColorVLRed}{rgb}{1,1,1}%
%\definecolor{ColorLRed}{rgb}{1,1,1}%
%\definecolor{ColorTRed}{rgb}{1,1,1}%
%\input Exercices.tex
\catcode`@=11\relax


%%%%%%%%%%%%%%%%%%%%%%%%%%%%%%%%%%%%%%%%%%%%%%%%%%%%%%%%%%%%%%%%%%
%															%
%						Revisions : Determinant, Polynome caracteristique			%
%															%
%%%%%%%%%%%%%%%%%%%%%%%%%%%%%%%%%%%%%%%%%%%%%%%%%%%%%%%%%%%%%%%%%%

\vglue-10mm\rightline{PT\hfill R\'evision 1 :  Determinant, polyn\^ome caract\'eristique\hfill}
\bigskip
\bigskip


\centerline{Determinant\strut}
\hrule\medskip
\Exercice{PTSIur}%
\medskip
\Exercice{PTSIuq}%
\medskip
\Exercice{PTfh}%


%\Exercice{PTSIwv}%
\hrule
\centerline{Polyn\^ome caract\'eristique\strut}
\hrule\medskip
Le polyn\^ome caract\'eristique $P$ d'une matrice carr\'ee $A=(a_{i,j})$ est l'unique polyn\^ome v\'erifiant
$$
\forall \lambda\in\ob C, \qquad P(\lambda)=\det(A-\lambda\hbox{I}_n).
$$

\noindent$\underline{{\bf Exercice}}${ \bf 4.}  Calculer et factoriser le plolyn\^ome caract\'eristique de $A:=\pmatrix{2&0&1\cr1&1&1\cr-2&0&-1\cr}$. 
\bigskip

\noindent$\underline{{\bf Exercice}}${ \bf 5.}  Calculer et factoriser le plolyn\^ome caract\'eristique de $A:=\pmatrix{
-7&3&1&-6
\cr
-6&2&1&-6
\cr
0&0&2&0
\cr
6&-3&-1&5
\cr}
$. 
\bigskip

\noindent$\underline{{\bf Exercice}}${ \bf 6.}  Calculer le polyn\^ome caract\'eristique $P_A$ de $A:=\pmatrix{0&1&&\ob O\cr
&\ddots&\ddots&\cr
\ob O&&0&1\cr
a_0&a_1&\ldots&a_{n-1}
\cr
}$ \pn \vskip-2em \noindent pour $(a_0,\cdots,a_{n-1})\in\ob C^n$. 
\bigskip

\def\addots{\setbox\olbox=\hbox{$\cdot$}%
\raise-0.10em\copy\olbox\kern0.1em\raise0.20em\copy\olbox\kern0.1em\raise0.50em\box\olbox}
\noindent$\underline{{\bf Exercice}}${ \bf 7.}   Diagonaliser la matrice $A:=\pmatrix{\ob O&&&1\cr
&&\addots&\cr
&\addots&&\cr
1&&&\ob O\cr}$. 
\medskip
\LD@Exo@Count=7\relax
\Exercice{PTfc}%




\bye









