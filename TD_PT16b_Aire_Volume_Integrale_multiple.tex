\catcode`@=11\relax

\input LD@Maths@TD.tex

\vglue-10mm\centerline{Spé PT\hfill Aires, volumes, intégrales multiples\hfill}
\bigskip

\Concept Aire d'une surface plane 

\Definition [] Soit $S$ une surface (quarrable et bornée) de~$\ob R^2$. 
Alors, l'aire de $S$ est le nombre 
$$
\sc A(S):=\int_S\d x\d y. 
$$

\Concept Aire d'une surface

\Definition []Soit $U$ un ouvert (borné) de $\ob R^2$ et $\vec f:U\to\ob R^3$ 
une fonction (bornée) de classe $\sc C^1$. Alors, l'aire de la surface 
$S:=\{f(x,y):(x,y)\in U\}$ 
paramétrée par $f$ est l'intégrale 
$$
\sc A(S):=\int_U\bigg|\!\bigg|{\partial \vec f\F\partial x}(x,y)\wedge{\partial\vec f\F\partial y}(x,y)\bigg|\!\bigg|\d x\d y. 
$$

\hrule
\centerline{Calculs d'aire}
\hrule
\Exercice{PTcb}
\bigskip

\Exercice{PTce}
\bigskip
\Exercice{PTcf}
\bigskip
\Exercice{PTnh}
\bigskip

\Exercice{PTnk}
\bigskip
\Exercice{PTnm}
\bigskip
\Exercice{PTnt}
\bigskip

\Exercice{PToa}
\bigskip
\Exercice{PTob}
\bigskip
\Exercice{PTol}
\bigskip

\Exercice{PTaco}
\bigskip
\Concept Volume

\Definition []Soit $U$ un ensemble (quarrable et borné) de $\ob R^3$. Alors, le volume de $U$ est le nombre
$$
\sc V\mbox{ol}(U):=\int_U\d x\d y\d z. 
$$

\hrule
\centerline{Calculs de Volume}
\hrule
\bigskip


\Exercice{PTci}
\bigskip

\Exercice{PTnj}
\bigskip
\Exercice{PTns}
\bigskip
\Exercice{PToc}
\bigskip
\Exercice{PTod}
\bigskip
\Exercice{PTacm}
\bigskip
\Exercice{PTacn}

\Concept Théorème de Fubini. 

\Theoreme [Title=Théorème de Fubini]
Soient $a< b$ et $c<d$ des nombres réels et $f:[a,b]\times[c,d]\to\ob R$ une fonction continue. 
$$
\int_a^b\Q(\int_c^df(x,y)\d y\W)\d x=\int\int\limits_{\llap{$\ss[a,b]\times[c,d]$}}f(x,y)\d x\d y=\int_c^d\Q(\int_a^bf(x,y)\d x\W)\d y.
$$

\Theoreme [Index=Theoreme@Théorème!de Fubini;Title=Théorème de Fubini {\it généralisé}]
Soit $A\subset \ob R^2$ et $f:A\to\ob R$ une fonction continue. S'il~existe 
des nombre $a< b$ et des fonctions continues $\varphi:[a,b]\to\ob R$ et $\phi:[a,b]\to\ob R$ vérifiant $\varphi\le \phi$ et 
$$
A=\B\{(x,y):a\le x\le b\mbox{ et }\varphi(x)\le y\le \phi(x)\B\}
$$
Alors, on a 
$$
\int\int_Af(x,y)\d x\d y=\int_a^b\Q(\int_{\varphi(x)}^{\phi(x)}f(x,y)\d y\W)\d x.
$$
S'il existe 
des nombres $c<d$ et des fonctions continues $\psi:[c,d]\to\ob R$ et $\Psi:[c,d]\to\ob R$ vérifiant $\psi\le\Psi$ et 
$$
A=\B\{(x,y):c\le y\le d\mbox{ et }\psi(y)\le x\le \Psi(y)\B\}
$$
Alors, on a 
$$
\int\int_Af(x,y)\d x\d y=\int_c^d\Q(\int_{\psi(y)}^{\Phi(y)}f(x,y)\d x\W)\d y.
$$


\Concept Théorème de changement de variable. 

\Definition []  Soit $f:x\mapsto\b(f_1(x),\cdots,f_p(x)\b)$ une application de $U\subset\ob R^n$ dans $\ob R^p$ 
dont toutes les dérivées partielles sont définies en $a\in U$. 
On appelle matrice jacobienne de $f$ en $a$ la matrice $J[f](a)\in\sc M_{p,n}(\ob R)$ 
définie par 
$$
J[f](a):=\Q({\partial f\F\partial x_1}(a),\ldots,{\partial f\F\partial x_n}(a)\W)
=\pmatrix{
{\partial f_1\F\partial x_1}(a)&\ldots&\ldots&{\partial f_1\F\partial x_n}(a)
\cr
{\partial f_2\F\partial x_1}(a)&\ldots&\ldots&{\partial f_2\F\partial x_n}(a)
\cr
\vdots&\ldots&{\partial f_i\F\partial x_j}(a)&\vdots
\cr
{\partial f_p\F\partial x_1}(a)&\ldots&\ldots&{\partial f_p\F\partial x_n}(a)
\cr
}
$$

\Definition []  Si $n=p$, on appelle Jacobien de $f$ au point $a$ et l'on note 
$$
\ds{\mbox{D}(f_1,\cdots,f_n)\F\mbox{D}(x_1,\cdots,x_n)}:=\det J[f](a)
$$ 
le déterminant 
de la matrice jacobienne de $f$ en $a$. 
\bigskip

\Theoreme [Title=Théorème de changement de variable] 
Soient $U,$ et $V$ des ouverts (quarrables) de $\ob R^n$, soit 
$f:V\to\ob R$ une fonction continue sur $V$ et son bord 
et soit $\phi:U\mapsto V$ un difféomorphisme de classe $\sc C^1$ de $U$ dans $V$. 
Alors, on a 
$$
\int_Vf(y_1,\cdots,y_n)\d y_1\cdots\d y_n
=\int_Uf\b(\underbrace{y_1,\cdots,y_n}_{\phi(x_1,\cdots,x_n)}\b)
\Q|{\mbox{D}(y_1,\cdots,y_n)\F\mbox{D}(x_1,\cdots,x_n)}\W|\d x_1\cdots\d x_n. 
$$

\centerline{Intégrales multiples}
\hrule

\Exercice{PTca}
\bigskip

\Exercice{PTcc}
\bigskip
\Exercice{PTcd}
\bigskip
\Exercice{PTcg}
\bigskip
\Exercice{PTch}
\bigskip
\Exercice{PTne}
\bigskip
\Exercice{PTni}
\bigskip
\Exercice{PTnl}
\bigskip
\bigskip
\bye
