\catcode`@=11\relax
\input LD@Header.tex
\input LD@Library.tex
\input LD@Typesetting.tex
\input LD@Exercices.tex

\ifHtml
	\input LD@Html
	\tikzstyle{every picture}=[svg text only=false,svg text css=Mathematicon]%
	\font\LD@Font@Tiny=cmr10\relax
	\font\LD@Font@Inferno=cmr10\relax
	\font\LD@Font@Arial=cmr10\relax
\else
	\LD@AFour@Book
	\font\LD@Font@Tiny=cmr5\relax
	\font\LD@Font@Inferno="Riot Act 2" at 8pt\relax
	\font\LD@Font@Arial="Arial" at 10pt\relax
\fi




%%%%%%%%%%%%%%%%%%%%%%%%%%%%%%%%%%%%%%%%%%%%%%%%%%%%%%%%%%%%%%%%%
%															%
%						Exo 01 : déterminant							%
%															%

\newcount\LD@Count@Temp


\newif\ifLD@Inferno@Master@\LD@Inferno@Master@true
\LD@Exo@Label@Show



\def\LD@Exercice@Display@Code{}%%\LD@Option@@Label\qquad\eightpts}%
\gdef\LD@Exercice@Solution@List{}%
\gdef\LD@Exercice@Indication@List{}%
\gdef\LD@Exercice@Notion@List{}%
\def\LD@Exercice@Display@Code@Post{%
	\ifcsname LD@Exo@@Solution\endcsname
		\unless\ifx\LD@Exo@@Solution\LD@Empty
			\ifLD@Inferno@Master@
				\pn{\eightpts Solution : \eightpts \LD@Exo@@Solution}%
			\else
				\ifx\LD@Exercice@Solution@List\LD@Empty
					\EA\gdef\EA\LD@Exercice@Solution@List\EA{\LD@Option@@Label}%
				\else
					\EA\EA\EA\gdef\EA\EA\EA\LD@Exercice@Solution@List\EA\EA\EA{\EA\LD@Exercice@Solution@List\EA ,\LD@Option@@Label}%
				\fi
			\fi
		\fi
	\fi
	\ifcsname LD@Exo@@Notion\endcsname
		\unless\ifx\LD@Exo@@Notion\LD@Empty
			\ifLD@Inferno@Master@
				\pn{\eightpts Notions intervenant dans la solution : \eightpts \LD@Exo@@Notion}%
			\else
				\ifx\LD@Exercice@Notion@List\LD@Empty
					\EA\gdef\EA\LD@Exercice@Notion@List\EA{\LD@Option@@Label}%
				\else
					\EA\EA\EA\gdef\EA\EA\EA\LD@Exercice@Notion@List\EA\EA\EA{\EA\LD@Exercice@Notion@List\EA ,\LD@Option@@Label}%
				\fi
			\fi	
		\fi
	\fi
	\ifcsname LD@Exo@@Indication\endcsname
		\unless\ifx\LD@Exo@@Indication\LD@Empty
			\ifLD@Inferno@Master@
				\pn{\eightpts Indication : \eightpts \LD@Exo@@Indication}%	
			\else
				\ifx\LD@Exercice@Indication@List\LD@Empty
					\EA\gdef\EA\LD@Exercice@Indication@List\EA{\LD@Option@@Label}%
					\else
					\EA\EA\EA\gdef\EA\EA\EA\LD@Exercice@Indication@List\EA\EA\EA{\EA\LD@Exercice@Indication@List\EA ,\LD@Option@@Label}%
				\fi
			\fi
		\fi
	\fi
	\medskip\penalty-100
}%
\def\LD@Display#1{%
	\LD@Count@Temp=#1\relax
	\ifcase\LD@Count@Temp
	\or
	Math. Sup.
	\or
	Math. Sp\'e
	\else
	\fi
}%

\newcount\LD@Exo@Total\LD@Exo@Total=0\relax

\def\LD@Exo@Theme@Display#1#2#3{%
	{\LD@Loop@For\LD@Theme=#2\WithSeparator\Do{%
%             Tests if there are exercices in the LD@Theme
		\gdef\LD@List@Empty{Y}%	
		{\LD@Loop@For\LD@Exo@Type=#3\WithSeparator ,\Do{\if Y\LD@List@Empty	
			{\LD@Loop@For\LD@Level=#1\WithSeparator ,\Do{\if Y\LD@List@Empty
				\ifcsname LD@Exo@Theme@@\LD@Level @@\LD@Theme @@\LD@Exo@Type\endcsname
					\EA\let\EA\LD@Exo@List\CS LD@Exo@Theme@@\LD@Level @@\LD@Theme @@\LD@Exo@Type\EC
					\unless\ifx\LD@Exo@List\LD@Empty
						\gdef\LD@List@Empty{N}%
					\fi
				\fi
			\fi}}%
		\fi}}%
%		If there are exercices in the LD@Theme, display them
		\if N\LD@List@Empty
			\edef\LD@Temp{null, \LD@Font@Arial\LD@Theme.}%
			\EA\Chapter\LD@Temp\PAR\medskip
			{\LD@Loop@For\LD@Exo@Type=#3\WithSeparator ,\Do{%
				{\LD@Loop@For\LD@Level=#1\WithSeparator ,\Do{%
					\ifcsname LD@Exo@Theme@@\LD@Level @@\LD@Theme @@\LD@Exo@Type\endcsname
%					\edef\LD@Temp{null, \LD@Font@Arial\LD@Exo@Type.}%
%					\EA\Section\LD@Temp\PAR
%					\medskip\noindent$\underline{\hbox{\LD@Font@Arial\LD@Exo@Type}}$\medskip\noindent
						\EA\let\EA\LD@Exo@List\CS LD@Exo@Theme@@\LD@Level @@\LD@Theme @@\LD@Exo@Type\EC
						{\LD@Loop@For\LD@Exo@Label=\LD@Exo@List\WithSeparator ,\Do{%
							\EA\Exercice\EA{\LD@Exo@Label}%
							\global\advance\LD@Exo@Total by1\relax
						}}%
					\fi
				}}%
			}}%
		\fi
	}}%
}%

\def\LD@Maths@Exercice@Text{%
	\underline{%
		\mbox{%
			\ifx\LD@Exo@@Solution\LD@Empty 
				\unless\ifx\LD@Exo@@Indication\LD@Empty 
					\it
				\fi 
			\else
				\ifx\LD@Exo@@Indication\LD@Empty 
					\bf
				\else
					\boldit
				\fi 
			\fi 
			Exercice%
		}%
	}%
}%

\def\LD@Exo@Sol@Display{%
	{\EA\LD@Loop@For\EA\LD@Exo@Label\EA=\csname LD@Exercice@Solution@List\endcsname\WithSeparator ,\Do{%
		\LD@Data@Def{Sol}\LD@Exo@Label\LD@Temp
		\noindent{\eightpts \ref{labelexo\LD@Exo@Label}. \LD@Temp}
		\medskip
	}}%

}%
\def\LD@Exo@Notion@Display{%
	{\EA\LD@Loop@For\EA\LD@Exo@Label\EA=\csname LD@Exercice@Notion@List\endcsname\WithSeparator ,\Do{%
		\LD@Data@Def{Notion}\LD@Exo@Label\LD@Temp
		\noindent\pn{\eightpts \ref{labelexo\LD@Exo@Label} : \LD@Temp}%
		\medskip
	}}%
}%


\def\LD@Exo@Indication@Display{%
	{\EA\LD@Loop@For\EA\LD@Exo@Label\EA=\csname LD@Exercice@Indication@List\endcsname\WithSeparator ,\Do{%
		\LD@Data@Def{Ind}\LD@Exo@Label\LD@Temp
		\noindent\pn{\eightpts \ref{labelexo\LD@Exo@Label}. \LD@Temp}%
		\medskip
	}}%

}%

%\long\def\Chapter #1, #2.{\init{\titredeux}\ifSecLabelEq\init{\eqnumber}\fi\ifHtml\writetocentry{chapter}{#2}\EndP\ifCenteredchapter\HCode{<div class="centerline">}\else\HCode{<div class="flushleft">}\fi
%\HCode{<span class="chapter">}{\fontetitreun\incr{\titreun}.\ \fontetitreun #2}\HCode{</span></div>}\PAR\else 
%\removelastskip\par\bigskip\bigskip\goodbreak\vskip0pt plus.01\vsize\penalty-200\vskip0pt plus-.01\vsize        
 %                                     \bigbreak\noindent\writetocentry{chapter}{#2}{\ifCenteredchapter\parindent0mm\raggedcenter\fi\fontetitreun\incr{\titreun}.\ \fontetitreun #2\par}\nobreak
%                                       \medskip\definexref{chap#1}{\the\titreun}{}%\definexref{#1}{#2}{}\definexref{ǧ§#1}{\the\pageno}{}% fix this
%\fi}


\overfullrule=0pt%
\nopagenumber
\newdimen\LD@dimena\LD@dimena=18cm\relax\advance\LD@dimena by-1cm\relax
\newdimen\LD@dimenb\LD@dimenb=28cm\relax\advance\LD@dimenb by-4cm\relax
 %
\centerline{%
	%\unless\ifHtml\font\SvgText=Papyrus\relax\fi
	\tikzpicture
	\ifHtml 
		\node[text height=2\LD@dimenb,text width=2\LD@dimenb,tex4ht node/escape=true] (a) {\noindent\smash{\Image[Height=2\LD@dimenb,Width=2\LD@dimenb]{F:/TeX/Mathematicon/Abime.jpg}}}%
	\else
		\ifLD@Inferno@Master@
			\node[text width=\LD@dimena,text height=\LD@dimenb%,draw,line width=2pt,inner sep=0pt
	] (a) at (0,-2) {\noindent\smash{\Image[Width=\LD@dimena,Height=\LD@dimenb]{F:/TeX/Mathematicon/AbimMaster.jpg}}}%
		\else
			\node[text width=\LD@dimena,text height=\LD@dimenb%,draw,line width=2pt,inner sep=0pt
	] (a) at (0,-2) {\noindent\smash{\Image[Width=\LD@dimena,Height=\LD@dimenb]{F:/TeX/Mathematicon/Abime.jpg}}}%
		\fi
	\fi
node [below=0.7cm,inner sep=1pt] (b) at (a.north) {};
\node [inner sep=1.5pt,above=1.7cm] (c) at (a.south) {};
\node [inner sep=1pt,below=1cm] (d) at (c) {};
\foreach\angle in {0,15,...,360}
{%
\node[color=black,scale=1.5] at (b.\angle) {Olus Livius Bindus};
\node[color=black,scale=5] at (c.\angle) {\LD@Font@Inferno Inferno}; 
\node[color=black,scale=1.5] at  (d.\angle)  {Abandon hope, all ye who enter here...};
}%
\node[color=gray,scale=1.5] at (b) {Olus Livius Bindus};
\node[color=gray,scale=5] at (c) {\LD@Font@Inferno Inferno};
\node[color=gray, scale=1.5] at (d) {Abandon hope, all ye who enter here...};
\endtikzpicture}%
\headline={\ifpagetitre\the\hautpagetitre
\else\ifodd\pageno\the\hautpagedroite\else\the\hautpagegauche\fi\fi }
\footline={\ifpagetitre\the\baspagetitre
\else\ifodd\pageno\the\baspagedroite
\else\the\baspagegauche\fi\fi \global\pagetitrefalse}
\hautpagegauche={\ifMidFolio\the\hautpagemilieu\else\tenrm\folio\hfil\the\auteurcourant\hfil\fi}
\hautpagedroite={\ifMidFolio\the\hautpagemilieu\else\hfil\the\titrecourant\hfil\tenrm\folio\fi}
\eject
\def\pgfutil@EveryShipout@Output{%
  \setbox255=\vbox{%
    \setbox0=\hbox{\pgfutil@abe\pgfutil@abc\global\let\pgfutil@abc\pgfutil@empty}%
    \wd0=0pt%
    \ht0=0pt%
    \dp0=0pt%
    \box0%
    \makeheadline
    \unvbox255%
    \makefootline
  }%
  \pgfutil@@EveryShipout@Org@Shipout\box\@cclv%
}
\hautspages{}{}%
\catcode`@=11\relax
\def\LD@AFour@Book{%
\hsize170truemm\vsize 250truemm\hoffset=-5truemm\voffset=-13truemm
\pretolerance=500\relax
\tolerance=1000\relax
\brokenpenalty=5000\relax
\parindent3mm
\paperwidth=210truemm
\paperheight=297truemm
\pdfpagewidth=210truemm
\pdfpageheight=297truemm
}
\LD@AFour@Book
\centerline{\seventeenbf Table des Mati\`eres}
\bigskip
\readtocfile

\eject
%%%%%%%%%%%%%%%%%%%%%%%%%%%%%%%%%%%%%%%%%%%%%%%%%%%%%%%%%%%%%%%%%%
%\formatpage

\Chapter sol, Legende.

L'ent\^ete de chaque exercise pr\'ecise le niveau \tikz\node[fill=green!30,inner sep=1pt,baseline]{$n$};, la difficult\'e \tikz\node[fill=red!30,inner sep=1pt,baseline]{$d$};, la richesse p\'edagogique \tikz\node[fill=blue!30,inner sep=1pt,baseline]{$r$}; (i.e. le nombre de nouveaux concepts, d'id\'ees int\'eressantes contenues qu'enseigne la r\'esolution de l'exercice) et parfois  \'egalement la provenance  \tikz\node[fill=purple!30,inner sep=0pt,baseline]{$o$}; de l'exercice. 
\medskip

Il est \'egalement indiqu\'e si l'exercice est associ\'e {\it \`a des indications}, $\underline{\hbox{\`a des concepts utiles \`a sa r\'esolution}}$ ou {\bf \`a une solution}, \unless\ifLD@Inferno@Master@
que vous trouverez respectivement aux sections \refn{chapIndications}, \refn{chapNotions} et \refn{chapSolutions}, \fi
selon le schema suivant : 
$$
\tikzpicture
	\node[fill=yellow!30] (exo) at (0,0) {Exercice};
	\node[below=0pt of exo.north east,inner sep=0.5pt,fill=green!30] (n) {$\scriptstyle n$};
	\node[above=1pt of exo.west,fill=red!30,inner sep=0.5pt] (d) {$\scriptstyle d$};
	\node[below=1pt of exo.west,fill=blue!30,inner sep=0.5pt] (r) {$\scriptstyle r$};
	\node[left=0.15cm of exo.west,fill=purple!30,inner sep=1pt] (o) {$\scriptstyle (o)$};
	\node[below right=-0.7cm and 0.7cm of n,matrix of nodes,fill=green!30,draw,font=\LD@Font@Tiny] (niveau) {%
%		\flat\flat& Lyc\'ee,\cr
		$\scriptscriptstyle n$ &Ann\'ee\\
		$\scriptscriptstyle\flat$&Math. Sup. (L1)\\
		&Math. Sp\'e. (L2) \\
%		$\scriptscriptstyle\sharp$&Licence (L3)\\
%		\sharp\sharp&Maitrise (M1),\cr
%		\sharp\sharp\sharp&DEA/DESS (M2).\cr
	};
	\node[above right=0.8cm and -3.85cm of d,matrix of nodes,fill=red!30,draw,font=\LD@Font@Tiny] (difficulte) {%
		$\scriptscriptstyle d$&Type&PT&PT*/MP&MP*\\
% 		-1&évident&Trop facile&Perte de temps&Bah\\
		&appliquer le cours&Facile&Trop facile&Perte de temps\\
		1&adapter le cours&Facile/Moyen&Facile&Trop facile\\
		2&ENSAM/CCP&Moyen/dur&Facile/Moyen&Facile\\
		3&Centrale/Mines&Dur/Baleze&Moyen/dur&Facile/Moyen\\
		4&Polytechnique&Baleze/Infernal&Dur/Baleze&Moyen/dur\\
		5&Ens Ulm&Infernal/Impossible&Baleze/Infernal&Dur/Baleze\\
	};
	\node [fill=purple!30,draw,font=\LD@Font@Tiny] (origin) at (-3.4,0) {origine de l'exercice};
	\node[below left=0.8cm and -0.7cm of r,matrix of nodes,fill=blue!20,draw,font=\LD@Font@Tiny] (richesse) {%
% 		-1&Perte de temps&\\	
		$\scriptscriptstyle r$&Type&Contenu\\
		&Entrainement&0 id\'ee nouvelle\\
		1&Formateur&1+ id\'ee nouvelle\\
		2&Classique&2+ id\'ees nouvelles\\
		3&Culte&3+ id\'ees nouvelles\\
	};
	\node[below right=0.7cm and -0.5cm of exo,matrix of nodes,fill=yellow!30,draw,font=\LD@Font@Tiny] (metadata) {%
		typographie&Contenu\\
		Exercice&\'enonc\'e\\
		{\sixit Exercice}& avec indication(s)\\
		{\fivebf Exercice}& avec solution\\
		$\underline{\mbox{Exercice}}$& avec notions utiles\\
	};
	\draw[->] (o) -- (origin);
	\draw[->] (d) -- (difficulte);
	\draw[->] (r) -- (richesse);
	\draw[->] (n) -- (niveau);
	\draw[->] (exo) -- (metadata);
\endtikzpicture
$$
\vfil
\eject

\LD@Exo@Theme@Display{1,2}\LD@Exo@Theme@List{%
	\Cours,\TravauxDirigés,\Exercices,\Colles%,\Problèmes,\Others,\Mathematica,\Maple,\LD@Empty
}%
\unless\ifLD@Inferno@Master@
	\eject
	\Chapter Indications, Indications.

	\LD@Exo@Indication@Display

	\Chapter Notions, Notions.

	\LD@Exo@Notion@Display

	\Chapter Solutions, Solutions.

	\LD@Exo@Sol@Display
\fi
\LD@dimenb=1cm\relax
\bye
	\LD@Loop@For\LD@Level=1,2\WithSeparator ,\Do{%
	\centerline{\seventeenbf \LD@Display\LD@Level}%
	\bigskip
	%\EA\let\EA\LD@List\CS LD@Exo@Theme@@\LD@Level\EC
	{
		\LD@Exo@Theme@Display\LD@Level\LD@Exo@Theme@List{\Cours,\TravauxDirigés,\Exercices,\Colles,\Problèmes,\Others,\Mathematica,\Maple,\LD@Empty}%
	}%
}%
\medskip\noindent
\the\LD@Exo@Total\LD@Space Exercices
\bye