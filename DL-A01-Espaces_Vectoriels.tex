%\def\Variables{MathsVariables}% 
\catcode`@=11\relax
\def\Api{Mathematicon@Api}%
\let\LD@Par\par
%%%% Newif  
%\def\@firstofone#1{#1}
\long\def\IGNORE#1\IGNORE{}%
\newif\ifexonumber
%%%% Switches
\exonumberfalse
\catcode`@=11\relax
\input LD@Header.tex
\input LD@Library.tex
\input LD@Typesetting.tex

%%%% Newif

\newif\ifexonumber

%%%% Switches
\exonumberfalse

%\input Preprocessing.tex

%\transparent
%\nomorecolors

\vglue-10mm\rightline{PT\hfill DS 1 (probl\`emes ind\'ependants) \hfill \date}
\bigskip
\medskip
\pn{\eightpts 
OLUS 6 : lire tout le sujet et rep\'erer/m\'emoriser les d\'efinitions (\`a surligner) et  les questions faciles 
\pn
OLUS 66 : Appliquer les m\'ethodes du cours et soigner la r\'edaction (cela vous aide et donne + de points)
\pn
OLUS 666 : Paniquez,  lisez mal les questions, comprenez de travers, zapez les hypoth\`eses, plantez vous et surtout surtout surtout faites nimportenawak (qu'on se marre un peu...)
}
\bigskip
\bigskip

\noindent{\twelvebf Probl\`eme 1 \it(Mines D'Albi, Al\`es, Douai, Nantes 98).}
\frenchspacing
%\font\twelverm=cmr10 at 12 pt
%\font\twelvebf=cmbx10 at 12pt
%\font\fifteenbf=cmbx10 at 15pt
%\font\twentybf=cmbx10 at 20pt
%\font\tenbb=msbm10     \font\sevenbb=msbm7   \font\fivebb=msbm5
\def\N{{\bb N}}
\def\Z{{\bb Z}}
\def\Arctan{\mathop{\rm Arctan}}
\def\Vect{\mathop{\rm Vect}}
\everymath{\displaystyle}
\def\prob#1
{\bigskip\centerline{\twelvebf#1}\bigskip\noindent}
\def\part#1
{\bigskip\noindent{\bf#1}\medskip\noindent}
\def\ques#1
{\medskip\item{\bf#1)}}
\def\dques#1
{\medskip\item{\bf#1)}\hangindent=2\parindent\textindent{\bf a)}}
\def\sques#1
{\smallskip\itemitem{\bf#1)}}
\def\ssques#1
{\smallskip\indent\indent\hangindent3\parindent\textindent{\bf#1)}}
\def\rc{\smallskip\noindent}
\def\rrc{\medskip\item{}}
\bigskip
\bigskip
Dans tout le probl\`eme, l'espace vectoriel $\ob R^3$ est muni de sa structure
euclidienne orient\'ee usuelle et rapport\'e \`a sa base canonique (orthonorm\'ee
directe) not\'ee $(e_1,e_2,e_3)$.
\rc
On note ${\cal L}(\ob R^3)$ la $\ob R$-alg\`ebre des endomorphismes de $\ob R^3$,
${\cal M}_3(\ob R)$ la $\ob R$-alg\`ebre des matrices d'ordre $3$ \`a coefficients
r\'eels et $I_3$ la matrice identit\'e.
\rc
{\it Il est demand\'e de faire figurer tous les calculs sur la copie.}
\part{Partie I {\it(les questions 2, 3 et 4 vous font calculer $S^n$ de trois fa\c cons diff\'erentes)}}  
Soit $s$ l'endomorphisme de $\ob R^3$ de matrice
$S={1\over3}\pmatrix{
\hfill5&-1&-1\cr
-1&\hfill5&-1\cr
-1&-1&\hfill5\cr}$ dans la base canonique.
\ques1
Montrer que $s$ est un automorphisme de $\ob R^3$.
\ques2
Soient $e'_1=(1,1,1)$,
$e'_2=(1,-1,0)$ et
$e'_3=(1,1,-2)$.
\sques a
Montrer que $(e'_1,e'_2,e'_3)$ est une base de $\ob R^3$.
\sques b
D\'eterminer la matrice $S'$ de $s$ dans la base $(e'_1,e'_2,e'_3)$.
\sques c
Calculer $(S')^n$ et donner une m\'ethode de calcul de $S^n$ (on ne demande pas
d'effectuer lesdits calculs).
\dques3
La famille $(I_3,S)$ est-elle libre dans ${\cal M}_3(\ob R)$ ?
\sques b
Montrer que $S^2$ peut s'exprimer sous forme de combinaison lin\'eaire de
$I_3$ et $S$.
\sques c
En d\'eduire que pour tout $n\in\N$, il existe un unique couple
$(a_n,b_n)$ de r\'eels tel que $S^n=a_nI_3+b_nS$
(on convient que : $\forall M\in{\cal M}_3(\ob R)\quad M^0=I_3$).
\sques d
Donner les valeurs de $a_0$, $b_0$, $a_1$, $b_1$, et exprimer, pour $n\in\N$,
$a_{n+1}$ et $b_{n+1}$ en fonction de $a_n$ et $b_n$.
\sques e
Montrer que la suite $(a_n+b_n)_{n\in\N}$ est constante, puis que la suite
$(b_n+1)_{n\in\N}$ est g\'eom\'etrique.
\sques f
En d\'eduire l'expression de $a_n$ et $b_n$ pour tout $n\in\N$.
\ques4
Soit $B=S-2I_3$.
\sques a
Calculer $B^n$ pour $n\in\N$.
\sques b
En d\'eduire l'expression de $S^n$ en fonction de $I_3$ et $B$ pour $n\in\N$
(on pourra, apr\`es justification, utiliser la formule du bin\^ome de Newton).
\sques c
Comparer avec le r\'esultat de la question 3).
\ques5
L'expression de $S^n$ obtenue aux questions 3) et 4) est-elle valable pour
$n\in\Z$ ?
\part{Partie II}
Soit $f$ l'endomorphisme de $\ob R^3$ de matrice
$A={1\over3}\pmatrix{
-1&-1&\hfill5\cr
\hfill5&-1&-1\cr
-1&\hfill5&-1\cr}$ dans la base canonique.
On pose : $u=f\circ s^{-1}$ et on note $U$ la matrice de $u$ dans la base
canonique.
\ques1
Calculer $U$ ; v\'erifier que $u$ est une rotation vectorielle et que
$u\circ s=s\circ u=f$.
\ques2
Soit $(e''_1,e''_2,e''_3)$ la famille obtenue en normant les vecteurs
$e'_1$, $e'_2$ et $e'_3$ de la question 2) de la premi\`ere partie.
\sques a
Montrer que $(e''_1,e''_2,e''_3)$ est une base orthonormale directe.
\sques b
Ecrire la matrice $U'$ de $u$ dans cette base et caract\'eriser g\'eom\'etriquement
$u$.
\dques3
Exprimer la matrice de $s$ dans la base $(e''_1,e''_2,e''_3)$ en fonction
de $S'$.
\sques b
En d\'eduire la matrice de $f$ dans la base $(e''_1,e''_2,e''_3)$.
\dques4
Quel est l'ensemble des vecteurs invariants par $f$ ?
\sques b
Soit $P=\Vect(e''_2,e''_3)$.
\ssques i
Montrer que $f(P)=P$.
\ssques{ii}
Soit $g$ l'endomorphisme de $P$ tel que pour tout $x$ de $P$, $g(x)=f(x)$.
Montrer que $g$ est la compos\'ee de deux applications lin\'eaires simples que
l'on reconna\^{\i}tra.
\ques5
On note ${\cal C}(f)$ l'ensemble des endomorphismes de $\ob R^3$ commutant avec
$f$, c'est-\`a-dire l'ensemble des endomorphismes $g$ tels que
$f\circ g=g\circ f$.
\sques a
Montrer que ${\cal C}(f)$ est une sous-alg\`ebre de ${\cal L}(\ob R^3)$.
\sques b
Soit $g\in{\cal C}(f)$.
\ssques i
Montrer que le vecteur $g(e''_1)$ est invariant par $f$.
Que peut-on en d\'eduire ?
\ssques{ii}
Soit $M$ la matrice de $g$ dans la base $(e''_1,e''_2,e''_3)$.
Montrer que $M$ commute avec $(S')^3$.
\ssques{iii}
En d\'eduire la forme g\'en\'erale de la matrice d'un endomorphisme de ${\cal C}(f)$
dans la base $(e''_1,e''_2,e''_3)$.
\sques c
Quelle est la dimension de l'espace vectoriel ${\cal C}(f)$ ?

\bigskip
\bigskip
\noindent{\twelvebf Probl\`eme 2 \it(ENSAE).}\bigskip
\medskip
\noindent $\sc M_2(\ob R)$ d\'esigne l'espace vectoriel des matrices carr\'ees de taille $2$ \`a coefficients r\'eels. \pn
Pour $(a,b)\in\ob R^2$, on pose 
$$
M(a,b):=\pmatrix{a&-b\cr b&a}.
$$
1.a. Montrer que l'ensemble $\sc C:=\Q\{M(a,b):(a,b)\in\ob R^2\W\}$ est un espace vectoriel. \pn
Pr\'eciser sa dimension et en donner une base. 
\medskip\noindent
On consid\`ere l'application 
$$
\eqalign{\Phi:\ob C&\to\sc C\cr z&\mapsto \pmatrix{\re z&-\im z\cr\im z&\re z}}.
$$
b) Montrer que, pour tous nombres complexes $z$ et $z'$ et tout nombre r\'eel $\lambda$, on a 
$$
\eqalign{
\Phi(\lambda.z)&=\lambda.\Phi(z)\cr
\Phi(z+z')&=\Phi(z)+\Phi(z')\cr
\Phi(z\times z')&=\Phi(z)\times\Phi(z')\cr
}
$$
En d\'eduire que 
$$
\forall z\in\ob C, \quad\forall p\in\ob N, \qquad \Phi(z^p)=\Phi(z)^p.
$$
c) L'application $\Phi$ est-elle un isomorphisme ? Entre quel genre de structure alg\'ebrique ? 
\bigskip
2.a. Soit $\theta\in[0,2\pi[$ et soit $A(\theta)$ la matrice 
$$
A(\theta):=\pmatrix{\cos(\theta)&-\sin(\theta)\cr\sin(\theta)&\cos(\theta)}.
$$
Pour $k\in\ob N$, calculer $A(\theta)^k$. Le r\'esultat obtenu est-il encore valable pour $k\in\ob Z$ ? 
\bigskip\noindent
b) Soit $p\in\ob N^*$. D\'eterminer une matrice $M\in\sc M_2(\ob R)$ telle que 
$M^p=\pmatrix{-2&0\cr0&-2}$.  \bigskip\noindent
3. On consid\`ere 
$$
\eqalign{f:\ob R[X]&\to\ob R[X]\cr
P\ &\mapsto (1+X^2)P''(X)-2XP'(X)}
$$
a) Pour $n\ge1$, justifier que la restriction $f_n$ de $f$ \`a l'espace $\ob R_n[X]$ est un endomorphisme. 
$$
\eqalign{f_n:\ob R_n[X]&\to\ob R_n[X]\cr
P\ &\mapsto (1+X^2)P''(X)-2XP'(X)}
$$
\bigskip\noindent
b) D\'eterminer le noyau de $f_n$ et en pr\'eciser la dimension. 
\bigskip
\noindent
c) D\'eterminer les valeurs propres de $f_n$ et pr\'eciser leur multiplicit\'e. 
\medskip\noindent
d) Peut-on trouver une base $\sc B$ de $\ob R_n[X]$ dans laquelle la matrice de $f_n$ est diagonale ? 
\medskip
\noindent
e) Soit $p\ge1$. D\'eterminer un endomorphisme $g_n$ de $\ob R_n[X]$ tel que $(g_n)^p=f_n$. \pn
{\it On pourra utiliser la question 2.b}
\bigskip
\bigskip
\noindent{\twelvebf Exercices ind\'ependants\it(Mines-Ponts).}\bigskip
\bigskip\noindent
{\twelvebf Exercice A)} Soit $E$ un $\ob R$-espace vectoriel  et soit $f\in\sc L(E)$ un endomorphisme de $E$.  \medskip
\noindent
1) D\'emontrer que la suite des noyaux des endomorphismes $f^k$ est une suite de sous-espaces vectoriels de $E$ emboit\'es, c'est-\`a-dire v\'erifiant $\ker f^k\subset\ker f^{k+1}$ pour $k\ge0$, 
$$
\ker f^0\subset\ker f\subset\ker f^2\subset\ker f^3\subset\cdots
$$
2) S'il existe $p\ge0$ tel que $\ker f^p=\ker f^{p+1}$, prouver que 
$$
\forall q\ge p, \qquad \ker f^q=\ker f^p. 
$$
3) Si $E$ est de dimension finie $n\ge1$, en d\'eduire que la suite des noyaux $u=(\ker f^p)_{p\ge0}$ est constante \`a partir d'un rang $q\le n$. 
\bigskip\noindent
4) De plus, s'il existe $d\ge1$ tel que $f^d=0$, prouver qu'il existe $s\le n$ 
tel que $f^s=0$. 
\bigskip\bigskip
\noindent
{\twelvebf Exercice B)} Soit $E$ un $\ob R$-espace vectoriel et soient $f$ et $g$ deux endomorphismes de $E$. \medskip
\noindent 1) s'il existe $\lambda\in\ob R$ tel que 
$f^2=\lambda\mbox{Id}_E+g$, prouver que $f$ et $g$ commuttent entre eux, i.e. 
$$
f\circ g=g\circ f. 
$$
On suppose d\'esormais que $f$ et $g$ commuttent entre eux. \medskip\noindent
\noindent
2) Soient $P$ et $Q$ des polyn\^omes \`a coefficients r\'eels. Prouver que $P(f)$ commutte avec $g$, puis que $P(f)$ commutte avec $Q(g)$. 
\bigskip
\noindent
3) Pour $\lambda\in\ob R$, prouver que l'espace $F_\lambda:=\ker(f-\lambda\mbox{Id}_E)$ est stable par $g$, c'est-\`a-dire que 
$$
g(F_\lambda)\subset F_\lambda
$$




\bye



\noindent{\bf Probl\`eme 1 \it(ENSAM, ESTP, ECRIN, ARCHIMEDE).}
\medskip
\noindent
1. Soit $u$ un endomorphisme de rang $1$ d'un $\ob C$-espace vectoriel $E$ de dimension finie $n\ge1$. 
\bigskip
\noindent
i) Selon la dimension de $\IM(u)\cap\ker(u)$, montrer que l'on a soit 
$E=\mbox{Im}(u)\oplus \ker(u)$ ou soit $\mbox{Im}(u)\subset\ker(u)$.
\bigskip
\noindent
ii) Soit $e\neq0$ dans $\mbox{Im}(u)$. Justifier l'existence d'une base de $E$ dont le $1^{\hbox{\sevenrm er}}$ vecteur est~$e$.  \pn
Dans le cas o\`u $\IM(u)\subset\ker(u)$, quelle est la forme de la matrice de $u$ sur une telle base\pn
Dans le cas o\`u $\IM(u)\subset\ker(u)$, montrer que $\mbox{Tr}(u)=0$. \pn  
{\it On rappelle que $\mbox{Tr}(u)$ est la trace de la matrice de $u$ (dans une base quelconque de~$E$).} 
\bigskip
\noindent
iii) Montrer l'\'equivalence des trois assertions suivantes : \medskip\noindent
a) il existe une base $\sc B$ de $E$ dans laquelle $\sc Mat_{\sc B}(u)$ est diagonale\smallskip\noindent
b) $E=\IM(u)\oplus\ker(u)$. \smallskip\noindent
c) $\mbox{Tr}(u)\neq0$. 
\bigskip\noindent
On note $\sc M_n(\ob C)$ le $\ob C$-espace vectoriel des matrices carr\'ees de taille $n$ \`a coefficients dans~$\ob C$ et  on note $\sc M_n(\ob C)^*$ le dual de $\sc M_n(\ob C)$, c'est-\`a-dire l'espace vectoriel   $\sc L\big(\sc M_n(\ob C), \ob C\big)$ des formes lin\'eaires sur $\sc M_n(\ob C)$.  \bigskip
\noindent
2. On fixe $A$ dans $\sc M_n(\ob C)$ et l'on pose 
$$
\forall X\in\sc M_n(\ob C), \qquad F_A(X):=\mbox{Tr}(AX), 
$$
o\`u $\mbox{Tr}(AX)$ d\'esigne la trace de la matrice $AX$. 
\bigskip
\noindent
i) Montrer que la relation pr\'ec\'edente d\'efinit une forme lin\'eaire $F_A$ sur $\sc M_n(\ob C)$. \bigskip
\noindent
ii) La matrice $A$ n'\'etant plus fix\'ee, on consid\`ere 
$$
\eqalign{F:\sc M_n(\ob C)&\to\sc M_n(\ob C)^*\cr A\quad &\mapsto F_A}
$$
Montrer que $F$ est une application lin\'eaire
\bigskip
\noindent
iii) Pour $(i,j)\in\{1, \cdots, n\}^2$, on note $E_{i,j}$ la matrice dont tous les coefficients sont nuls, except\'e le coefficient de la ligne $i$ et de la colonne $j$, qui vaut $1$. 
\bigskip
\noindent
Pour $(i,j)\in\{1, \cdots, n\}^2$, exprimer $F_A(E_{i,j})$ en fonction des coefficients de $A=(a_{i,j})_{1\le i\le n\atop1\le j\le n}$. \pn En d\'eduire que $F$ est injective. 
\bigskip
\noindent
iv) Montrer que $F$ est un isomorphisme. 
\bigskip
\noindent
3. Soit $J$ une matrice non nulle de $\sc M_n(\ob C)$ et soit $f$ une forme lin\'eaire non nulle sur $\sc M_n(\ob C)$, i.e. une forme lin\'eaire qui n'est pas la forme lin\'eaire nulle $M\mapsto0$. 
On consid\`ere 
$$
\eqalign{\psi_f:\sc M_n(\ob C)&\to\sc M_n(\ob C)\cr
X\quad&\mapsto f(X)J}
$$
Prouver que $\psi_f$ est un endomorphisme de $\sc M_n(\ob C)$. \bigskip
\noindent
i) Justifier l'existence d'une unique matrice $A$ de $\sc M_n(\ob C)$ telle que : 
$$
\forall X\in\sc M_n(\ob C), \qquad f(X)=\mbox{Tr}(AX).
$$
ii) Comparer le noyau de $\psi_f$ et le noyau de $f$. Que valent l'image et le rang de $\psi_f$ ? 
\bigskip
\noindent
iii) Exprimer la trace de $\psi_f$ en fonction de $A$ et $J$. \bigskip
\noindent
iv) En d\'eduire une condition n\'ec\'essaire et suffisante pour qu'il existe une base $\sc B$ de $\sc M_n(\ob C)$ dans laquelle la matrice de $\psi_f$ est diagonale. 
\bigskip
