\catcode`@=11\relax

\input LD@Maths@TD.tex
%%% TD 11.
\centerline{Rappels utiles sur limites, DL, \'equivalents, fonctions n\'egligeables}
\hrule
\medskip
$$
\eqalign{
 f&=o_a\big(g(x)\big)\ssi   \mbox{$f$ est n\'egligeable devant $g$ en $a$}\ssi \lim_{x\to a}{f(x)\F g(x)}=0.\cr
 f&\mathop{\sim}_ag(x)\ssi   \mbox{$f$ est \'equivalente \`a $g(x)$ en $a$}\ssi f(x)=g(x)+o_a\big(g(x)\big)
}
$$
Propri\'et\'e utile faisant le lien dans un cas simple (limites finies) entre les notions de limites, de DL, d'\'equivalents et de fonctions n\'egligeables : 
$$
\eqalign{
&\mbox{si $\ell\neq0$}\qquad \lim_{x\to a}f(x)=\ell\ \Longleftrightarrow\   f(x)=\ell+o_a(1)\ \Longleftrightarrow\   \mbox{$f$ est \'equivalente \`a $\ell$ en $a$}\cr
&\mbox{si $\ell=0$}\qquad \lim_{x\to a}f(x)=0\ \Longleftrightarrow\  f(x)=o_a(1)\ \Longleftrightarrow\  \mbox{$f$ est n\'egligeable devant $1$ en $a$}
}
$$
{\it Remarquer la subtile diff\'erence entre les cas $\ell\neq0$ et $\ell=0$} \qquad(danger !  boulettes ! ) \medskip
\medskip
\centerline{Op\'erations l\'egales}
\hrule
\medskip
{\it Remarquer qu'on ne peut pas faire grand chose avec les \'equivalents. 
On les utilisera surtout pour obtenir une approximation grossi\`ere ou pour avoir un ordre de grandeur d'une fonction (g\'en\'eralement en fin de raisonnement, apr\`es avoir utilis\'e limites et DL). }
\medskip
{\offinterlineskip
\tabskip=0pt
\halign{
\vrule\quad\hfil#\hfil\quad&\vrule\quad \hfil#\hfil\quad&\vrule\quad \hfil#\hfil\quad&\vrule\quad \hfil#\hfil\quad\vrule\cr
\noalign{\hrule}\cr
Op\'erations & Limites & DL ou D\'evelopement asymptotique& \'Equivalents\cr
\noalign{\hrule}\cr
$f(x)+g(x)$\vbox to 0.5cm{} & oui & oui & NON !!!!\cr
\noalign{\hrule}\cr
$f(x)-g(x)$ \vbox to 0.5cm{}& oui & oui & NON !!!!\cr\noalign{\hrule}\cr
$f\circ g(x)$\vbox to 0.5cm{} & oui & DL de $g$ en $a$ et de $f$ en $g(a)$ & NON !!!!\cr\noalign{\hrule}\cr
$\lambda f(x)$ \vbox to 0.5cm{}& oui & oui & oui\cr\noalign{\hrule}\cr
$f(x)*g(x)$ \vbox to 0.5cm{}& oui & oui & oui\cr\noalign{\hrule}\cr
$ {f(x)\F g(x)}$ \vbox to 0.5cm{}& Si $\ds\lim\limits_{x\to a}g(x)\neq 0$& Via le DL de $\ds{1\F 1-u}$& oui\cr\noalign{\hrule}\cr
}}


$$\hbox{A connaitre}
\Q\{\eqalign{
{1\F 1-x}&=1+x+x^2+\cdots+x^n+o_0(x^n)
\cr
\ln(1+x)&=x-{x^2\F 2}+\cdots+(-1)^{n+1}{x^n\F n}+o_0(x^n)
\cr
\e^x&=1+x+{x^2\F 2}+\cdots+{x^n\F n!}+o_0(x^n)
\cr
\sin(x)&=x-{x^3\F 6}+\cdots+(-1)^n{x^{2n+1}\F(2n+1)!}+o_0(x^{2n+1})
\cr
\cos(x)&=1-{x^2\F 2}+\cdots+(-1)^n{x^{2n}\F(2n)!}+o_0(x^{2n})
\cr
(1+x)^\alpha&=1+\alpha x+o_0(x).
}\W.
$$
\medskip
\centerline{Th\'eor\`eme de Taylor-Lagrange {\it(\`a utiliser en dernier recours seulement)}}
\hrule
\medskip
Si $f$ est de classe $\sc C^n$ sur un intervalle contenant $a$,  $f$ admet un DL \`a l'ordre $n$ en $a$  et 
$$\eqalign {
f(x)=&f(a)+f'(a)(x-a)+{f''(a)\F 2!}(x-a)^2+\cdots+{f^{(n)}(a)\F n!}(x-a)^n+o_a\Big((x-a)^n\Big)\cr &
=\sum_{0\le k\le n}{f^{(n)}(a)\F n!}(x-a)^n+o_a\Big((x-a)^n\Big)\qquad\qquad\hbox{{\bf (Taylor Lagrange)}}}
$$
\eject\pagetitretrue
\vglue-10mm\rightline{Sp\'e PT\hfill TD 4: D\'eveloppements limit\'es}%\hfill \date}
\bigskip
\bigskip\noindent
0) Calculer un d\'eveloppement \`a l'ordre $2$ des fonctions suivantes en $x={\pi\F 4}$ : 
$$
f(x)=\sin(x), \qquad g(x)=\e^x, \qquad h(x)=\ln(x), \qquad k(x)={1\F x}.
$$\bigskip
\Exercice{PThj}
\bigskip
\Exercice{PThk}
\bigskip
\Exercice{PTia}
\bigskip
\Exercice{PThz}
\vfill\null



% 
% \Chapter Indications, Indications.
% 
% \LD@Exo@Indication@Display
% 
% \Chapter Notions, Notions.
% 
% \LD@Exo@Notion@Display

%\Chapter Solutions, Solutions.

%\LD@Exo@Sol@Display

\bye