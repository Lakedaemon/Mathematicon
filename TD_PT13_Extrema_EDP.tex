%\def\Variables{MathsVariables}%
\catcode`@=11\relax
\def\Api{Mathematicon@Api}%
%%%% Newif
%\def\@firstofone#1{#1}

\newif\ifexonumber
%%%% Switches
\exonumberfalse
\catcode`@=11\relax
\input LD@Header.tex
\input LD.tex
\input LD@Typesetting.tex
\input LD@Exercices.tex
\input LD@Exercices.tex
\def\LD@Exercice@Display@Code{\eightpts}%
% debug tikz
\let\@firstofone\pgfutil@firstofone
\let\@ifnextchar\pgfutil@ifnextchar

\olspept
\DefineRGBcolor F0F9E3=VLGreen.
\DefineRGBcolor E5F9D1=LGreen.
\DefineRGBcolor DAF9BE=TGreen.
\DefineRGBcolor 5DA93B=Green.
\DefineRGBcolor F6DCCA=VLRed.
\DefineRGBcolor F6D4BD=LRed.
\DefineRGBcolor DAF9BE=TRed.
\DefineRGBcolor B5F9A1=TTRed.
\DefineRGBcolor F6B080=Red.
\DefineRGBcolor F9F5E3=VLOrange.
\DefineRGBcolor F9F5D0=LOrange.
\DefineRGBcolor DAF9BE=TOrange.
\DefineRGBcolor B5F9A1=TTOrange.
\DefineRGBcolor D7A93B=Orange.
\DefineRGBcolor EEEEEE=VLBlack.
\DefineRGBcolor DDDDDD=LBlack.
\DefineRGBcolor CCCCCC=TBlack.
\DefineRGBcolor B5F9A1=TTBlack.
\DefineRGBcolor 000000=Black.

\def\transparent{%
	\CS long\EC\def\Demonstration##1\CQFD{}%
}%
%\transparent
\def\Students{%
	\DefineRGBcolor FFFFFF=VLGreen.
	\DefineRGBcolor FFFFFF=LGreen.
	\DefineRGBcolor FFFFFF=TGreen.
	\DefineRGBcolor 000000=Green.
	\DefineRGBcolor FFFFFF=VLRed.
	\DefineRGBcolor FFFFFF=LRed.
	\DefineRGBcolor FFFFFF=TRed.
	\DefineRGBcolor FFFFFF=TTRed.
	\DefineRGBcolor 000000=Red.
	\DefineRGBcolor FFFFFF=VLOrange.
	\DefineRGBcolor FFFFFF=LOrange.
	\DefineRGBcolor FFFFFF=TOrange.
	\DefineRGBcolor FFFFFF=TTOrange.
	\DefineRGBcolor 000000=Orange.
	\DefineRGBcolor FFFFFFF=VLBlack.
	\DefineRGBcolor FFFFFF=LBlack.
	\DefineRGBcolor FFFFFF=TBlack.
	\DefineRGBcolor FFFFFF=TTBlack.
	\DefineRGBcolor 000000=Black.
}
\Students
\def\red{}
\def\blue{}
\def\Red#1{#1}%%%% Fix this !
\def\Blue#1{#1}%
\def\Font #1@#2pt{\font\olbi=cmr10\olbi}
\font\SvgText=cmr10\relax
%
%
%\catcode`@=11\relax
%\def\Api{Mathematicon@Api}%
%
%\input LD@Header.tex
%\input LD.tex
%\input LD@Exercices.tex
%\input LD@Typesetting.tex
%
%\catcode`@=11\relax
\font\LD@Font@Arial="Arial" at 10pt
%%%%%%%%%%%%%%%%%%%%%%%%%%%%%%%%%%%%%%%%%%%%%%%%%%%%%%%%%%%%%%%%%%
%															%
%					Extrema et équations aux dérivées partielles				%
%															%
%%%%%%%%%%%%%%%%%%%%%%%%%%%%%%%%%%%%%%%%%%%%%%%%%%%%%%%%%%%%%%%%%%
\newcount\LD@Count@Temp
\def\LD@Exercice@Display@Code{}%%\LD@Option@@Label\qquad\eightpts}%
\def\LD@Exercice@Display@Code@Post{%
	\ifcsname LD@Exo@@Solution\endcsname
		\unless\ifx\LD@Exo@@Solution\LD@Empty
			\pn{\eightpts Solution : \eightpts \LD@Exo@@Solution}%
		\fi
	\fi
}%
\def\LD@Display#1{%
	\LD@Count@Temp=#1\relax
	\ifcase\LD@Count@Temp
	\or
	Math. Sup.
	\or
	Math. Sp\'e
	\else
	\fi
}%
\newcount\LD@Exo@Total\LD@Exo@Total=0\relax

%%% TD 9.
\vglue-10mm\rightline{Sp\'e PT\hfill TD 13 : Extrema et \'equations aux d\'eriv\'ees partielles\hfill}%\date}
\bigskip

\centerline{\bf Pour localiser les extrema locaux, cherchez les points critiques ! }
\hrule
\bigskip\noindent

\Definition [$a\in D\subset\ob R^n$] 
Le point $a$ est un point critique d'une fonction $f:D\to \ob R$, si et~seulement~si $f$ admet des d\'eriv\'ees partielles en $a$, 
qui sont nulles, i. e. 
$$
\forall k\in\{1,\cdots, n\}, \qquad {\partial f\F\partial x_k}(a)=0.
$$

\Theoreme [$a\in U$ ouvert de $\ob R^n$] 
Si $f\in\sc C^1(U,\ob R)$ admet un extremum relatif en un point $a$ de l'{\bf ouvert} $U$, 
alors $a$ est un point critique de $f$.   
\medskip

\centerline{\bf Pour tester si un point critique est un extemum local}
\hrule
\medskip\noindent
{\bf Utiliser un DL, des in\'egalit\'es, l'astuce, etc... mais d'abord le th\'eor\`eme suivant  : }
\bigskip

\Theoreme [$U$ ouvert de $\ob R^2$, $a\in U$ point critique de $f\in\sc C^2(U,\ob R)$] 
Soient $r$, $s$ et  $t$ les nombres r\'eels d\'efinis par 
$$
\ds r={\partial^2f\F\partial x^2}(a),\qquad 
\ds s={\partial^2f\F\partial x\partial y}(a)\quad \hbox{ et }\quad 
\ds t={\partial^2f\F\partial y^2}(a). \leqno{(*)}
$$
Alors, quatre cas se pr\'esentent : 
\medskip
\noindent 
1) Si $rt-s^2>0$ et $r<0$, la fonction $f$ admet un maximum relatif en $a$. \pn
2) Si $rt-s^2>0$ et $r>0$, la fonction $f$ admet un minimum relatif en $a$. \pn
3) Si $rt-s^2<0$, la fonction $f$ n'admet pas d'extremum relatif en $a$ {\it (selle de cheval).} \pn
4) Si $rt-s^2=0$, on ne peut conclure. {\it Il faudrait affiner l'estimation de $f$ en $a$.} 



\vfill\noindent
\Exercice{PTajt}
\vfill\noindent
\Exercice{PTajq}
\vfill\noindent
\Exercice{PTajw}
\vfill \noindent
\Exercice{PTajs}
\vfill \noindent
\Exercice{PTajx}
\vfill \noindent
\Exercice{PTajr}
\vfill \noindent
\Exercice{PTakb}
\vfill \noindent
\Exercice{PTakc}
\vfill \noindent
\Exercice{PTajy}
\vfill\vfill\vfill\vfil\vfill\vfill\vfill\vfill\vfill\vfill\vfill\vfill\null
\bye