%\def\Variables{MathsVariables}%
\catcode`@=11\relax
\def\Api{Mathematicon@Api}%
%%%% Newif
%\def\@firstofone#1{#1}

\newif\ifexonumber
%%%% Switches
\exonumberfalse
\catcode`@=11\relax
\input LD@Header.tex
\input LD.tex
\input LD@Typesetting.tex
\input LD@Exercices.tex
\input LD@Exercices.tex
\def\LD@Exercice@Display@Code{\eightpts}%
% debug tikz
\let\@firstofone\pgfutil@firstofone
\let\@ifnextchar\pgfutil@ifnextchar

\olspept
\DefineRGBcolor F0F9E3=VLGreen.
\DefineRGBcolor E5F9D1=LGreen.
\DefineRGBcolor DAF9BE=TGreen.
\DefineRGBcolor 5DA93B=Green.
\DefineRGBcolor F6DCCA=VLRed.
\DefineRGBcolor F6D4BD=LRed.
\DefineRGBcolor DAF9BE=TRed.
\DefineRGBcolor B5F9A1=TTRed.
\DefineRGBcolor F6B080=Red.
\DefineRGBcolor F9F5E3=VLOrange.
\DefineRGBcolor F9F5D0=LOrange.
\DefineRGBcolor DAF9BE=TOrange.
\DefineRGBcolor B5F9A1=TTOrange.
\DefineRGBcolor D7A93B=Orange.
\DefineRGBcolor EEEEEE=VLBlack.
\DefineRGBcolor DDDDDD=LBlack.
\DefineRGBcolor CCCCCC=TBlack.
\DefineRGBcolor B5F9A1=TTBlack.
\DefineRGBcolor 000000=Black.

\def\transparent{%
	\CS long\EC\def\Demonstration##1\CQFD{}%
}%
%\transparent
\def\Students{%
	\DefineRGBcolor FFFFFF=VLGreen.
	\DefineRGBcolor FFFFFF=LGreen.
	\DefineRGBcolor FFFFFF=TGreen.
	\DefineRGBcolor 000000=Green.
	\DefineRGBcolor FFFFFF=VLRed.
	\DefineRGBcolor FFFFFF=LRed.
	\DefineRGBcolor FFFFFF=TRed.
	\DefineRGBcolor FFFFFF=TTRed.
	\DefineRGBcolor 000000=Red.
	\DefineRGBcolor FFFFFF=VLOrange.
	\DefineRGBcolor FFFFFF=LOrange.
	\DefineRGBcolor FFFFFF=TOrange.
	\DefineRGBcolor FFFFFF=TTOrange.
	\DefineRGBcolor 000000=Orange.
	\DefineRGBcolor FFFFFFF=VLBlack.
	\DefineRGBcolor FFFFFF=LBlack.
	\DefineRGBcolor FFFFFF=TBlack.
	\DefineRGBcolor FFFFFF=TTBlack.
	\DefineRGBcolor 000000=Black.
}
\Students
\def\red{}
\def\blue{}
\def\Red#1{#1}%%%% Fix this !
\def\Blue#1{#1}%
\def\Font #1@#2pt{\font\olbi=cmr10\olbi}
\font\SvgText=cmr10\relax
%
%
%\catcode`@=11\relax
%\def\Api{Mathematicon@Api}%
%
%\input LD@Header.tex
%\input LD.tex
%\input LD@Exercices.tex
%\input LD@Typesetting.tex
%
%\catcode`@=11\relax
\font\LD@Font@Arial="Arial" at 10pt
%%%%%%%%%%%%%%%%%%%%%%%%%%%%%%%%%%%%%%%%%%%%%%%%%%%%%%%%%%%%%%%%%%
%															%
%					 Intégrales généralisées à un paramètre					%
%															%
%%%%%%%%%%%%%%%%%%%%%%%%%%%%%%%%%%%%%%%%%%%%%%%%%%%%%%%%%%%%%%%%%%
\newcount\LD@Count@Temp
\def\LD@Exercice@Display@Code{}%%\LD@Option@@Label\qquad\eightpts}%
\def\LD@Exercice@Display@Code@Post{%
	\ifcsname LD@Exo@@Solution\endcsname
		\unless\ifx\LD@Exo@@Solution\LD@Empty
			\pn{\eightpts Solution : \eightpts \LD@Exo@@Solution}%
		\fi
	\fi
}%
\def\LD@Display#1{%
	\LD@Count@Temp=#1\relax
	\ifcase\LD@Count@Temp
	\or
	Math. Sup.
	\or
	Math. Sp\'e
	\else
	\fi
}%
\newcount\LD@Exo@Total\LD@Exo@Total=0\relax

%%% TD 14.
\vglue-10mm\rightline{Sp\'e PT\hfill TD 14 : S\'eries\hfill}%\date}
\bigskip
\vfill
%
%\centerline{Convergence des s\'eries}
%\hrule
%\medskip\noindent
%
%
%\Definition [$(u_k)_{k\in\ob N}$ suite complexe] 
%La s\'erie $\sum_{k=0}^\infty u_k$ converge vers $S\in\ob C\Leftrightarrow$ 
%la suite des sommes partielles d\'efinie par 
%$$
%\forall n\in\ob N, \qquad S_n:=\sum_{k=0}^nu_k 
%$$
%converge vers un nombre complexe $S$,  qui est appel\'e somme de la s\'erie. On \'ecrit alors  
%$$
%\sum_{k=0}^\infty u_k=\sum_{k\ge0}u_k=\lim_{n\to\infty}\sum_{0\le k\le n}u_k=\lim_{n\to\infty}S_n=S. 
%$$ 
%Si la s\'erie de terme g\'en\'eral $(u_n)$ ne converge vers aucun nombre $S\in\ob K$, i.e. si la suite~$(S_n)$ des sommes partielles de $(u_n)$ n'admet aucune limite finie, 
%on dit que la s\'erie diverge. 
%\bigskip
%\Remarque : si l'on sait calculer les sommes  $S_n$, \'etudier la nature de $\sum_{k=0}^\infty u_k$ est facile. \bigskip



\centerline{S\'eries t\'elescopiques}
\hrule
\medskip\noindent

\Propriete [$(u_n)_{n\in\ob N}$ suite complexe]
La s\'erie $u_0+\sum_{n=1}^\infty(u_n-u_{n-1})$ converge $\ssi$ la suite 
$(u_n)_{n\in\ob N}$ converge. De plus, en~cas de convergence, on a 
\Equation [\bf S\'eries t\'elescopiques]
$$
u_0+\sum_{n=1}^\infty(u_n-u_{n-1})=\lim_{n\to\infty}u_n. 
$$
{\it lorsque les sommes partielles sont t\'el\'escopiques, on en trouve facilement une expression et la limite}. 

\noindent
\Exercice{PTpm}
\vfill
\noindent
\Exercice{PTpn}
\vfill
\noindent
\Exercice{PTpl}
\vfill

\centerline{Th\'eor\`eme special des s\'eries altern\'ees}
\hrule
\medskip\noindent


\Theoreme [$(u_n)_{n\in\ob N}$ suite r\'eelle altern\'ee]
Si la suite $|u_n|$ est d\'ecroissante, de limite nulle et si $(-1)u_n$ est de signe constant \`a partir d'un certain rang, 
alors la s\'erie $\sum_{n=0}^\infty u_n$ converge et 
\Equation [\bf Th\'eor\`eme sp\'ecial]
$$
\forall k\ge0, \qquad  \sum_{n=k}^\infty u_n \sbox { est  compris entre } 0 \sbox{ et }u_k \sbox{ inclus}.
$$ 

\bigskip
\Exercice{PTpk}
\bigskip
\noindent
\Exercice{PTpj}
\bigskip
\noindent
\Exercice{PTon}
\bigskip
\Exercice{PTom}
\bigskip
\Exercice{PToq}
\bigskip
\Exercice{PTama}
\medskip\noindent
\centerline{Convergence absolue}
\hrule
\medskip\noindent

\Propriete [$(u_n)_{n\in\ob N}$ suite complexe] 
Si la s\'erie $\sum_{n=0}^\infty|u_n|$ converge, alors la s\'erie $\sum_{n=0}^\infty u_n$ converge et l'on a 
\Equation [\bf Convergence absolue]
$$
\Q|\sum_{n=0}^\infty u_n\W|\le \sum_{n=0}^\infty|u_n|. 
$$

\Theoreme [$f:[0,\infty[\to\ob R$ continue par morceaux sur $[0,\infty[$]
Si $f$ est positive et d\'ecroissante sur $[0, \infty[$, alors l'int\'egrale $\int_0^\infty f(t)\d t$ 
et la s\'erie $\sum_{n=0}^\infty f(n)$ ont la m\^eme nature. 

\centerline{Exercices fun sur les s\'eries}
\hrule
\medskip\noindent
\bigskip
\Exercice{PTux}
\bigskip
\Exercice{PTpi}
\vfill\null











\bye