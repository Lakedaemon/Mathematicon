%%%%%       Banque d'exercices 
%
\catcode`@=11\relax

\exo [Level=1,Fight=3,Learn=3,Field=\Groupes,Type=\Colles,Origin=\MP] a. 
Soit $G$ un groupe tel que $x^2=\e$ pour $x\in G$. \pn 
a) Démontrer que $G$ est abélien. \pn 
b) Soient $H\neq G$ un sous-groupe de $G$ et $a\in G\ssm H$. 
Que peut on dire de $H\cup aH$ ? \pn
c) Lorsque $G$ est fini, démontrer que son cardinal est une puissance de $2$. 

\exo [Level=1,Fight=2,Learn=1,Field=\RécurrencesLinéaires,Type=\Colles,Origin=\MP,Solution={$u_n=-1+{5+3\sqrt 5\F10}\Q({1+\sqrt5\F 2}\W)^n+{5-3\sqrt 5\F10}\Q({1-\sqrt5\F 2}\W)^n$},Indication={Résoudre la récurrence linéaire sans second membre, puis trouver une solution particulière constante.}] b. 
Calculer la suite $(u_n)_{n\in\ob N}$ 
définie par $u_0=0$, $u_1=1$ et 
$$
u_{n+2}=u_{n+1}+u_n+1\qquad(n\in\ob N).
$$

\exo [Level=1,Fight=4,Learn=1,Field=\Matrices,Type=\Others,Origin=\MP] c. 
Soit $A\in\sc M_n(\ob C)$ une matrice de trace nulle. Démontrer que $A$ 
est semblable à une matrice dont tous les éléments diagonaux sont nuls. 

\exo [Level=1,Fight=2,Learn=1,Field=\DéveloppementsLimités,Type=\Colles,Origin=\MP] d. 
Soit $f\in\sc C^1\b([0,1],\ob R\b)$ telle que $f(0)=0$. 
Déterminer la limite $\ds \lim\limits_{n\to\infty}\sum_{k=1}^nf\Q({k\F n^2}\W)$.  

\exo [Level=1,Fight=3,Learn=1,Field=\DéveloppementsLimités,Type=\Colles,Origin=\MP] e. 
Soit $f\in\sc C\b([0,1],\ob R\b)$ une fonction 
pour laquelle la limite $\lim_{t\to0^+}{f(2t)-f(t)\F t}$ existe. 
Démontrer que $f$ est dérivable en~$0$. 

\exo [Level=2,Fight=3,Learn=2,Field=\FonctionsDéfiniesParUneIntégrale,Type=\Colles,Origin=\MP] f. 
Soit $(a,b)\in\ob R^2$ tel que $a<b$. On pose $f(x):=\Q\{
\eqalign{0&\hbox{ si } x\notin\Q]a,b\W[\cr
\ds\exp{1\F(x-a)(x-b)}&\hbox{ si }x\in\Q]a,b\W[
}\W.$. 
\vskip-1em\noindent
Prouver que $f\in\sc C^\infty(\ob R)$.

\exo [Level=2,Fight=4,Learn=1,Field=\Dérivation,Type=\Others,Origin=] g. %% A faire %%
Démontrer que $\ds{\d^{n+1}\hfill\F\d x^{n+1}}\b(x^n\ln(1+x)\b)
=\sum_{0\le k\le n}{n!\F(1+x)^{k+1}}$ pour $n\in\ob N$ et $x>-1$. 

\exo [Level=1,Fight=3,Learn=2,Field=\DéveloppementsLimités,Type=\Convexite,Origin=\MP] h. 
Soient $f:[0,1]\to\ob R$ une fonction convexe et $a\in\Q]0,1\W[$. \pn
a) Démontrer que $f$ est continue en $a$. \pn
b) Démontrer que $f$ admet en $a$ une dérivée à gauche et à droite. 

\exo [Level=2,Fight=2,Learn=2,Field=\Normes,Type=\Cours,Origin=\MP] i. 
Soit $N:\sc M_n(\ob R)\to\ob R$ l'application définie par 
$N(A):=\sqrt{\mbox{\rm tr}(\NULL^{\mbox{\rm t}}AA)}$ pour $A\in\sc M_n(\ob R)$. \pn
a) Démontrer que $N$ est une norme de $\sc M_n(\ob R)$. \pn
b) Pour $(A,B)\in\sc M_n(\ob R)^2$, prouver que $N(AB)\le N(A)N(B)$. 

\exo [Level=2,Fight=2,Learn=1,Field=\Matrices,Type=\Colles,Origin=\MP] j. 
Soit $n\ge3$ un entier et soit $M_n(X)$ la matrice $\pmatrix{
X & 1 & 0 & \ldots & 0 \cr
{X^2\F2!}& X & 1 & 0 & \vdots\cr
\vdots & \ddots& \ddots& \ddots & 0\cr
\vdots & & \ddots& \ddots &1\cr
{X^n\F n!} & {X^{n-1}\F(n-1)!} & \ldots & {X^2\F 2!}& X 
}
$. 
\vskip-3em \noindent  
Calculer $\det M_n'(X)$ puis $\det M_n(X)$. 
\vskip3em

\exo [Level=1,Fight=3,Learn=1,Field=\Intégration,Type=\Colles,Origin=\MP] k. 
Soit $f\in\sc C\b([0,1],\ob R\b)$ une fonction 
telle que $\int_0^1f(t)\d t=\int_0^1tf(t)\d t=\int_0^1t^2f(t)\d t=0$. \pn
Quel est le nombre minimal de zéros de $f$ sur $[0,1]$ ? 

\exo [Level=1,Fight=2,Learn=1,Field=\Intégration,Type=\Colles,Origin=\MP] l. 
Déterminer $\lim_{n\to\infty}\int_0^1\b|\sin(nt)\b|\d t$. 

\exo [Level=1,Fight=2,Learn=2,Field=\Intégration,Type=\Colles,Origin=\MP] m. 
Soit $\{I_n\}_{n\in\ob N}$ la suite définie par $I_n:=\int_0^1\ln(1+t^n)\d t$ 
pour $n\in\ob N$. \pn
a) Démontrer que $\lim_{n\to\infty}I_n=0$. \pn
b) Lorsque $n\to+\infty$, démontrer que $I_n\sim K/n$ pour une certaine constante $K>0$. 

\exo [Level=2,Fight=3,Learn=2,Field=\Dérivation,Type=\Colles,Origin=\MP] n. 
Soit $f\in\sc C(\ob R,\ob C)$ une fonction 
vérifiant $\forall(x,y)\in\ob R^2, \ds2yf(x)=\int_{x-y}^{x+y}f(t)\d t$ 
\vskip-0.7em\noindent
a) Démontrer que $f$ est de classe $\sc C^\infty$. \pn
b) Déterminer $f$. 

\exo [Level=1,Fight=2,Learn=2,Field=\Intégration,Type=\Colles,Origin=\MP,Indication={On pourra calculer
$J_n:=I_{n+1}-I_n$ et $K_n:=J_{n+1}-J_n$ pour $n\in\ob N$.}] o. 
Pour chaque entier $n\ge0$, déterminer $I_n:=\ds\int_0^\pi{\sin^2(nt)\F\sin^2t}\d t$. 

\exo [Level=1,Fight=3,Learn=1,Field=\Intégration,Type=\Colles,Origin=\MP] p. 
Soit $f\in\sc C^1\b([a,b],\ob C\b)$ telle que $f(a)=f(b)=0$ 
et soit $M_1:=\sup_{a\le t\le b}\b|f'(t)\b|$. \pn
a) Démontrer que $\b|\int_a^xf(t)\d t\b|\le {M_1\F2}(x-a)^2$ pour $a\le x\le b$. \pn
b) En déduire que $\b|\int_a^bf(t)\d t\b|\le {M_1\F4}(b-a)^2$. 

\exo [Level=1,Fight=4,Learn=2,Field=\Intégration,Type=\Others,Origin=\MP] q. 
Soit $f\in\sc C^3\b([0,1],\ob C\b)$. Déterminer la limite $\ds\lim_{n\to\infty}n^2\Big(\int_0^1f(x)\d x-{1\F n}\sum_{0\le k<n}f\Q({k\F n}\W)\Big)$. 

\exo [Level=1,Fight=3,Learn=2,Field=\Intégration,Type=\Colles,Origin=\MP] r. 
Soit $f\in\sc C^2(\ob R,\ob C)$ une fonction telle que $f$ et $f''$ soient bornées. \pn
a) Posant $M_0:=\sup_{x\in\ob R}|f(x)|$ et $M_2:=\sup_{x\in\ob R}|f''(x)|$, 
démontrer que $|f'(t)|\le {2M_0\F a}+{a\F2}M_2$ pour $a>0$ et $t\in\ob R$. \pn
b) Posant $M_1:=\sup_{x\in\ob R}|f'(x)|$, en d'eduire que $M_1\le 2\sqrt{M_0M_2}$. 


\exo [Level=1,Fight=2,Learn=1,Field=\Intégration,Type=\Exercices,Origin=\MP] s. 
Déterminer $\ds \lim_{n\to\infty}\int_0^1{n\e^{-x}+x^2\F n+x}\d x$. 

\exo [Level=1,Fight=1,Learn=1,Field=\Intégration,Type=\Exercices,Origin=\MP] t. 
Soit $f\in\sc C^2([0,1],\ob C)$. 
Donner un développement limité à l'ordre $2$, selon les puissances de $1/n$, 
de la suite de terme général $u_n:=\int_0^1t^nf(t)\d t\ \,(n\in\ob N)$. 

\exo [Level=1,Fight=3,Learn=2,Field=\Intégration,Type=\Others,Origin=\MP] u. 
Soit $f\in\sc C\b([a,b],\ob R^+\b)$. Démontrer que 
$$
\lim_{n\to\infty}\Q(\int_a^bf(t)^n\d t\W)^{1/n}=\sup_{a\le x\le b}|f(x)|. 
$$

\exo [Level=2,Fight=1,Learn=1,Field=\FonctionsDéfiniesParUneIntégrale,Type=\Colles,Origin=\MP] v. 
Soient $\alpha>0$ et $F_\alpha$ l'application définie 
par $F_\alpha(x):=\int_0^\pi\exp\b(-x^\alpha\sin t\b)\d t$ 
pour $x\ge 0$. \pn 
Allure des courbes représentatives des fonctions $F_\alpha$. 

\exo [Level=2,Fight=3,Learn=2,Field=\FonctionsDéfiniesParUneIntégrale,Type=\TravauxDirigés,Origin=\MP,Indication={c) étudier $I(1/x)$.}] w. 
Pour chaque nombre réel $x\notin\{-1,1\}$, 
on pose $I(x):=\int_{-\pi}^\pi\ln(1-2x\cos t+x^2)\d t$. \pn
a) Vérifier que $I$ est bien définie pour $|x|\neq1$. \pn 
b) Calculer $I'(x)$ puis $I(x)$ pour $x\in\Q]-1,1\W[$. \pn
c) Calculer $I(x)$ pour $|x|>1$. 

\exo [Level=2,Fight=4,Learn=2,Field=\IntégralesGénéralisées,Type=\Cours,Origin=] x. 
Convergence et calcul de $\int_1^\infty\ds{\d x\F x\sqrt{x^{10}+x^5+1}}$. 

\exo [Level=2,Fight=1,Learn=1,Field=\IntégralesGénéralisées,Type=\TravauxDirigés,Origin=] y. 
Représentez l'ensemble des points $(\alpha,\beta)$ du plan pour lesquels 
$\ds\int_0^1{\d x\F|1-x^\alpha|^\beta}$ converge, avec $\alpha\neq0$. 

\exo [Level=1,Fight=3,Learn=2,Field=\Intégration,Type=\Exercices,Origin=\MP] z. 
Soit $f:[a,b]\to\ob C$ une application. Démontrer que $\lim_{n\to\infty}\int_a^bf(t)\e^{int}\d t=0$. \pn
a) En supposant que $f$ est constante par morceaux. \pn
b) En supposant que $f$ est de classe $\sc C^1$ par morceaux. \pn
c) {\it Théorique et difficile : }en supposant que $f$ est continue par morceaux.  

\exo [Level=2,Fight=3,Learn=3,Field=\IntégralesGénéralisées,Type=\TravauxDirigés,Origin=] aa. 
Pour chaque entier $n\ge0$, on pose $a_n:=\int_0^{\pi/2}{\sin(2nx+x)\F\sin x }\d x$ et 
$b_n:=\int_0^{\pi/2}{\sin(2nx+x)\F x }\d x$. \pn
a) Étudier $a_{n+1}-a_n$ puis calculer $a_n$. \pn
b) Déterminer $\lim_{n\to\infty}b_n$ 
en utilisant le résultat de l'exercice \eqrefn{labelexoPTz} et la fonction $g:[0,1]\to\ob R$ 
définie par $g(x):={1\F\sin x}-{1\F x}$ pour $0<x\le \pi/2$ et par $g(0):=0$. \pn
c) En déduire que $\int_0^\infty{\sin x\F x}\d x={\pi/2}$. 

\exo [Level=2,Fight=1,Learn=1,Field=\IntégralesGénéralisées,Type=\Exercices,Origin=] ab. 
Convergence et calcul de $\int_0^\infty\int_x^\infty\exp(-t^2)\d t\d x$. 

\exo [Level=2,Fight=3,Learn=1,Field=\IntégralesGénéralisées,Type=\Colles,Origin=\MP] ac. 
Soit $f\in\sc C(\ob R^+,\ob R^+)$ une fonction décroissante telle que $\int_0^\infty f(x)\d x$ converge. \pn
Démontrer que $\lim_{x\to\infty}xf(x)=0$. 

\exo [Level=2,Fight=3,Learn=2,Field=\IntégralesGénéralisées,Type=\Colles,Origin=\MP] ad. 
Soit $f\in\sc C\Q(\Q]0,1\W],\ob R\W)$ une fonction décroissante 
telle que $I:=\int_0^1 f(x)\d x$ converge. \pn
Démontrer que $\lim_{n\to\infty}{1\F n}\sum_{1\le k\le n}f(k/n)=I$. 

\exo [Level=2,Fight=1,Learn=1,Field=\FonctionsDéfiniesParUneIntégrale,Type=\Cours,Origin=\MP] ae. 
Soit $f\in\sc C(\ob R^+,\ob C)$ telle que l'application $t\mapsto f(t)\e^{-\alpha t}$ 
soit bornée sur $\ob R^+$ pour chaque $\alpha>0$. \pn
Démontrer que l'application $F:x\mapsto\int_0^\infty f(t)\e^{-xt}\d t$ 
est définie et de classe $\sc C^\infty$ sur $\Q]0,\infty\W[$. 

\exo [Level=2,Fight=2,Learn=2,Field=\FonctionsDéfiniesParUneIntégrale,Type=\Exercices,Origin=\MP] af. 
Pour chaque nombre réel $x>0$, on pose $f(x):=\int_1^\infty{t^{-x}\F 1+t}\d t$. \pn
a) Démontrer que $f$ est définie, continue et décroissante sur $\Q]0,\infty\W[$. \pn
b) Relation entre $f(x)$ et $f(x+1)$ ? \pn
c) Equivalents de $f$ en $0^+$ et en $+\infty$ ?

\exo [Level=2,Fight=4,Learn=3,Field=\FonctionsDéfiniesParUneIntégrale,Type=\Colles,Origin=\MP] ag. 
Pour chaque nombre réel $x>0$, on pose $f(x):=\int_0^1\ds{\d t\F\sqrt{t(1-t)(t+x)}}$. \pn
a) Démontrer que $f$ est définie et continues sur $\Q]0,\infty\W[$. \pn
b) Démontrer que $f(x)\sim \pi/\sqrt x$ lorsque $x\to+\infty$. \pn
c) Démontrer que $f(x)\sim\int_0^1\ds{\d t\F\sqrt t(t+x)}\d t\sim-\ln x$ lorsque $x\to0^+$. 

\exo [Level=2,Fight=2,Learn=1,Field=\Séries,Type=\Colles,Origin=\MP] ah. 
Nature de la série $\sum_{n=1}^\infty\rho(n)^\alpha/n$ où $\rho(n)$ désigne le nombre de chiffres de $n$. 

\exo [Level=2,Fight=2,Learn=1,Field=\Séries,Type=\Colles,Origin=\MP] ai. 
Soient $\{u_n\}_{n\in\ob N}$ et $\{v_n\}_{n\in\ob N}$ deux suites réelles positives. \pn
a) Démontrer que la série $\sum_{n\in\ob N}\sqrt{u_nv_n}$ converge 
si les séries $\sum_{n\in\ob N}u_n$ et $\sum_{n\in\ob N}v_n$ convergent. \pn
b) Lorsque $v_n+n^2u_nv_n=1\ \,(n\in\ob N)$, montrer que les séries de terme général $u_n$ et $v_n$ ne peuvent converger simultanément. 

\exo [Level=2,Fight=3,Learn=1,Field=\Séries,Type=\Colles,Origin=\MP] aj. 
Soient $(a,b)\in\Q]0,\infty\W[^2$ et $\{\alpha_n\}_{n\ge0}$ 
la suite définie par $u_n:=\sum_{0\le k\le n}1/(k+b)$ pour $n\in\ob N$. \pn 
Déterminer la nature de la série de terme général $a^{u_n}$. 

\exo [Level=2,Fight=2,Learn=3,Field=\Séries,Type=\TravauxDirigés,Origin=\MP] ak. 
Soient $u_0\in\Q]0,1\W[$ et $\{u_n\}_{n\ge1}$ la suite définie par $u_{n+1}:=u_n-u_n^2$ pour $n\ge1$. \pn
a) Démontrer que $\lim_{n\to\infty}u_n=0$. \pn
b) Nature des séries de terme général $u_n^2$, $\ln(1-u_n)$ et $u_n$. 
 
\exo [Level=2,Fight=1,Learn=1,Field=\Séries,Type=\Exercices,Origin=\MP] al. 
Soit $\{u_n\}_{n\ge1}$ la suite définie par $u_n:=2\sqrt n-\sum_{1\le k\le n}1/\sqrt k$ pour $n\ge1$. 
En étudiant la série de terme général $u_{n+1}-u_n$ montrer que la suite $\{u_n\}_{n\ge1}$ converge vers un nombre réel $\ell$ 
et fournir un équivalent simple de $u_n-\ell$. 

\exo [Level=2,Fight=2,Learn=2,Field=\Séries,Type=\Colles,Origin=\MP] am. 
Soient $\{u_n\}_{n\ge1}$ une suite positive et $\{v_n\}_{n\ge1}$ 
la suite définie par $v_n=u_n\prod_{1\le k\le n}(1+u_k)^{-1}$ pour $n\ge1$. \pn
a) Démontrer que la série $\sum_{n\ge1}v_n$ converge. \pn
b) Démontrer que $\sum_{n\ge1}v_n<1$ si $\sum_{n\ge1}u_n$ converge 
et que $\sum_{n\ge1}v_n=1$ sinon. 

\exo [Level=2,Fight=1,Learn=1,Field=\DéveloppementsLimités,Type=\Exercices,Origin=\MP] an. 
Démontrer que $\ds \sum_{2\le k\le n}{1\F\ln k}\sim {n\F\ln n}$ lorsque $n\to+\infty$. 

\exo [Level=2,Fight=2,Learn=2,Field=\Séries,Type=\Colles,Origin=\MP] ao.
Nature de la série $\ds\sum_{n=2}^\infty{(-1)^n\F n+(-1)^nn/\ln n}$.

\exo [Level=2,Fight=2,Learn=2,Field=\Séries,Type=\Colles,Origin=\MP] ap. 
Soit $\{u_n\}_{n\ge1}$ une suite complexe telle que $\sum_{n\ge1}u_n$ converge 
et soit $\{v_n\}_{n\ge 1}$ la suite définie par $v_n:=\sum_{1\le k\le n}ku_k$ pour $n\ge1$. \pn
a) Démontrer que $\lim_{n\to\infty}{v_n\F n}=0$. \pn
b) Convergence et somme de la série $\sum_{n\ge1}{v_n\F n^2+n}$. 

\exo [Level=2,Fight=1,Learn=1,Field=\SériesEntières,Type=\Exercices,Origin=] aq. 
Rayon de convergence $R$ de la série entière $\sum_{n\ge1}n^{\cos(n\pi)}x^n$. 

\exo [Level=2,Fight=0,Learn=0,Field=\SériesEntières,Type=\Exercices,Origin=] ar. 
Développez la fonction $f(x)={x^2+1\F(x+1)^3}$ en série entière au voisinage de $0$. 

\exo [Level=2,Fight=2,Learn=2,Field=\SériesEntières,Type=\TravauxDirigés,Origin=] as. 
Soit $f$ l'unique fonction $f\in\sc C^1(\ob R)$ vérifiant $f(0)=0$ 
et $f'(x)-2xf(x)=1$ pour $x\in\ob R$. \pn
a) Prouvez que $f$ est de classe $\sc C^\infty$ sur $\ob R$. \pn
b) Développer $f$ en série entière de deux manières différentes.\pn 
c) En déduire que $\ds\sum_{0\le k\le n}(-1)^k{c_n^k\F 2k+1}={2.4\ldots (2n)\F3.5\ldots(2n+1)}$. 

\exo [Level=2,Fight=2,Learn=1,Field=\SériesEntières,Type=\Exercices,Origin=] at. 
Déterminer la suite réelle définie par $u_0=1$ 
et $u_{n+1}=\sum_{0\le k\le n}u_ku_{n-k}$ pour $n\ge0$. 

\exo [Level=2,Fight=3,Learn=1,Field=\SériesEntières,Type=\Colles,Origin=] au. 
Déterminer un équivalent en $1^-$ de la série entière 
$\sum_{n\ge0}x^{n^2}$.  

\exo [Level=2,Fight=3,Learn=1,Field=\SériesEntières,Type=\Others,Origin=] av. 
Soit $a\in\Q]-1,1\W[$. Pour chaque nombre réel $x$, on pose $f(x):=\sum_{n\ge0}\sh(a^nx)$. \pn
a) Démontrer que $f$ est bien définie sur $\ob R$. \pn
b) Développer $f$ en série entière. 
  

\exo [Level=2,Fight=2,Learn=1,Field=\SériesEntières,Type=\Colles,Origin=] aw. 
Soit $f:\Q]-\pi/2,\pi/2\W[$ l'application définie par $f(x):=1/(\cos x)$ pour $|x|<\pi/2$. \pn
a) Démontrer qu'il existe une suite $\{P_n\}_{n\ge0}$ de polynômes à coefficients positifs tels que 
$$
f^{(n)}(x)={P_n(\cos x)\F\cos^{n+1}x}\qquad(n\ge0,-\pi/2<x<\pi/2). 
$$
b) Calculez $P_n(1)$ \pn
c) Démontrez que $f$ est développable en série entière dans un voisinage de $0$. 

\exo [Level=2,Fight=0,Learn=0,Field=\SériesEntières,Type=\Exercices,Origin=] ax. 
Rayon de convergence et somme de $\sum_{n\in\ob N}x^{3n}/(3n)!$. 

\exo [Level=2,Fight=0,Learn=0,Field=\SériesDeFourier,Type=\Exercices,Origin=] ay. 
Soit $f$ l'application 
définie par $f_\alpha(x):=\b|\sin (x)\b|$ pour $x\in\ob R$. \pn
a) Série de Fourier de $f$ \pn
b) Montrer que 
$$
|\sin x|={8\F\pi}\sum_{n\ge1}{\sin^2nx\F4n^2-1}\qquad(x\in\ob R).
$$

\exo [Level=2,Fight=1,Learn=0,Field=\SériesDeFourier,Type=\Exercices,Origin=] az. 
Soit $\alpha\in\Q]0,1\W[$ et soit $f_\alpha$ une fonction 
$2\pi$-périodique telle que $f_\alpha(x)=\e^{-i\alpha x}$ pour $-\pi<x<\pi$. \pn
a) Série de Fourier de $f_\alpha$. \pn
b) Calculer $A_\alpha:=\sum_{n\in\ob Z}1/(n+\alpha)^2$ et $B_\alpha:=\sum_{n\in\ob Z}1/(n+\alpha)^3$.

\exo [Level=2,Fight=1,Learn=0,Field=\SériesEntières,Type=\Exercices,Origin=] ba. 
Soit $f(z):=\sum_{n\ge0}a_nz^n$ une série entière de rayon de convergence $R>0$. Prouver que 
$$
{1\F2\pi}\int_0^{2\pi}\Q|f\Q(r\e^{i\theta}\W)\W|^2\d\theta=\sum_{n\ge0}|a_n|^2r^{2n}\qquad(0<r<R).
$$

\exo [Level=2,Fight=1,Learn=1,Field=\SériesDeFourier,Type=\Exercices,Origin=] bb. 
Soit $f\in\sc C^1(\ob R,\ob R)$ une fonction $2\pi$-périodique 
telle que $\int_0^{2\pi}f(t)\d t=0$. \pn
a) Démontrer que $\int_0^{2\pi}f(t)^2\d t\le \int_0^{2\pi}f'(t)^2\d t$. \pn
b) Cas dégalité ?

\exo [Level=2,Fight=2,Learn=1,Field=\SériesDeFonctions,Type=\Exercices,Origin=] bc. 
Soit $a>0$, soit $f$ une fonction paire, $2\pi$-périodique, telle que $f(x)=a^2x^2/2$ pour $0\le x\le \pi$ 
et soit $g:\ob R\to\ob R$ l'application définie par 
$$
g(x):=2a^2\sum_{n\ge1}(-1)^n{\cos nx\F n^2+a^2}\qquad(x\in\ob R).
$$ 
a) Série de Fourier de $g$. \pn
b) Démontrer que $g-f$ est deux fois dérivable sur $\ob R$. \pn
c) Étudier $g''-f''$ et en déduire une équation différentielle 
satisfaite par $g$ sur $\Q]-\pi,\pi\W[$. \pn
d) Calculer $A:=\sum_{n\ge0}(-1)^n/(n^2+a^2)$ et $B:=\sum_{n\ge0}1/(n^2+a^2)$. 

\exo [Level=2,Fight=1,Learn=1,Field=\SériesDeFourier,Type=\Exercices,Origin=,Indication=Utiliser le résultat a) de l'exercice \eqrefn{labelexoPTay}.] bd. 
Résoudre sur $\ob R$ l'équation différentielle $y''-4y'+4y=|\sin x|$ 
en cherchant une solution particulière deux fois dérivable, 
développable en série trigonométrique. 


\exo [Level=2,Fight=1,Learn=0,Field=\SystèmesDifférentiels,Type=\Exercices,Origin=] be. 
Résoudre sur $\ob R$ le système différentiel : 
$\ds \Q\{\eqalign{
x'(t)
=&x(t)-y(t)+\e^{2t}\cr
y'(t)=&x(t)+3y(t)+t
}\W.
$

\exo [Level=2,Fight=1,Learn=0,Field=\SystèmesDifférentiels,Type=\Exercices,Origin=] bf. 
Résoudre sur $\Q]0,\infty\W[$ le système différentiel 
$\ds
\Q\{\eqalign{
x'(t)=&4x(t)+2z(t)
\cr
y'(t)=&-2x(t)+3y(t)-z(t)
\cr
z'(t)=&-2x(t)-z(t)
\cr
u'(t)=&4x(t)+2z(t)+3u(t)
}\W.
$

\exo [Level=2,Fight=2,Learn=1,Field=\SystèmesDifférentiels,Type=\Exercices,Origin=] bg. 
Résoudre dans $\sc C^2(\ob R,\ob R)$ l'équation $f''(x)+f(-x)=x+\cos x$. 

\exo [Level=2,Fight=0,Learn=0,Field=\EquationsDifférentiellesLinéairesDuSecondOrdre|\Intégration,Type=\Colles,Origin=] bh. 
Déterminer les applications $f\in\sc C(\ob R,\ob R)$ vérifiant 
$$
f(x)f(y)=\int_{x-y}^{x+y}f(t)\d t. 
$$

\exo [Level=2,Fight=2,Learn=1,Field=\SériesEntières,Type=\Exercices,Origin=] bi. 
Soit $S:\Q]-R,R\W[\to\ob C$ une série entière 
de rayon de convergence $R>0$. Démontrer que toute solution 
sur $\Q]-R,R\W[$ de l'équation différentielle $y""+y=S(x)$ 
est développable en série entière sur $\Q]-R,R\W[$. 

\exo [Level=2,Fight=0,Learn=0,Field=\EquationsDifférentielles,Type=\Others,Origin=] bj. 
Soien $\ell\in\ob R$ et $f\in\sc C^2(\ob R,\ob R)$ 
telle que $\lim_{x\to\infty}\b(f(x)+f'(x)+f''(x)\b)=\ell$. \pn
Démontrer que $\lim_{x\to\infty}f(x)=\ell$. 

\exo [Level=1,Fight=2,Learn=1,Field=\Matrices,Type=\Exercices,Origin=\MP] bk. 
Soient $A$ et $B$ deux matrices réelles semblables dans $\sc M_n(\ob C)$. 
Démontrer qu'elles sont également semblables dans $\sc M_n(\ob R)$. 

\exo [Level=1,Fight=3,Learn=2,Type=\Colles,Field=\Trigonométrie|\Récurrences,Origin=] bl. 
Montrez que $\ds\prod_{1\le k\le n}\cos\Q({x\F2^k}\W)={\sin x\F 2^n\sin(x/2^n)}$ pour $n\ge1$ et $x\in\ob R$. 

\exo[Level=2,Fight=2,Learn=1,Field=\Déterminant,Type=\Colles,Origin=\MP]  bm. 
Pour $n\ge2$ et $(x_1,\cdots, x_n)\in\ob R^n$, 
calculer le déterminant de la matrice $A:=\Q(\cos(ix_j-x_j)\W)_{1\le i,j\le n}$.  
 
\exo [Level=2,Fight=3,Learn=1,Field=\Diagonalisation,Type=\Others,Origin=\MP] bn. 
Soit $A\in\sc M_n(\ob R)$ telle que $A^3=A+I_n$. Prouver que $\det A>0$. 

\exo [Level=2,Fight=2,Learn=2,Field=\PolynômesCaractéristiques,Type=\Exercices,Origin=\MP] bo. 
Soit $(A,B)\in \sc M_n(\ob R)$, soit $P$ 
le polynôme caractéristique de la matrice $AB$ 
et soit $Q$ le polynôme caractéristique de la matrice $BA$. \pn
a) Lorsque $A$ est inversible, prouver que $P=Q$. \pn
b) Lorsque $A$ n'est pas inversible, 
remarquer que l'ensemble $\{x\in\ob R,A+xI_n\rm{ inversible}\}$ est infini 
et en déduire que $P=Q$. 

\exo [Level=2,Fight=3,Learn=1,Field=\Diagonalisation,Type=\Exercices,Origin=\MP] bp. 
Soit $A\in\sc M_n(\ob R)$ ayant $n$ valeurs propres distinctes. 
Démontrer que $M\in\sc M_n(\ob R)$ commute avec $A$ 
si, et seulement si, $M$ est une combinaison linéaire de $I_n,A,A^2, \cdots A^{n-1}$. 

\exo [Level=2,Fight=2,Learn=1,Field=\Diagonalisation,Type=\Colles,Origin=\MP] bq. 
Soient $u\in\sc L(\ob R^n)$ un endomorphisme diagonalisable 
et $v\in\sc L(\ob R^n)$. \pn
a) Démontrer que $u$ et $v$ commuttent si, et seulement si, 
chaque sous-espace propre de $u$ est stable par $v$. \pn
b) Dimension de $\{v\in\sc L(\ob R^n):v\circ u=u\circ v\}$. 

\exo [Level=2,Fight=1,Learn=1,Field=\Diagonalisation,Type=\Exercices,Origin=\MP] br. 
Soient $(f,u,v)\in\sc L(\ob R^n)^3$ des endomorphismes diagonalisables 
et $F\subset\ob R^n$ un sous-espace vectoriel stable par $f$. \pn
a) Démontrer que la restriction de $f$ à $F$ est diagonalisable. 
\pn
b) Lorsque $u\circ v=v\circ u$, prouver que $u$ et $v$ sont diagonalisable 
dans la même base \pn
(i.e. il existe une base $B$ de $\ob R^n$ dans laquelle les matrices de $u$ et $v$ sont diagonales). 

\exo [Origin=MP*,Level=1,Fight=3,Learn=1,Type=\Others,Field=\Matrices|\Groupes] bs. 
Soit $G:=\{M_1,\cdots,M_k\}$ un sous groupe fini de $\sc Gl_n(\ob R)$ 
et soit $A$ la matrice 
$$
A:={1\F k}\big(M_1+M_2+\cdots+M_k\big).
$$
a) Montrer que $A$ est la matrice d'un projecteur. \pn
b) En déduire que $\rm{tr }A=0\Rightarrow A=0$. 

\exo [Level=2,Fight=1,Learn=1,Field=\Diagonalisation,Type=\TravauxDirigés,Origin=] bt. 
Pour $A:=\pmatrix{1&2\cr2&1\cr}$, diagonaliser la matrice $B:=\pmatrix{A&A&A\cr A&A&A\cr A&A&A\cr}$. 

\exo [Level=2,Fight=1,Learn=0,Field=\Diagonalisation,Type=\Colles,Origin=\MP] bu. 
Condition nécéssaire et suffisante sur $(x_1,\cdots, x_n)\in\ob R^n$ 
pour que la matrice $(a_{i,j})_{1\le i,j\le n}$ 
définie par $a_{i,j}:=x_i$ si $i+j=n+1$ et $a_{i,j}=0$ sinon soit diagonnalisable. 

\exo [Level=2,Fight=2,Learn=1,Field=\Diagonalisation,Type=\Colles,Origin=\MP] bv. 
Soit $(A,B)\in\sc M_n(\ob C)^2$ telles que $AB=BA$ %% utilité de cette hypothèse ?
et soit $\ds M:=\pmatrix{A&0\cr B&A\cr}$. \pn
\vskip-0.5em\noindent
Trouver une CNS pour que $M$ soit diagonalisable. 

\exo [Level=2,Fight=3,Learn=1,Field=\Diagonalisation,Type=\Others,Origin=\MP] bw. 
Soient $(n,p)\in\ob N^*$, $(a_1,\cdots, a_p)\in\ob C^p$, $(u_1,\cdots,u_p)\in\sc L(\ob R^n)^p$ et $f\in\sc L(\ob R^n)$ tel que 
$$
\sum_{1\le i\le p}a_i^ku_i=f^k\qquad(0\le k\le p).
$$
a) Démontrer que $f$ est diagonalisable. \pn
b) Lorsque les nombres $a_1,\cdots,a_p$ sont distincts deux à deux, démontrer que 
$$
\sum_{1\le i\le p}a_i^ku_i=f^k\qquad(k\in\ob N).
$$

\exo [Level=2,Fight=2,Learn=1,Field=\MatricesSymétriques,Type=\Others,Origin=\MP]  bx. 
Pour chaque couple $(A,B)\in\sc M_n(\ob R)$, on pose $(A|B):=\mbox{\rm tr}(\NULL^tAB)$. \pn
a) Démontrer que l'on définit un produit scalaire sur $\sc M_n(\ob R)$ ce faisant. 
b) Pour chaque matrice symétrique réelle $(a_{i,j})_{1\le i,j\le n}$ de valeurs propres $\lambda_1,\cdots,\lambda_n$, prouver que 
$$
\sum_{1\le i,j\le n}a_{i,j}^2=\sum_{1\le k\le n}\lambda_k^2.
$$

\exo [Level=2,Fight=1,Learn=0,Field=\MatricesSymétriques,Type=\Exercices,Origin=] by. 
Soient $k\ge2$ et $(A,B)\in\sc M_n(\ob R)^2$ un couple de matrices symétriques 
positives telles que $A^k=B^k$. \pn
Démontrer que $A=B$. 

\exo [Level=2,Fight=2,Learn=1,Field=\Coniques|\Roulement,Type=\TravauxDirigés,Origin=] bz. 
Une parabole roule sans glisser sur une droite. \pn
a) Lieu du foyer. 
b) Enveloppe de l'axe de symétrie. 

\exo [Level=2,Fight=2,Learn=1,Field=\IntégralesMultiples,Type=\Exercices,Origin=] ca. 
Calculer $\int\int_D\exp{x^3+y^3\F xy}\d x\d y$ pour $p>0$ et 
$D:=\{(x,y)\in\ob R^2:y^2\le 2px\hbox{ et }x^2\le 2py\}$. 

\exo [Level=2,Fight=2,Learn=1,Field=\Aires,Type=\Exercices,Origin=] cb. 
Aire du domaine 
plan $\{(x,y)\in\ob R^2:{x^2\F a}\le y\le {x^2\F b}\hbox{ et }{c^2\F x}\le y\le {d^2\F x}\}$ 
pour $0<b<a$ et $0<c<d$. 

\exo [Level=2,Fight=2,Learn=1,Field=\IntégralesMultiples,Type=\Exercices,Origin=] cc. 
Soit $f\in\sc C^4(\ob R^2,\ob R)$. 
Calculer $\int_D xy \partial_1^2\partial_2^2f(x,y)\d x\d y$ 
pour $D:=[0,a]\times[0,b]$. 

\exo [Level=2,Fight=2,Learn=1,Field=\IntégralesMultiples,Type=\Exercices,Origin=] cd. 
Calculer $\ds \int\int_D x^2y\d x\d y$ 
pour $D:=\{(x,y)\in\ob R^2:x^2-y\le0\hbox{ et }x^2+y^2\le 2x\}$. 

\exo [Level=2,Fight=2,Learn=1,Field=\Aires,Type=\Exercices,Origin=] ce. 
$S$ surface d'équation $xy-az=0$. 
Aire de la partie de $S$ dont la projection orthogonale 
sur $Oxy$ est à l'intérieur de la boucle de la courbe 
d'équation polaire $\rho^2=a\cos(2\theta)$ avec $|\theta|\le\pi/4$. 

\exo [Level=2,Fight=2,Learn=1,Field=\Aires|\Quadriques,Type=\Exercices,Origin=] cf. 
Soient $a>0$, $S:=\{(x,y,z)\in\Q[0,+\infty\W[^3:x^2+y^2+z^2=a^2\}$, $t\in[0\pi/2]$, 
$H:=\{(a\cos t,a\sin t,at/(2\pi)):t\in\ob R\}$ et $E$ l'hélicoïde 
droit d'axe $Oz$ et de directrice $H$. 
Aire de la partie de $S$ située au dessus de $E$. 

\exo [Level=2,Fight=2,Learn=1,Field=\IntégralesMultiples|\FonctionsDePlusieursVariables,Type=\Exercices,Origin=]  cg. 
Soit $f$ l'application 
définie par $f(x,y,z):=\d s{z^3\F (x+y+z)(y+z)}$ pour $(x,y,z)\in\Q]0,\infty\W[^3$. \pn
a) Démontrer que $f$ est prolongeable par continuité sur 
$D:=\{(x,y,z)\in\Q[0,+\infty\W[^3:x+y+z\le 1\}$. \pn
b) Calculer $\int\int\int_Df(x,y,z)\d x\d y\d z$. 

\exo [Level=2,Fight=2,Learn=1,Field=\IntégralesMultiples,Type=\Exercices,Origin=] ch. 
Centre d'inertie d'un domaine homogene limité 
par une sphère et deux plans parallèles. 

\exo [Level=2,Fight=2,Learn=1,Field=\Volumes,Type=\Exercices,Origin=] ci. 
Soit $p>0$. 
Volume limité par les surfaces d'équation $x^2+y^2=2pz$ et $z^2=x^2+y^2$. 

\exo [Level=1,Fight=2,Learn=1,Field=\GéométriePlane,Type=\Exercices,Origin=]  cj. 
Soit $ABC$ un triangle rectangle en $C$. Pour chaque droite $D$ passant par $C$, 
on note $A'$ et $B'$ les projections orthogonales de $A$ et $B$ sur $D$. 
Démontrer que le cercle de diamètre $A'B'$ passe par un point fixe. 

\exo [Level=2,Fight=2,Learn=1,Field=\FonctionsDePlusieursVariables,Type=\Exercices,Origin=] ck. 
Soit $f\in\sc C^2(\ob R^2,\ob R)$ et soit $F:\ob R^2\to\ob R$ l'application 
définie pour $(x,y)\in\ob R^2$ par 
$F(x,y):={f(x)-f(y)\F x-y}$ si $x\neq y$ et par $F(x,y):=f'(x)$ si $x=y$. 
Étudier la continuité et la différentiabilité de $F$ sur $\ob R^2$. 

\exo [Level=2,Fight=0,Learn=0,Field=\SériesEntières,Type=\Exercices,Origin=] cl. 
Déterminer le rayon de convergence de la  série entière $\ds\sum_{n\ge0}\e^{-n}z^{2n+1}$. 

\exo [Level=2,Fight=1,Learn=0,Field=\SériesEntières,Type=\Exercices,Origin=] cm. 
Calculer la somme de la série entière $\ds\sum_{n\ge0}(n+3)x^n$.


\exo [Level=2,Fight=1,Learn=0,Field=\SériesEntières,Type=\Exercices,Origin=] cn. 
Soit $\sum_{n\ge0}a_nz^n$ une série entière de rayon de convergence $1$. 
Que peut on dire du rayon de convergence $R$ des séries entières suivantes : \pn
a) $\sum_{n\ge0}(-2)^na_nz^{n+1}$\qquad
b) $\sum_{n\ge0}{a_n\F n!}z^n$\qquad
c) $\sum_{n\ge0}{2^na_n\F n^2+1}$\qquad
d) $\sum_{n\ge0}n^2a_nz^n$\qquad
e) $\sum_{n\ge0}n^{\log n}a_n z^n$\qquad
f) $\sum_{n\ge0}(1+i^n)a_nz^n$. 

\exo [Level=2,Fight=2,Learn=2,Field=\SériesEntières,Type=\Exercices,Origin=] co. 
En développant $t\mapsto(1+t^2)/(1+t^4)$ en série entière, prouver que 
$$
\int_0^1{1+t^2\F1+t^4}\d t=4\sum_{k=0}^\infty(-1)^k{2k+1\F(4k+1)(4k+3)}.
$$

\exo [Level=2,Fight=2,Learn=2,Field=\SériesEntières,Type=\Exercices,Origin=] cp. 
A l'aide de la formule de Taylor, prouver que la fonction $x\mapsto 1/(1+x)$ 
est développable en série entière sur $\Q]-1,1\W[$. 

\exo [Level=2,Fight=2,Learn=2,Field=\FonctionsDePlusieursVariables,Type=\TravauxDirigés,Origin=] cq. 
Existence et calcul de la limite $\ds \lim_{(x,y)\to(0,0)\atop(x,y)\neq(0,0)}{xy\F2x^2+y^2}$. 

\exo [Level=2,Fight=2,Learn=2,Field=\FonctionsDePlusieursVariables,Type=\TravauxDirigés,Origin=] cr. 
Étudier si la fonction  $f(x,y):=\ds{xy\F(2x^2+y^3)}$ est bornée sur $\Q]0,\infty\W[^2$. 

\exo [Level=2,Fight=2,Learn=2,Field=\FonctionsDePlusieursVariables|\FonctionsDéfiniesParUneIntégrale,Type=\TravauxDirigés,Origin=] cs. 
Soit $f\in\sc C^1(\ob R,\ob R)$. Prouver que l'on définit une fonction continue sur $\ob R^2$ en posant 
$$
g(x,y):=\ds \Q\{
\eqalign{
{f(x)-f(y)\F x-y} \sbox{ si }\ x\neq y
\cr 
f'(x)\qquad\sbox{ si }\ x=y^{\strut}}\W.
$$

\exo [Level=2,Fight=1,Learn=1,Type=\Cours,Field=\SériesEntières,Origin=\Lakedaemon] ct. 
Résoudre dans $\ob C$ l'équation $\cos z=2$. 

\exo [Level=2,Fight=1,Learn=1,Type=\Exercices,Field=\SériesEntières, Origin=] cu. 
Pour chaque nombre complexe $z$, prouver que $\b|\e^z-1\b|\le\e^{|z|}-1\le |z|\e^{|z|}$.  

\exo [Level=2,Fight=1,Learn=1,Type=\Exercices,Field=\FonctionsDePlusieursVariables, Origin=] cv. 
Déterminer si l'on définit une fonction de classe $\sc C^1$ sur $\ob R^2$ en posant  
$$
f(x,y):=\Q\{\eqalign{
&{xy^2\F x^2+y^2}\hbox{ pour }(x,y)\neq(0,0)\cr
&0\hbox{ sinon}
}\W.
$$

\exo [Level=2,Fight=1,Learn=1,Type=\Exercices,Field=\FonctionsDePlusieursVariables, Origin=] cw. 
Soit $E:=\ob R^2\ssm\{(t,0):t\le0\}$. En procédant à un changement de variable polaire, 
déterminer les solutions $f\in\sc C^1(E,\ob R)$ de l'équation aux dérivées partielles 
$$
\partial_1^2f(x,y)+\partial_2f^2(x,y)=0\qquad\big((x,y)\in E\big).
$$ 

\exo [Level=2,Fight=1,Learn=1,Type=\Exercices,Field=\FonctionsDePlusieursVariables, Origin=] cx. 
Soit $f:\ob R^2\to\ob R$ l'application définie par $f(x,y):=x^3+3xy^2-15x-12y$ pour $(x,y)\in\ob R^2$. \pn
a) Déterminer les points critiques de $f$. \pn
b) Parmi ces points critiques, lesquels sont des extrémums locaux/globaux de $f$. 

\exo [Level=2,Fight=1,Learn=1,Type=\Exercices,Field=\FonctionsDePlusieursVariables, Origin=] cy. 
Soit $E:=\ob R^3\ssm\{(0,s,t):s\le 0,t\in\ob R\}$. En procédant à un changement de variable sphérique, 
déterminer les solutions $f\in\sc C^1(E,\ob R)$ de l'équation aux dérivées partielles 
$$
\partial_1^2f(x,y,z)+\partial_2f^2(x,y),z+\partial_3f^2(x,y,z)=0
\qquad\big((x,y,z)\in E\big).
$$ 

\exo [Level=2,Fight=1,Learn=1,Type=\Exercices,Field=\FonctionsDePlusieursVariables, Origin=]  cz. 
Soient $(f,g)\in\sc C^1(\ob R^2,\ob R)^2$. Pour $(x,y)\in\ob R^2$, on pose 
$\ds h(x,y):=f\b(x^2-y^2,y^2-g(x,y)^2\b)$. \pn
a) Calculer $\partial_1h$ et $\partial_2h$ en fonction de $\partial_1f$, 
$\partial_2f$, $\partial_1g$ et $\partial_2g$. \pn
b) Si $f$ ne possède aucun point critique et si $h(x,y)=0$ 
pour $(x,y)\in\ob R^2$, établir~que 
\medskip
\hfill
$\ds 
yg(x,y){\partial g\F\partial x}(x,y)+xg(x,y){\partial g\F\partial y}(x,y)=xy\qquad\b((x,y)\in\ob R^2\b). 
$\hfill\null

\exo [Level=2,Fight=3,Learn=1,Type=\Exercices,Field=\FonctionsDePlusieursVariables, Origin=] da. 
Soient $p\in\ob N^*$ et $f:\ob R^p\ssm\{0\}\to\ob R$ 
une application de classe $\sc C^1$. Pour $\alpha\in\ob R$, prouvez que 
\medskip\hfill
$\ds 
f(\lambda x)=\lambda^\alpha f(x)\quad\ (\lambda>0,x\neq 0)\quad\Longleftrightarrow\quad 
 \sum_{1\le i\le p}x_i\partial_if(x)=\alpha f(x)\quad\ \b(x:=(x_1,\cdots,x_p)\neq0\b)
$\hfill\null

\exo [Level=2,Fight=2,Learn=2,Type=\Exercices,Field=\EquationsAuxDérivéesPartielles, Origin=] db. 
A l'aide des coordonnées polaires, déterminer 
les solutions $f\in\sc C^1\b(\Q]0,\infty\W[\times \ob R,\ob R\b)$ de l'équation aux dérivés partielles $x\partial_2f(x,y)-y\partial_1(x,y)=kf(x,y)\ \,(x>0,y\in\ob R)$. 

\exo [Level=2,Fight=1,Learn=1,Type=\Exercices,Field=\SériesEntières, Origin=] dc. 
Soit $S$ la série entière définie par $S(x):=\sum_{n\ge0}x^{3n+2}/(3n+2)$. 
a) Rayon de convergence et somme de $S$. \pn
b) En déduire la valeur de la série numérique $\sum_{n\ge0}(-1)^n/(3n+2)$. 

\exo [Level=2,Fight=2,Learn=2,Type=\Exercices,Field=\SériesEntières, Origin=] dd. 
On pose $F(x):=\ds\int_0^1{1-t\F1-xt^2}\d t$ et $S(x):=\ds\sum_{n\ge0}{x^n\F(2n+1)(2n+2)}$. \pn
a) Montrer que $S$ est définie et continue sur $[-1,1]$. \pn
b) prouver que $F(x)=S(x)$ pour $|x|<1$. \pn
c) En déduire la valeur de la série numérique $\ds\sum_{n\ge0}{(-1)^n\F(2n+1)(2n+2)}$. 

\exo [Level=2,Fight=2,Learn=2,Type=\Exercices,Field=\SériesEntières, Origin=] de. 
Pour $x\in\ob R$, on pose $J_0(x):={1\F2\pi}\int_{-\pi}^\pi\e^{ix\sin t}\d t$ (fonction $J_0$ de Bessel). \pn
a) Développer $J_0(x)$ en série entière de $x$ à l'aide 
de l'intégrale de Wallis ${1\F2\pi}\int_{-\pi}^\pi\sin^{2k}t\d t=\ds{(2k)!\F4^k(k!)^2}\ \,(k\ge0)$. \pn
b) Prouver que $xJ_0''(x)+J_0'(x)+xJ_0(x)=0$ pour $x\in\ob R$. 


\exo [Level=2,Fight=2,Learn=2,Field=\FonctionsDéfiniesParUneIntégrale,Type=\Exercices,Origin=] df. 
Soit $f:\ob R\to\ob R$ 
l'application définie par $f(x):=\int_{-\infty}^\infty\e^{-t^2/2-itx}\d t$ pour $x\in\ob R$. \pn
a) Montrer que $f$ est solution de l'équation différentielle $f'(x)+xf(x)=0\ \,(x\in\ob R)$. \pn
b) En remarquant que $f(0)^2=\int\int_{\ob R^2}\e^{-x^2-y^2}\d x\d y$ 
prouver que $f(x)=\sqrt{2\pi}\e^{-x^2/2}$ pour $x\in\ob R$. 

\exo [Level=2,Fight=1,Learn=1,Type=\Exercices,Field=\FonctionsDéfiniesParUneIntégrale, Origin=] dg. 
Prouver que la fonction $\Gamma:\Q]0,\infty\W[\to\ob R$ 
définie par $\Gamma(x):=\int_0^\infty t^{x-1}\e^{-t}\d t$ pour $x>0$ 
est de classe $\sc C^\infty$ sur $\Q]0,\infty\W[$. 

\exo [Level=1,Fight=3,Learn=2,Type=\Others,Field=\Anneaux,Origin=\MP] dh. 
Soit $A$ un anneau intégre fini de caractéristique $c$. \pn
a) Démontrer que $c$ est un nombre premier divisant $|A|$. \pn
b) Démontrer que $(x+y)^c=x^c+y^c$ pour $(x,y)\in A^2$.  

\exo [Level=1,Fight=3,Learn=1,Type=\Others,Field=\Trigonométrie,Origin=\Capaces] di. 
Démontrer qu'il existe une infinité de points de $\ob Q^2$ 
sur le cercle unité 
$$
\{(x,y)\in\ob R^2: x^2+y^2=1\}.
$$ 

\exo [Level=1,Fight=3,Learn=2,Type=\Others,Field=\Anneaux,Origin=\MP] dj. 
Soient $n\in\ob N^*$, $p$ un nombre premier et $P:=X^p-X$. \pn
a) En étudiant $P$ dans $(\ob Z/p\ob Z)[X]$, prouver que $(p-1)!+1\equiv0\quad[p]$. \pn
b) Lorsque $(n-1)!+1\equiv0\quad[n]$, prouver que $n$ est un nombre premier (théorème de Wilson). 

\exo [Level=1,Fight=2,Learn=1,Type=\Colles,Field=\EspacesVectoriels,Origin=] dk. 
Soit $E$ un espace vectoriel de dimension finie et soient $F,G$ 
deux sous espaces vectoriels de~$E$ de même dimension. Démontrer que $F$ et $G$ 
admettent un même supplémentaire. 

\exo [Level=1,Fight=2,Learn=2,Type=\Colles,Field=\EspacesVectoriels,Origin=] dl. 
Soient $(m,n,p)\in\ob N_*^3$, $u\in\sc L(\ob R^m,\ob R^n)$ et $v\in\sc L(\ob R^n,\ob R^p)$. 
Démontrer que 
\smallskip
\hfill
$\ds
\hbox{\rm rg}(u)+\hbox{\rm rg}(v)-n\le\hbox{\rm rg}(v\circ u)\le\min\b\{\hbox{\rm rg}(u),\hbox{\rm rg}(v)\b\}.
$\hfill\null
\PAR

\exo [Level=1,Fight=3,Learn=2,Type=\Colles,Field=\Groupes,Origin=\MP] dm. 
Soient $E$ un espace vectoriel de dimension finie et 
$\{f_1,\cdots,f_n\}$ un sous groupe fini de $\sc Gl(E)$. \pn
a) Démontrer que $f:=(f_1+\cdots+f_n)/n$ est un projecteur (i.e. $f^2=f$). \pn
b) Démontrer que $\rm{Im}\ f=\cap_{1\le i\le n}\rm{Ker}(f_i-I_E)$. 

\exo [Level=1,Fight=3,Learn=1,Type=\Others,Field=\EspacesVectoriels,Origin=\MP] dn. 
Soient $n\ge2$ et $u_1,\cdots u_n$ des endomorphismes nilpotents de $\ob R^n$ commutant deux à deux. 
Démontrer que $u_1\circ u_2\circ\cdots\circ u_n=0$. 

\exo [Level=1,Fight=2,Learn=1,Type=\Others,Field=\EspacesVectoriels,Origin=\MP] do. 
Soient $E$ un espace vectoriel de dimension finie et $F,G$ 
deux sous-espaces vectoriels~de~$\sc L(E)$ 
vérifiant $\sc L(E)=F+G$ et $f\circ g+g\circ f=0$ 
pour $(f,g)\in F\times G$. Montrer que $F=\{0\}$ ou $G=\{0\}$. 

\exo [Level=2,Fight=2,Learn=1,Type=\Colles,Field=\Déterminant,Origin=\MP,Indication={Faire $C_k-C_1\to C_k$ pour $\le k\le n$ puis $L_k+L_n\to L_k$ pour $1\le k<n$},Solution={$(-2)^{n-2}(1-n)$}] dp. 
Calculer le déterminant de la matrice $A:=\big(|j-i|\big)_{1\le i,j\le n}$.

\exo [Level=1,Fight=1,Learn=1,Type=\Exercices,Field=\SystèmesLinéaires,Origin=] dq. 
Résoudre et discuter selon les paramètres le système  $\ds
\Q\{\eqalign{
x+y+z&=a\cr
\alpha x+\beta y+\gamma z&=b\cr
\alpha^2x+\beta^2y+\gamma^2z&=c\cr
}\W.
$

\exo [Level=1,Fight=3,Learn=2,Type=\Exercices,Field=\Anneaux,Origin=\MP,Indication={c) On pourra se ramener au cas d'une matrice $M(x,y,z)$ pour lesquels les nombres $x$, $y$ et $z$ sont des entiers n'ayant aucun facteur premier en commun.}]  dr. 
Pour $(x,y,z)\in\ob R^3$, on pose 
$$
M(x,y,z):=\pmatrix{x&y&z\cr 2z&x&y\cr 2y&2z&x\cr}.
$$ 
a) Montrer que $K:=\{M(x,y,z):(x,y,z)\in\ob Q^3\}$ forme un sous anneau de $\Q(\sc M_3(\ob R),+,\times\W)$. \pn
b) Montrer que la multiplication est commutative dans $K$. \pn
c) Montrer que $(K,+,\times)$ est un corps, c'est à dire que chaque matrice non nulle $M\in K$ est inversible. 




\exo [Level=2,Fight=3,Learn=1,Field=\ValeursPropres,Type=\Exercices,Origin=\MP] ds. 
Etant donné un entier $n\ge3$ et des nombres réels $a_n>\cdots>a_2>a_1>0$, 
on pose 
$A:=\pmatrix{
0& a_2&\cdots &a_n\cr
a_1&0&\vdots&\vdots\cr
\vdots&a_2&\ddots&a_n\cr
a_1& a_2&\cdots&0\cr
}$. \pn
a) Démontrer que $\lambda$ est valeur propre de $A$ si, et seulement si, 
$\sum_{1\le k\le n}{a_k\F\lambda+a_k}=1$. \pn
b) En déduire que $A$ possède $n$ 
valeurs propres réelles deux à deux distinctes. 

\exo [Level=1,Fight=2,Learn=1,Type=\Colles,Field=\EspacesVectoriels,Origin=] dt. 
Démontrer que les endomorphismes non nuls de $\ob R^3$ tels que $f(x\vec y)=f(x)\vec f(y)$ pour $x\in\ob R^3$ et $y\in\ob R^3$ 
sont les rotations de $\ob R^3$. 

\exo [Level=2,Fight=2,Learn=1,Field=\MatricesSymétriques,Type=\Exercices,Origin=\MP] du. 
Soient $A\in\sc M_n(\ob R)$, $B:=^tAA$, $\alpha$ la plus petite valeur propre de $B$ 
et $\beta$ la plus grande. \pn
a) Démontrer que $B$ est symétrique, positive, de même rang que $A$. \pn
b) Pour chaque valeur propre de $A$, démontrer que $\alpha\le \lambda^2\le \beta$. 
 
\exo [Level=1,Fight=2,Learn=2,Type=\Exercices,Field=\Connexité,Origin=] dv. 
Soit $f\in\sc C\b([0,1],\ob R\b)$ telle que $f(0)=f(1)$. \pn
a) Démontrer qu'il existe $x\in[0,1/2]$ tel que $f(x+1/2)=f(x)$. \pn
b) Pour chaque entier $p\ge3$, montrer 
qu'il existe $x\in[0,1-1/p]$ tel que $f(x+1/p)=f(x)$. 

\exo [Level=2,Fight=2,Learn=1,Type=\Exercices,Field=\DéveloppementsLimités,Origin=] dw.  
Développement limité au voisinage de $0$ de $f(x):=\arcsin\b(\exp(-x^2)\b)$. 

\exo [Level=1,Fight=2,Learn=1,Type=\Exercices,Field=\DéveloppementsLimités,Origin=] dx. 
Soit $f:\ob R\to\ob R$ l'application définie par $f(0):=0$ 
et par $\ds f(x):={\e^{x^2}-1\F x}$ pour $x\neq0$. \vskip-0.5em\noindent
a) Démontrer que $f$ admet une fonction réciproque $g$. \pn
b) Déterminer un développement limité à l'ordre $5$ de $g$ en $0$. 

\exo [Level=1,Fight=2,Learn=2,Type=\Exercices,Field=\ThéorèmeDeRolle|\DéveloppementsLimités,Origin=] dy. 
Soit $f:\ob R\to\ob R$ l'application définie par $f(x):=x\sin x$ pour $x\in\ob R$. \pn
Démontrer que $f'$ s'annule pour une suite $\{x_n\}_{n\ge0}$ de valeurs 
vérifiant $\forall n\in\ob N,|x_n-n\pi|<\pi/2$. \pn
Posant $u_n:=x_n-n\pi+\pi/2$ pour $n\ge0$, démontrer que $\ds\lim_{n\to0}u_n=0$ et donner un équivalent~simple
\vskip-0.5em\noindent de la suite $\{u_n\}_{n\ge0}$. 

\exo [Level=1,Fight=2,Learn=2,Type=\Exercices,Field=\DéveloppementsLimités,Origin=] dz. 
Soit $x_0\in[0,\pi/2]$ et soit $\{x_n\}_{n\ge1}$, la suite définie par $x_{n+1}:=\sin x_n$ pour $n\ge0$. \pn
a) Pour chaque suite convergente $\{u_n\}_{n\in\ob N}$ de limite $\ell$, déterminer la limite 
$\lim_{n\to\infty}{1\F n}\sum_{1\le k\le n}u_k$. 
b) Déterminer un nombre réel $\alpha$ tel que la quantité $x_{n+1}^\alpha-x_n^\alpha$ ait une limite non nulle. \pn
c) En déduire un équivalent simple de la suite $\{x_n\}_{n\ge0}$. 

\exo [Level=1,Fight=2,Learn=2,Type=\Exercices,Field=\DéveloppementsLimités,Origin=] ea. 
Soit $\{u_{a,n}\}_{a>1,n\ge0}$ 
la famille de nombres réels définie par $u_{a,0}:=\e$ et 
$u_{a,n+1}:=\log_{u_{a,n}}a$ pour $a>1$ et $n\ge0$. \pn
a) Pour $a>\e^\e$, démontrer que la limite $\ell_a:=\lim_{n\to\infty}u_{a,n}$ existe. \pn
b) Démontrer que l'application $a\mapsto \ell_a$ est continue sur $\Q]\e^\e,+\infty\W[$. 

\exo [Origin=,Level=1,Fight=1,Learn=2,Type=\Colles,Field=\Suites] eb. Pour $z_0\in\ob C$, établir la convergence de 
la suite $\{z_n\}_{n\ge1}$ définie par $z_{n+1}:={z_n+|z_n|\F2}$ pour $n\ge0$ 
et déterminer sa limite. 

\exo [Origin=,Level=1,Fight=2,Learn=2,Type=\Exercices,Field=\Suites]  ec. Soient $u_0>0$ et $v_0>0$.  
Prouver que les suites $\{u_n\}_{n\ge1}$ et $\{v_n\}_{n\ge1}$ 
définie par 
$$
\forall n\in\ob N^*,\quad u_{n+1}:=\sqrt{u_nv_n}\qquad\hbox{et}\qquad\forall n\in\ob N^*, \quad 
v_{n+1}:={u_n+v_n\F2}
$$ 
convergent vers la même limite.

\exo [Level=2,Fight=1,Learn=1,Type=\Colles,Field=\Diagonalisation,Origin=] ed. 
Soit $A\in\sc M_n(\ob C)$ et soient $\lambda_1,\cdots,\lambda_n$ les valeurs propres de $A$. \pn 
a) Lorsque $A$ est diagonalisable, 
prouver que $\lim_{p\to\infty}A^p=0\Longleftrightarrow \max_{1\le p\le n}|\lambda_p|<1$. \pn 
b) Soient $S\in\sc M_n(\ob C)$ une matrice triangulaire supérieure, 
$D$ une matrice diagonale de même diagonale principale que $S$ et $N:=S-D$. 
Vérifier que $DN=ND$ et que $N^n=0$. \pn
c) Pour $A\in\sc M_n(\ob C)$, prouver 
que $\lim_{p\to\infty}A^p=0\Longleftrightarrow \max\limits_{1\le p\le n}|\lambda_p|<1$.  

\exo [Level=2,Fight=1,Learn=1,Type=\Others,Field=\Connexité,Origin=] ef. 
Démontrer que $\sc Gl_n(\ob C)$ est une partie connexe de $\sc M_n(\ob C)$. 

\exo [Origin=,Level=1,Fight=2,Learn=1,Type=\Others,Field=\Continuité] eg. 
Soit $f\in\sc C\b([0,1],\ob R^+\b)$ tel que $f(x)<x$ pour $0<x\le 1$. 
Prouver que $$
\lim\limits_{n\to\infty}\sup\limits_{0\le x\le 1}\b|f^{\circ n}(x)\b|=0.
$$ 

\exo [Level=1,Fight=3,Learn=1,Type=\Colles,Field=\Continuité,Origin=] eh. 
Soit $f\in\sc C\b([0,1],\Q]0,\infty\W[\b)$. Déterminer la limite $\ds \lim_{x\to0^+}\Q(\int_0^1f(t)^x\d t\W)^{1/x}$. 

\exo [Level=1,Fight=2,Learn=2,Type=\TravauxDirigés,Field=\ThéorèmeDeRolle,Origin=] ei. 
Soit $a\in\ob R$ et $f\in\sc C^1\b(\Q[a,+\infty\W[,\ob R\b)$ telle que $\lim_{x\to\infty}f(x)=f(a)$. \pn
a) Démontrer qu'il existe $c\in\Q]a,+\infty\W[$ tel que $f'(c)=0$. \pn
b) En déduire que la dérivée $n^{\hbox{\sevenrm ième}}$ de la fonction $\arctan$ 
s'annule $n-1$ fois et calculer ces zéros. 

\exo [Level=1,Fight=2,Learn=2,Type=\Exercices,Field=\SommesDeRiemann,Origin=] ej. 
Soit $f\in\sc C^1\b([0,1],\ob C\b)$. Déterminer 
la limite $\lim_{n\to\infty}{1\F n}\sum_{0\le k< n}f(k/n)f'\b((k+1)/n\b)$. 

\exo [Level=1,Fight=2,Learn=2,Type=\Exercices,Field=\Intégration,Origin=] ek. 
Soit $f\in\sc C\b([0,1]\ob R\b)$ et soit $S:\Q[0,+\infty\W[\to\ob R$ l(application définie par $S(t):=4[t]-2[2t]+1$ pour $t\ge0$. 
Démontrer que $\lim_{n\to\infty}\int_0^1f(u)S(nu)\d u=0$. 

\exo [Level=1,Fight=2,Learn=2,Type=\Exercices,Field=\ThéorèmeDeRolle,Origin=] el. 
Soit $f\in\sc C^2\b([0,1],\ob R\b)$ telle que $f(0)=f'(0)=f'(1)=0$ et $f(1)=1$. 
Démontrer qu'il existe $a\in[0,1]$ tel que $|f''(a)|\ge4$. 

\exo [Level=1,Fight=2,Learn=2,Type=\Exercices,Field=\ThéorèmeDeRolle,Origin=,Indication={On pourra vérifier que $\ds\forall x\in\ob R, g(x)=\int_0^x{(x-t)^{n-1}\F(n-1)!}f(t)\d t$.}] em. 
Soient $n,p\in\ob N^*$ et $f\in\sc C(\ob R,\ob R^p)$. 
Montrer qu'il existe une unique fonction $g\in\sc C^n(\ob R,\ob R^p)$ 
vérifiant $g^{(n)}=f$ et $g^{(k)}(0)=0\ \,(0\le k<n)$. \pn


\exo [Level=1,Fight=2,Learn=2,Type=\Exercices,Field=\DéveloppementsLimités|\Intégration,Origin=] en. 
Déterminer un équivalent simple lorsque $n$ tends vers $+\infty$ de la suite $\{I_n\}_{n\ge0}$ définie par 
$\ds I_n:=\int_0^1{\d t\F1+t+\cdots+t^n}$ pour $n\ge0$. 

\exo [Level=1,Fight=2,Learn=1,Type=\Exercices,Field=\Suites,Origin=] eo. 
Soient $f\in\sc C\b([0,1],\ob R\b)$, $f_0:=f$ et $\{f_n\}_{n\ge1}$ 
la suite de fonctions définie par $f_{n+1}(x):=\b(f_n(x/2)+f_n(x/2+1/2)\b)/2$ 
pour $n\ge0$ et $0\le x\le 1$. Démontrer qu'il existe une constante $\ell\in\ob R$ 
pour laquelle $\lim_{n\to\infty}\sup_{0\le x\le 1}\b|f_n(x)-\ell\b|=0$. 

\exo [Level=1,Fight=2,Learn=2,Type=\Exercices,Field=\IntégralesGénéralisées,Origin=] ep. 
Soit $f\in\sc C^2(\ob R^+,\ob R)$ telle que $\int_0^\infty f(t)^2\d t$ et $\int_0^\infty f''(t)^2\d t$ convergent. \pn
Démontrer que $\int_0^\infty f(x)f''(x)\d x$ et que $\int_0^\infty f'(x)^2\d x$ convergent. 

\exo [Level=1,Fight=1,Learn=1,Type=\Exercices,Field=\Rang,Origin=] eq. 
Soient $n\ge2$ et $u\in\sc L(\ob R^n)$ un endomorphisme de rang $n-1$. 
Démontrer que $u^{n-1}\neq0$. 

\exo [Level=1,Fight=2,Learn=1,Type=\Exercices,Field=\Matrices,Origin=] er. 
Caractériser les matrices $A\in\sc Gl_n(\ob R)$ telles que $A$ et $A^{-1}$ soient à coefficients positifs. 

\exo [Level=1,Fight=3,Learn=1,Type=\Exercices,Field=\Matrices|\Anneaux,Origin=\MP] es. 
Lorsque $(a,b,c)\in\ob Q^3\ssm\{(0,0,0)\}$, montrer que $A:=\pmatrix{a&c&b\cr b&a+c&b+c\cr c&b&a+c}$ est inversible. 
 
\exo [Level=1,Fight=2,Learn=2,Type=\Exercices,Field=\Matrices,Origin=] et. 
Soient $n\ge2$ et $A:=(a_{i,j})_{1\le i,j\le n}\in\sc M_n(\ob C)$ telle que $|a_{i,i}|>\sum_{j\neq i}|a_{i,j}|$ pour $1\le i\le n$. 
Démontrer que $A$ est inversible. 

\exo [Level=1,Fight=3,Learn=2,Type=\Others,Field=\Réduction,Origin=\MP] eu. 
Soient $(a,b,c)\in\ob R^3$. 
Démontrer que $\pmatrix{a&b&c\cr c&a&b\cr b&c&a\cr}$ est une matrice de rotation 
si et seulement si $a$, $b$ et $c$ sont racines d'une équation du type $t^3-t^2+k=0$ 
avec $0\le k\le 4/27$. 

\exo [Level=1,Fight=3,Learn=1,Type=\Exercices,Field=\Rang,Origin=] ev. 
Soit $A\in\sc M_n(\ob R)$ une matrice antisymétrique. 
Démontrer que le rang de $A$ est pair. 

\exo [Level=1,Fight=3,Learn=2,Type=\Others,Field=\InégalitéDeCauchySchwarz,Origin=] ew. 
Soit $A:=(a_{i,j})_{1\le i,j\le n}\in\sc O(n)$. Démontrer que $\b|\sum_{1\le i,j\le n}a_{i,j}\b|\le n$. 
 
\exo [Level=2,Fight=2,Learn=2,Type=\Others,Field=\EndomorphismesOrthogonaux,Origin=] ex. 
Soit $p$ un projecteur de $\ob R^n$ euclidien tel que $\|p(x)\|\le \|x\|$ 
pour $x\in\ob R^n$. 
Démontrer que $p$ est un projecteur orthogonal. 

\exo [Level=2,Fight=2,Learn=1,Type=\Others,Field=\MatricesSymétriques,Origin=] ey. 
Soit $A\in\sc M_n(\ob R)$ une matrice symétrique positive. 
Démontrer que la comatrice de $A$ est également symétrique positive. 

\exo [Level=2,Fight=1,Learn=1,Type=\Exercices,Field=\VecteursPropres,Origin=] ez. 
Soient $A\in\sc M_n(\ob C)$ et $B:=\pmatrix{0&I_n\cr A&0\cr}$. 
Déterminer valeurs propres et vecteurs propres de $B$ \pn
\vskip -0.6em\noindent en fonction de ceux de $A$. 

\exo [Level=2,Fight=-1,Learn=0,Type=\Exercices,Field=\TravauxDirigés,Origin=\Lakedaemon,Solution={$P:=\pmatrix{1&1\cr1&2}$ et $D:=\pmatrix{1&0\cr0&2\cr}$}] fa.
Diagonaliser la matrice $A:=\pmatrix{0&1\cr-2&3\cr}$. 


\exo [Level=2,Fight=2,Learn=1,Type=\Exercices,Field=\Déterminant,Origin=,Indication={Montrer que $D_n(\lambda_1,\cdots,\lambda_n)=\lambda_1D_{n-1}(\lambda_2-\lambda_1,\cdots,\lambda_n-\lambda_1)$ puis intuiter une proposition de récurrence},Notion=Opérations élémentaires|déterminant par blocs|Récurrence,Solution={$D_n(\lambda_1,\cdots,\lambda_n)=\lambda_1\prod_{i=2}^n(\lambda_i-\lambda_{i-1})$}] fb. 
Pour $n\ge1$ et $(\lambda_1,\cdots,\lambda_n)\in\ob C^n$, calculer le déterminant 
$D_n(\lambda_1,\cdots,\lambda_n):=\Q|\matrix{
\lambda_1&\ldots&\ldots&\lambda_1
\cr
\vdots&\lambda_2&\ldots&\lambda_2
\cr
\vdots&\vdots&\ddots&
\cr
\lambda_1&\lambda_2&&\lambda_n
\cr
}\W|$. 

\exo [Origin=,Level=2,Fight=2,Learn=2,Type=\Cours,Field=\Déterminant] fc. 
Pour $(A,B)\in\sc M_n(\ob C)^2$, montrer que 
$$
\det\pmatrix{A&B\cr -B&A}=\det(A+iB)\det(A-iB).
$$ 

\exo [Origin=,Level=2,Fight=2,Learn=2,Type=\Cours,Field=\Déterminant,Indication={développer par rapport à la première colonne puis procéder par récurrence.},Notion={développement par rapport à une colonne|récurrence|$n$-linéarité}] fd. 
Pour $A\in\sc M_n(\ob C)$, démontrer que $\overline{\det A}=\det\overline{A}$. 

\exo [Origin=,Level=2,Fight=3,Learn=2,Type=\Exercices,Field=\Déterminant,Indication={Remarquer que l'ensemble $\{\lambda\in\ob C^*:\lambda D+I_n\in\sc Gl_n(\ob C)\}$ 
est non vide.}] fe. 
Soit $(A,B,C,D)\in\sc M_n(\ob C)^4$ tel que $CD=DC$. \pn
a) Lorsque $D$ est inversible, prouver que $\det(AD-BC)=\det\pmatrix{A&B\cr C&D}$. \pn
b) Prouver que cette identité est encore vraie si $D$ n'est pas inversible. 

\exo [Origin=,Level=2,Fight=2,Learn=2,Type=\Cours,Field=\Déterminant,Solution={%
	$D=\Q\{\eqalign{%
		&{c\F c-b}\prod_{i=1}^n(\lambda_i-b)-{b\F c-b}\prod_{i=1}^n(\lambda_i-c)\mbox{ si }c\neq b\cr
		&\prod_{i=1}^n(\lambda_i-b)+b\sum_{i=1}^n\prod_{j\neq i}(\lambda_j-b) \mbox{ si }c=b\cr
		}\W.$
},Indication={Notant $M:=(1)_{1\le i,j\le n}$, Introduire le polynôme $P:=\det(A+XM)$, dont l'on determinera le degré puis l'expression.},Notion={Polynômes|Opérations élémentaires|Développement par rapport à une colonne|degré}] ff. 
Pour $a,b,\lambda_1,\cdots,\lambda_n\in\ob C$, calculer 
le déterminant $D$ de la matrice $A:=\!\pmatrix{
\lambda_1&b&\ldots&b
\cr
c&\ddots&\ddots&\vdots
\cr
\vdots&\ddots&\ddots&b
\cr
c&\ldots& c& \lambda_n\cr
}$.\pn

\exo[Origin=\Lakedaemon,Level=2,Fight=0,Learn=0,Type=\Exercices,Field=\Déterminant] fg. 
Calculer $\Delta:=\det\pmatrix{1&2&3&4\cr2&3&4&5\cr3&4&5&6\cr4&5&6&7}$. 



\exo [Origin=,Level=2,Fight=2,Learn=3,Type=\Cours,Field=\Déterminant,Solution={$V_n(a_0,\cdots,a_n)=\prod_{i<j}(a_j-a_i)$},Indication={Introduire le polynôme $P:=\det\pmatrix{
1&X&X^2&\ldots& X^n
\cr
1&a_1&a_1^2&\ldots& a_1^n
\cr
\vdots&\vdots&\vdots&\ldots&\vdots
\cr
1&a_n&a_n^2&\ldots &a_n^n
\cr}$ dont on cherchera le degré, puis les racines et l'expression factorisée, avant de faire une récurrence},Notion=Polynômes|Racines|Factorisation|Récurrence] fh. 
Pour $n\in\ob N$ et $(a_0,\cdots, a_n)\in\ob C^{n+1}$, calculer 
$$
V_n(a_0,\cdots, a_n):=\det\pmatrix{
1&a_0&a_0^2&\ldots& a_0^n
\cr
1&a_1&a_1^2&\ldots& a_1^n
\cr
\vdots&\vdots&\vdots&\ldots&\vdots
\cr
1&a_n&a_n^2&\ldots &a_n^n
\cr}.\leqno{\mbox{(Déterminant de Vandermond)}}
$$

\exo [Origin=,Level=2,Fight=2,Learn=2,Type=\Cours,Field=\Déterminant,Indication={Développer par rapport à la première colonne pour obtenir une récurrence linéaire},Notion={Développement par rapport à une colonne|Déterminant par bloc|Récurrence linéaire|Sommes geometriques},Solution={$D_n(a)=\Q\{%
	\eqalign{%
	{1-a^{2n+2}\F 1-a^2}&\mbox{ si }a^2\neq1\cr
	n+1&\mbox{ si }a^2=1\cr
	}\W.$}] fi. 
Pour $n\ge3$ et $a\in\ob C$, calculer le déterminant $D_n(a):=\Q|\matrix{
1+a^2& a&&\ob O
\cr
a&\ddots&\ddots&
\cr
&\ddots&\ddots&a
\cr
\ob O&&a&1+a^2
\cr}\W|$. 

\exo [Level=1,Fight=0,Learn=0,Type=\Cours,Field=\Déterminant,Origin=] fj. 
Etablir l'identité $(\vec X\Lambda\vec Y).\vec Z=\det_{\textstyle\sc B}(X,Y,Z)$ 
pour chaque triplet $(X,Y,Z)\in\sc M_{3,1}(\ob R)^3$, 
le~symbôle $\sc B$ désignant la base canonique de l'espace $\sc M_{3,1}(\ob R)$. 

\exo [Origin=\Fac,Level=2,Fight=0,Learn=0,Type=\TravauxDirigés,Field=\Diagonalisation,Annee=2007,Solution={$P:=\pmatrix{1&-1&2&2\cr0&1&0&1\cr1&-2&1&0\cr0&2&0&1}$ et $D:=\pmatrix{-1&0&0&0\cr0&0&0&0\cr0&0&1&0\cr0&0&0&1\cr}$}] fk. 
Diagonaliser la matrice $A:=\pmatrix{3&-3&-4&-1\cr 0&2&0&-1\cr 2&-4&-3&0\cr0&2&0&-1}$.  
 
\exo [Level=2,Fight=1,Learn=1,Type=\Cours,Field=\Déterminant,Origin=] fl. 
Soit $A$ une matrice carrée de taille $n$ dont tous les coefficients appartiennent à $\{-1,+1\}$. \pn
Prouver que $\det A$ est un multiple de $2^{n-1}$. 

\exo [Level=2,Fight=2,Learn=1,Type=\Exercices,Field=\Déterminant,Origin=] fm. 
Montrer que la matrice $A:=\Q(c_{j-1}^{i-1}\W)_{1\le i,j\le n+1}$ est inversible ( avec $c_p^q:=0$ si $p<q$) et  
déterminer~$A^{-1}$ en~utilisant l'endomorphisme de $\ob R_n[X]$ associé à $A$ 
pour la base canonique.  

\exo [Level=1,Fight=1,Learn=2,Type=\TravauxDirigés,Field=\Matrices,Origin=] fn. 
Soit $A\in\sc M_n(\ob C)$ et $P\in\ob C[X]$ un polynôme vérifiant $P(0)\neq0$ et $P(A)=0$. 
Prouver que $A$ est inversible. 

\exo [Level=2,Fight=0,Learn=0,Field=\SériesEntières,Type=\Exercices,Origin=] fo. 
Développez la fonction $f(x)=\ln^2(1+x)$ en série entière au voisinage de $0$. 
 
\exo [Level=2,Fight=0,Learn=0,Type=\Colles,Field=\Diagonalisation,Origin=] fp. 
Pour quel triplet $(a,b,c)\in\ob C^3$ 
la matrice $A:=\pmatrix{0&0&a\cr 0&0&b\cr a&b&c\cr}$ est elle diagonalisable ? 

\exo [Level=2,Fight=2,Learn=1,Type=\Exercices,Field=\Réduction,Origin=] fq. 
Résoudre dans $\ob M_n(\ob C)$ l'équation $A^2=\pmatrix{11&-5&-5\cr-5&3&3\cr-5&3&3\cr}$. 

\exo [Level=2,Fight=2,Learn=1,Type=\Exercices,Field=\Réduction,Origin=] fr. 
Trouver toutes les matrices $M\in\sc M_3(\ob C)$ vérifiant $M^2=\pmatrix{0&1&1\cr1&0&1\cr1&1&0\cr}$. 

\exo [Level=2,Fight=1,Learn=1,Type=\Exercices,Field=\Diagonalisation,Origin=] fs. 
Lorsque c'est possible, diagonaliser la matrice $M:=\pmatrix{a&b&a&b\cr b&a&b&a\cr a&b&a&b\cr b&a&b&a\cr}$. 

\exo [Level=2,Fight=2,Learn=1,Type=\Exercices,Field=\Réduction,Origin=] ft. 
Trouver toutes les matrices $M\in\sc M_3(\ob C)$ 
telles que $M^2=\pmatrix{3&0&0\cr-5&2&0\cr4&0&1\cr}$. 

\exo [Level=2,Fight=2,Learn=1,Type=\Colles,Field=\Diagonalisation,Origin=] fu. 
Soit $M\in\sc Gl_n(\ob C)$ telle que $M^2$ soit diagonalisable. 
Montrer que $M$ est diagonalisable. 

\exo [Level=2,Fight=2,Learn=1,Type=\Exercices,Field=\Réduction,Origin=] fv. 
Résoudre l'équation $X^2=\pmatrix{1&0&0\cr1&1&0\cr1&0&4\cr}$ dans $\sc M_3(\ob R)$ . 

\exo [Level=2,Fight=1,Learn=1,Type=\TravauxDirigés,Field=\ValeursPropres,Origin=] fw. 
Montrer qu'il n'existe pas de matrice $A\in\sc M_3(\ob R)$ telle que $A^2+I_3=0$. 

\exo [Level=1,Fight=0,Learn=0,Type=\TravauxDirigés,Field=\RécurrencesLinéaires,Origin=] fx. 
Soit $u$ la suite définie 
par $u_0:=1$, $u_1:=2$ et $u_{n+2}=\ds{u_n+u_{n+1}\F2}$ pour $n\ge0$. \pn
\vskip-0.1em\noindent
Calculer explicitement $u_n$ pour $n\in\ob N$ et déterminer $\lim\limits_{n\to\infty}u_n$. 

\exo [Level=2,Fight=1,Learn=1,Type=\Colles,Field=\Réduction,Origin=] fy. 
Pour $X\in\sc M_4(\ob R)$, on pose $f(X):=X-\strut^tX$. \pn
a) Montrer que $f$ est un endomorphisme de $\sc M_4(\ob R)$. \pn
b) Déterminer les sous espaces propres de $f$. L'application $f$ est elle diagonalisable ?

\exo [Level=2,Fight=2,Learn=1,Type=\Exercices,Field=\Réduction,Origin=] fz. 
La matrice $\pmatrix{
a&c&&\ob O
\cr 
b&\ddots&\ddots&
\cr
&\ddots&\ddots &c
\cr
\ob O&& b&a
\cr}$ est elle diagonalisable ? 
Déterminer ses éléments~propres. 

\exo [Level=2,Fight=1,Learn=1,Type=\Colles,Field=\Réduction,Origin=] ga. 
Calculer la limite de $A_n:=\pmatrix{1&-\alpha/n\cr\alpha/n&1\cr}^n$. 

\exo [Level=2,Fight=1,Learn=0,Type=\Exercices,Field=\Diagonalisation,Origin=] gb. 
Trouver une condition nécéssaire et suffisante sur $\theta\in\Q[-\pi,\pi\W[$ 
pour que $\pmatrix{0&\sin\theta&\sin2\theta
\cr
\sin\theta&0&\sin2\theta
\cr
\sin2\theta&\sin\theta&0
\cr}$ \vskip-1em\noindent soit diagonalisable. Réduire cette matrice. 

\exo [Level=2,Fight=0,Learn=0,Type=\Exercices,Field=\Diagonalisation,Origin=] gc. 
Soit $M\in\sc M_2(\ob R)$ telle que $\det M=1$. 
Montrer que $M$ est semblable à l'une des matrices $\pmatrix{1&1\cr0&1\cr}$, 
$\pmatrix{-1&1\cr0&-1\cr}$, 
$\pmatrix{\cos\theta&-\sin\theta\cr \sin\theta&\cos\theta\cr}$ ou 
$\pmatrix{t&0\cr0&t^{-1}\cr}$. 
  

\exo [Level=2,Fight=0,Learn=0,Type=\Colles,Field=\Trigonalisation,Origin=,Solution={
$T=\pmatrix{2&0&0\cr0&1&-1\cr0&0&1}$ et $P=\pmatrix{0&0&-1\cr1&0&1\cr2&1&0}$}] gd. 
Trigonaliser sur $\ob R$ la matrice $\pmatrix{1&0&0\cr1&2&0\cr3&2&1\cr}$. 

\exo [Level=1,Fight=2,Learn=1,Type=\Colles,Field=\NombresComplexes,Origin=\Fac] ge. 
Soient $(a,b)\in\ob C^2$ tels que $a\neq b$. \pn 
1) Montrez que $|a|=|b|$ si, et seulement si, il existe $k\in\ob R$ tel que $a+b=ki(a-b)$. \pn
2) En déduire que le polynôme $X^2-sX+p$ admet $2$ racines distinctes 
de même module si, et~seulement~si, il existe 
$\lambda\ge0$ tel que $s^2=\ds{4\lambda\F1+\lambda}p$. 
 
\exo [Level=2,Fight=1,Learn=1,Field=\SériesEntières,Type=\Exercices,Origin=] gf. 
Développez en série entière au voisinage de $0$ la fonction 
$$
f(x):=\Q\{\eqalign{
&{\arctan\sqrt{-x}\F\sqrt{-x}} \hbox{ si }x\le 0\cr
&{Argth\sqrt x\F\sqrt x}\hbox{ si }x>0.
}\W. 
$$

\exo [Level=1,Fight=1,Learn=1,Type=\Colles,Field=\Trigonométrie|Récurrences,Origin=] gg. 
Montrer que $\ds\sum\limits_{0\le k<n}{1\F\cos(kx)\cos(kx+x)}={\tan nx\F\sin x}$ 
pour $n\ge1$ et $x\notin\pi\ob Q$. 

\exo [Level=2,Fight=2,Learn=2,Type=\Exercices,Field=\Topologie,Origin=] gh. 
Prouver que $\ob Z$ est un ensemble fermé de $\ob R$ et de $\ob C$. 

\exo [Level=2,Fight=1,Learn=1,Type=\Cours,Field=\Topologie,Origin=] gi. 
Soient $\b(E,\|\cdot\|\b)$ un espace vectoriel normé, 
$x\in E$, $r>0$ et $\rho\ge0$. \pn
Prouver que la boule ouverte $B(x,r):=\{y\in E:\|x-y\|<r\}$ 
est une partie ouverte de $E$  
et que la boule fermée $\ol{B(x,\rho)}:=\{y\in E:\|x-y\|\le\rho\}$ 
est une partie fermée de $E$. 

\exo [Level=2,Fight=2,Learn=1,Type=\Exercices,Field=\Continuité,Origin=] gj. 
Soient $f,g:\ob R^p\to\ob R$ deux applications continues en $a\in\ob R^p$. \pn
Prouver que l'application $x\mapsto\min\{f(x),g(x)\}$ est continue en $a$. 

\exo [Level=2,Fight=1,Learn=1,Type=\Exercices,Field=\Normes,Origin=] gk. 
Prouver que l'application $P\mapsto\|P\|:=\sup\limits_{0\le t\le1}\b|P(t)-P'(t)\b|$ 
définit une norme de $\ob R[X]$. 

\exo [Level=2,Fight=1,Learn=1,Type=\Exercices,Field=\Normes,Origin=] gl. 
Prouver que l'application $(x,y)\mapsto\ds\sup_{t\in\ob R}{|x+ty|\F1+t+t^2}$ 
définit une norme de l'espace $\ob R^2$ et déssiner la boule unité. 

\exo [Level=1,Fight=2,Learn=1,Type=\Exercices,Field=\Suites,Origin=] gm. 
Pour $n\ge2$ et $(x_1,\cdots, x_n)\in\ob R^n$, montrer que 
$\ds \lim_{p\to\infty}\Q(\sum_{k=1}^n|x_k|^p\W)^{1/p}=\max_{1\le k\le n}|x_k|$.

\exo [Level=2,Fight=2,Learn=2,Type=\Cours,Field=\Topologie,Origin=] go. Montrer que $F\subset \ob R^p$ est fermé si, 
et seulement si, pour toute suite convergente $\{u_n\}_{n\in\ob N}\in F^{\ob N}$, on a $\lim_{n\to\infty}u_n\in A$ 
(i.e. $F$ est stable par passage à la limite). % Complément de cours

\exo [Level=2,Fight=0,Learn=0,Type=\TravauxDirigés,Field=\Continuité,Origin=] gn. 
Prouver que 
l'application $\ds(x,y,z,t)\mapsto \Q({\cos (zy)\F1+t^2+x^2},1,txyz\W)$ 
est continue sur $\ob R^4$. 

\exo [Level=1,Fight=2,Learn=2,Type=\Exercices,Field=\Suites,Origin=] gp. 
Étudier la limite de la suite $u$ définie par $u_0\in\ob R$ et $u_{n+1}:=\sin(u_n)$ pour $n\ge0$.  

\exo [Level=2,Fight=2,Learn=2,Type=\Cours,Field=\Normes,Origin=\MP] gq. 
Soit $\|.\|$ une norme de $\ob R^n$. Pour $A\in\sc M_n(\ob R)$, 
on pose $$
|||A|||:=\sup_{\Q|X\W|=1}\|AX\|=\sup{X\neq0}{\|AX\|\F \|X\|}.
$$ 
a) Montrer que l'appliation $A\mapsto|||A|||$ définit 
une norme de l'espace $\sc M_n(\ob R)$. \pn
b) Pour $(A,B)\in\sc M_n(\ob R)^2$, prouver que $|||AB|||\le |||A|||\times|||B|||$. % Complément de cours

\exo [Level=2,Fight=1,Learn=1,Type=\Cours,Field=\Normes,Origin=] gr. 
Prouver que l'application $f\mapsto\int_0^1\b|f(t)\b|\d t$ définit une norme sur l'espace $\sc C\b([0,1],\ob R\b)$. 

\exo [Level=2,Fight=0,Learn=0,Type=\Exercices,Field=\Continuité,Origin=] gs. 
Étudier l'existence de la limite $\ds \lim_{(x,y)\to(0,1)\atop x>0}\arctan{y\F x}$. 

\exo [Level=2,Fight=1,Learn=1,Type=\Colles,Field=\Diagonalisation,Origin=] gt. 
Soient $(A,B)\in\sc M_4(\ob C)^2$ telles que $A^2=B^2=\rm I_4$ et $AB=-BA$. \pn
a) En calculant $\rm{tr}(ABA)$ de deux fa\c con, montrer que $\rm{tr}\ A=\rm{tr}\ B=0$. \pn
b) Montrer que $A$ et $B$ sont diagonalisables. \pn
c) Déterminer les valeurs propres de $A$ et $B$ ainsi que leur ordre de multiplicité. \pn
d) On note $C:=iAB$. Vérifier que $C^2=\rm I_4$, que $AC=-CA$ et que $BC=-CB$. \pn
e) En déduire les valeurs propres de $iAB$ et $\rm{tr }AB$. 

\exo [Level=2,Fight=1,Learn=1,Type=\Exercices,Field=\Diagonalisation,Origin=] gu. 
Montrer que l'application $u:M\mapsto \rm{com}(M)$ 
définit un endomorphisme de $\sc M_2(\ob R)$. Diagonaliser $u$. 

\exo [Level=2,Fight=1,Learn=1,Type=\Colles,Field=\Diagonalisation,Origin=] gv. 
Soient $M,N$ deux matrices diagonalisables de $\sc M_n(\ob R)$ admettant le même polynôme caractéristique. \pn
Prouver qu'il existe un couple $(A,B)\in\sc M_n(\ob R)$ tel que $M=AB$ et $N=BA$. 

\exo [Level=1,Fight=2,Learn=2,Type=\Exercices,Field=\Trigonométrie,Origin=] gw. 
Trouver une expression simple de 
$\ds\sum_{0\le k\le n}\Q({n\atop k}\W)\cos(kx)$ pour $n\in\ob N^*$ et $x\in\ob R$. 

\exo [Level=1,Fight=2,Learn=1,Type=\Exercices,Field=\Trigonométrie|\Polynômes,Origin=] gx. 
Expliciter l'unique polynôme unitaire 
$P\in\ob R_3[X]$ vérifiant  $P\Q(\cos{2k\pi\F7}\W)=0$ pour $k\in\{2,4,6\}$. 

\exo [Level=2,Fight=2,Learn=2,Type=\Cours,Field=\Continuité,Origin=\MP] gy. %(Complément de cours) 
Soient $p,q$ deux entiers supérieurs à $1$ et $f:\ob R^p\to\ob R^q$. \pn
1) Montrer que $f\in\sc C(\ob R^p,\ob R^q)\Longleftrightarrow f^{-1}(O)$ 
est un ouvert de $\ob R^p$ pour chaque ouvert $O$ de $\ob R^q$. \pn
2) En déduire que $\sc Gl_n(\ob R)$ est ouvert dans $\sc M_n(\ob R)$. \pn
3) Prouver que $f\in\sc C(\ob R^p,\ob R^q)\Longleftrightarrow f^{-1}(F)$ 
est un fermé de $\ob R^p$ pour chaque fermé $O$ de $\ob R^q$. \pn
4) En déduire que l'ensemble des matrices symétriques est fermé. 
 
\exo [Level=1,Fight=1,Learn=1,Type=\Exercices,Field=\Trigonométrie,Origin=] gz. 
Trouver une expression simple de $\ds\sum_{0\le k\le n}\cos(3x)$ 
pour $n\in\ob N^*$ et $x\in\ob R$. 

\exo [Level=2,Fight=2,Learn=2,Type=\Cours,Field=\Normes,Origin=\MP] ha. 
Soient $n\ge1$, $N$ une norme de l'espace $E:=\ob R^n$ et $\|.\|_\infty$ la norme infinie. \pn
a) Prouver que $\b|N(x)-N(y)\b|\le N(x-y)$ pour $(x,y)\in E^2$. \pn
b) En déduire que $x\mapsto N(x)$ est une application continue de $(E,N)$ 
dans $(\ob R,|.|)$. \pn
c) Montrer que la sphére $\{x\in E:\|x\|_\infty=1\}$ est un compact de $(E,N)$. \pn
d) En déduire que les normes $N$ et $\|.\|_\infty$ sont équivalentes sur $E$.  
 
\exo [Level=2,Fight=2,Learn=2,Type=\TravauxDirigés,Field=\Compacité,Origin=\MP] hb. 
Soient $n\ge1$, $K$ un compact de $\ob R^n$ et $f:\ob R^n\to\ob R^n$ une application vérifiant $f(K)\subset K$ et  
\medskip
\hfill
$\ds
\|f(y)-f(x)\|_2<\|y-x\|_2\qquad(x\neq y).
$\hfill\null\medskip
\noindent
a) Montrer que $f$ admet au plus un point fixe dans $K$. \pn
b) Montrer que $f$ est continue sur $\ob R^n$. \pn
c) En déduire que l'application $g:x\mapsto\|x-f(x)\|_2$ est continue sur $\ob R^n$. \pn
d) Montrer que $f$ admet un unique point fixe $a$ dans $K$. 

\exo [Level=1,Fight=1,Learn=1,Type=\Exercices,Field=\NombresComplexes,Origin=] hc. 
Soit $a>0$. Montrer que les solutions de l'équation $\ds{a-z\F a+z}=\e^{2z}$ vérifient $\re z=0$.  

\exo [Level=1,Fight=1,Learn=1,Type=\Exercices,Field=\NombresComplexes,Origin=] hd. 
Soient $a>0$ et $n\ge1$. Expliciter en fonction de $a$ les solutions de l'équation $(1-ia)(1-z)^n=(1+ia)(1+z)^n$. 

\exo [Level=2,Fight=2,Learn=1,Type=\Exercices,Field=\Continuité,Origin=] he. 
Trouver une condition sur $(a,b,c,\alpha,\beta,\gamma)\in\Q[0,+\infty\W[^2$ pour 
que $\ds\lim_{\ss(x,y,z)\to(0,0,0)\atop\ss x>0,y>0,z>0}{x^\alpha y^\beta z^\gamma\F x^a+2y^b+3z^c}$ existe.  

\exo [Level=1,Fight=1,Learn=1,Type=\Exercices,Field=\Trigonométrie,Origin=] hf. 
Calculer $\cos (2\pi/5)+\cos(4\pi/5)$ et $\cos(2\pi/5)\cos(4\pi/5)$. 
En déduire $\cos(2\pi/5)$. 

\exo [Level=2,Fight=2,Learn=1,Type=\Exercices,Field=\DéveloppementsLimités,Origin=] hg. 
Développement limité au voisinage de $0$ de $f(x)=\arccos\ds{\sin x\F x}$ 
à l'ordre $4$. 

\exo [Level=1,Fight=2,Learn=1,Type=\Exercices,Field=\DéveloppementsLimités,Origin=] hh. 
Pour $x\in\Q]0,\e\W[$, on pose $f(x):=x^{1/x}$. \pn 
a) Prouver que $f$ est une bijection de $\Q]0,\e\W[$ dans $\Q]0,\e^{1/\e}\W[$. \pn
b) Existence et calcul du développemet limité en $1^+$ de $g$ à l'ordre $2$. 

\exo [Level=2,Fight=2,Learn=1,Type=\Exercices,Field=\Continuité,Origin=] hi. 
Soit $a\in\ob R^n$. Déterminer les applications $f\in\sc C(\ob R^n,\ob R^p)$ 
telles que $\forall x\in\ob R^n, f(2x+a)=f(x)$. 

\exo [Level=1,Fight=0,Learn=0,Type=\Exercices,Field=\DéveloppementsLimités,Origin=] hj. 
Calculer un développement limité à l'ordre $2$ de $f(x)=\sqrt{2+\sin x}$ en $0$. 

\exo [Level=1,Fight=0,Learn=0,Type=\Exercices,Field=\Equivalents,Origin=] hk. 
Déterminer la limite (ou un équivalent simple en cas de limite nulle ou infinie) pour $f(x)=(1-\ln x)^x$ en $0^+$. 

\exo [Level=1,Fight=1,Learn=0,Type=\Exercices,Field=\Fonctions,Origin=\CCP] hl. 
Calculer $\arctan 2+\arctan 5+\arctan 8$. \pn 
En déduire les solutions de l'équation 
$\arctan(x-3)+\arctan x+\arctan(x+3)=\ds{3\pi\F4}$. 

\exo [Level=1,Fight=1,Learn=0,Type=\Exercices,Field=\Fonctions,Origin=] hm. 
Lorsque elle est définie, simplifier l'expression $\ds\argth{1+xy+yz+xz\F x+y+z+xyz}$. 

\exo [Level=1,Fight=0,Learn=0,Type=\Exercices,Field=\Courbes,Origin=] hn. 
Construire la courbe d'équation $y=\ln(\ch x)-x$. 

\exo [Level=1,Fight=1,Learn=0,Type=\Exercices,Field=\Limites,Origin=] ho. 
Calculer $\ds\lim_{x\to0}\Q({1\F x\e^x(x+1)}-{1\F x\cos x}\W)$. 

\exo [Level=1,Fight=1,Learn=0,Type=\Exercices,Field=\Limites,Origin=] hp. 
Calculer $\ds\lim_{x\to1}\Q({2\F1-x^2}-{3\F1-x^3}\W)$. 

\exo [Level=1,Fight=1,Learn=0,Type=\Exercices,Field=\Limites,Origin=]  hq. 
Calculer $\ds\lim_{x\to+\infty}\Q(\sqrt{\ln(x^2+1)}-\sqrt{\ln(x^2-1)}\W)$. 

\exo [Level=1,Fight=1,Learn=0,Type=\Exercices,Field=\Limites,Origin=]  hr. 
Calculer $\ds\lim_{x\to\e^-}(\ln x)^{\ln(\e-x)}$. 

\exo [Level=1,Fight=1,Learn=0,Type=\Exercices,Field=\Limites,Origin=]  hs. 
Calculer $\lim_{x\to\infty}\Q(\ch\sqrt{x+1}-\ch\sqrt x\W)^{1/\sqrt x}$. 

\exo [Level=1,Fight=1,Learn=0,Type=\Exercices,Field=\Limites,Origin=]  
ht. Calculer $\lim_{x\to0}x^2\Q(\e^{1/x}-\e^{1/(x-1)}\W)$ et $\lim_{x\to+\infty}x^2\Q(\e^{1/x}-\e^{1/(x-1)}\W)$. 

\exo [Level=1,Fight=1,Learn=0,Type=\Exercices,Field=\Limites,Origin=]  hu. 
Calculer $\ds\lim_{x\to1}\Q({4\F\pi-4\arctan x}+{2\F x-1}\W)$. 

\exo [Level=1,Fight=1,Learn=0,Type=\Exercices,Field=\Equivalents,Origin=]  hv. 
Trouver un équivalent simple de $\tan x-\sqrt{\tan x^2}$ en $0^+$. 

\exo [Level=1,Fight=1,Learn=0,Type=\Exercices,Field=\Limites,Origin=]  hw. 
Calculer la limite $\lim_{x\to{\pi\F2}^-}\Q(\Q({\pi\F2}-x\W)\tan x\W)^{\tan x}$. 

\exo [Level=1,Fight=1,Learn=0,Type=\Exercices,Field=\Limites,Origin=] hx. 
Calculer la limite $\lim_{x\to+\infty}\b(\sh(\ch x)-\ch(\sh x)\b)$. 

\exo [Level=1,Fight=1,Learn=0,Type=\Exercices,Field=\Limites,Origin=]  hy. 
Soit $(a,b)\in\ob R^2$ tel que $a\neq0\neq b$. Calculer $\lim_{x\to1}\Q({a\F1-x^a}-{b\F1-x^b}\W)$. 

\exo [Level=1,Fight=1,Learn=0,Type=\Exercices,Field=\Equivalents,Origin=] hz. 
Trouver un équivalent simple de $\ds{1\F x}-{1\F2}\Q({1\F\sin x}+{1\F\sh x}\W)$ en $0$. 

\exo [Level=1,Fight=1,Learn=0,Type=\Exercices,Field=\Equivalents,Origin=] ia. 
Trouver un équivalent de $\sin(\sh x)-\sh(\sin x)$ en $0$. 

\exo [Level=1,Fight=1,Learn=0,Type=\Exercices,Field=\Limites,Origin=] ib. 
Calculer $\ds\lim_{x\to+\infty}\Q({\arctan(x+1)\F\arctan x}\W)^{x^2}$. 

\exo [Level=1,Fight=1,Learn=0,Type=\Exercices,Field=\Limites,Origin=] ic. 
Determiner la limite de la suite de terme général $u_n=\Q(3\times2^{1/n}-2\times3^{1/n}\W)^n$

\exo [Level=1,Fight=1,Learn=0,Type=\Exercices,Field=\Limites,Origin=] id. 
Déterminer la limite $\lim_{x\to 0\atop x\neq0}{\ln\cos ax\F\ln\cos bx}$ pour $a,b\in\ob R^*$. 

\exo [Level=1,Fight=1,Learn=0,Type=\Exercices,Field=\DéveloppementsLimités,Origin=] ie. 
Déterminer la limite $\ds\ell=\lim_{x\to0^+}{(\sin x)^x-1\F x^x-1}$. \pn
Trouver une meilleure estimation que $o(1)$ pour la  vitesse de convergence 
vers $\ell$. 

\exo [Level=1,Fight=1,Learn=0,Type=\Exercices,Field=\DéveloppementsLimités,Origin=] if. 
Effectuer un développement limité au voisinage de $0$ pour $f(x)={1\F\cos x}$ à l'ordre $5$.
 
\exo [Level=1,Fight=1,Learn=0,Type=\Exercices,Field=\Limites,Origin=] ig. Calculer $\lim_{x\to\infty}{\ln(x+1)\F\ln x}$. 

\exo [Level=2,Fight=1,Learn=1,Type=\Exercices,Field=\Diagonalisation,Origin=] ih. Soit $M:=\ds{1\F2}\pmatrix{-1&1&1\cr0&0&1\cr0&0&1\cr}$. 
Déterminer la limite de la suite de terme général $\ds u_n:=\sum_{1\le k\le n}M^k$. 

\exo [Level=2,Fight=1,Learn=0,Type=\Cours,Field=\Topologie,Origin=] ii. 
Montrer que $\{(x,y)\in\Q]0,\infty\W[^2: 2x+5y>10\hbox{ et } x^2+y^2<9\}$ 
est une partie ouverte de $\ob R^2$. 

\exo [Level=1,Fight=1,Learn=0,Type=\Exercices,Field=\Récurrences,Origin=] ij. 
Montrer que $\ds\prod_{0\le k\le n}\Q(1+{1\F 2k+1}\W)>\sqrt{2n+3}$ pour $n\ge0$. 

\exo [Level=1,Fight=1,Learn=0,Type=\Exercices,Field=\Récurrences,Origin=] ik. 
Soit $x\in\ob R\ssm\pi\ob Q$. 
Montrer que $\ds\sum_{1\le k\le n}{1\F\sin(2^kx)}=\cot x-\cot(2^nx)$ pour $n\ge0$. 

\exo [Level=1,Fight=2,Learn=2,Type=\Cours,Field=\Fonctions,Origin=] il. %Complément de cours 
Soit $f:E\to F$ une application. Montrer que : \pn
a) $f\hbox{ est injective } \Leftrightarrow\forall A\subset E, A=f^{-1}\b(f(A)\b)$. \pn
b) $f\hbox{ est surjecive } \Leftrightarrow\forall B\subset F, B=f\b(f^{-1}(B)\b)$. 

\exo [Level=1,Fight=3,Learn=1,Type=\Cours,Field=\Fonctions,Origin=\MP] im. 
Pour $(i,j)\in\ob N^2$, on pose $f(i,j)=\ds{(i+j)(i+j+1)\F2}+j$. 
\medskip\noindent
a) Montrer que $f$ définit une bijection de $\ob N^2$ dans $\ob N$. Faire un dessin. 
\pn
b) En déduire qu'il existe une surjection de $\ob N$ dans $\ob Q$. 

\exo [Level=2,Fight=2,Learn=1,Type=\Exercices,Field=\Continuité,Origin=] in. 
Etudiez la continuité sur $\ob R^2$ de l'application 
$f:(x,y)\mapsto \ds\Q\{\eqalign{
0&\quad\hbox{si }x=y,\cr
{\sin(xy)\F x+y}&\quad\hbox{si } x\neq y.}\W.$

\exo [Level=2,Fight=0,Learn=1,Type=\Cours,Field=\Topologie,Origin=] io. 
Soient $A$ un ouvert de $\ob R$ et $B$ un ouvert de $\ob R^2$. 
Montrer que $A\times B$ est un ouvert de $\ob R^3$. 

\exo [Level=2,Fight=2,Learn=1,Type=\Exercices,Field=\Continuité,Origin=] ip. 
Calculer $\ds\lim_{\ss(x,y)\to(0,0)\atop\ss x\neq y}{\sin x-\sin y\F\sh x-\sh y}$. 

\exo [Level=2,Fight=0,Learn=1,Type=\Cours,Field=\Topologie,Origin=] iq. 
Prouver que l'ensemble $\{(x,y,z)\in\ob R^3:x^2+2y^2+3z^2<1\}$ est ouvert dans $\ob R^3$. 

\exo [Level=2,Fight=1,Learn=2,Type=\Cours,Field=\Dérivation,Origin=] ir. % Complément de cours
On pose $f(0,0):=0$ et $$
\ds f(x,y)=xy{x^2-y^2\F x^2+y^2}\qquad \hbox{si}\quad (x,y)\neq(0,0).
$$
a) Montrer que $f$ est de classe $\sc C^\infty$ sur $\ob R^2\ssm\{(0,0)\}$. \medskip\noindent
b) Montrer que $f$ est de classe $\sc C^1$ sur $\ob R^2$. \medskip\noindent
c) Calculer $\ds{\partial^2f\F\partial x\partial y}(0,0)$ et $\ds{\partial^2f\F\partial y\partial x}(0,0)$. \medskip\noindent
d) En déduire que $f$ n'est par de classe $\sc C^2$ sur $\ob R^2$. 


\exo [Level=2,Fight=0,Learn=1,Type=\TravauxDirigés,Field=\Dérivation,Origin=] is. 
Déterminer si l'on définit une fonction de classe $\sc C^1$ sur $\ob R^2$ en posant 
$$
f(x,y)=\sqrt{x^2+y^2}\qquad (x,y)\in\ob R^2.
$$ 


\exo [Level=2,Fight=2,Learn=2,Type=\TravauxDirigés,Field=\EquationsAuxDérivéesPartielles,Origin=] it. 
Soit $f:\ob R^2\ssm\{(0,0)\}\to\ob R$ une application de classe $\sc C^2$ 
vérifiant 
\medskip\hfill
$\ds
\Delta f(x,y):={\partial^2f\F\partial x^2}(x,y)+{\partial^2f\F\partial y^2}(x,y)=0\qquad\b((x,y)\neq(0,0)\b). 
$\hfill\null\medskip\noindent
On suppose qu'il existe une application $g:\Q]0,\infty\W[\to\ob R$ telle que 
\medskip\hfill$\ds 
f(x,y)=g\Q(\sqrt{x^2+y^2}\W)\qquad\b((x,y)\neq(0,0)\b).
$\hfill\null\medskip\noindent
a) Prouver que $g$ est de classe $\sc C^2$ sur $\Q]0,\infty\W[$. \smallskip
\noindent
b) Exprimer $\Delta f(x,y)$ en fonction des dérivées $g$ au point $r:=\sqrt{x^2+y^2}$. 
\smallskip\noindent 
c) En déduire une équation différentielle satisfaite par $g$
\smallskip\noindent
d) En déduire $f$. 

\exo [Level=2,Fight=2,Learn=2,Type=\TravauxDirigés,Field=\EquationsAuxDérivéesPartielles,Origin=] iu. 
Soit $f:\ob R^3\ssm\{(0,0,0)\}\to\ob R$ une application de classe $\sc C^2$ 
vérifiant 
\medskip\hfill$\ds 
\Delta f(x,y,z):={\partial^2f\F\partial x^2}(x,y,z)
+{\partial^2f\F\partial y^2}(x,y,z){\partial^2f\F\partial y^2}(x,y,z)=0\qquad\b((x,y,z)\neq(0,0,0)\b). 
$\hfill\null\medskip\noindent
On suppose qu'il existe une application $g:\Q]0,\infty\W[\to\ob R$ telle que 
\medskip\hfill$\ds 
f(x,y,z)=g\Q(\sqrt{x^2+y^2+z^2}\W)\qquad\b((x,y,z)\neq(0,0,0)\b).
$\hfill\null\medskip\noindent
a) Prouver que $g$ est de classe $\sc C^2$ sur $\Q]0,\infty\W[$. \smallskip
\noindent
b) Exprimer $\Delta f(x,y,z)$ en fonction des dérivées $g$ au point $r:=\sqrt{x^2+y^2+z^2}$. 
\smallskip\noindent 
c) En déduire une équation différentielle satisfaite par $g$
\smallskip\noindent
d) En déduire $f$. 

\exo [Level=2,Fight=2,Learn=2,Type=\TravauxDirigés,Field=\EquationsAuxDérivéesPartielles,Origin=] iv. 
Soit $\alpha\in\ob R$ et soit $f\in\sc C^1\b(\Q]0,\infty\W[\times\ob R,\ob R\b)$ une solution 
de l'équation aux dérivées partielles 
\medskip\noindent($*$)\hfill$\ds 
x{\partial f\F\partial y}(x,y)-y{\partial f\F\partial x}=\alpha f(x,y)\qquad\b(x>0,y\in\ob R\b).
$\hfill\null\medskip\noindent 
a) Procéder au changement de variable polaire, en posant $f(x,y)=g(r,\theta)$. 
\medskip\noindent
b) exprimer $\ds{\partial f\F\partial x}(x,y)$ et $\ds{\partial f\F\partial y}(x,y)$ en fonction de 
$\ds{\partial g\F\partial r}(r,\theta)$ et $\ds{\partial g\F\partial\theta}(r,\theta)$. 
\medskip\noindent
c) En déduire une équation aux dérivées partielles simple satisfaite par $g$
\medskip\noindent
d) La résoudre et en déduire qu'il existe 
une fonction $\varphi\in\sc C^2(\Q]0,\infty\W[,\ob R)$ telle que 
\medskip\hfill$\ds 
f(x,y)=\varphi(\sqrt{x^2+y^2})\exp\Q(\alpha\arctan{y\F x}\W)\qquad(x>0,y\in\ob R). 
$\hfill\null\medskip\noindent 
e) Réciproquement, remarquer que les fonction de ce type sont solutions de ($*$). 

\exo [Level=2,Fight=1,Learn=1,Type=\Exercices,Field=\Extrema,Origin=] iw. 
Pour $(x,y)\in\ob R^2$, on pose $f(x,y)=x^3+3xy^2-15x-12y$. \pn
a) Déterminer les points critiques de $f$. \pn
b) Lesquels sont des extrema locaux et lesquels 
sont des extrema absolus/globaux. 

\exo [Level=2,Fight=1,Learn=1,Type=\Exercices,Field=\Extrema,Origin=] ix. 
Étudier les extrema de $f:(x,y)\mapsto 4x^4+4y^4-(x-y)^2$ sur $\ob R^2$. 

\exo [Level=2,Fight=1,Learn=1,Type=\Exercices,Field=\Extrema,Origin=] iy. 
Étudier les extremas de la fonction $f:(x,y)\mapsto2\sqrt{x^2+y^2}-x-y$ sur la boule fermée 
de centre $(0,0)$ et de rayon $2$. 

\exo [Level=2,Fight=1,Learn=1,Type=\Exercices,Field=\Extrema,Origin=] iz. 
Étudier les extrema locaux de $\ds f:(x,y)\mapsto\e^{x\sin y}$ sur $\ob R^2$.  

\exo [Level=2,Fight=2,Learn=1,Type=\Exercices,Field=\Extrema,Origin=] ja. 
Étudier les extrema locaux de $f:(x,y)\mapsto x\e^y+y\e^x$ sur $\ob R^2$. 

\exo [Level=2,Fight=3,Learn=1,Type=\Others,Field=\Extrema,Origin=] jb. 
Étudier l'existence et la continuité de $f:(x,y,z)\mapsto\sin|xyz|$ 
et de ses dérivées partielles. 

\exo [Level=2,Fight=2,Learn=2,Type=\Cours,Field=\Extrema,Origin=] jc. 
Calculer l'aire maximale d'un triangle inscrit dans un cercle. 

\exo [Level=2,Fight=1,Learn=1,Type=\Exercices,Field=\Extrema,Origin=] jd. 
Trouver les extrema locaux de $f:(x,y)\mapsto x^2+y^2+\cos(x^2+y^2)$ sur $\Q]-1/2,1/2\W[^2$. 

\exo [Level=2,Fight=1,Learn=1,Type=\Exercices,Field=\Extrema,Origin=] je. 
Trouver les extrema locaux de $\ds f:(x,y)\mapsto{xy\F(1+x)(1+y)(x+y)}$sur $\Q]0,\infty\W[^2$ 

\exo [Level=2,Fight=2,Learn=1,Field=\EquationsAuxDérivéesPartielles,Type=\Exercices,Origin=] jf. 
Trouver les applications $f\in\sc C^1([a,b],\ob R)$ telles 
que la fonction $g:(x,y)\mapsto \int_x^yf(t)\d t$ 
vérifie 
$$
\Delta g(x,y):={\partial^2 g\F\partial x^2}(x,y)+{\partial^2 g\F\partial y^2}(x,y)=0\qquad\b((x,y)\in[a,b]^2\b). 
$$

\exo [Level=2,Fight=2,Learn=2,Field=\FonctionsDePlusieursVariables,Type=\Exercices,Origin=] jg. 
On pose $\ds f(x,y):=\Q\{\eqalign{
{x\sin y-y\sin x\F x^2+y^2}&(x,y)\neq(0,0),\cr
0&(x,y)=(0,0).}
\W.$. \medskip\noindent
a) Continuité de $f$ sur $\ob R^2$ ? \pn 
b) Existence des dérivées partielles ? \pn 
c) $f$ est elle de classe $\sc C^1$ ?

\exo [Level=2,Fight=2,Learn=2,Type=\Exercices,Field=\Extrema,Origin=] jh. 
Montrer qu'il existe un minimum global et un maximum global de l'application $f:(x,y)\mapsto(y-x)^3+6xy$ sur l'ensemble $\Delta=\{(x,y)\in[-1,1]^2:x\le y\}$. 
Les calculer. 

\exo [Level=2,Fight=1,Learn=1,Type=\Exercices,Field=\Extrema,Origin=] ji. 
Extrema globaux et locaux de $(x,y)\mapsto -(x-1)^2-(x-\e^y)^2$ sur $\ob R^2$. 

\exo [Level=2,Fight=2,Learn=1,Field=\EquationsAuxDérivéesPartielles,Type=\Exercices,Origin=] jj. 
Déterminer les applications $f\in\sc C^2( \Q[0,+\infty\W[,\ob R)$ telles que
la fonction $g_f:(x,y)\mapsto f\Q({x^2+y^2\F z^2}\W)$ vérifie $\Delta g_f(x,y,z)=0$ 
pour $(x,y)\in\ob R^2$ et $z>0$. 

\exo [Level=2,Fight=1,Learn=1,Field=\EquationsAuxDérivéesPartielles,Type=\Cours,Origin=] jk. 
Déterminer les applications $f\in\sc C^1(\ob R^2,\ob R)$ vérifiant 
$$
\ds{\partial f\F \partial x}(x,y)=0\qquad\b((x,y)\in\ob R^2\b).
$$

\exo [Level=2,Fight=1,Learn=1,Field=\EquationsAuxDérivéesPartielles,Type=\Cours,Origin=] jl. 
Déterminer les applications $f\in\sc C^2(\ob R^3,\ob R)$ vérifiant \medskip
$$
{\partial^2 f\F\partial x\partial y}(x,y,z)=0\qquad\b((x,y,z)\in\ob R^3\b).
$$

\exo [Level=2,Fight=1,Learn=1,Field=\FonctionsDePlusieursVariables,Type=\Exercices,Origin=] jm. 
Prouver que $f:(x,y,z)\mapsto(x,xy,xyz)$ est un difféomorphisme 
de classe $\sc C^\infty$ de $\Q]0,\infty\W[^3$ dans $\Q]0,\infty\W[^3$. 

\exo [Origin=,Level=1,Fight=0,Learn=0,Type=\Cours,Field=\Déterminant,Solution={$D=1$.}] jn. 
Calculer le déterminant $D:=\Q\vert\matrix{10&7&8&7\cr7&5&6&5\cr8&6&10&9\cr7&5&9&10\cr}\W\vert$. 

\exo [Level=1,Fight=3,Learn=1,Field=\Fonctions,Type=\Exercices,Origin=] jo. 
Montrer qu'il n'existe pas d'application $f:\ob N\to\ob N$ vérifiant 
$\forall n\in\ob N, f\circ f(n)=n+2003$. 

\exo [Level=1,Fight=3,Learn=1,Field=\NombresEntiers,Type=\Exercices,Origin=] jp. 
Soient $m$ et $n$ deux entiers strictements positifs. \pn
a) Si $n\ge m$, combien y a t'il 
d'applications $f:\{1,\cdots, m\}\to\{1,\cdots,n\}$ strictement croissantes ? \pn
b) Combien y a t-il d'applications $f:\{1,\cdots, m\}\to\{1,\cdots,n\}$ croissantes (au sens large) ? 

\exo [Level=1,Fight=2,Learn=1,Field=\NombresEntiers,Type=\Exercices,Origin=] jq. 
Soit $n\in\ob N^*$. Montrer que le produit de $n$ entiers consécutifs est divisible par $n!$. 

\exo [Level=1,Fight=2,Learn=1,Field=\NombresEntiers,Type=\Exercices,Origin=] jr. 
Trouver une expression simple de la somme $\ds\sum_{0\le k\le n}\Q({n\atop k}\W)kx^k$ 
pour $n\in\ob N$ et $x\in\ob R$. 

\exo [Level=2,Fight=2,Learn=1,Field=\FonctionsDePlusieursVariables,Type=\Exercices,Origin=] js. 
Prouver que l'application $\ds (x,y)\mapsto{\sin x-\sin y\F\sh x-\sh y}$ est de classe $\sc C^1$ sur $\ob R^2$. 

\exo [Level=2,Fight=1,Learn=1,Field=\FonctionsDePlusieursVariables,Type=\Exercices,Origin=] jt. 
Pour $(x,y)\in\ob R^2$, on pose $r:=\sqrt{x^2+y^2}$ 
et $f(x,y):=\Q\{\eqalign{
-r^2& r\le1\cr 1-2r &r>1^{\strut}}\W.$. \pn
Quel est le plus grand entier $k$ tel que $f\in\sc C^k(\ob R^2,\ob R)$ ?   

\exo [Level=2,Fight=0,Learn=0,Field=\FonctionsDePlusieursVariables,Type=\Exercices,Origin=] ju. 
Pour $x\in\Q]0,2\pi\W[$ et $y\in\ob R$, on pose $P(x,y):=\ds{\sin x\F\ch y-\cos x}$ 
et $\ds Q(x,y):={\sin x\sh y\F(\ch y-\cos x)^2}$. \pn
Calculer $\ds\Delta P:={\partial^2P\F\partial x^2}+{\partial^2P\F\partial y^2}$ 
et $\ds\Delta Q:={\partial^2Q\F\partial x^2}+{\partial^2Q\F\partial y^2}$. 

\exo [Level=2,Fight=1,Learn=1,Field=\FonctionsDePlusieursVariables,Type=\Exercices,Origin=] jv. 
Soit $f\in\sc C^3(U,\ob R)$ une fonction harmonique sur un ouvert $U\subset\ob R^2$ 
(i.e. tq $\forall(x,y)\in U,\Delta f(x,y)=0$). \pn
Prouver que les fonctions $\ds {\partial f\F\partial x}$, $\ds{\partial f\F\partial y}$ 
et $\ds(x,y)\mapsto x{\partial f\F\partial y}(x,y)-y{\partial f\F\partial x}(x,y)$ sont harmoniques sur $U$. 

\exo [Level=2,Fight=1,Learn=1,Field=\FonctionsDéfiniesParUneIntégrale,Type=\Exercices,Origin=] jw. 
Existence, continuité, dérivabilité et calcul de 
$\ds g(x):=\int_0^1{\e^{-(t^2+1)x^2}\F t^2+1}\d t$. 

\exo [Level=2,Fight=2,Learn=2,Field=\FonctionsDéfiniesParUneIntégrale,Type=\Exercices,Origin=] jx. 
Pour $n\in\ob Z$, on pose $\ds J_n(x):={1\F\pi}\int_0^\pi\cos(nt-x\sin t)\d t$. \pn
a) Montrer que $J_n$ est définie sur $\ob R$. Quelle est sa parité ? \pn
b) Exprimer $J_{-n}$ en fonction de $J_n$. \pn
c) Montrer que $J_n$ est de classe $\sc C^\infty$ sur $\ob R$. \pn
d) Montrer que $J_n$ est solution de l'équation différentielle homogène 
$$
x^2J_n''(x)+xJ_n'(x)+(x^2-n^2)J_n(x)=0\qquad(x\in\ob R).
$$

\exo [Level=2,Fight=0,Learn=1,Field=\FonctionsDéfiniesParUneIntégrale,Type=\Exercices,Origin=] jy. 
Calculer $f(x)=\int_0^1\cos(xt)\d t$ pour $x\in\ob R$. En déduire que 
$$
g(x):=\Q\{\eqalign{{\sin x\F x}&\hbox{ si }x\neq0,\cr 1&\hbox{ si } x=0}\W.
$$ 
est une application de classe $\sc C^\infty$ sur $\ob R$. 

\exo [Level=2,Fight=2,Learn=2,Field=\FonctionsDéfiniesParUneIntégrale,Type=\Exercices,Origin=] jz. 
Soit $f\in\sc C\b([0,1),\ob C\b)$. Pour $x\in\ob R$, on pose $F(x):=\int_0^1 xf(t)\e^{-xt}\d t$. \pn
a) Montrer que $F$ est continue sur $\ob R$. \pn
b) Posant $\ds M(u) = \sup_{0\le t\le u}\b|f(t) - f(0)\b|$, établir la majoration
$$
\Q|F(x)-f(0)\int_0^1 x\e^{-xt}\d t\W| \le M\Q({1\F \sqrt x}\W) + M(1)\int_{1\F\sqrt x}^1 x\e^{-xt}\d t\qquad(x>0).
$$
c) Calculer les intégrales $\int_0^1 x\e^{-xt}\d t$ et $\int_{1\F\sqrt x}^1 x\e^{-xt}\d t$ ainsi que leur limite lorsque $x\to+\infty$. \pn
d) Prouver (par un raisonnement epsilonnique) que $\ds\lim_{u\to 0^+}M(u)=0$. \pn
e) En déduire la limite $\ds\lim_{x\to+\infty}F(x)$. 

\exo [Level=2,Fight=2,Learn=2,Field=\FonctionsDéfiniesParUneIntégrale,Type=\Exercices,Origin=] ka. 
Pour $x>0$, on pose $F(x):=\int_0^{\pi/2}\ln\b(x\cos^2t+\sin^2t\b)\d t$. \pn
a) Montrer que $F$ est définie et dérivable sur $\Q]0,\infty\W[$. \pn
b) Calculer $F'$, simplifier et en déduire $F(x)$. \pn
c) Pour $(a,b)\in\Q]0,\infty\W[^2$, calculer $\int_0^{\pi/2}\ln\b(a^2\cos^2t+b^2\sin^2t\b)\d t$. 


\exo [Level=2,Fight=2,Learn=2,Field=\FonctionsDéfiniesParUneIntégrale,Type=\Exercices,Origin=] kb. 
Pour $x\in\ob R$, on pose $f(x):=\ds\int_0^1{\e^{-tx}\d t\F1+t^2}$. \pn
a) Montrer que $f$ est de classe $\sc C^\infty$ sur $\ob R$. \pn
b) Trouver une équation diférentielle vérifiée par $f$. \pn
c) Montrer que $\ds f(x)=\int_0^xg(t)\cos(x-t)\d t+{\pi\F4}\cos x+{\ln 2\F 2}\sin x$ pour $x\in\ob R$ où $g$ est une fonction que l'on déterminera (Utiliser la méthode de variation de la constante). 

\exo [Level=2,Fight=2,Learn=2,Field=\FonctionsDePlusieursVariables,Type=\Exercices,Origin=] kc. 
Déterminer les fonctions $f\in\sc C^1(\ob R^2,\ob R)$ pour lesquelles 
la matrice~ja\-co\-bi\-en\-ne de l'application $\varphi:(x,y)\mapsto (f(x,y),2xy)$ est 
de la forme $\pmatrix{v_{x,y}&-w_{x,y}\cr w_{x,y}&v_{x,y}\cr}$  en chaque point~$(x,y)\in\ob R^2$.

\exo [Level=2,Fight=0,Learn=0,Field=\FonctionsDePlusieursVariables,Type=\Exercices,Origin=] kd. 
Pour chaque $(x,y,z)\in\Q]0,\infty\W[^3$, on pose $f(x,y,z)=\arctan(x/y)+\arctan(y/z)+\arctan(z/x)$. 
Calculer $\Delta f(x,y,z):=\ds{\partial^2f\F\partial x^2}(x,y,z)
+{\partial^2f\F\partial y^2}(x,y,z)+{\partial^2f\F\partial z^2}(x,y,z)$. 

\exo [Level=2,Fight=1,Learn=1,Field=\EquationsAuxDérivéesPartielles,Type=\Exercices,Origin=,Indication={Procéder au changement de variable $u=xy$ et $v=x/y$ (poser $f(x,y)=g(u,v)$\b).}] ke. 
Déterminer les solutions $f\in\sc C^2\b(\Q]0,\infty\W[^2,\ob R\b)$ de l'équation aux dérivées partielles 
$$
x^2{\partial^2f\F\partial x^2}(x,y)-y^2{\partial^2f\F\partial y^2}(x,y)=0.
$$


\exo [Level=1,Fight=1,Learn=1,Field=\Groupes,Type=\Exercices,Origin=] kf. 
a) Montrer que l'intervalle $I=\Q]-1,1\W[$ muni de la loi\medskip
\hfill$\ds x\star y={x+y\F1+xy}$\hfill\null
forme un groupe abélien. \pn
b) Trouver un isomorphisme de groupe entre $(I,\star)$ et $(\ob R,+)$. 

\exo [Level=1,Fight=1,Learn=1,Field=\Anneaux,Type=\Exercices,Origin=] kg. 
On munit l'ensemble $E=\sc F(\ob N^*,\ob C)$ de la loi $\star$ définie par 
$$
(f\star g)(n)=\sum_{d|n}f(d)g(n/d)\qquad(n\in\ob N^*). 
$$
a) Montrer que $(E,+,\star)$ est un anneau. Quel est son élément unité ? \pn
b) Montrer que $f$ est inversible si, et seulement si, $f(1)\neq0$. 

\exo [Level=1,Fight=1,Learn=1,Field=\Anneaux,Type=\Exercices,Origin=] kh. 
Soit $(A,+,.)$ un anneau vérifiant $\forall x\in A, x^2=x$ (anneau de Boole). \pn
a) Donner un exemple simple d'anneau de Boole. \pn
b) Montrer que $2x=0$ pour chaque $x\in A$. \pn
c) En déduire que $A$ est commutatif. \pn
d) Montrer que $\hbox{Card}\ A\neq3$. \pn
e Lorsque que $\hbox{Card}\ A>2$, montrer que $A$ n'est pas intégre. 

\exo [Level=1,Fight=1,Learn=1,Field=\Anneaux,Type=\Exercices,Origin=] ki. 
On munit l'ensemble $A:=\b\{a+b\sqrt2:(a,b)\in\ob Z^2\b\}$ 
de l'addition et de la multiplication usuelles. \pn
a) Montrer que $(A,+,.)$ est un anneau, commutatif et intégre, inclus dans $\ob R$. \pn
b) Pour $x=(a+b\sqrt2)$, on pose $N(x)=a^2-2b^2$. Montrer que $N(xy)=N(x)N(y)$ pour $(x,y)\in A^2$. \pn
c) Prouver que $x$ est inversible dans $A$ si, et seulement si, $N(x)=\pm1$. \pn
d) Trouver une infinité d'éléments inversibles. 

\exo [Level=1,Fight=2,Learn=2,Field=\Groupes,Type=\Exercices,Origin=] kj. 
Soit $G$ un groupe (noté multiplicativement). Pour $a\in G$, on note $\phi_a$ l'application 
$$
\eqalign{\phi_a:G&\to G\cr x&\mapsto axa^{-1}}.
$$
a) Pour $a\in G$, montrer que l'application $\phi_a$ est un automorphisme de $G$. \pn
b) Prouver que l'ensemble $\mbox{\rm Aut}(G)$ des automorphismes de $G$ 
forme un groupe pour la loi~$\circ$. \pn
c) Prouver que l'application $\eqalign{\Phi: (G,*)&\to(\mbox{\rm Aut}(G),\circ)\cr a&\mapsto\phi_a}$ 
est un morphisme de groupe. \smallskip\noindent
d) Que vaut l'ensemble $\{a\in G:\Phi(a)=1\}$ ?

\exo [Level=2,Fight=1,Learn=1,Type=\Colles,Field=\Diagonalisation,Origin=] kk. 
Trouver une condition nécéssaire et suffisante sur les nombres réels 
$a$, $b$ et $c$ pour que la matrice $M(a,b,c)$ soit diagonalisable. 
Dans ce cas, trouver une matrice $Q\in\sc Gl_5(\ob R)$ 
telle que $QM(a,b,c)Q^{-1}$ soit diagonale.  
$$
M(a,b,c):=\pmatrix{a&0&0&0&b\cr 0&a&1&b&0\cr0&0&c&0&0\cr0&b&1&a&0\cr b&0&0&0&a\cr}
$$ 

\exo [Level=2,Fight=1,Learn=1,Type=\Maple,Field=\Diagonalisation,Origin=] kl. 
Trouver une condition sur $a$ et $b$ pour que $M(a,b):=\pmatrix{0&0&0&1\cr 0&0&0&0\cr0&b&a&0\cr1&0&0&0\cr}$ 
soit diagonalisable. 
Dans ce cas, trouver une matrice $Q\in\sc Gl_4(\ob R)$ 
telle que $Q^{-1}M(a,b)Q$ soit diagonale.  

\exo [Level=2,Fight=1,Learn=1,Type=\Mathematica,Field=\Programmation,Origin=] 
km. a) Programmer une fonction "CarreeQ" à valeurs booléennes 
testant si un objet/une matrice $M$ est carrée\pn
b) Programmer une fonction "DiagonnalisableQ" à valeurs booléennes testant si 
un object/une matrice carrée $M$ est diagonalisable. 

\exo [Level=2,Fight=1,Learn=1,Type=\Maple,Field=\VecteursPropres,Origin=] kn. 
Condition sur $a$, $b$, $c$, $d$, $e$ et $f$ 
pour que $(1,1,0)$, $(1,2,1)$ et $(1,1,2)$ 
forment une base de vecteurs propres de 
$$
A:=\pmatrix{a&1&b\cr1&c&d\cr e&f&-1\cr}
$$ 

\exo [Level=2,Fight=1,Learn=1,Type=\Mathematica,Field=\Diagonalisation,Origin=] ko. 
Calculer $A^n$ avec 
$A=\pmatrix{2&1&1&1&1\cr1&2&1&1&1\cr1&1&2&1&1\cr1&1&1&2&1\cr1&1&1&1&2\cr}$
en utilisant l'instruction $MatrixPower[A,n]$ puis en diagonalisant $A$...

\exo [Level=2,Fight=1,Learn=1,Type=\Mathematica,Field=\Anneaux,Origin=] kp. 
Pour $(A,B,C)\in\sc M_3(\ob R)$, on pose $[A,B]:=AB-BA$. Prouver que 
$$
\Q[A,[B,C]\W]+\Q[B,[C,A]\W]+\Q[C,[A,B]\W]=0.
$$
 
\exo [Level=1,Fight=0,Learn=0,Type=\Exercices,Field=\Primitives,Origin=] kq. 
Calculer les primitives $\ds\int{\sqrt{x+2}\d x\F(\sqrt{x+2}+1)(x+3)}$. 

\exo [Level=1,Fight=1,Learn=0,Type=\Exercices,Field=\Primitives,Origin=] kr. 
Calculer les primitives $\ds\int{\d x\F\cos x-\cos 3x}$. 

\exo [Level=1,Fight=1,Learn=0,Type=\Exercices,Field=\Primitives,Origin=] ks. 
Calculer les primitives $\ds\int{\d x\F\sqrt{x^2+2x}+x}$


\exo [Level=2,Fight=0,Learn=0,Field=\IntégralesGénéralisées,Type=\Exercices,Origin=\Fac,Indication={En $+\infty$ : majorer par ${1\F x^2}$.},Solution={L'intégrale converge}] kt. 
Convergence de $\int _0^\infty\e^{-\sqrt x}\d x$. 

\exo [Level=2,Fight=2,Learn=1,Type=\Exercices,Field=\IntégralesGénéralisées,Origin=] kt. 
Calculer $\ds\int_0^1(1-x^2)\ln{1+x\F1-x}\d x$. 

\exo [Level=1,Fight=1,Learn=0,Type=\Exercices,Field=\Intégrales,Origin=] ku. 
Calculer $\ds\int_0^{\pi/6}{\d x\F\cos^4x+\sin^4x}$

\exo [Level=1,Fight=1,Learn=0,Type=\Exercices,Field=\Intégrales,Origin=] kv. 
Calculer $\ds\int_{-1}^1{\d x\F\sqrt{1+x^2}+\sqrt{1-x^2}}$

\exo [Level=2,Fight=1,Learn=0,Type=\Exercices,Field=\IntégralesGénéralisées,Origin=] kw. 
Nature de $\ds\int_0^\infty\sin(x^2)\d x$. 

\exo [Level=2,Fight=1,Learn=0,Type=\Exercices,Field=\IntégralesGénéralisées,Origin=] kx. 
Nature de $\ds\int_0^\infty{\sin t\F\sqrt t}\d t$. 

\exo [Level=2,Fight=1,Learn=0,Type=\Exercices,Field=\IntégralesGénéralisées,Origin=] ky. 
Nature de $\ds\int_0^\infty\e^{-\alpha x}x^\beta\d x$. 

\exo [Level=2,Fight=1,Learn=0,Type=\Exercices,Field=\IntégralesGénéralisées,Origin=] kz. 
Nature de $\ds\int_2^\infty x^\alpha(\ln x)^\beta$ pour $(\alpha,\beta)\in\ob R^2$. 

\exo [Level=2,Fight=1,Learn=0,Type=\Exercices,Field=\IntégralesGénéralisées,Origin=] la. 
Nature de $\ds\int_0^1{\d t\F1-\sqrt{1-t}}$. 

\exo [Level=2,Fight=1,Learn=0,Type=\Exercices,Field=\IntégralesGénéralisées,Origin=] lb. 
Nature de $\ds\int_1^\infty{|\sin t|\F t}\d t$. 

\exo [Level=2,Fight=1,Learn=0,Type=\Exercices,Field=\IntégralesGénéralisées,Origin=] lc. 
Nature et calcul de $\ds I:=\int_0^{\pi/2}\ln(\sin x)\d x$ et $\ds J:=\int_0^{\pi/2}\ln(\cos x)\d x$

\exo [Level=2,Fight=1,Learn=0,Type=\Exercices,Field=\IntégralesGénéralisées,Origin=] ld. 
Nature et calcul de $\ds\int_0^\infty{\arctan^2x\F x^2}\d x$. 

\exo [Level=2,Fight=1,Learn=0,Type=\Exercices,Field=\IntégralesGénéralisées,Origin=] le. 
Nature et calcul de $\ds\int_1^\infty\Q({1\F x}-\arctan{1\F x}\W)\d x$. 

\exo [Level=2,Fight=1,Learn=0,Type=\Exercices,Field=\IntégralesGénéralisées,Origin=] lf. 
Nature et calcul de $\ds\int_{-\infty}^{\infty}\Q(\arctan(x+a)-\arctan x\W)\d x$

\exo [Level=2,Fight=1,Learn=0,Type=\Exercices,Field=\IntégralesGénéralisées,Origin=X] lg. 
Nature et calcul de $\ds\int_0^1{\d x\F(x^2-x^3)^{1/3}}$. 

\exo [Level=2,Fight=1,Learn=0,Type=\Exercices,Field=\IntégralesGénéralisées,Origin=] lh. 
Nature et calcul de $\ds\int_0^\infty{\ln x\F(1+x^2)^2}\d x$. 

\exo [Level=2,Fight=1,Learn=0,Type=\Exercices,Field=\IntégralesGénéralisées,Origin=] li. 
Pour $n\in\ob N^*$, nature et calcul de $\ds I_n:=\int_{-\infty}^\infty t^n\e^{-t^2}\d t$ 
sachant que $I_0={\pi/2}$. 

\exo [Level=2,Fight=1,Learn=0,Type=\Exercices,Field=\IntégralesGénéralisées,Origin=] lj. 
Nature de l'intégrale $\ds\int_\e^\infty{\d t\F(\ln t)^{\ln t}}$. 

\exo [Level=1,Fight=2,Learn=1,Type=\Exercices,Field=\Groupes,Origin=] lk. 
Pour $\theta\in\ob R$, on pose $\Gamma_\theta:=\pmatrix{\cos^2\theta&-\sin2\theta&\sin^2\theta\cr\cos\theta\sin\theta&\cos2\theta&-\sin\theta\cos\theta\cr\sin^2\theta&\sin2\theta&\cos^2\theta\cr}$. 
\pn
a) Calculer $\det \Gamma_\theta$. \pn
b) Montrer que $\{\Gamma_\theta\}_{\theta\in\ob R}$ forme un groupe. 

\exo [Level=2,Fight=2,Learn=2,Type=\Colles,Field=\DimensionFinie,Origin=MP*] ll. 
Soient $E,F$ deux espaces vectoriels de dimension finie $n$ et $(u,v)\in\sc L(E,F)^2$. \pn
a) Montrer que $\hbox{\rm rg}(u+v)\le \hbox{\rm rg}\  u+\hbox{\rm rg}\  v$. \pn
b) Montrer que $|\hbox{\rm rg}\ u-\hbox{\rm rg}\ v|\le \hbox{\rm rg}(u+v)$. \pn
c) Lorsque $E=F$, $u+v$ est inversible et $u\circ v=0$, montrer que $\hbox{\rm rg}\ u+\hbox{\rm rg}\ v=n$. \pn
d) Montrer que $\sc I\hbox{\rm m}\ u\subset\sc I\hbox{\rm m}\ v$ si, et seulement si, il existe $w\in\sc L(E,F)$ tel que $u=v\circ w$. 

\exo [Level=2,Fight=1,Learn=0,Type=\Exercices,Field=\Déterminant,Origin=] lm. 
Pour $(x_1,\cdots, x_n)\in\ob C^n$, calculer le déterminant de $\ds A:=\b((x_i+x_j)^2\b)_{1\le i,j\le n}$. 

\exo [Level=1,Fight=4,Learn=1,Type=\Others,Field=\EspacesVectoriels,Origin=] ln. % dur ! 
Montrer que la famille $\{f_{a,b}:x\mapsto \arctan(ax+b)\}_{a>0,b\in\ob R}$ est libre dans $\sc C(\ob R,\ob R)$. 

\exo [Level=1,Fight=1,Learn=1,Type=\Colles,Field=\SystèmesLinéaires,Origin=,Solution={Comme $(E)$ admet une solution si, et seulement si ($\lambda\notin\{1,-3\}$) ou ($\lambda=1$ et $a=1$) ou ($\lambda=-3$ et $a\in\{i,-1,-i\}$)},Notion=Déterminant] lo. 
Condition nécéssaire et suffisante sur $\lambda$ et $a$ pour que le  système 
$$
\Q\{\eqalign{
\lambda x+y+z+t&=1\cr
x+\lambda y+z+t&=a\cr
x+y+\lambda z+t&=a^2\cr
x+y+z+\lambda t&=a^3
	}\W.\leqno{(E)}
$$
admette au moins une solution. 

\exo [Level=1,Fight=1,Learn=1,Type=\Colles,Field=\EspacesVectoriels,Origin=] lp. 
Prouver que la famille $\{x\mapsto|x-a|\}_{a\in\ob R}$ 
est libre dans $\sc C(\ob R,\ob R)$. 

\exo [Origin=\Fac,Level=2,Fight=0,Learn=0,Type=\TravauxDirigés,Field=\Diagonalisation,Solution={$P:=\pmatrix{1&0&1\cr0&1&1\cr1&1&0}$ et $D:=\pmatrix{3&0&0\cr0&-1&0\cr0&0&1\cr}$}] lq. 
Diagonaliser la matrice $A:=\pmatrix{2&-1&1\cr1&0&-1\cr2&-2&1}$.  


\exo [Level=2,Fight=2,Learn=1,Type=\Colles,,Field=\Déterminant,Origin=] lr. 
Soit $A\in\sc M_n(\ob C)$ et soit $\hbox{\rm Com}(A)$ la comatrice de $A$. \pn
a) Calculer le produit matriciel $A\times\hbox{\rm Com}(A)$.\pn
b) En déduire le rang de la comatrice de $A$. 

\exo [Level=2,Fight=1,Learn=1,Type=\Exercices,Field=\Extrema,Origin=] ls. 
Trouver à volume égal les parallélépipèdes de moindre surface. 

\exo [Level=2,Fight=1,Learn=1,Type=\Exercices,Field=\Extrema,Origin=] lt. 
Étudier les extrema de $f:(x,y)\mapsto\sqrt{x^2+(1-y)^2}+\sqrt{y^2+(1-x)^2}$ sur $\ob R^2$. 

\exo [Level=2,Fight=1,Learn=1,Type=\Exercices,Field=\Extrema,Origin=] lu. 
Étudier les extrema de l'application $f:(x,y,z)\mapsto x^2+y^2+z^2-2xyz$ sur $\ob R^2$. 

\exo [Level=2,Fight=1,Learn=1,Type=\Exercices,Field=\Extrema,Origin=] lv. 
Étudier les extrema de l'application $f:(x,y)\mapsto x\b((\ln x)^2 +y^2\b)$ sur $\Q]0,\infty\W[\times\ob R$. 

\exo [Level=1,Fight=1,Learn=1,Type=\Exercices,Field=\EspacesVectoriels,Origin=] lw. 
Soient $E$ un $\ob K$-espace vectoriel et $F,G$ deux sous-espaces vectoriels de $E$. Enoncer une condition nécéssaire et suffisante pour que $F\cup G$ 
soit un espace vectoriel. 

\exo [Level=1,Fight=1,Learn=1,Type=\Exercices,Field=\EspacesVectoriels,Origin=] lx. 
a) Soit $E:=\sc F(\ob R,\ob R)$. Montrer que l'ensemble $F$ des fonction paires $f:\ob R\to\ob R$ 
et que l'ensemble $G$ des fonctions $g:\ob R\to\ob R$ impaires
forment des espaces vectoriels. \pn 
b) Prouver que $E=F\oplus G$ (i.e. que $E=F+G$ et que $F\cap G=\{0\}$). 

\exo [Origin=,Level=1,Fight=1,Learn=2,Type=\Colles,Field=\EspacesVectoriels] ly. 
Soient $a_0<...<a_n$ et soit $F$ l'ensemble $\{f\in\sc F(\ob R,\ob R):f(a_0)=...=f(a_n)=0\}$. \pn
a) Prouver que $F$ est un espace vectoriel. \pn
b) Trouver un suplémentaire de $F$ dans $\sc F(\ob R,\ob R)$. 


\exo [Level=1,Fight=1,Learn=1,Type=\Exercices,Field=\Groupes,Origin=] lz. 
Pour $(x,y)\in\ob R^2$, on pose $\ds x*y:=(x^3+y^3)^{1/3}$. 
Prouver que $(\ob R,*)$ forme un groupe. 

\exo [Level=1,Fight=1,Learn=1,Type=\Exercices,Field=\Groupes,Origin=] ma. 
Pour $(x,y)\in\Q]-1,1\W[$, on pose $\ds x*y:={x+y\F1+xy}$. Prouver que $\b(\Q]-1,1\W[,*\b)$ forme un groupe. 

\exo [Level=1,Fight=1,Learn=1,Type=\Exercices,Field=\Groupes,Origin=] mb. 
Soit $(G,*)$ un groupe. Prouver que l'ensemble $\{x\in G:\forall y\in G, x*y=y*x\}$ muni de la loi $*$ 
forme un groupe. 

\exo [Level=2,Fight=1,Learn=1,Type=\Exercices,Field=\FonctionsDéfiniesParUneIntégrale,Origin=] mc. 
Trouver un équivalent de $\ds \int_1^\e{\ln t\F\sqrt{x+t}}\d t$ lorsque $x\to+\infty$

\exo [Level=1,Fight=3,Learn=1,Type=\Exercices,Field=\Intégration,Origin=] md. 
Calculer la limite $\lim_{x\to 0^+}\Q(\int_0^1f(t)^x\d t\W)^{1/x}$. \pn
a) Pour les applications $t\mapsto t^\alpha$ et $t\mapsto \e^{\alpha t}$ avec $\alpha\ge0$. \pn
b) lorsque $f\in\sc C\b([0,1],\Q]0,\infty\W[\b)$. 

\exo [Level=2,Fight=2,Learn=1,Type=\Exercices,Field=\IntégralesGénéralisées,Origin=] me. 
Pour $f\in\sc C\Q([0,1],\ob R\W)$, calculer $\ds \lim_{n\to\infty}\int_0^1f(t)t^n\ln t\d t$

\exo [Level=2,Fight=2,Learn=1,Type=\Exercices,Field=\IntégralesGénéralisées,Origin=] mf. 
Calculer $\ds\lim_{n\to\infty}\int_0^{\pi/4}{\sin t\F 1+t+t^2+\cdots+t^n}$. 

\exo [Level=1,Fight=1,Learn=1,Type=\Maple,Field=\DéveloppementsLimités,Origin=] mg. 
à l'aide des commandes ``series'' et ``solve'', déterminer $a,b,c,d,e$ pour que $\ds f:x\mapsto\sh x-{ax+bx^3+cx^5\F1+dx^2+ex^4}$ soit un infiniment petit 
d'ordre le plus élevé possible au voisinage de $x=0$ (c'est à dire de la forme $\alpha x^\beta$ avec $\beta$ le plus grand possible) 
et donner alors un équivalent de $f(x)$. 

\exo [Level=2,Fight=2,Learn=1,Type=\Exercices,Field=\FonctionsDePlusieursVariables,Origin=] mh. 
Soit $f:\Q]0,\infty\W[\times\ob R\to\ob R$ de classe $\sc C^2$ 
telles que $f(x,y)=F(y/x)$ pour $x>0$ et $y\in\ob R$. \pn
a) Calculer $\Delta f$. \pn
b) Lorsque $\Delta f(x,y)=0$ pour $x>0$ et $y\in\ob R$, déterminer $f$. \pn
c) Même question en supposant que $F(0)=0$ et $F(1)=1$. 

\exo [Level=2,Fight=2,Learn=1,Type=\Exercices,Field=\Orthonormalisation,Origin=] mi. 
Minimum de l'application $(u,v)\mapsto\int_0^1(\sh x-ux-v)^2\d x$. 

\exo [Level=1,Fight=1,Learn=1,Type=\Maple,Field=\DéveloppementsLimités,Origin=] mj. 
Trouver les coefficients $a$, $b$, $c$ pour lesquels la fonction 
$$
f(x)={a\F\sin x}+{b\F\tan x}+{cx\F\cos x}+{1\F x}
$$ 
est prolongeable par continuité en $0$ et possède en $0$ le plus petit équivalent possible. 

\exo [Level=2,Fight=2,Learn=1,Type=\Exercices,Field=\FonctionsDePlusieursVariables,Origin=] mk. 
Trouver toutes les fonctions $\phi:\ob R\to\ob R$ de classe $\sc C^2$ telle que la fonction $f$ 
définie pour~$x>0$ et~$y\in\ob R$ par $f(x,y)=\phi(x^2+y^2)$ vérifie l'équation 
$$
{\partial^2f\F\partial x\partial y}(x,y)-3y{\partial f\F\partial x}=f(x,y)\qquad(x>0,y\in\ob R). 
$$

\exo [Level=1,Fight=1,Learn=1,Type=\Maple,Field=\Intégrales,Origin=] ml. 
Trouver $(a,b,c)$ tels que $\ds \int{ax^4+bx^2+c\F(x-1)^3(x+2)^5}\d x$ 
soit une fraction rationelle de $2$ fa\c cons différentes (avec $2$ commandes distinctes). 

\exo [Level=2,Fight=1,Learn=0,Type=\Exercices,Field=\FonctionsDePlusieursVariables,Origin=] mm. 
Calculer la limite de  de $\ds (x,y)\mapsto {2y^3+2y^2-2xy^2+3x^2y+x^2+x^3\F x^2+2y^2}$ en $(0,0)$. 

\exo [Level=1,Fight=1,Learn=0,Type=\Exercices,Field=\Fonctions,Origin=] mn. 
Graphe de la fonction $\ds x\mapsto {2x-\e^{-x}\F\ln\Q(1+\ch(x)\W)+\e^{-x}}$ et asymptotes

\exo [Level=2,Fight=2,Learn=2,Type=\Exercices,Field=\FonctionsDéfiniesParUneIntégrale,Origin=] mo. 
On pose $\ds f(x)=\int_0^\infty{\e^{-xt}\F1+t^2}\d t$ et $\ds g(x):=\int_0^\infty{\sin t\F x+t}\d t$. \pn
a) Montrer que $f$ est de classe $\sc C^2$ sur $\Q]0,\infty\W[$. \pn
b) En intégrant au besoint par partie, prouver que $g$ est de clase $\sc C^2$ sur $\Q]0,\infty\W[$. \pn
c) Montrer que $f$ et $g$ vérifient $y''+y={1\F x}$ sur $\Q]0,\infty\W[$. \pn
d) Montrer que $f$ et $g$ sont continues en $0$. \pn
e) En déduire que $\ds\int_0^\infty{\sin u\F u}\d u={\pi\F 2}$. 

\exo [Level=2,Fight=2,Learn=2,Type=\Exercices,Field=\FonctionsDéfiniesParUneIntégrale,Origin=] mp. 
On pose $\ds f(x)=\int_0^{\pi/2}(\sin t)^x\d t$. \pn
a) Domaine de définition, de continuité, de dérivabilité de $f$ ? \pn
b) On pose $\phi(x):=xf(x)f(x-1)$. Prouver que $\phi$ est $1$-périodique. \pn
c) Montrer que $\ds x\mapsto{\phi(x)\F x}$ est décroissante. \pn
d) Montrer que $\phi$ est constante. \pn
e) Equivalent de $f$ en $-1^+$ et en $+\infty$ ? 

\exo [Level=2,Fight=1,Learn=2,Type=\Exercices,Field=\FonctionsDéfiniesParUneIntégrale,Origin=] mq. 
On pose $\ds G(x):=\int_0^\infty{1-\cos(xt)\F t^2}\d t$. \pn
a) Prouver que $G$ est définie et continue sur $\ob R$. \pn
b) Dérivabilité de $G$ ?

\exo [Level=2,Fight=1,Learn=2,Type=\Exercices,Field=\FonctionsDéfiniesParUneIntégrale,Origin=] mr. 
On pose $\ds f(x):=\int_0^1{t^x-1\F\ln t}\d t$. \pn
a) Pour quels $x$ la fonction $f$ est elle définie ? \pn
b) Calculer $f'$ et en déduire une expression simple pour $f(x)$. 

\exo [Level=2,Fight=1,Learn=2,Type=\Exercices,Field=\FonctionsDéfiniesParUneIntégrale,Origin=] ms. 
On pose $f(x):=\int_0^\infty\ch(2xt)\e^{-t^2}\d t$. \pn
a) Prouver que $f$ est de classe $\sc C^1$ sur $\ob R$ et satisfait $y'-2xy=0$. \pn
b) En déduire que $\ds \int_0^\infty\ch(2xt)\e^{-t^2}\d t={\sqrt\pi\F 2}\e^{x^2}$ pour $x\in\ob R$. 

\exo [Level=2,Fight=1,Learn=2,Type=\Exercices,Field=\FonctionsDéfiniesParUneIntégrale,Origin=] mt. 
Soit $f$ continue et bornée sur $\Q[0,+\infty\W[$. Déterminer $\ds\lim_{x\to0}\int_0^\infty{xf(t)\F x^2+t^2}\d t$. 

\exo [Level=2,Fight=2,Learn=2,Type=\Exercices,Field=\FonctionsDéfiniesParUneIntégrale,Origin=] mu. 
Pour $x>0$, on pose $\Gamma(x):=\int_0^\infty\e^{-t}t^{x-1}\d t$. \pn
a) Ensemble de définition $\sc D\Gamma$ de $\Gamma$ ? \pn 
b) Montrer que $\Gamma(x+1)=x\Gamma(x)$ pour $x\in\sc D\Gamma$. \pn
c) Prouver que $\Gamma$ est de classe $\sc C^\infty$ sur $\sc D\Gamma$. 

\exo [Level=2,Fight=1,Learn=2,Type=\Exercices,Field=\FonctionsDéfiniesParUneIntégrale,Origin=] mv. 
Soit $f:\ob R^2\to\ob R$ une fonction de classe $\sc C^\infty$. 
On pose $F(x):=\int_0^xf(t,x)\d t$. \pn
Calculer $F'(x)$ et $F''(x)$ pour $x\in\ob R$. 
 
\exo [Level=2,Fight=2,Learn=2,Type=\Exercices,Field=\FonctionsDéfiniesParUneIntégrale,Origin=] mw. 
On pose $\ds F(x):=\int_0^1{\d t\F\sqrt{t(1-t)(t+x)}}$. \pn
a) Démontrer que $F$ est définie sur $\Q]0,\infty\W[$. \pn
b) Démontrer que $F$ est équivalent en $+\infty$ à $\ds{\pi\F\sqrt x}$. \pn
c) Démontrer que $F(x)$ est équivalent en $0^+$ à $\ds \int_0^1{\d t\F{t(t+x)}}$ et donc à $-\ln x$. 

% Redondant mx. 

\exo [Origin=,Level=1,Fight=1,Learn=1,Type=\Cours,Field=\EspacesVectoriels] my. 
Soient $E$ un espace vectoriel et  
$E_1,\cdots, E_n$ des sous-espaces vectoriels de $E$. \pn
Prouver que $\hbox{\rm Vect\ }(E_1\cup\cdots\cup E_n)
=E_1+\cdots+E_n$. 

\exo [Origin=,Level=1,Fight=2,Learn=1,Type=\Exercices,Field=\EspacesVectoriels] mz. 
Soient $E$ un $\ob R$-espace vectoriel et $f\in\sc L(E)$ tel que $f^2-5f-6=0$. \pn
Prouver que $E=\hbox{\rm Ker}(f-6\hbox{\rm Id}_E)\oplus\hbox{\rm Ker}(f+\hbox{\rm Id}_E)$. 

% Duplicate % na. %

% Duplicate % nb. %

\exo [Level=1,Fight=1,Learn=1,Type=\Exercices,Field=\EspacesVectoriels,Origin=] nc. 
Soient $E$ un $\ob R$-espace vectoriel et $p$ un projecteur de $E$. 
Prouver que $u\in\sc L(E)$ commutte avec~$p$~si, et~seulement si, 
$u(\hbox{\rm Im\ }p)\subset\hbox{\rm Im\ }p$ et $u(\hbox{\rm Ker\ }p)\subset\hbox{\rm Ker\ }p$. 

\exo [Origin=,Level=1,Fight=3,Learn=2,Type=\Colles,Field=\EspacesVectoriels] nd. 
Soient $E,F,G$ trois espaces vectoriels et $g\in\sc L(F,G)$. On pose 
$$
\forall f\in\sc L(E,F), \qquad \Phi_g(f):=g\circ f.
$$ 
a) Prouver que l'application $\Phi_g:\sc L(E,F)\to\sc L(E,G)$ est linéaire. \pn
b) Si $g$ est injective, qu'en est-il de $\Phi_g$ ?

\exo [Level=2,Fight=1,Learn=1,Type=\Exercices,Field=\IntégralesMultiples,Origin=] ne. 
Pour $V:=\{(x,y,z)\in\ob R^3:x^2+y^2+z^2\le R^2\}$ et $a>R$,  
calculer le potentiel Newtonnien 
$$
V(a):=\int\!\!\int\!\!\int_V{\d x\d y\d z\F{x^2+y^2+(z-a)^2}}.
$$

\exo [Level=2,Fight=2,Learn=2,Type=\Exercices,Field=\FonctionsDéfiniesParUneIntégrale,Origin=] nf. 
Soit $f:\ob R\to\ob C$ une application continue. Pour $x\in\ob R$, 
on pose 
$$
\ds F(x):={1\F 2}\int_{x-1}^{x+1}f(t)\d t.
$$
a) On suppose que la limite $\ds\lim_{t\to\infty}f(t)$ existe et vaut $\ell\in\ob C$. Démontrer que 
$\ds\lim_{x\to+\infty}F(x)=\ell$. \pn
b) On suppose que $\ds\int_{-\infty}^{\infty}f(t)\d t$ converge. Existence et calcul de 
$\ds\int_{-\infty}^{\infty}F(x)\d x$. 

\exo [Level=2,Fight=2,Learn=2,Type=\Exercices,Field=\FonctionsDéfiniesParUneIntégrale,Origin=] ng. 
{\bf Intégrales Eulériennes. }On pose 
$$
\Gamma(s)=\int_0^{\infty}\e^{-t}t^{s-1}\d t\qquad\hbox{et}\qquad B(x,y):=\int_0^1t^{x-1}(1-t)^{y-1}\d t.
$$
a) Quels sont les ensembles de définition de $\Gamma$ et de $B$. \pn
b) Effectuer le changement de variable $t=u^2$ dans les deux intégrales. \pn
c) Effectuer ensuite le changement de variable $u=\cos \theta$ dans $B$. \pn
Pour $k\in\{1,2,3\}$, $\Delta_1:=[0,a/\sqrt2]\times[0,a/\sqrt2]$, 
$\Delta_2:=\{(x,y)\in\Q[0,+\infty\W[^2:x^2+y^2\le a^2\}$ et $\Delta_3:=[0,a]\times[0,a]$, 
on pose 
$$
\Gamma_a(s)=\int_0^a\e^{-t}t^{s-1}\d t\qquad\hbox{et}\qquad 
I_k=\int\!\!\int_{\Delta_k}\e^{-(x^2+y^2)}x^\alpha y^\beta\d x\d y.
$$  
d) Démontrer que $I_1\le I_2\le I_3$ puis calculer $I_1$, $I_2$, $I_3$. \pn
e) En passant à la limite lorsque $a\to+\infty$, trouver une relation 
entre $\Gamma(x)$, $\Gamma(y)$ et $\Gamma(x+y)$. \pn
f) Calculer $\Gamma(1/2)$ et $\ds\int_{-\infty}^\infty\exp-{x^2\F2\sigma}\d x$ pour $\sigma>0$. 

\exo [Level=2,Fight=1,Learn=1,Type=\Exercices,Field=\Aires,Origin=] nh. 
Aire du domaine plan 
$\ds\Q\{(x,y):{x^2\F a^2}+{y^2\F b^2}\le 1, {x^2\F b^2}+{y^2\F a^2}\le 1\W\}$ pour $b\ge a>0$. 

\exo [Level=2,Fight=1,Learn=1,Type=\Exercices,Field=\IntégralesMultiples,Origin=] ni. 
Centre d'inertie d'un domaine tridimensionnel homogène limité par une sphère et deux plans parallèles ? 

\exo [Level=2,Fight=1,Learn=1,Type=\Exercices,Field=\Volumes,Origin=] nj. 
Volume limité par les surfaces $x^2+y^2=2pz$ et $z^2=x^2+y^2$ ? 

\exo [Level=2,Fight=1,Learn=1,Type=\Exercices,Field=\Aires,Origin=] nk. 
$S$ surface d'équation $xy=az$. Aire de la partie de $S$ dont 
la projection orthogonale sur $Oxy$ est à l'intérieur de la boucle de la courbe d'équation 
polaire $\rho^2=a\cos(2\theta)$ avec $|\theta|\le{\pi\F4}$. 

\exo [Level=2,Fight=1,Learn=1,Type=\Exercices,Field=\IntégralesMultiples,Origin=] nl. 
$\ds f(x,y,z)={z^3\F(x+y+z)(y+z)}$ et $\ds D:=\b\{(x,y,z)\in\Q[0,+\infty\W[^3:x+y+z\le1\b \}$. \pn
a) Démontrer que $f$ est continue ou prolongeable par continuité sur $D$. \pn
b) Calculer $\ds \int\!\!\int\!\!\int_Df(x,y,z)\d x\d y\d z$. 

\exo [Level=2,Fight=1,Learn=1,Type=\Exercices,Field=\IntégralesMultiples,Origin=] nm. 
Soit $(\Gamma)$ courbe du plan $Oxy$ d'équation polaire $\rho^2=a^2\cos(2\theta)$ 
limitée à $x\ge0$ et soit $S$ la surface (homogène) engendrée par la révolution de $(\Gamma)$ autour de $Ox$. \pn
a) Aire de $S$. \pn
b) Centre d'inertie géomètrique de $(S)$. \pn
c) Moment d'inertie de $(S)$ par rapport à $ox$. 

\exo [Level=2,Fight=1,Learn=1,Type=\Exercices,Field=\IntégralesGénéralisées|\IntégralesMultiples,Origin=] nn. 
Convergence et calcul de $\ds \int_0^\infty\Q(\int_x^\infty\e^{-t^2}\d t\W)\d x$. 

\exo [Level=2,Fight=1,Learn=1,Type=\Exercices,Field=\FonctionsDéfiniesParUneIntégrale,Origin=] no. 
On pose $\ds f(x)=\int_1^\infty{t^{-x}\F1+t}\d t$. \pn 
a) Montrer que $f$ est définie, continue et décroissante sur $\Q]0,\infty\W[$. \pn
b) Relation entre $f(x)$ et $f(x+1)$ ? \pn
c) Equivalents de $f$ en $0^+$ et en $+\infty$ ? 

\exo [Level=2,Fight=1,Learn=1,Type=\Exercices,Field=\IntégralesGénéralisées,Origin=] np. 
Existence et calcul de $\ds\int_1^\infty\Q({1\F x}-\arcsin{1\F x}\W)\d x$. 

\exo [Level=2,Fight=1,Learn=1,Type=\Exercices,Field=\IntégralesGénéralisées,Origin=] nq. 
Convergence et calcul de $\ds\int_0^\infty{x-\arctan x\F x(x^2+1)\arctan x}\d x$. 

\exo [Level=2,Fight=1,Learn=1,Type=\Exercices,Field=\FonctionsDéfiniesParUneIntégrale,Origin=] nr. 
Pour $x>0$, on pose $\ds F(x):=\int_0^\infty{\sin t\F t^2+x}\d t$. \pn
a) Existence de $F$ ? \pn
b) Déterminer $\lim_{x\to\infty}F(x)$. \pn
c) Déterminer $\lim_{x\to0^+}F(x)$.  

\exo [Level=2,Fight=1,Learn=1,Type=\Exercices,Field=\Volumes,Origin=] ns. 
Pour $a>0$, calculer le volume de l'ensemble 
$$
D_a:=\Q\{(x,y,z)\in\ob R^3:x^2+y^2\le a^2
\hbox{ et }y^2+z^2\le a^2\W\}. 
$$ 

\exo [Level=2,Fight=1,Learn=1,Type=\Exercices,Field=\Aires,Origin=] nt. 
Soient $a<b$ et $c<d$ des nombres réels strictement positifs. 
Calculer l'aire du domaine 
$$
\Q\{(x,y)\in\ob R^2:ax\le y^2\le bx\hbox{ et }cx\le x\W\}.
$$ 

\exo [Level=2,Fight=0,Learn=0,Type=\Exercices,Field=\IntégralesMultiples,Origin=] n. 
Soit $D$ l'ensemble défini par les inégalités $(x+y)^2\le 2x$ et $y\ge0$. 
Calculer l'intégrale double 
$$
I:=\int\!\!\int_Dxy\d x\d y. 
$$

\exo [Level=2,Fight=1,Learn=1,Type=\Exercices,Field=\IntégralesMultiples,Origin=] nv. 
Calculer l'intégrale $\ds \int\!\!\int_D{x\d x\d y\F\sqrt{x^2-y^2}}$ 
pour $\ds D:=\Q\{(x,y)\in\ob R^2:0\le y\le x\le 1\W\}$.  

\exo [Level=2,Fight=1,Learn=1,Type=\Exercices,Field=\IntégralesMultiples,Origin=] nw. 
Pour $a>0$ et $b>0$, calculer l'intégrale double $\ds I:=\int\!\!\int_D(x^2-y^2)\d x\d y$ 
pour le domaine  
$$
D:=\Q\{(x,y)\in\ob R^2:{x^2\F a^2}+{y^2\F b^2}\le 1\W\}.
$$ 
 
\exo [Level=2,Fight=1,Learn=1,Type=\Exercices,Field=\IntégralesMultiples,Origin=] nx. 
Pour $a>0$, calculer l'intégrale $\ds\int\!\!\int_D{y\d x\d y\F a^2+y^2}$ pour le domaine 
$$
D:=\{(x,y)\in\Q[0,+\infty\W[^2: x^2+y^2\le a^2\}.
$$ 

\exo [Level=2,Fight=0,Learn=1,Type=\Exercices,Field=\IntégralesCurvilignes,Origin=] ny. 
Calculer $\int_\gamma\omega$ pour $\omega:=(x^2+y^2)\d x+2xy\d y$ et $\gamma$ 
la courbe d'équation paramétrique 
$$
\Q\{\eqalign{
x(t)=\cos t\cr
y(t)=\sin(2t)}\W.\qquad(0\le t\le\pi). 
$$

\exo [Level=2,Fight=1,Learn=1,Type=\Exercices,Field=\IntégralesMultiples,Origin=] nz. 
Pour $D:=\ds\Q\{(x,y)\in\ob R^2:2\le x^2+xy+y^2\le 4\W\}$. Calculer $I:=\ds \int\!\!\int_D x^2\d x\d y$. 

\exo [Level=2,Fight=1,Learn=1,Type=\Exercices,Field=\Aires,Origin=] oa. 
Calculer l'aire de la partie du cone $x^2+y^2=k^2z^2$ intérieur à 
la sphère d'équation $x^2+y^2+z^2=2az$. 

\exo [Level=2,Fight=1,Learn=1,Type=\Exercices,Field=\Aires,Origin=] ob. 
Pour $a>b>0$, 
calculer l'aire de la partie de la sphère $S$ d'équation $x^2+y^2+z^2=a^2$ intérieure au cylindre d'équation 
$$
{x^2\F a^2}+{y^2\F b^2}=1.
$$

\exo [Level=2,Fight=1,Learn=1,Type=\Exercices,Field=\Volumes,Origin=] oc. 
Déterminer un plan qui coupe une boule en deux domaines 
dont l'un a un volume double de l'autre. 

\exo [Level=2,Fight=1,Learn=1,Type=\Exercices,Field=\Volumes,Origin=] od. 
Dans $\ob R^3$, on considère deux cylindres de révolution égaux 
dont les axes se rencontrent 
et font un angle $\alpha\in\Q]0,\pi/2\W]$. Calculer le volume de leur intersection. 

\exo [Level=1,Fight=1,Learn=1,Type=\Exercices,Field=\EspacesVectoriels,Origin=] oe. 
Soit $N\in\ob N$. La famille de fonctions $\{x\mapsto \sin(nx)\}_{1\le n\le N}$ 
est elle libre dans $\sc F(\ob R,\ob C)$ ?

\exo [Origin=,Level=1,Fight=2,Learn=1,Type=\Exercices,Field=\EspacesVectoriels] of. 
Soit $n\ge1$. La famille de fonctions $\{x\mapsto \sin^kx\}_{0\le k\le n}$ 
est elle libre dans $\sc F(\ob R,\ob C)$ ?

\exo [Origin=,Level=1,Fight=2,Learn=2,Type=\Exercices,Field=\EspacesVectoriels] og. 
Soient $E$ un espace vectoriel, $f\in\sc L(E)$ et $n\in\ob N^*$ 
tels que $f^n=0$ et $f^{n-1}\neq0$. \pn
a) Prouver qu'il existe $x\in E$ tel que la famille 
$\{x,f(x),f^2(x),\cdots,f^{n-1}(x)\}$ soit libre dans $E$. \pn
b) En déduire que $\{Id_E,f,f^2,\cdots,f^{n-1}\}$ est libre dans $\sc L(E)$.   

\exo [Origin=,Level=1,Fight=2,Learn=2,Type=\Colles,Field=\EspacesVectoriels] oh. Pour $P\in \ob R_3[X]$, on pose 
$$
\Delta(P):=P(x+1)-P(x)\qquad\hbox{ et }\qquad d(P):=P'.
$$
a) prouver que $\Delta$ et $d$ sont des endomorphismes de $\ob R_3[X]$. \pn
b) Injectivité/surjectivité ?\smallskip\noindent
c) Prouver que $\ds d=\Delta-{\Delta^2\F 2}+{\Delta^3\F3}$. \smallskip\noindent
d) Prouver que $\ds\Delta=d+{d^2\F 2}+{d^3\F 6}$. 

\exo [Origin=,Level=1,Fight=1,Learn=1,Type=\Colles,Field=\EspacesVectoriels] oi.  Pour $P\in \ob R_3[X]$, 
on pose 
$$
f(P):=\B((X+1)P(X)-P(X+1)\B)'.
$$ 
L'application $f:P\mapsto f(P)$ est-elle un endomorphisme de $\ob R_3[X]$ ? 
Si oui, est-elle bijective ? 

\exo [Level=2,Fight=1,Learn=1,Type=\Exercices,Field=\IntégralesMultiples,Origin=] oj. 
Calculer $\ds I=\int\!\!\int_D{x\d x\d y\F(x^2+y^2+1)^2}$ à l'aide du théorème 
de Green-Riemann pour 
$$
D:=\{(x,y)\in\ob R^2:x\ge0\hbox{ et }x^2+y^2\le 1\}.
$$ 
 
\exo [Level=2,Fight=1,Learn=1,Type=\Exercices,Field=\IntégralesCurvilignes,Origin=] ok. 
Calculer $\int_\Gamma(z\d x+x\d y+y\d z)$, où $\Gamma$ est un paramètrage 
de la courbe déterminée par 
$$
\Q\{\eqalign{x^2+y^2+z^2=1,\cr x+y=1.\cr}\W.
$$

\exo [Level=2,Fight=1,Learn=1,Type=\Exercices,Field=\Aires,Origin=] ol. 
Calculer l'aire de l'ellipse $\ds{x^2\F a^2}+{y^2\F b^2}\le1$ à l'aide du théorème de Green-Riemann. 

\exo [Level=2,Fight=0,Learn=0,Type=\Exercices,Field=\Séries,Origin=] om. 
Nature de la série $\ds\sum_{n=2}^\infty{(-1)^n\F(-1)^n+\sqrt n}$ ?

\exo [Level=2,Fight=0,Learn=0,Type=\Exercices,Field=\Séries,Origin=] on. 
Nature de la série $\ds\sum_{n=1}^\infty{(-1)^n\F n+(-1)^n\ln n}$ ?

\exo [Level=2,Fight=0,Learn=0,Type=\Exercices,Field=\Séries,Origin=] oo. 
Nature des séries $\ds\sum_{n=1}^\infty\e^{-\sqrt{\ln n}}$ 
et $\ds\sum_{n=1}^\infty(-1)^n\e^{-\sqrt{\ln n}}$ ?

\exo [Level=2,Fight=0,Learn=0,Type=\Exercices,Field=\Séries,Origin=] op. 
Nature de la série $\ds\sum_{n=0}^\infty\sin\Q(\pi\sqrt{n^2+2n+2}\W)$ ?

\exo [Level=2,Fight=0,Learn=0,Type=\Exercices,Field=\Séries,Origin=] oq. 
Nature de la série $\ds\sum_{n=2}^\infty(-1)^n{\sqrt n\sin(1/\sqrt n)\F\sqrt n+(-1)^n}$ ?

\exo [Level=2,Fight=0,Learn=0,Type=\Exercices,Field=\Séries,Origin=]  or. 
Pour $a>0$, nature de la série $\ds\sum_{n=2}^\infty a^{-(\ln n)^2}$ ?

\exo [Level=2,Fight=0,Learn=0,Type=\Exercices,Field=\Séries,Origin=] os. 
Nature de la série $\ds\sum_{n=1}^\infty {(\ln n)^n\F n!}$ ?

\exo [Level=2,Fight=0,Learn=0,Type=\Exercices,Field=\Séries,Origin=] ot. 
Pour $a>0$, nature de la série $\ds\sum_{n=1}^\infty {\ln n!\F n^a}$ ?

\exo [Level=2,Fight=0,Learn=0,Type=\Exercices,Field=\Séries,Origin=] ou. 
Nature de la série $\ds\sum_{n=2}^\infty \Q({n+3\F2n+1}\W)^{n\ln n}$ ?

\exo [Level=2,Fight=0,Learn=0,Type=\Exercices,Field=\Séries,Origin=] ov. 
Pour $\alpha\in\ob R$, nature de la série $\ds\sum_{n=2}^\infty{1\F n(\ln n)^\alpha}$ ? 

\exo [Level=2,Fight=0,Learn=0,Type=\Exercices,Field=\Séries,Origin=] ow. 
Pour $\alpha>0$, nature de la série $\ds\sum_{n=2}^\infty\b(n\sin(1/n)\b)^{n^\alpha}$ ?

\exo [Level=1,Fight=2,Learn=2,Type=\Exercices,Field=\Suites,Origin=] ox. 
Soit $u_0\in\Q]0,\pi/2\W[$ et soit $(u_n)_{n\ge1}$ la suite définie par $u_{n+1}:=\sin u_n$ pour $n\ge0$. \pn 
a) Prouver que $(u_n)_{n\in\ob N}$ est décroissante puis qu'elle tends vers $0$. \pn
b) En déduire la nature des séries $\ds\sum_{n=0}^\infty u_n^3$,\ 
$\ds\sum_{n=0}^\infty\ln\Q({u_{n+1}\F u_n}\W)$ \  et \ $\ds\sum_{n=0}^\infty u_n^2$. \pn
c) Trouver un nombre $\alpha>0$ et une constante $\beta>0$ tels que 
$$
{1\F(u_n)^\alpha}-{1\F(u_{n+1})^\alpha}\sim\beta\qquad(n\to\infty).
$$
d) En sommant la relation précédente, 
trouver un équivalent simple de la suite $u_n$ lorsque $n\to\infty$. 

\exo [Level=2,Fight=2,Learn=2,Type=\Exercices,Field=\SériesNumériques,Origin=] oy. 
a) Montrer que la série de terme général $u_n=\e^{-\sqrt{\ln n}}$ diverge. \pn
b) Prouver que $u_n\sim nu_n-(n-1)u_{n-1}$ et en déduire un équivalent 
de $\sum_{n=1}^ku_n$ lorsque $k\to\infty$. 

\exo [Level=2,Fight=0,Learn=0,Type=\Exercices,Field=\SériesNumériques,Origin=] oz. 
Nature de la série $\ds\sum_{n=1}^\infty {2^nn!\F n^n}$ ?

\exo [Level=2,Fight=0,Learn=0,Type=\Exercices,Field=\SériesNumériques,Origin=] pa. 
Démontrer que $\ds\sum_{k=2}^n{1\F\ln k}$ est équivalent en $+\infty$ à $\ds{n\F\ln n}$. 

\exo [Level=2,Fight=0,Learn=0,Type=\Exercices,Field=\SériesNumériques,Origin=] pb. 
Nature de la série $\ds\sum_{n=1}^\infty2^{-1/n}$ ? 

\exo [Level=2,Fight=0,Learn=0,Type=\Exercices,Field=\SériesNumériques,Origin=] pc. 
Nature de la série $\ds\sum_{n=2}^\infty{1\F(\ln n)^n}$ ?

\exo [Level=2,Fight=0,Learn=0,Type=\Exercices,Field=\SériesNumériques,Origin=] pd. 
Nature de la série $\ds\sum_{n=0}^\infty\e^{-\sqrt n}$ ?

\exo [Level=2,Fight=0,Learn=0,Type=\Exercices,Field=\SériesNumériques,Origin=] pe. 
Nature de la série $\ds\sum_{n=2}^\infty{1\F n}\ln\Q(1+{1\F n^2}\W)$ ?

\exo [Level=2,Fight=0,Learn=0,Type=\Exercices,Field=\SériesNumériques,Origin=] pf. 
Nature de la série $\ds\sum_{n=2}^\infty(\ln n)^{-\ln n}$ ?

\exo [Level=2,Fight=0,Learn=0,Type=\Exercices,Field=\SériesNumériques,Origin=] pg. 
Nature de la série $\ds\sum_{n=2}^\infty{1\F(1+\sqrt n)^n}$ ?

\exo [Level=2,Fight=0,Learn=0,Type=\Exercices,Field=\SériesNumériques,Origin=] ph. 
Nature de la série $\ds\sum_{n=1}^\infty\arccos\Q({n^3+1\F n^3+2}\W)$ ?

\exo [Level=2,Fight=0,Learn=0,Type=\Exercices,Field=\SériesNumériques,Origin=] pi. 
Soit $\alpha\in\Q]0,\pi\W[$. Montrer que la série $\ds\sum_{n=1}^\infty\sin(\alpha n)$ 
diverge grossièrement. 

\exo [Level=2,Fight=2,Learn=1,Type=\Exercices,Field=\SériesNumériques,Origin=] pj. 
Étudier la nature des séries $\ds\sum_{n=1}^\infty u_n$ et $\ds\sum_{n=1}^\infty(-1)^nu_n$ pour
$$
u_n:=\arccos{1\F n}-\arccos{1\F n^2}\qquad(n\in\ob N^*).
$$ 

\exo [Level=2,Fight=0,Learn=0,Type=\Exercices,Field=\SériesNumériques,Origin=] pk. 
Pour $\alpha>0$, montrer que la série $\sum_{n=1}^\infty{(-1)^n\F n^\alpha}$ converge. 

\exo [Level=2,Fight=0,Learn=0,Type=\Exercices,Field=\SériesNumériques,Origin=] pl. 
Calculer si elle existe la somme de la série $\ds\sum_{k=1}^\infty{1\F k(k+1)(k+2)}$. 

\exo [Level=2,Fight=0,Learn=0,Type=\Exercices,Field=\SériesNumériques,Origin=] pm. 
Calculer si elle existe la somme de la série $\ds\sum_{k=2}^\infty{1\F k^2-1}$. 

\exo [Level=2,Fight=0,Learn=0,Type=\Exercices,Field=\SériesNumériques,Origin=] pn. 
Calculer si elle existe la somme de la série $\ds\sum_{k=2}^\infty\ln\Q(1-{1\F k^2}\W)$. 

\exo [Level=2,Fight=0,Learn=0,Type=\Exercices,Field=\SériesNumériques,Origin=] po. 
Pour $\theta\in\Q[0,\pi/2\W[$, 
calculer si elle existe la somme de la série $\ds\sum_{k=0}^\infty\ln\Q(\cos\Q({\theta\F2^k}\W)\W)$. 

\exo [Level=2,Fight=0,Learn=0,Type=\Exercices,Field=\SériesNumériques,Origin=] pp. 
Soit $F$ une fraction rationelle sans pôles dans $\ob N$. 
Nature de $\ds\sum_{n=0}^\infty F(n)$. 

\exo [Level=2,Fight=0,Learn=0,Type=\Exercices,Field=\SériesNumériques,Origin=] pq. 
Soit $P$ un polynôme. Nature de $\ds\sum_{n=0}^\infty P(n)\e^{-n}$ ?

\exo [Level=1,Fight=2,Learn=1,Type=\Exercices,Field=\EspacesVectoriels,Origin=] pr. 
Soit $E$ un espace vectoriel de dimension finie $n$ 
et soit $u\in\sc L(E)$ de rang $r$. Calculer la dimension 
des sous-espaces vectoriels $C:=\{f\in\sc L(E): u\circ f=0\}$, 
$D:=\{f\in\sc L(E):f\circ u=0\}$ et $C\cap D$. 

\exo [Origin=,Level=1,Fight=3,Learn=1,Type=\Colles,Field=\EspacesVectoriels] ps. 
Soient $n\ge1$ et $f\in\sc L(\ob R^n)$ une application linéaire de rang $1$. 
Prouver qu'il existe un vecteur non-nul $\vec v\in\ob R^n$ 
et une forme linéaire $g\in\sc L(\ob R^n,\ob R)$ telle que 
$$
\forall x\in\ob R^n,\quad f(x)= g(x)\vec v.  
$$ 

\exo [Level=2,Fight=0,Learn=0,Type=\Exercices,Field=\IntégralesMultiples,Origin=] pt. 
Déterminer le centre d'inertie du solide homogène délimité par le tore 
(un pneu à section circulaire) de rayons $R>r$ et par deux demi-plans méridiens 
(dont le bord est l'axe de révolution du pneu). 

\exo [Level=2,Fight=0,Learn=0,Type=\Exercices,Field=\Intégrales Multiples,Origin=] pu. 
Calculer l'intégrale $\ds I(S):=\int\!\!\int_S{(x\d y-y\d x)\wedge\d z\F x^2+y^2+z^2}$ 
lorsque  : \pn
a) $S$ est la sphère de centre $0$ et de rayon $1$. \pn
b) $S$ est la sphère d'équation $x^2+y^2+z^2-2z=0$. 

\exo [Level=2,Fight=0,Learn=0,Type=\Exercices,Field=\IntégralesMultiples,Origin=] pv. 
Soit $T$ le tétraèdre défini par les inégalités $x\ge0$, $y\ge0$, $z\ge0$ et $x+y+z\le 1$. 
\pn
a) Calculer $\ds\int\!\!\int_Sx^2\d y\wedge\d z$ en prenant successivement pour $S$ 
chacune des faces du tétraèdre. \pn
b) En déduire le centre d'inertie du tétraèdre. 

\exo [Level=2,Fight=0,Learn=0,Type=\Exercices,Field=\FonctionsDePlusieursVariables|\PotentielsScalaires,Origin=] pw.  
Déterminer une fonction $f(x,y)$ de classe $\sc C^1$ sur $\ob R^2$ 
(un potentiel scalaire) telle que 
$$
\d f=(2x\cos y-y^2\sin x)\d x+(2y\cos x-x^2\sin y)\d y.
$$ 

\exo [Level=2,Fight=0,Learn=0,Type=\Exercices,Field=\IntégralesMultiples,Origin=] px. 
Déterminer le centre d'inertie du huitième 
de boule unité $D:=\{(x,y,z)\in\Q[0,+\infty\W[^3:x^2+y^2+z^2\le1\}$. 

\exo [Level=2,Fight=0,Learn=0,Type=\Exercices,Field=\FonctionsDePlusieursVariables|\PotentielsVecteurs,Origin=] py. 
a) Déterminer un potentiel vecteur $f=(P,Q,R)$ 
tel que 
$$
\vec{\hbox{\rm rot}} f=\b(x^2,y^2,-(2z+a)(x+y)\b)\qquad\Q((x,y,z)\in\ob R^3\W).
$$ 
b) Si $S$ désigne une surface orientée de bord $\partial S$, en déduire que  
$$
\int\!\!\int_S\b(x^2\d y\Lambda\d z+y^2\d z\Lambda\d x-(2z+a)(x+y)\d x\Lambda\d y\b)
=\int_{\partial S}(P\d x+Q\d y+R\d z).
$$ 

\exo [Level=2,Fight=0,Learn=0,Type=\Exercices,Field=\SériesNumériques,Origin=] pz. 
Nature de la série $\ds\sum_{n=0}^\infty{\e^n\F n!}$ ? 

\exo [Level=2,Fight=0,Learn=0,Type=\Exercices,Field=\SériesEntières,Origin=] qa. 
Rayon de convergence de la série entière $\ds\sum_{n=1}^\infty n^{(-1)^n}z^n$ ?

\exo [Level=2,Fight=0,Learn=0,Type=\Exercices,Field=\SériesEntières,Origin=] qb. 
Rayon de convergence de la série entière $\ds\sum_{n=0}^\infty{\sh(n)\F\ch^2(n)}z^n$ ?

\exo [Level=2,Fight=0,Learn=0,Type=\Exercices,Field=\SériesEntières,Origin=] qc. 
Pour $a>0$, rayon de convergence 
de la série entière $\ds\sum_{n=1}^\infty{z^n\F\ln(\ch na)}$ ?

\exo [Level=2,Fight=0,Learn=0,Type=\Exercices,Field=\SériesEntières,Origin=] qd. 
Pour $\alpha\ge0$, rayon de convergence de la série entière 
$\ds\sum_{n=1}^\infty\b(\cos(1/n)\b)^{n^\alpha}z^{n^2}$ ?

\exo [Level=2,Fight=0,Learn=0,Type=\Exercices,Field=\SériesEntières,Origin=] qe. 
Pour $a>0$, rayon de convergence de la série entière 
$\ds\sum_{n=0}^\infty[a^n]z^n$ ?

\exo [Level=2,Fight=2,Learn=1,Type=\Exercices,Field=\SériesEntières,Origin=] qf. 
Soit $(a_n)$ une suite de nombres non nuls telle que 
$\ds\sum_{n=0}^\infty a_nz^n$ soit de rayon de convergence~$R>0$. \vskip-1em\noindent
a) Prouver que 
le rayon de convergence $R'$ de la série entière $\ds\sum_{n=0}^\infty{z^n\F a_n}$ 
vérifie $RR'\le 1$. \pn
b) Pour $\alpha\in\Q]0,1\W[$, donner un exemple de suite $(a_n)_{n\in\ob N}$ 
pour lequel $RR'=\alpha$. 

\exo [Level=2,Fight=0,Learn=0,Type=\Exercices,Field=\SériesEntières,Origin=] qg. 
Rayon de convergence de la série entière $\ds\sum_{n=0}^\infty{n!\F(n+1)^n}z^n$ ?

\exo [Level=2,Fight=0,Learn=0,Type=\Exercices,Field=\SériesEntières,Origin=] qh. 
Pour $a\in\ob R$, rayon de convergence de la série entière 
$\ds\sum_{n=1}^\infty{a^n+2\F n}z^n$ ?

\exo [Level=2,Fight=0,Learn=0,Type=\Exercices,Field=\SériesEntières,Origin=] qi. 
Pour $a>1$, rayon de convergence de la série entière 
$\ds\sum_{n=0}^\infty a^{n(-1)^n}z^n$ ?

\exo [Level=2,Fight=0,Learn=0,Type=\Exercices,Field=\SériesEntières,Origin=] qj. 
Soit $\ds\sum_{n=0}^\infty a_nz^n$ de rayon de convergence $R$. 
Rayon de convergence de $\ds\sum_{n=0}^\infty a_nz^{2n}$ ?

\exo [Level=2,Fight=0,Learn=0,Type=\Exercices,Field=\SériesEntières,Origin=] qk. 
Pour $a>0$, rayon de convergence de $\ds\sum_{n=0}^\infty a^{n^2}z^n$ ?

\exo [Level=2,Fight=0,Learn=0,Type=\Exercices,Field=\SériesEntières,Origin=] ql. 
Rayon de convergence de la série entière $\ds\sum_{n=0}^\infty\Q({2n\atop n}\W)z^n$ ?

\exo [Level=2,Fight=0,Learn=0,Type=\Exercices,Field=\SériesEntières,Origin=] qm. 
Rayon de convergence de la série $\ds\sum_{n=1}^\infty {n^{2n}\F(2n)!}z^n$ ?

\exo [Level=2,Fight=0,Learn=0,Type=\Exercices,Field=\SériesEntières,Origin=] qn. 
Rayon de convergence et somme de la série entière $\sum_{n=0}^\infty n^2x^n$ ?

\exo [Level=2,Fight=0,Learn=0,Type=\Exercices,Field=\SériesEntières,Origin=] qo. 
On pose $T(x):=\sum_{n=0}^\infty(-1)^nx^{2n}$. \pn
a) Domaine de définition et calcul de $T(x)$. \pn
b) En déduire l'égalité 
$$
\arctan x=\sum_{n=0}^\infty(-1)^n{x^{2n+1}\F2n+1}\qquad(-1<x<1). 
$$
c) Trouver une série absolument ayant pour somme $\pi$. \pn
d) Donner une estimation rationnelle de $\pi$ à $10^{-6}$ près. 

\exo [Level=2,Fight=0,Learn=0,Type=\Exercices,Field=\SériesEntières,Origin=] qp. 
Pour $a>0$, développer la fonction $f(x)=\log(a+x)$ 
en série entière au voisinage de $0$.  

\exo [Level=2,Fight=0,Learn=0,Type=\Exercices,Field=\SériesEntières,Origin=] qq. 
Calculer la somme de la série entière $\ds\sum_{n=0}^\infty{x^n\F n+2}$. 

\exo [Level=2,Fight=0,Learn=0,Type=\Exercices,Field=\SériesEntières,Origin=] qr. 
Calculer la somme de la série entière $\ds\sum_{n=0}^\infty(n+3)x^n$. 

\exo [Level=2,Fight=0,Learn=0,Type=\Exercices,Field=\SériesEntières,Origin=] qs. 
Calculer la somme de la série entière $\ds\sum_{n=1}^\infty{x^n\F n(n+1)}$. 

\exo [Level=2,Fight=0,Learn=0,Type=\Exercices,Field=\SériesEntières,Origin=] qt. 
Calculer la somme de la série entière $\ds \sum_{n=0}^\infty{3+(-1)^n\F2^{n+1}}x^n$. 

\exo [Level=2,Fight=0,Learn=0,Type=\Exercices,Field=\SériesEntières,Origin=] qu. 
Calculer la somme de la série entière $\ds\sum_{n=0}^\infty{x^{4n}\F4n+1}$. 

\exo [Level=2,Fight=2,Learn=1,Type=\Exercices,Field=\SériesEntières,Origin=] qv. 
Soit $\sum_{n=0}^\infty a_nz^n$ une série entière de rayon de convergence $1$. 
Que peut-on dire du rayon de convergence $R$ des séries entières suivantes : \pn
a) $\ds\sum_{n=0}^\infty2^na_nz^n$\qquad\qquad b) $\ds\sum_{n=0}^\infty{a_n\F n!}z^n$\qquad\qquad c) $\ds\sum_{n=0}^\infty n!a_nz^n$\pn
d) $\ds\sum_{n=0}^\infty{a_n\F n^2+1}z^n$\qquad\qquad e) $\ds\sum_{n=1}^\infty n^2a_nz^n$\qquad\qquad 
f) $\ds\sum_{n=2}^\infty n^{\ln n}a_nz^n$\pn
g) $\ds\sum_{n=0}^\infty(1+i^n)a_nz^n$

\exo [Level=2,Fight=0,Learn=0,Type=\Exercices,Field=\SériesEntières,Origin=] qw. 
On pose $\ds F(x)=\int_0^1{1-t\F 1-xt^2}\d t$ et $\ds S(x)=\sum_{n=0}^\infty{x^n\F(2n+1)(2n+2)}$. \pn
a) Montrer que $S$ est définie et continue sur $[-1,1]$. \pn
b) Quel est le domaine de définition de $F$ ?\pn
c) Montrer que $F(x)=S(x)$ pour $x\in\Q]-1,1\W[$. \pn
d) Montrer que $F$ est continue en $-1$. \pn
e) En déduire la valeur de la série numérique $\ds\sum_{n=0}^\infty{(-1)^n\F(2n+1)(2n+2)}$. 

\exo [Level=2,Fight=2,Learn=2,Field=\FonctionsDéfiniesParUneIntégrale|\SériesEntières,Type=\Exercices,Origin=] qx. 
On pose $\ds J_0(x)={1\F2\pi}\int_{-\pi}^\pi\e^{ix\sin t}\d t$ (fonction $J_0$ de Bessel). \pn
a) Développer $J_0(x)$ en série entière de $x$ à l'aide de l'intégrale de Wallis
$$
{1\F2\pi}\int_{-\pi}^\pi\sin^{2k}t\d t={(2k)!\F4^k(k!)^2}\qquad(k\in\ob N).
$$
b) Quel est le rayon de convergence de la série entière obtenue ?\pn
c) En dérivant terme à terme, calculer $\ds J_0''(x)+J_0(x)+{1\F x}J_0'(x)$ et en déduire 
que $J_0$ vérifie une équation différentielle du second ordre. 

\exo [Level=2,Fight=0,Learn=0,Type=\Exercices,Field=\SériesEntières,Origin=] qy. 
On pose $\ds S(x)=\sum_{n=0}^\infty{x^{3n+2}\F 3n+2}$. \pn
a) Pour $-1<x<1$, montrer que 
$$
S(x)={1\F6}\ln{1+x+x^2\F(1-x)^2}-{1\F\sqrt3}\arctan{x\sqrt3\F2+x}.
$$
b) En déduire la valeur de la série numérique $\ds\sum_{n=0}^\infty{(-1)^n\F3n+2}$. 

\exo [Level=2,Fight=0,Learn=0,Type=\Exercices,Field=\SériesEntières,Origin=] qz. 
En développant $\ds\int_0^x{1+t^2\F1+t^4}\d t$ en série entière pour $-1<x<1$, 
établir l'égalité suivante : 
$$
\int_0^1{1+t^2\F1+t^4}\d t=4\sum_{k=0}^\infty(-1)^k{2k+1\F(4k+1)(4k+3)}. 
$$

\exo [Level=2,Fight=0,Learn=0,Type=\Exercices,Field=\SériesEntières,Origin=] ra. 
Calculer la somme de la série $\ds S(x)=\sum_{n=1}^\infty(-1)^n{x^{4n-1}\F4n}$. 

\exo [Level=2,Fight=0,Learn=0,Type=\Exercices,Field=\SériesEntières,Origin=] rb. 
Déterminer le rayon de convergence et la somme 
de la série entière $\ds\sum_{n=0}^\infty n^2{x^n\F n!}$. 

\exo [Level=2,Fight=0,Learn=0,Type=\Exercices,Field=\SériesEntières,Origin=] rc. 
Déterminer le rayon de convergence et la somme 
de la série entière $\ds\sum_{n=0}^\infty(n^2+n+1)x^n$. 

\exo [Level=2,Fight=0,Learn=0,Type=\Exercices,Field=\SériesEntières,Origin=] rd. 
Déterminer le rayon de convergence et 
la somme de $\sum_{n=0}^\infty\b(2+(-1)^n\b)^nx^n$. 

\exo [Level=2,Fight=0,Learn=0,Type=\Exercices,Field=\SériesEntières,Origin=] re. 
Déterminer le rayon de convergence 
et la somme de $\ds\sum_{n=2}^\infty{x^n\F n^2-1}$. 

\exo [Level=2,Fight=0,Learn=0,Type=\Exercices,Field=\SériesEntières,Origin=] rf. 
Pour $\alpha>0$, Déterminer le rayon de convergence et la somme de 
$\sum_{n=1}^\infty{\sin n\alpha\F n}x^n$. 

\exo [Level=2,Fight=0,Learn=0,Type=\Exercices,Field=\SériesEntières,Origin=] rg. 
On pose $f(t)=0$ si $t\le 0$ et $f(t)=\e^{-1/t}$ si $t>0$. 
Montrer que $f$ n'est pas développable en série entière au voisinage de $0$. 

\exo [Level=2,Fight=0,Learn=0,Type=\Exercices,Field=\SériesEntières,Origin=] rh. 
Développer la fonction $f(x)=\ln(x^2-5x+6)$ en série entière au voisinage de $0$. 

\exo [Level=2,Fight=0,Learn=0,Type=\Exercices,Field=\SériesEntières,Origin=] ri. 
On pose $f(0):=-{1\F6}$ et $\ds f(x):={1\F x}-{1\F\sin x}$ si $x\in\Q]-\pi,\pi\W[\ssm\{0\}$. 
Prouver que la fonction $f$ est de classe $\sc C^\infty$ sur l'intervalle $\Q]-\pi,\pi\W[$. 

\exo [Level=2,Fight=0,Learn=0,Type=\Exercices,Field=\SériesEntières,Origin=] rj. 
Déterminer les solutions $f$ développables en séries entières sur $\ob R$ de 
$$
4xf''(x)+2f'(x)-f(x)=0\qquad(x\in\ob R).
$$

\exo [Level=2,Fight=0,Learn=0,Type=\Exercices,Field=\SériesEntières,Origin=] rk. 
Etude de la série $S(x)=\sum_{n=0}^\infty2^{\sqrt n}x^n$. \pn
a) Rayon de convergence de $S$. \pn
b) Donner une valeur de $N$ pour que 
$$
S_N(x):=\sum_{0\le n\le N}2^{\sqrt n}x^n
$$ 
soit une valeur approchée de $S(x)$ à $10^{-8}$ près 
pour tout $|x|\le1/3$. \pn
c) Graphe de $x\mapsto S_N(x)$ pour $x\in[-1/3,1/3]$. 

\exo [Level=2,Fight=0,Learn=0,Type=\Maple,Field=\SériesEntières,Origin=] rl. 
Condition sur $a$, $b$, $c$ et $d$ pour que la série de terme général 
$$
u_n=\sin\Q({a\F\sqrt{n+b}}\W)-\tan\Q({c\F\sqrt{n+d}}\W)
$$
soit convergente. 

\exo [Level=2,Fight=0,Learn=0,Type=\Maple,Field=\SériesEntières,Origin=] rm. 
Trouver tous les polynômes $P$ tels que $\ds\lim_{n\to+\infty}P(n)=+\infty$ et 
tels que 
$$
u_n=(n^6+3n^4)^{1/6}-P(n)^{1/3}
$$
soit le terme général d'une série convergente. 

\exo [Level=1,Fight=0,Learn=0,Type=\Maple,Field=\Suites,Origin=] rn. 
Pour chaque entier $n\ge1$, on pose 
$$
u_n:=2\sqrt n-\sum_{k=1}^n{1\F\sqrt k}\quad\hbox{ et }\quad v_n:=2\sqrt{n+1}-\sum_{k=1}^n{1\F\sqrt k}. 
$$
a) Montrer que les deux suites sont adjacentes. \pn
b) Tracer sur un graphique les deux suites pour $1\le n\le 100$. \pn
c) Calculer $\lim_{n\to\infty} u_n$. 

\exo [Level=2,Fight=0,Learn=0,Type=\Exercices,Field=\IntégralesMultiples,Origin=] ro. 
Soient $a>0$, $O$ le point $(0,0,0)$ et $M_0$ le point $(0,0,a)$ de $\ob R^3$. 
Pour $(x,y,z)\in\ob R^3$, on note $M$ le point $(x,y,z)$ et on pose 
$$
F(x,y,z)={OM^2-OM_0^2\F M_0M^3}. 
$$
Calculer l'intégrale $\int\!\!\int\!\!\int_Bf(x,y,z)\d x\d y\d z$ où $B$ 
désigne la Boule de centre $O$ et de rayon $a$. 

\exo [Level=2,Fight=0,Learn=0,Type=\Exercices,Field=\SériesEntières,Origin=] rp. 
Déterminer le rayon de convergence et étudier 
la convergence en $x=-R$ et $x=R$ de la série entière 
$$
\sum_{n=0}^\infty\ln\Q({(-1)^n+\sqrt n\F \sqrt{n+1}}\W)x^n.
$$
 
\exo [Level=2,Fight=2,Learn=1,Type=\Exercices,Field=\FonctionsDéfiniesParUneIntégrale,Origin=] rq. 
En dérivant(sous l'intégrale), établir l'identité suivante  
$$
\int_0^{\pi/2}\e^{-x\cos t}\cos(x\sin t)\d t={\pi\F 2}-\int_0^x{\sin t\F t}\d t\qquad(w\in\ob R).
$$
En déduire la valeur de $\int_0^\infty{\sin t\F t}\d t$. 

\exo [Level=2,Fight=2,Learn=2,Field=\FonctionsDéfiniesParUneIntégrale,Type=\Exercices,Origin=] rr. 
On pose $\ds f(x)=\int_0^\infty{\e^{-t}\F\sqrt t}\cos(xt)\d t$ et $\ds g(x)=\int_0^\infty{\e^{-t}\F\sqrt t}\sin(xt)\d t$. \pn
a) Dériver $h=f+ig$ sur $\ob R$\pn
b) En déduire une équation différentielle vérifiée par $h=f+ig$. \pn
c) En déduire $f$ et $g$. 

\exo [Level=1,Fight=0,Learn=0,Type=\Exercices,Field=\Groupes,Origin=] rs. 
Soit $G$ un sous groupe de $(\ob R,+)$. \pn
a) Montrer que l'une des deux propriétés suivantes est satisfaite : 
$$
\eqalignno{
&\exists\delta\ge0,\quad G=\{n\delta:n\in\ob Z\}=\delta\ob Z\quad\hbox{($G$ est discret dans $\ob R$)}&(i)\cr
&\forall (x,\epsilon)\in\ob R\times\Q]0,\infty\W[, \quad G\cap \Q]x-\epsilon,x+\epsilon\W[\neq\emptyset\quad 
\hbox{($G$ est dense dans $\ob R$)}&(ii)
\cr
}
$$ 
b) En déduire que $\{a+b\sqrt2:(a,b)\in\ob Z^2\}$ est dense dans $\ob R$. \pn
c) Que peut on en déduire pour les sous groupes $H$ de $\b(\Q]0,\infty\W[,\times\b)$ ? 

\exo [Level=1,Fight=0,Learn=0,Type=\Exercices,Field=\EspacesVectoriels,Origin=] rt. 
On note $E$ l'ensemble des fonctions $f:\ob N\to\ob R$ vérifiant 
$$
8f(n+3)-12f(n+2)+6f(n+1)-f(n)=0\qquad(n\in\ob N)
$$
et on note $\Phi[f](n)=f(n+1)$ et $\Psi[f](n)=2nf(n+1)-nf(n)$ pour $f\in E$ et $n\in\ob N$.\pn
a) Prouver que $E$ est un $\ob R$-espace vectoriel de dimension finie 
et calculer sa dimension. \pn
b) Prouver que $\Phi:f\mapsto\Phi[f]$ et $\Psi:f\mapsto\Psi[f]$ sont des endomorphismes de $E$. \pn
c) $\Phi$ et $\Psi$ sont elles bijectives ? 

\exo [Level=1,Fight=1,Learn=0,Type=\Exercices,Field=\Fonctions,Origin=] ru. 
Soient $f,g:\ob R\to\ob R$ deux fonctions bornées et $\lambda\in\ob R$. 
Prouver que 
$$
\sup_{x\in\ob R}|f(x)+g(x)|\le \sup_{x\in\ob R}|f(x)|+\sup_{x\in\ob R}|g(x)|
$$
et que 
$$
\sup_{x\in\ob R}|\lambda f(x)|=|\lambda|\sup_{x\in\ob R}|f(x)|. 
$$


\exo [Level=2,Fight=0,Learn=0,Type=\Exercices,Field=\SériesEntières,Origin=] rv. 
Déterminer le rayon de convergence 
et la somme de $\ds\sum_{n=0}(n^2+n+1){x^n\F n!}$. 

\exo [Level=2,Fight=0,Learn=0,Type=\Exercices,Field=\SériesEntières,Origin=] rw. 
Pour $\alpha>0$ et $x\in\Q]-1,1\W[$, calculer la somme $\ds\sum_{n=0}^\infty\cos( n\alpha)x^n$. 

\exo [Level=2,Fight=0,Learn=0,Type=\Exercices,Field=\SériesEntières,Origin=] rx. 
Pour $a\in\ob R$, développer la fonction $x\mapsto\ch(x+a)$ en série entière 
au voisinage de $0$. Quel est le rayon de convergence de la série entière obtenu ?
 
\exo [Level=1,Fight=2,Learn=2,Type=\Exercices,Field=\SériesEntières,Origin=] ry. 
Pour chaque entier $n\in\ob N$, on pose $\ds I_n:=\int_0^\pi\cos(t)^{2n}\d t$. \pn
a) Trouver une relation entre $I_n$ et $I_{n-1}$ en intégrant par partie. \pn
b) En procèdant par récurence sur $n$, établir que 
$$
I_n={(2n)!\F(2^nn!)^2}\pi\qquad(n\in\ob N)
$$
c) En déduire un développement en série entière au voisinage de $0$ 
de $\ds x\mapsto\int_0^\pi\cos(x\cos t)\d t$. 

\exo [Level=2,Fight=1,Learn=0,Type=\Exercices,Field=\SériesEntières,Origin=] rz. 
Calculer un développement en série entière de $f(x)=\int_0^x\cos(t^2)\d t$ 
au voisinage de $0$. 

\exo [Level=2,Fight=2,Learn=2,Field=\FonctionsDéfiniesParUneIntégrale|\SériesEntières,Type=\Exercices,Origin=] sa. 
Pour $x\in\ob R$, on pose $\ds f(x)=\int_0^x\e^{t^2-x^2}\d t$. \pn
a) Prouver que $f$ satisfait l'équation différentielle 
$$
f'(x)+2xf(x)=1\qquad(x\in\ob R). \leqno{(E)}
$$
b) On suppose qu'il existe une suite $(a_n)_{n\in\ob N}$ telle que 
$$
f(x)=\sum_{n=0}^\infty a_nx^n\qquad(x\in\ob R).
$$
b) déterminer la suite $(a_n)_{n\in\ob N}$. \pn
c) Réciproquement, démontrer que la série entière associée à la suite du b) 
est de rayon de convergence infini et satisfait $(E)$. \pn
d) En déduire un développement en série entière de $f$. 
 
\exo [Level=2,Fight=0,Learn=0,Type=\Exercices,Field=\SériesEntières,Origin=] sb. 
Pour $\alpha\in\ob R$, développement en série entière 
au voisinage de $0$ de $\ds x\mapsto{1\F x^2-2x\cos\alpha+1}$. 

\exo [Level=2,Fight=0,Learn=0,Type=\Exercices,Field=\SériesEntières,Origin=] sc. 
Développer en série entière au voisinage de $0$ 
la fonction $x\mapsto\cos^4x$. 

\exo [Level=2,Fight=0,Learn=0,Type=\Exercices,Field=\SériesEntières,Origin=] sd. 
On pose $\ds S(x)=\sum_{n=0}^\infty{(-1)^nx^{2n+1}\F(2n+1)(2n-1)}$. \pn
a) Rayon de convergence et domaine de définition de $S$ ?\pn
b) Calculer la somme sur $\Q]-1,1\W[$ et en déduire $\ds \sum_{n=0}^\infty{(-1)^n\F 4n^2-1}$. 

\exo [Level=2,Fight=0,Learn=0,Type=\Exercices,Field=\SériesEntières,Origin=] se. 
a) Développer la fonction $x\mapsto\arctan x$ en série entière au voisinage de $0$. 
\pn
b) En déduire que $\ds {\pi\F 4}=\sum_{n=0}^\infty{(-1)^n\F 2n+1}$. \pn
c) Déterminer la nature de la série 
$$
\sum_{n=10}^\infty\ln\Q(\tan\sum_{k=0}^n{(-1)^k\F2k+1}\W).
$$

\exo [Level=2,Fight=0,Learn=0,Type=\Exercices,Field=\SériesEntières,Origin=] sf. 
Pour $\alpha\in\ob R$, on pose 
$$
\ds \Q({\alpha\atop n}\W):={1\F n!}
\prod_{k=0}^{n-1}(\alpha-k)\qquad(n\in\ob N).
$$
a) Rayon de convergence de la série entière 
$\ds S_\alpha(x):=\sum_{n=0}^\infty\Q({\alpha\atop n}\W)x^n$ ? \pn
b) Prouver que $S_\alpha$ satisfait l'équation différentielle  
$$
(1+x)S_\alpha'(x)=\alpha S_\alpha(x)\qquad(-R<x<R). 
$$
c) En déduire $S_\alpha$ pour $\alpha\in\ob R$. \pn
d) Que se passe-t-il lorsque $\alpha\in\ob N$ ?

\exo [Level=2,Fight=0,Learn=0,Type=\Exercices,Field=\SériesEntières,Origin=] sg. 
Soit  $(a_n)_{n\in\ob N}$ une suite 
de nombres positifs tels que $\ds\lim_{n\to\infty}{a_{2n+1}\F a_{2n}}=\ell_1$ 
et $\ds\lim_{n\to\infty}{a_{2n+2}\F a_{2n+1}}=\ell_2$. \pn 
Lorsque $(\ell_1,\ell_2)\in\Q]0,\infty\W[^2$, quel est 
la rayon de convergence de $\ds\sum_{n=0}^\infty a_nx^n$ ?

\exo [Level=2,Fight=0,Learn=0,Type=\Exercices,Field=\SériesEntières,Origin=] sh. 
Calculer le rayon de convergence de la série entière 
$$
\sum_{n=0}^\infty{nx^{2n+1}\F 1.3.5...(2n+1)}
$$

\exo [Level=2,Fight=0,Learn=0,Type=\Exercices,Field=\SériesEntières,Origin=] si. 
On pose $\ds S(x)=\sum_{n=0}^\infty{x^{3n}\F(3n)!}$. \pn
a) Calculer le rayon de convergence $R$ de $S$ et montrer que 
$$
S'''(x)=S(x)\qquad(-R<x<R)\leqno(E')
$$
b) Déterminer $S$. (les solutions de $(E')$ 
sont de la forme $x\mapsto\lambda\e^x+\mu\e^{jx}+\nu\e^{j^2x}$ où $j=\e^{2i\pi/3}$). 

\exo [Level=2,Fight=0,Learn=0,Type=\Exercices,Field=\SériesEntières,Origin=] sj. 
Déterminer le rayon de convergence de la série entière 
$\ds \sum_{n=0}^\infty\Q(\cos(\pi/\sqrt n)\W)^{n^3}x^{n^2}$ ?

\exo [Level=2,Fight=0,Learn=0,Type=\Exercices,Field=\Suites,Origin=] sk. 
Pour $n\ge3$, on pose $\ds S_n:=\sum_{k=3}^\infty{1\F k\ln k}$. \pn
a) Prouver que $S_n\sim\ln(\ln n)$ lorsque $n\to\infty$. \pn
b) On pose $S_n=\ln(\ln n)+r_n$ pour $n\ge3$. Prouver que la suite $r_n$ est bornée. \pn
c) Prouver que la suite $r_n$ converge vers une limite $\ell$. 

\exo [Level=2,Fight=0,Learn=0,Type=\Exercices,Field=\SériesNumériques,Origin=] sl. 
Nature de la série $\ds\sum_{n=1}^\infty\sin\Q(\sum_{k=1}^n{\pi\F 2^k}\W)$ ? 

\exo [Level=2,Fight=0,Learn=0,Type=\Exercices,Field=\SériesNumériques,Origin=] sm. 
Les séries $\ds\sum_{n=0}^\infty|a_n|^2$ et $\ds\sum_{n=0}^\infty|b_n|^2$ étant convergente, 
prouver que la série $\ds\sum_{n=0}^\infty a_nb_n$ converge. 

\exo [Level=2,Fight=0,Learn=0,Type=\Exercices,Field=\SériesNumériques,Origin=] sn. 
Soit $(u_n)_{n\in\ob N^*}$ une suite de limite $\ell$ et soient $(v_n)_{n\in\ob N^*}$ 
et $(w_n)_{n\in\ob N^*}$  
les suites définies par 
$$
\forall n\in\ob N, \quad v_n={1\F n}\sum_{k=1}^n u_k\qquad\hbox{et}\qquad\forall n\in\ob N^*, \quad w_n={1\F n^2}\sum_{k=1}^nku_k.
$$
Étudier la convergence de la suite $(v_n)$ puis de la suite $(w_n)$. 

\exo [Level=1,Fight=0,Learn=0,Type=\Exercices,Field=\Suites,Origin=] so. 
Étudier la convergence de la suite $(u_n)_{n\in\ob N}$ définie par 
$\ds u_n={\sqrt{1+4n^2}\F n+3}$ pour $n\ge0$. 

\exo [Level=1,Fight=0,Learn=0,Type=\Exercices,Field=\Suites,Origin=] sp. 
Soit $\alpha\in\Q]0,\pi\W[$. Prouver la divergence de la suite $(u_n)_{n\in\ob N}$ 
définie par $u_n=\sin(n\alpha)$ pour $n\ge0$. 

\exo [Level=1,Fight=0,Learn=0,Type=\Exercices,Field=\Suites,Origin=] sq. 
Pour $n\in\ob N$, on pose $\ds u_n=\sum_{k=1}^n{1\F k!}$ et $\ds v_n=u_n+{1\F n.n!}$. \pn
a) Prouver que $u_n$ et $v_n$ sont adjacentes. \pn 
b) Notant $\e$ leur limite commune, trouver une approximation rationnelle de $\e$ à $10^{-3}$ près. 

\exo [Level=2,Fight=0,Learn=0,Type=\Exercices,Field=\SériesEntières,Origin=] sr. 
Prouver que l'application $\ds x\mapsto {x\F \e^x-1}$ est prolongeable en une fonction de classe $\sc C^\infty$ sur $\ob R$. 

\exo [Level=2,Fight=0,Learn=0,Type=\Exercices,Field=\NombresComplexes,Origin=] ss. 
Pour chaque nombre complexe $z$, prouver que $|\e^z-1|\le \e^{|z|}-1\le |z|\e^{|z|}$. 

\exo [Level=2,Fight=0,Learn=0,Type=\Exercices,Field=\SériesEntières,Origin=] st. 
Pour $\alpha\ge0$, rayon de convergence de la série entière 
$\ds\sum_{n=1}^\infty \b(\arctan n\b)^{n^\alpha}z^{n^2}$ ?

\exo [Level=2,Fight=0,Learn=0,Type=\Exercices,Field=\SériesDeFourier,Origin=] su. 
A l'aide de la fonction $2\pi$-périodique $f$ 
définie par $f(x)=x^2$ pour $|x|\le \pi$, calculer $\ds\sum_{n=1}^\infty{1\F n^4}$. 

\exo [Level=2,Fight=0,Learn=0,Type=\Exercices,Field=\SériesDeFourier,Origin=] sv. 
Soit $f$ et $g$ les fonction $2\pi$ périodiques définies par 
$$
f(x)=\Q\{\eqalign{
-1\qquad(-\pi<x\le0)
\cr
\quad1\qquad\hfill(0<x\le\pi)
}\W.
\qquad\hbox{et}\qquad
g(x)=\Q\{\eqalign{
\pi+x\qquad(-\pi<x\le0)
\cr
\pi-x\qquad\hfill(0<x\le\pi)
}\W.
$$
a) Faire un dessin. \pn
b) Calculer les coefficients de Fourier des fonctions $f$ et $g$. \pn
c) En déduire les sommes des séries $S_1=\ds\sum_{n=0}^\infty{(-1)^n\F(2n+1)}$, 
$S_2=\ds\sum_{n=0}^\infty{1\F(2n+1)^2}$ et $\ds S_3=\sum_{n=0}^\infty{1\F(2n+1)^4}$. \pn
d) En déduire les sommes des séries 
$\ds S_4=\sum_{n=1}^\infty{1\F n^2}$, $\ds S_5=\sum_{n=1}^\infty{(-1)^n\F n^2}$ 
et $\ds S_6=\ds\sum_{n=1}^\infty{1\F n^4}$ en bidouillant. 

\exo [Level=2,Fight=0,Learn=0,Type=\Exercices,Field=\SériesDeFourier,Origin=] sw. 
En utilisant la fonction $2\pi$-périodique $f$ définie 
par $f(x)=x^2(2\pi-x)^2$ pour $0\le x\le 2\pi$, 
calculer les sommes des séries $\ds\sum_{n=1}^\infty\ds{1\F n^8}$, 
$\ds\sum_{n=1}^\infty{1\F n^4}$ et $\ds\sum_{n=1}^\infty{(-1)^{n+1}\F n^4}$. 

\exo [Level=2,Fight=0,Learn=0,Type=\Exercices,Field=\SériesDeFourier,Origin=] sx. 
A l'aide de la fonction $x\mapsto\ch(\alpha x)$, calculer les sommes suivantes : 
$$
\sum_{n=1}^\infty\ds{1\F n^2+\alpha^2}\qquad  
\sum_{n=1}^\infty{(-1)^n\F n^2+\alpha^2}\quad\hbox{ et }\quad\sum_{n=1}^\infty{1\F(n^2+\alpha^2)^2}. 
$$

\exo [Level=2,Fight=0,Learn=0,Type=\Exercices,Field=\SériesDeFourier,Origin=] sy. 
On pose $f(t)=|\cos t|$ pour $t\in\ob R$. \pn
a) Calculer les coefficients de Fourier de $f$. \pn 
b) En déduire la somme des séries suivantes : 
$$
\sum_{n=1}^\infty{1\F4n^2-1}\qquad\sum_{n=1}^\infty{(-1)^n\F4n^2-1}\quad\hbox{et}\quad
\sum_{n=1}^\infty{1\F(4n^2-1)^2}. 
$$

\exo [Level=2,Fight=0,Learn=0,Type=\Exercices,Field=\SériesDeFourier,Origin=] sz. 
Soit $f:\ob R\to\ob R$ une fonction de classe $\sc C^1$ et de période $T$ 
sur $\ob R$. Prouver que 
$$
\int_0^T\b|f(t)\b|^2\d t\le {T^2\F4\pi}\int_0^T\b[f'(t)\b|^2\d t+{1\F T}\Q|\int_0^Tf(t)\d t\W|^2. 
$$
Question subsidiaire : étudier les cas d'égalité. 

\exo [Level=2,Fight=0,Learn=0,Type=\Exercices,Field=\SériesDeFourier,Origin=] ta. 
Soit Montrer que les seules fonctions $f:\ob R\to\ob R$ de classe $\sc C^\infty$ 
et $2\pi$-périodique. On suppose qu'il existe deux constantes $\lambda>1$ et $M>1$ telles que 
$$
\Q|f^{(n)}(x)\W|\le M\lambda^n\qquad(n\in\ob N, x\in\ob R). 
$$
Montrer que $f$ est un polynôme trigonométrique 
(i.e. une somme finie de fonctions du type $x\mapsto\sin(nx)$ et $x\mapsto\cos(nx)$). Réciproque ?

\exo [Level=2,Fight=0,Learn=0,Type=\Exercices,Field=\SériesDeFourier,Origin=] tb. 
Soit $\alpha\in\ob R\ssm\ob Z$ et soit $f$ la fonction $2\pi$-périodique 
définie par $f(t)=\cos(\alpha t)$ pour $|t|\le\pi$. 
Étudier la convergence de la sériee de Fourier de $f$ et en déduire que 
$$
\forall x\in\ob R\ssm\ob Z, \quad \sum_{n=1}^\infty{2x\F\pi^2n^2-x^2}={1\F x}-{1\F\tan x} \qquad\hbox{ et }
\qquad \sum_{n=1}^\infty{(-1)^{n+1}\F\pi^2n^2-x^2}={1\F\sin x}-{1\F x}. 
$$

\exo [Level=2,Fight=0,Learn=0,Type=\Exercices,Field=\SériesDeFourier,Origin=] tc. 
Soit $r\in\Q]-1,1\W[$ et soient $f_r$ et $g_r$ les fonctions définies par 
$$
f_r(t)={1-r\cos t\F 1-2r\cos t+r^2}\quad\hbox{et}\quad g_r(t)={r\sin t\F 1-2r\cos t+r^2}\qquad(t\in\ob R)
$$
a) Sans faire de calculs d'intégrales, montrer que 
$$
f_r(t)=\sum_{n=0}^\infty r^n\cos(nt)\qquad\hbox{et}\qquad g_r(t)=\sum_{n=0}^\infty r^n\sin(nt)\qquad(t\in\ob R). 
$$
b) En déduire les coefficients de Fourier de $f_r$ et $g_r$. \pn
c) On pose $G_r(t):={1\F2}\ln(1-2r\cos t+r^2)$ pour $t\in\ob R$. Calculer $\int_0^{2\pi}G_r(t)\cos(nt)\d t$ pour $n\ge 0$. 


\exo [Level=2,Fight=1,Learn=0,Type=\Exercices,Field=\SériesDeFourier,Origin=] td. 
Pour deux applications $f,g:\ob R\to\ob R$ continues et $2\pi$-périodiques, on pose 
$$
f\star g(x)={1\F2\pi}\int_0^{2\pi}f(x-t)g(t)\d t\qquad(x\in\ob R)
$$
a) Montrer que la fonction $f\star g$ est définie, 
continue et $2\pi$-périodique sur $\ob R$.  \pn
b) Calculer les coefficients de Fourier de $f\star g$ en fonctions de ceux de $f$ et de $g$. \pn


\exo [Level=2,Fight=1,Learn=2,Type=\Exercices,Field=\SériesDeFourier,Origin=] te. 
Soit $f:\ob R\to\ob R$ une application $2\pi$-périodique, 
de classe $\sc C^1$ sur $\ob R$. \pn
a) Trouver une relation entre les coefficients de Fourier de $f$ et de $f'$. \pn
b) Lorsque $\int_0^{2\pi}f(t)\d t=0$, prouver que 
$$
\int_0^{2\pi}f(t)^2\d t\le \int_0^{2\pi}f'(t)^2\d t. 
$$
c) Étudier les cas d'égalités. 

\exo [Level=2,Fight=0,Learn=0,Type=\Exercices,Field=\SériesDeFourier,Origin=] tf. 
Soit $f$ l'application impaire et $2\pi$-périodique définie par $f(x)=x(\pi-x)$ 
pour $0\le x\le\pi$. \pn
a) Étudier le développement en série de Fourier de $f$. \pn
b) En déduire la somme des séries 
$$
\sum_{n=0}^\infty{(-1)^n\F(2n+1)^3}\qquad\sum_{n=0}^\infty{1\F(2n+1)^3}
\quad\hbox{et}\quad\sum_{n=1}^\infty{1\F n^6}. 
$$

\exo [Level=2,Fight=0,Learn=0,Type=\Exercices,Field=\SériesDeFourier,Origin=] tg. 
Pour chaque nombre réel $x\in\Q]0,{\pi\F2}\W]$, on note $f_x$ 
l'unique fonction paire et $2\pi$-périodique vérifiant $f(t)=\max\{0, 1-{t\F2x}\}$ lorsque $0\le t\le \pi$. \pn
a) Développer $f_x$ en série de Fourier.\pn 
b) En déduire les sommes des séries 
$$
\sum_{k=1}^\infty{\sin^2(kx)\F k^2}\qquad\hbox{et}\qquad\sum_{k=1}^\infty{\sin^4(kx)\F k^4}.
$$

\exo [Level=2,Fight=0,Learn=0,Type=\Exercices,Field=\SériesDeFourier,Origin=] th. 
a) Etablir l'identité 
$$
|\sin x|={2\F\pi}+\sum_{n=1}^\infty{2\F\pi}\Q({1\F2n+1}-{1\F2n-1}\W)\cos(nx)\qquad(x\in\ob R).
$$
b) En déduire que 
$$
|\sin x|={8\F\pi}\sum_{n=1}^\infty{\sin^2(nx)\F4n^2-1}\qquad(x\in\ob R).
$$  

\exo [Level=2,Fight=0,Learn=0,Type=\Exercices,Field=\SériesDeFourier,Origin=] ti. 
a) En justifiant soigneusement, calculer la série
$$
f(x)=\sum_{n=0}^\infty{\sin nx\F n}
$$
b) En utilisant la fonction impaire et $2\pi$-périodique $g$ définie par 
$g(x)=xf(1)$ pour $0\le x\le 1$ et par $g(x)=f(x)$ pour $1\le x\le \pi$, montrer que 
$$
\sum_{n=1}^\infty{\sin n\F n}=\sum_{n=1}^\infty{\sin^2n\F n^2}. 
$$ 
c) Calculer $\ds \sum_{n=1}^\infty{\sin^4 n\F n^4}$. 

\exo [Level=2,Fight=0,Learn=0,Type=\Exercices,Field=\SériesDeFourier,Origin=] tj. 
Soit $\alpha\in\Q[0,+\infty\W[$ et soit $f$ la fonction $2\pi$-périodique 
définie par $f(x)=\cos (\alpha x)$ 
pour $-\pi< x\le \pi$. \pn
a) Montrer que $f$ est paire, continue sur $\ob R$ et de classe $\sc C^1$ 
par morceaux. \pn
b) Déterminer le développement en série de Fourier de $f$ (faire $2$ 
cas selon que $\alpha\in\ob N$). \pn
c) Pour $\alpha\notin\ob N$, déterminer les valeurs de $\ds \sum_{n=1}^\infty{1\F n^2-\alpha^2}$ 
et de 
$\ds \sum_{n=1}^\infty{(-1)^n\F n^2-\alpha^2}$. \pn
d) En déduire que 
$$
{1\F\tan t}={1\F t}-2\sum_{n=1}^\infty{t\F (n\pi)^2-t^2}\qquad(0<t<\pi)
$$

\exo [Level=2,Fight=0,Learn=0,Type=\Exercices,Field=\SériesEntières,Origin=] tk. 
Déterminer le rayon de convergence $R$ de la série entière 
$\ds S(x)=\sum_{n=0}^\infty{n^3+n+3\F n+1}x^n$. Calculer sa somme $S(x)$ pour $-R<x<R$ 
et étudier $S(x)$ en $x=\pm R$. 

\exo [Level=2,Fight=0,Learn=0,Type=\Exercices,Field=\SériesEntières,Origin=] tl. 
Rayon de convergence de $\ds\sum_{n=1}^\infty {n+\e^n\F1+ n^{\sqrt n}}x^n$ ?

\exo [Level=2,Fight=0,Learn=0,Type=\Exercices,Field=\SériesEntières,Origin=] tm. 
Rayon de convergence de $\ds\sum_{n=1}^\infty{n^{n!}\F n^{\exp n\ln n}}x^n$ ? 

\exo [Level=2,Fight=0,Learn=0,Type=\Exercices,Field=\SériesNumériques,Origin=] tn. 
a) Prouver que
$$
{1\F k+1}\le \ln(k+1)-\ln k\le {1\F k}\qquad(k\in\ob N^*).
$$ 
b) En déduire que la suite $(u_n)_{n\in\ob N^*}$ 
définie par $\ds u_n=\sum_{k=1}^n{1\F k}-\ln n$ converge. \pn
c) En déduire un équivalent de la suite $(u_n)_{n\in\ob N}$ à l'infini. 


\exo [Level=1,Fight=2,Learn=2,Type=\Exercices,Field=\Suites,Origin=] to. 
Soit $(u_n)_{n\in\ob N}$ la suite définie par $u_0=0$ et $u_{n+1}:=\sqrt{2+u_n}$ 
pour $n\ge0$. \pn
a) Prouver que $u_n\in[0,2]$ pour $n\in\ob N$. \pn
b) Prouver que la suite $(u_n)_{n\in\ob N}$ est croissante et converge vers $\ell$ à déterminer. \pn
c) En étudiant la suite $x_n=\arccos(u_n/2)$
retrouver le résultat précédent. \pn
d) trouver une constante $a\in\Q[0,1\W[$ telle que $\Q|u_{n+1}-2\W|\le a\Q|u_n-2\W|$ pour $n\in\ob N$  
et retrouver le résultat des questions b) et c). 

\exo [Level=2,Fight=0,Learn=0,Type=\Exercices,Field=\IntégralesGénéralisées,Origin=] tp. 
Prouver la convergence de l'intégrale $\ds \int_{-1}^1{\arctan(t+1)\F\sqrt{1-t^2}}\d t$. 

\exo [Level=1,Fight=1,Learn=0,Field=\EspacesPréHilbertiens,Type=\Exercices,Origin=] tq. 
Soit $E$ un $\ob R$-espace vectoriel 
muni d'un produit scalaire $\langle\cdot,\cdot\rangle$ et de sa norme associée. Prouver que 
$$
\forall (x,y)\in E^2,\quad 2+\|x+y\|^2\le2\Q(1+\|x\|^2\W)\Q(1+\|y\|^2\W)
$$
 
\exo [Level=2,Fight=1,Learn=0,Field=\Orthonormalisation,Type=\Exercices,Origin=] tr. 
Calculer la valeur minimale de $\int_0^1(t\ln t-a-bt)^2\d t$ et préciser 
pour quels couples $(a,b)$ cette valeur est atteinte. 

\exo [Level=1,Fight=1,Learn=1,Field=\EspacesPréHilbertiens,Type=\Exercices,Origin=] ts. 
Soit $n\in\ob N$. Montrer que l'on munit $E:=\ob R_n[X]$ 
d'un produit scalaire réel en posant 
$$
\forall(P,Q)\in E^2,\qquad\langle P, Q\rangle:=\sum_{k=0}^nP(k)Q(k)
$$ 

\exo [Level=2,Fight=1,Learn=0,Field=\Orthonormalisation,Type=\Exercices,Origin=] tt. 
Montrer que l'on munit $E:=\ob R[X]$ d'un produit scalaire réel en posant 
$$
\forall (P,Q)\in E^2,\qquad\langle P,Q\rangle:=\int_0^\infty P(t)Q(t)\e^{-t}\d t
$$
Déterminer $a$, $b$, $c$ et $d$ pour que les fonctions polynômes 
définies sur $\Q[0,+\infty\W[$ par $P_0(t):=a$, $P_1(t):=t+b$ 
et $P_2(t):= t^2+ct+d$ soient orthogonales $2$ à $2$. \pn
En déduire une famille orthonormée de trois vecteurs de $E$. 

\exo [Level=1,Fight=1,Learn=0,Field=\EspacesPréHilbertiens,Type=\TravauxDirigés,Origin=] tu. 
Soit $E$ un espace pré-Hilbertien réel et soit $f:E\to E$ une application vérifiant $f(0)=0$ et conservant les distances, i.e. telle que 
$$
\forall (x,y)\in E^2,\quad \|f(x)-f(y)\|=\|x-y\|.
$$
a) Montrer que $f$ préserve la norme, c'est-à-dire que $\|f(x)\|=\|x\|$ pour $x\in E$. \pn
b) Montrer que $f$ préserve le produit scalaire, c'est-à-dire que $\langle f(x),f(y)\rangle=\langle x,y\rangle$ pour $(x,y)\in E^2$. \pn
c) Prouver que $f$ est linéaire en étudiant pour $(\lambda,\mu)\in\ob R^2$ et $(x,y)\in E^2$ 
la quantité 
$$
\|f(\lambda x+\mu y)-\lambda f(x)-\mu f(y)\|^2. 
$$

\exo [Level=1,Fight=2,Learn=1,Field=\EspacesPréHilbertiens,Type=\Exercices,Origin=] tv. 
Soit $E$ un espace pré-Hilbertien réel et soit $f:E\to E$ 
une application vérifiant 
$$
\forall (x,y)\in E^2,\quad\langle f(x),y\rangle=\langle x,f(y)\rangle. 
$$
a) Pour $(\lambda,\mu)\in \ob R^2$ et $(x,y,z)\in E^3$, 
prouver que l'on a 
$$
\langle f(\lambda x+\mu y),z\rangle=\langle \lambda f(x)+\mu f(y),z\rangle
$$
b) En déduire que l'application $f$ est linéaire. 

\exo [Level=2,Fight=1,Learn=0,Field=\Orthonormalisation,Type=\Exercices,Origin=] tw. 
Soit $n\in\ob N^*$. pour $(x_1,\cdots, x_n)\in \ob R^n$, on pose 
$$
f(x_1,\cdots,x_n):=\int_0^1(1+x_1t+\cdots x_nt^n)^2\d t
$$
a) Prouver que la fonction $f$ admet un minimum absolu sur $\ob R^n$. \pn
b) Déterminer ce minimum pour $n=3$. 

\exo [Level=2,Fight=1,Learn=0,Field=\Orthonormalisation,Type=\Exercices,Origin=] tx. 
Soit $n\in\ob N^*$. pour $(x_1,\cdots, x_n)\in \ob R^n$, on pose 
$$
f(x_1,\cdots,x_n):=\int_0^1(1+x_1t+\cdots x_nt^n)^2\e^{-t}\d t
$$
a) Prouver que la fonction $f$ admet un minimum absolu sur $\ob R^n$. \pn
b) Déterminer ce minimum pour $n=3$. 

\exo [Level=2,Fight=1,Learn=0,Field=\Orthonormalisation,Type=\Exercices,Origin=] ty. 
Calculer la valeur minimale de $\int_0^\pi(\sin x-ax^2-bx)^2\d x$
pour $(a,b)\in\ob R^2$. 

\exo [Level=2,Fight=1,Learn=1,Field=\Orthonormalisation,Type=\Exercices,Origin=] tz. 
L'espace $\ob R^3$ étant muni du produit scalaire canonique, déterminer l'expression analytique de la projection orthogonale $p$ sur le plan $P$ d'équation $x-y+z=0$ puis calculer la distance du plan $P$ au point $A$ de coordonnées $(-1,2,1)$. 

\exo [Level=2,Fight=1,Learn=1,Field=\Orthonormalisation,Type=\Exercices,Origin=] ua. 
On note $F$ le sous-espace vectoriel de $\ob R^4$ d'équation 
$$
\Q\{\eqalign{x+y+z+t=0\cr
x+2y+3z+4t=0\cr}\W.
$$
Déterminer la projection orthogonale sur $F$ et calculer $d(x,F)$ pour $x\in\ob R^4$. 

\exo [Level=2,Fight=1,Learn=1,Field=\Orthonormalisation,Type=\Exercices,Origin=] ub. 
On note $F$ le sous-espace vectoriel de $\ob R^4$ d'équation 
$$
\Q\{\eqalign{x+y+z+t=0\cr
x-y+z-t=0\cr}\W.
$$
Déterminer la projection orthogonale sur $F$ et calculer $d(x,F)$ pour $x\in\ob R^4$. 

\exo [Level=2,Fight=2,Learn=2,Field=\InégalitéDeCauchySchwarz,Type=\Exercices,Origin=] uc. 
Pour $n\ge1$, on pose $S_n:=\sum_{k=0}^nk$ et $T_n:=\sum_{k=0}^nk^2$. \pn
a) Calculer $S_n$. En cas de panne, on pourra d'abbord prouver que $S_n=\sum_{k=0}^n(n-k)$ pour $n\ge0$. \pn
b) Montrer que $T_n=\sum_{k=0}^n(n-k)^2$ pour $n\in\ob N$, développer et en déduire 
une formule pour $T_n$ ne dépendant que de $n$. \pn 
b) A l'aide de Cauchy Schwarz, en déduire que  
$$
\sum_{k=0}^nk\sqrt k\le {n(n+1)\F 2\sqrt3}\sqrt{2n+1}\qquad(n\in\ob N).
$$

\exo [Level=2,Fight=2,Learn=1,Field=\EspacesPréHilbertiens,Type=\Exercices,Origin=] ud. 
Soit $E$ un espace préhilbertien réel et soient $f,g:E\to E$ telles que 
$$
\langle f(x),y\rangle=\langle x,g(y)\rangle\qquad(x,y)\in E^2. 
$$
Montrer que $f$ et $g$ sont linéaires. 

\exo [Level=1,Fight=0,Learn=1,Field=\EspacesPréHilbertiens,Type=\Cours,Origin=\Lakedaemon] ue. 
Soit $E$ un $\ob R$-espace vectoriel muni d'un produit scalaire 
$\langle\cdot,\cdot\rangle$ et de sa norme associée $x\mapsto\|x\|:=\sqrt{\langle x,x\rangle}$. 
Prouver que 
$$
\forall(x,y)\in E^2, \qquad
\langle x,y\rangle={\|x+y\|^2-\|x\|^2-\|y\|^2\F2}={\|x+y\|^2-\|x-y\|^2\F4}.
$$

\exo [Level=2,Fight=0,Learn=0,Type=\Exercices,Field=\IntégralesGénéralisées,Origin=] uf. 
Prouver la convergence de l'intégrale 
$\ds \int_{-1}^1{\arctan(1+1/x)\F\sqrt{|x|}}\d t$. 

\exo [Level=2,Fight=0,Learn=0,Type=\Exercices,Field=\IntégralesGénéralisées,Origin=] ug. 
Prouver la convergence de l'intégrale $\ds\int_0^\infty{\arctan t\F t(\ln t)^2}\d t$. 

\exo [Level=2,Fight=0,Learn=0,Type=\Exercices,Field=\IntégralesGénéralisées,Origin=] uh. 
Nature de l'intégrale $\ds\int_{-\infty}^\infty{(\th t)^2\F t\ln (1+t^2)}\d t$ ? 

\exo [Level=2,Fight=0,Learn=0,Type=\Exercices,Field=\SériesDeFourier,Origin=] ui. 
a) Pour $r\in\Q]-1,1\W[$, calculer $\ds1+2\sum_{n=1}^\infty r^n\cos(nx)$. \pn
b) Soit $a>0$. Utiliser le calcul précédent pour trouver le développement 
en série de Fourier de 
$$
f(x)={1\F\cos(x)+\ch a}\qquad(x\in\ob R).
$$
c) En déduire $\ds I_n=\int_0^\pi{\cos(nt)\F\cos t+\ch a}\d t$ pour $n\in\ob N$. 


\exo [Level=2,Fight=0,Learn=0,Type=\Exercices,Field=\SériesDeFourier,Origin=] uj. 
Pour $x\in\ob R$, on pose $f(x):=\max\{0,\sin x\}$. \pn
a) Série de Fourier de $f$ ?\pn
b) Calculer $\ds\sum_{n=1}^\infty{1\F(4n^2-1)^2}$. 

\exo [Level=2,Fight=0,Learn=0,Type=\Maple,Field=\SériesDeFourier,Origin=] uk. On note $f$ l'unique fonction paire et $2$-périodique 
telle que $f(x)=x^3$ pour $0\le x\le1$. \medskip\noindent
a) Définir la fonction $f$ en utilisant les commandes $:=$, $\backslash;$, $\le$, $\&\&$, 
IntegerPart ou FractionalPart. \pn
Il existe plusieurs fa\c cons de procéder... \medskip\noindent
b) Déssiner le graphe de $f$ sur $[-1.1,3.2]$. 
On pourra utiliser l'option PlotRange$\rightarrow\{y_{min},y_{max}\}$. \medskip\noindent
c) Quel est la classe de $f$ sur $[0.5,1.5]$ ? 
On pourra utiliser $P(x):=x^3\ \,(x\in\ob R)$ pour les calculs.\medskip\noindent 
d) Quel est la classe de $f$ sur $[-0.5,0.5]$ ? 
\medskip\noindent
e) quel est le plus grand entier $k$ pour lequel $f$ 
est de classe $\sc C^k$ sur $\ob R$ ? \medskip\noindent
f) Calculer les coefficients de Fourier de $f$. Utiliser $P$ si $f$ 
ne se laisse pas intégrer ($\%\&!\star\#$). Substituer $0$ à $\sin(\pi n)$ 
et $(-1)^n$ à $\cos(\pi n)$ 
pour simplifier. \medskip\noindent
g) Appliquer le théorème de Dirichlet pour les valeurs $x=0$, $x=1/2$ et $x=1/4$. 
\medskip\noindent
h) Appliquer le théorème de Parseval à la fonction $f$. 
\medskip\noindent
i) faire $5$ dessins (ou 1 seul ?) faisant apparaitre $f$ 
et respectivement $S_1[f]$, $S_2[f]$, $S_3[f]$, $S_5[f]$ et $S_{30}[f]$ 
sur l'intervalle $[-1,1]$. 

\exo [Level=2,Fight=0,Learn=0,Type=\Maple,Field=\SériesDeFourier,Origin=] ul. 
Soient $(a,b,c)\in\ob R^3$ et $f$ la fonction paire, $2\pi$-périodique, définie par 
$$
f(x)=ax^2+bx+c\qquad(0\le x\le \pi).
$$
a) Déterminer les coefficients de Fourier de $f$. \medskip\noindent
b) Montrer que l'on peut choisir $a$, $b$ et $c$ pour que la série de Fourier de $f$ soit 
$$
S[f](x)=\sum_{n=1}^\infty{\cos nx\F n^2}\qquad(x\in\ob R).
$$
Représenter alors le graphe de $f$ sur $[-\pi,\pi]$. \medskip\noindent
c) En déduire la valeur de $\sum_{n=1}^\infty{1\F n^2}$. 

\exo [Level=2,Fight=0,Learn=0,Type=\Maple,Field=\SériesDeFourier,Origin=] um. 
a ) Développer $f_0(x)=\exp x$ en série entière au voisinage de $0$. 
On pourra comparer les commandes "$Series...$" et "$SeriesTerm[f[x],{x,0,n}]$" 
(package Rsolve) et au besoin, on rebootera. \medskip\noindent
b) Rayon de convergence de la série entière $\sum_{n=0}^\infty a_nx^n$ obtenue (Dalembert...). \medskip\noindent
c) Pour $x=1$, trouver un entier $N$ tel que 
$|f_0(x)-\sum_{n=0}^Na_nx^n|\le 10^{-6}$.\medskip\noindent
d) mêmes questions (Trouver $N_1$ pour c...) pour $f_1(x)=\ln(1+x)$ et $x=1$. \pn
{\it Chercher dans l'aide la signification 
des fonctions zarbis apparaissant dans le résultat de d) et e).} \pn
e) mêmes questions (Trouver $N_2$ pour c...) pour $f_2(x)=\arctan x$ et $x=1$. \pn
f) mêmes questions (Trouver $N_3$ pour c...) pour $f_3(x)=6\arcsin x$ et $x=1/2$. \pn
g) Comparer $N_2$ et $N_3$..."Timing" des calculs ... conclusions ?

\exo [Level=2,Fight=2,Learn=2,Field=\EquationsDifférentielles,Type=\Maple,Origin=] un. 
Trouver UNE série entière $S[x]=\sum_{n=0}^\infty a_nx^n$ non triviale
solution de l'équation différentielle 
$$
x^2y''+x(x+1)y'-y=0
$$
Avec les instructions "SeriesTerm[  ]", "Gf[a][x]", "Rsolve" ou "GeneratingFunction". \pn
Retrouver le résultat avec "Dsolve". 

\exo [Level=1,Fight=0,Learn=0,Type=\Exercices,Field=\Suites,Origin=] uo. 
Soient $(a,b)\in\Q]0,\infty\W[^2$ et soit $(u_n)_{n\in\ob N}$ la suite définie par $u_0=a$, $u_1=b$ et 
$$
u_{n+1}=\sqrt{u_nu_{n-1}}\qquad(n\ge0).
$$
Montrer que $(u_n)_{n\in\ob N}$ converge et calculer sa limite. 

\exo [Level=1,Fight=0,Learn=0,Type=\Exercices,Field=\DéveloppementsLimités,Origin=] up. 
Montrer que $\ds\sum_{k=0}^n\e^{k^2}\sim\e^{n^2}$ lorsque $n\to\infty$. 

\exo [Level=1,Fight=0,Learn=0,Type=\Exercices,Field=\DéveloppementsLimités,Origin=] uv. Donner un équivalent simple de $\ds\sum_{k=n}^{2n}{1\F k^k}$ lorsque $n\to\infty$. 
 
\exo [Level=1,Fight=0,Learn=0,Type=\Exercices,Field=\DéveloppementsLimités,Origin=] uw. 
Soient $a\in\ob R$ et soit $(u_n)_{n\in\ob N}$ une suite vérifiant 
$\ds\lim_{n\to\infty}(u_{n+1}-u_n)=a$. Calculer $\ds\lim_{n\to\infty}{u_n\F n}$. 

\exo [Level=2,Fight=0,Learn=0,Type=\Exercices,Field=\SériesNumériques,Origin=] ux. 
Soit $u_0\in\ob R$. Pour $n\ge1$, on pose $u_n=\sqrt{1+u_{n-1}^2}$ 
et $v_n={1\F v_1}+\cdots+{1\F v_n}$. Étudier les suites $u_n$, $v_n$, $v_n-2u_n$ et $v_n/u_n$. 

\exo [Level=1,Fight=0,Learn=0,Type=\Exercices,Field=\Suites,Origin=] uy. 
a)Etablir une condition nécéssaire et suffisante 
sur deux suites $u_n$ et $v_n$ pour que $\e^{u_n}\sim\e^{v_n}$. 
\pn
b) trouver deux suites $u_n\not\sim v_n$ telles que $\e^{u_n}\sim\e^{v_n}$ et deux suites $u_n'\sim v_n'$ 
telles que $\e^{u_n'}\not\sim\e^{v_n'}$. 

\exo [Level=1,Fight=0,Learn=0,Type=\Exercices,Field=\Suites,Origin=] uz. 
Démontrer que les suites $u_n=\Q(1+{1\F1^2}\W)\cdots\Q(1+{1\F n^2}\W)$ 
et $v_n=u_n(1+{1\F n})$ sont adjacentes. 

\exo [Level=2,Fight=2,Learn=2,Field=\Orthonormalisation,Type=\Colles,Origin=] va. 
a) Montrer que l'on définit un produit scalaire sur $E=\ob R[X]$ en posant 
$$
\langle P,Q\rangle:=\int_0^\infty P(x)Q(x)\e^{-2x}\d x
$$ 
b) Pour $n\in\ob N$ et $x\in\ob R$, on pose $\ds L_n(x):={\e^x\F n!}{\d ^n\F\d x^n}\Q(x^n\e^{-x}\W)$. 
Prouver que $L_n$ est une famille orthonormale de vecteurs de $E$ (telle que $\|L_n\|=1$ pour $n\ge0$ 
et $\langle L_m, L_n\rangle=0$ pour $m\neq n$). \pn
c) Soit $\alpha>0$ et $f:x\mapsto \e^{-\alpha x}$. Calculer $\|f\|^2$, $\langle f,L_n\rangle$ 
pour $n\in\ob N$ et $\ds\sum_{n=0}^\infty\langle f,L_n\rangle^2$. Remarques ?

\exo [Level=2,Fight=1,Learn=0,Field=\SériesDeFourier,Type=\Exercices,Origin=] vb. 
a) Développer la fonction $f:x\mapsto\arccos(\sin x)$ en série de Fourier. \pn
b) En déduire $\sum_{n=1}^\infty{1\F n^2}$ et $\sum_{n=1}^\infty{1\F n^4}$. 

\exo [Level=2,Fight=2,Learn=1,Field=\Normes,Type=\Exercices,Origin=] vc. 
On pose $E:=\sc C^1\Q([-1,1],\ob R\W)$ et
$$
N(f):=\sqrt{2f(0)^2+\int_{-1}^1{f'(t)^2\F\sqrt{1-t^2}}\d t}
$$
a) Prouver que $N$ est une norme de $E$. \pn
b) $N$ est elle une norme euclidienne ?

\exo [Level=2,Fight=2,Learn=2,Type=\Cours,Field=\Orthonormalisation,Origin=\Lakedaemon] ve. 
On note $E:=\sc C_{2\pi}(\ob R)$ l'ensemble des fonctions continues et $2\pi$-périodiques 
sur $\ob R$ et   
$$
\forall(f,g)\in E^2, \qquad\langle f,g\rangle:=\int_0^{2\pi}f(t)g(t)\d t
$$
Pour $N\in\ob N$, Prouver que l'application $\phi:f\mapsto S_N[f]$ est : \pn
a) un endomorphisme de $E$.\pn 
b) un projecteur de $E$. (i.e. que $\phi\circ\phi=\phi$). \pn
c) Déterminer $\hbox{Im}\ \phi$ et prouver que $S_N(f)$ 
est la projection orthogonale de $f$ sur $\hbox{Im}\ \phi$, i.e. que 
$$
\forall g\in\hbox{Im}\ \phi, \qquad\langle f-S_N(f),g\rangle=0. 
$$

\exo [Level=2,Fight=1,Learn=0,Field=\Orthonormalisation,Type=\Exercices,Origin=] vf. 
Déterminer une base orthonormée de $\ob R_3[X]$ pour le produit scalaire 
$$
\langle P,Q\rangle:=\int_{-1}^1 P(t)Q(t)\d t.
$$

\exo [Level=2,Fight=2,Learn=1,Field=\MatricesOrthogonales|\InégalitéDeCauchySchwarz,Type=\Exercices,Origin=,Indication={Utiliser Cauchy-Schwarz pour le produit scalaire 
$\sum_{\ell=1}^nx_\ell y_\ell$ en remarquant que les colonnes de $A$ forment 
une famille orthonormale}] vg. 
Pour $n\in\ob N^*$ et $A\in\sc M_n(\ob R)$ matrice orthogonale 
(i.e. vérifiant $\null^{\hbox{t}}AA=\hbox{I}_n$),  
établir que 
$$
\Q|\sum_{i=1}^n\sum_{j=1}^na_{i,j}\W|\le n.
$$

\exo [Level=2,Fight=2,Learn=1,Field=\MatricesOrthogonales|\InégalitéDeCauchySchwarz,Type=\Exercices,Origin=,Indication={Utiliser Cauchy-Schwarz pour le produit scalaire $\sum_{i=1}^n\sum_{j=1}^nx_{i,j}y_{i,j}$}] vh. 
Pour $n\in\ob N^*$ et $A\in\sc M_n(\ob R)$ matrice orthogonale 
(i.e. vérifiant $\null^{\hbox{t}}AA=\hbox{I}_n$),  
établir que 
$$
\sum_{i=1}^n\sum_{j=1}^n|a_{i,j}|\le n\sqrt n. 
$$


\exo [Level=2,Fight=1,Learn=0,Field=\Orthonormalisation,Type=\Exercices,Origin=] vi. 
Calculer la valeur minimale de $\int_0^1(t\ln t-a-bt)^2\d t$ pour $(a,b)\in\ob R^2$. 

\exo [Level=2,Fight=2,Learn=1,Field=\EspacesPréHilbertiens,Type=\Exercices,Origin=] vj. 
Démontrer que la norme de l'espace $\sc C\Q([0,1],\ob R\W)$ définie par $\|f\|:=\int_0^1|f(t)|\d t$ 
n'est pas une norme euclidienne. 

\exo [Level=2,Fight=4,Learn=2,Field=\EspacesPréHilbertiens,Type=\Exercices,Origin=] vk. 
Pour $E:=\sc M_n(\ob R)$, on pose 
$$
\forall(A,B)\in E^2,\quad \langle A,B\rangle:=\mbox{\rm tr}\Q(\NULL^tAB\W).
$$
a) Démontrer que $\langle\cdot,\cdot\rangle$ est un produit scalaire de $E$. \pn
b) {\it difficile : }Notant $N$ la norme associée à $\langle\cdot,\cdot\rangle$, prouver que $N(AB)\le N(A)N(B)$ 
pour $(A,B)\in\sc M_n(\ob R)^2$. 


\exo [Level=2,Fight=2,Learn=1,Field=\IntégralesGénéralisées,Type=\Exercices,Origin=] vl. 
On pose $E:=\Q\{f\in\sc C\Q(\Q]0,1\W],\ob R\W):\ds\int_0^1{f(t)^2\F \ln t}\d t\hbox{ converge }\W\}$. \pn
a) Prouver que $E$ est un $\ob R$-espace vectoriel. \pn
b) Prouver que l'on définit un produit scalaire sur $E$ en posant 
$$
\forall (f,g)\in E^2, \qquad\langle f,g\rangle=-\int_0^1{f(t)g(t)\F\ln  t}\d t. 
$$
c) prouver que 
$$
\Q(\int_0^1|\ln t|^\alpha\d t\W)^2\le\int_0^1|\ln t|^{\alpha-1}\d t\int_0^1|\ln t|^{\alpha+1}\d t\qquad(\alpha>0.) 
$$

\exo [Level=2,Fight=1,Learn=2,Field=\EspacesPréHilbertiens,Type=\Cours,Origin=] vm. 
Soit $\b(E,\langle\cdot,\cdot\rangle\b)$ un espace pré-Hilbertien réel. 
Pour chaque sous-ensemble $A$ de $E$, on~pose~$A^{\per}:=\{x\in E:\forall y\in A, x\Per y\}$. 
\pn
a) Prouver que $A^\per$ est un sous espace vectoriel de $E$. \pn
b) Montrer que $A\subset(A^\per)^\per$ et que $A\cap A^\per=\{0\}$. \pn
c) Lorsque $A$ est un sous-espace vectoriel de dimension finie de $E$, 
prouver que $E=A+A^\per$ (autrement dit, on a $E=A\oplus A^\per$). 

\exo [Level=2,Fight=1,Learn=0,Field=\InégalitéDeCauchySchwarz,Type=\Cours,Origin=] vn. 
Soient $n\in\ob N^*$, $(u_n)_{n\in\ob N}$ 
une suite de vecteurs de $\ob R^n$ convergeant vers $\ell\in\ob R^n$ et $x\in E$ tel que 
$$
x\Per u_n\qquad(n\in\ob N),  
$$ 
pour la structure euclidienne canonique de $\ob R^n$. Prouver que $x\Per\ell$. 

\exo [Level=2,Fight=1,Learn=1,Field=\EspacesPréHilbertiens,Type=\Exercices,Origin=] vo. 
Soit $E$ un espace vectoriel euclidien et soit $u\in\sc L(E)$ un endomorphisme de $E$ 
conservant l'orthogonalité, i.e. vérifiant 
$$
\forall (x,y)\in E^2,\quad \langle x,y\rangle=0\Longrightarrow \langle u(x),u(y)\rangle=0. 
$$
Prouver qu'il existe un nombre réel $\lambda\ge0$ tel que 
$$
\forall x\in E, \quad\|u(x)\|=\lambda\|x\|.
$$

\exo [Level=2,Fight=2,Learn=1,Field=\EspacesPréHilbertiens,Type=\Colles,Origin=] vp. 
Soit $E$ un espace euclidien de dimension $n\ge1$ et soit $e_1,\cdots,e_n$ une famille de vecteurs unitaires vérifiant
$$
\forall x\in E,\quad \|x\|^2=\sum_{k=1}^n\langle x,e_k\rangle^2. 
$$
Prouver que $e_1,\cdots,e_n$ est une base orthonormale de $E$. 

\exo [Level=2,Fight=1,Learn=1,Field=\EspacesPréHilbertiens,Type=\Exercices,Origin=] vq. 
Soit $E$ un espace euclidien et $f\in\sc L(E)$ tel que 
$$\forall x\in E, \qquad \langle x, f(x)\rangle=0.
$$ 
a) Prouver que $\forall (x,y)\in E^2, \quad \langle f(x),y\rangle=-\langle x,f(y)\rangle$. \pn
b) En déduire que $E=\hbox{Ker\ }f\mathop{+}\limits^\per\hbox{Im\ }f$. 

\exo [Level=2,Fight=0,Learn=1,Field=\MatricesOrthogonales,Type=\Exercices,Origin=] vr. 
Compléter la matrice ${1\F 7}\pmatrix{6&3&\cdots\cr-2&6&\cdot\cr3&\cdot&\cdot\cr}$ 
en une matrice $A$ de $SO(n)$. \pn 
Nature géométrique de l'endomorphisme dont la matrice dans la base canonique de $\ob R^3$ 
est $A$ ?  

\exo [Level=2,Fight=0,Learn=0,Field=\MatricesOrthogonales,Type=\Exercices,Origin=] vs. 
Déterminer la nature géométrique de l'application linéaire $f$ dont la matrice 
dans la base canonique de $\ob R^3$ est
$$
-{1\F5}\pmatrix{0&5&0\cr3&0&4\cr4&0&-3\cr}.
$$ 

\exo [Level=2,Fight=1,Learn=0,Field=\EndomorphismesOrthogonaux,Type=\Exercices,Origin=] vt. 
Soit $E$ un espace euclidien et $f\in\sc O(E)$. Montrer que $f$ est diagonalisable sur $\ob R$ si, 
et~seulement~si, $f$ est une symétrie orthogonale. 

\exo [Level=2,Fight=0,Learn=0,Field=\MatricesOrthogonales,Type=\Exercices,Origin=] vu. 
Soit $u$ un vecteur unitaire d'un espace euclidien $E$ 
et soit $U$ sa matrice dans une base orthonormale $B$ de $E$. \pn
a) Montrer que $U\null^tU$ est la matrice de la projection orthogonale 
sur $\hbox{Vect}\langle u\rangle$. \pn
b) Trouver la matrice de la symétrie orthogonale 
par rapport à $\hbox{Vect}\langle u\rangle$ et par rapport à $\hbox{Vect}\langle u\rangle^\per$

\exo [Level=2,Fight=1,Learn=1,Type=\Others,Field=\MatricesOrthogonales,Origin=] vv. 
Soient $A\in\sc O(n)$ et $X$ un vecteur propre de $A$ 
pour la valeur propre complexe $\lambda$. 
Pourver que $|\lambda|=1$ en calculant de deux fa\c cons 
$\null^{\hbox{t}}(\overline{AX})AX$, 
où $\overline M$ désigne la conjuguée 
de la matrice $M$ c'est-à-dire la matrice dont les coefficients 
sont les conjugués des coefficients de $M$. 

\exo [Level=2,Fight=1,Learn=1,Field=\EndomorphismesOrthogonaux,Type=\Exercices,Origin=] vw. 
Soit $E$ un espace vectoriel euclidien, soit $f\in\sc O(E)$ et soit $V$ un sous-espace vectoriel de $E$ stable par $f$. 
Prouver que $f(V)=V$ et que $f(V^\per)=V^\per$. 

\exo [Level=2,Fight=3,Learn=1,Field=\EndomorphismesOrthogonaux,Type=\Exercices,Origin=]  vx. Soit $E$ un espace euclidien et $p, q$ deux projecteurs orthogonaux de $E$. 
Prouver que $p\circ q=0\Longleftrightarrow q\circ p=0$. 

\exo [Level=2,Fight=0,Learn=0,Field=\MatricesOrthogonales,Type=\Exercices,Origin=] vy. 
Déterminer la nature géométrique de l'endomorphisme de $\ob R^3$ 
dont la matrice dans la base canonique est 
$$
M={1\F 7}\pmatrix{-2&6&-3\cr6&3&2\cr-3&2&6\cr}
$$

\exo [Level=2,Fight=2,Learn=1,Field=\EspacesPréHilbertiens,Type=\Colles,Origin=] vz. 
Soit $E$ un espace euclidien de dimension $n$. 
On dit que les familles $\{e_1,\cdots,e_b\}$ et $\{f_1,\cdots,f_n\}$ 
sont $bi-orthogonale$ si, et seulement si, elles vérifient : 
$$\forall(i,j)\in\{1,\cdots,n\}^2,\quad
\langle e_i,f_j\rangle=\Q\{\eqalign{
1\quad\hbox{ si }i=j\cr
0\quad\hbox{ si }i\neq j.}\W.
$$
a) Soient $\{e_1,\cdots,e_b\}$ et $\{f_1,\cdots,f_n\}$ deux familles bi-orthogonales. 
Montrer qu'elles sont libres. \pn
b) Soit $B$ une base de $E$. Montrer qu'il existe une unique base $B'$ de $E$ telle que $B$ et $B'$ soient bi-orthogonales. \pn
c) trouver la base $B'$ associée par bi-orthogonalité à 
$$
B:=\Q\{\pmatrix{1\cr0\cr1\cr0\cr},\pmatrix{0\cr1\cr2\cr0\cr},\pmatrix{0\cr0\cr1\cr0\cr},\pmatrix{0\cr0\cr3\cr1\cr}\W\}.
$$

\exo [Level=2,Fight=0,Learn=0,Field=\MatricesOrthogonales,Type=\Exercices,Origin=] wa. 
Soit $f$ l'endomorphisme de $\ob R^3$ dont la matrice représentative 
dans la base canonique de $\ob R^3$ est 
$$
A={1\F3}\pmatrix{1&-2&-2\cr-2&1&-2\cr2&2&-1\cr}
$$
Montrer que $f$ est une isométrie dont on précisera les caractéristiques. 

\exo [Level=2,Fight=2,Learn=1,Field=\EspacesPréHilbertiens,Type=\Exercices,Origin=] wb. 
a) Prouver que $\langle X,Y\rangle:=\mbox{\rm tr}(\NULL^{\mbox{\rm t}}XY)$ est un produit scalaire de $\sc M_n(\ob R)$. \pn
b) En déduire que $\b|\hbox{tr\ }A\b|\le\sqrt n\sqrt{\NULL^{\mbox{\rm t}}AA}$ pour $A\in\ob M_n(\ob R)$. 

\exo [Level=2,Fight=2,Learn=1,Field=\Orthonormalisation,Type=\Exercices,Origin=] wc. 
a) Prouver que l'on définit un produit scalaire 
sur $\ob R[X]$ en posant 
$$
\langle P,Q\rangle:=\int_0^\infty P(t)Q(t)\e^{-t}\d t
$$
b) Montrer qu'il existe une (unique) famille orthonormale 
de polynômes unitaires $(P_n)_{n\in\ob N}$ 
telle que $\deg P_n=n$ pour $n\in\ob N$. \pn
c) Pour $n\in\ob N$, prouver que $P_n$ possède exactement $n$ 
racines simples et strictement positives.  

\exo [Level=2,Fight=2,Learn=2,Field=\Orthonormalisation,Type=\Exercices,Origin=] wd. 
Montrer que lon munit l'espace $\ob R[X]$ d'un produit scalaire en posant 
$$
\langle P,Q\rangle:=\int_{-1}^1{P(t)Q(t)\F\sqrt{1-t^2}}
$$
b) Expliquer pourquoi il existe une unique base orthonormée $(P_n)_{n\in\ob N}$ 
telle que $\deg P_n=n$ et telle que le coefficient 
dominant de $P_n$ soit positif pour $n\in\ob N$. \pn
c) Montrer que $\ds P_n(t)={\alpha_n\F 2^{n-1}}\cos(n\arccos t)$ pour $n\in\ob N$ et $t\in[-1,1]$. \pn
d) Déterminer les zéros de $P_n$ pour $n\in\ob N$. \pn
e) Soit $U$ l'ensemble des polynômes de $\ob R_n[X]$ de degré $n$ 
et de coefficient égal à $1$. Montrer que la borne inférieure
$$
\inf_{P\in U}\int_{-1}^1{P(t)^2\F\sqrt{1-t^2}}\d t
$$
est atteinte pour $P=P_n$. 

\exo [Level=2,Fight=1,Learn=0,Field=\MatricesOrthogonales,Type=\Exercices,Origin=] we. 
Soit $B$ une base orthonormée d'un espace euclidien
$E$ de dimension $3$ et soit $f$ l'endomorphisme de $E$ dont la matrice dans $B$ est 
$$
\pmatrix{0&0&-1\cr1&0&0\cr0&1&0\cr}.
$$
Montrer que $f$ est la composée de deux transformations simples 
que l'on déterminera. 

\exo [Level=1,Fight=1,Learn=1,Field=\EspacesPréHilbertiens,Type=\Exercices,Origin=] wf. 
Soit $E$ un espace euclidien et $p$ un projecteur de $E$. Montrer que $p$ est un projecteur orthogonal si, 
et seulement si, $\|p(x)\|\le\|x\|$ pour $x\in E$. 

\exo [Level=1,Fight=1,Learn=1,Field=\Continuité,Type=\Exercices,Origin=] wh. 
Étudier la continuité sur $\ob R$ de la fonction $f:x\mapsto E(x)+\sqrt{x-E(x)}$. 

\exo [Level=1,Fight=2,Learn=1,Field=\Continuité,Type=\Exercices,Origin=] wi. 
Montrer que la fonction $x\mapsto\sin{1\F x}$ n'est pas prolongeable par continuité en $0$, contrairement à 
$x\mapsto\sqrt{|x|}\sin(1/x)$. 

\exo [Level=1,Fight=0,Learn=1,Field=\Continuité,Type=\Exercices,Origin=] wj. 
Que peut-on dire d'une fonction périodique $f$ de période $T>0$ 
admettant une limite en $+\infty$ ?

\exo [Level=1,Fight=0,Learn=0,Field=\DéveloppementsLimités,Type=\Exercices,Origin=]  wk. 
Pour $(a,b)\in\ob R^2$, calculer $\ds\lim_{x\to\infty}\Q(\sqrt{x^2+x+1}-ax-b\W)$. 

\exo [Level=1,Fight=1,Learn=1,Field=\DéveloppementsLimités,Type=\Exercices,Origin=] wl. 
Calculer la limite $\ds\lim_{x\to\infty}\Q(\sh x\F \ch x-1\W)^x$. 

\exo [Level=1,Fight=1,Learn=1,Field=\Continuité,Type=\Exercices,Origin=] wm. 
étudier la continuité de la fonction $x\mapsto\arccos(\sin x)$. 

\exo [Level=1,Fight=0,Learn=0,Field=\DéveloppementsLimités,Type=\Exercices,Origin=] wn. 
Pour $x\in\ob R$, calculer la limite $\ds\lim_{n\to\infty}\Q(1-{x\F n}\W)^n$. 

\exo [Level=1,Fight=0,Learn=0,Field=\Limites,Type=\Exercices,Origin=] wo. 
Calculer la limite $\ds\lim_{x\to1}{\arctan(x)^2-\pi^2/16\F x^2-1}$. 

\exo [Level=1,Fight=0,Learn=1,Field=\DéveloppementsLimités,Type=\Exercices,Origin=] wp. 
Calculer la limite $\ds\lim_{x\to 1}{(x-1)\cos{\pi x}\F\sin(3\pi x)}$

\exo [Level=2,Fight=1,Learn=1,Field=\MatricesSymétriques,Type=\Colles,Origin=] wq. 
Soit $A\in\ob M_n(\ob R)$ telle que $B=A+\null^{\hbox{t}}A$ soit nilpotente ($\exists n\in\ob N, B^n=0$). 
Prouver que $A$ est anti-symétrique (que $\null^{\hbox{t}}A=-A$). 

\exo [Level=2,Fight=2,Learn=3,Field=\MatricesSymétriques,Type=\Exercices,Origin=] wr. 
Soit $A\in\sc M_n(\ob R)$. Montrer qu'il existe $M\in\sc M_n(\ob R)$ telle que $A=\null^{\hbox{t}}MM$ si, et seulement si, 
$A$ est symétrique et positive. 

\exo [Level=2,Fight=2,Learn=2,Field=\EndomorphismesSymétriques,Type=\Exercices,Origin=\MP] ws. 
Soit $f$ un endomorphisme symétrique d'un espace euclidien $E$, 
de valeurs propres $\lambda_1\le\cdots\le \lambda_n$ et soit $x$ un vecteur unitaire de $E$ (tel que $\|x\|=1$). Prouver que : \pn
a) $\lambda_1\le \langle x,f(x)\rangle \le \lambda_n$. \pn
b) $\langle x,f(x)\rangle=\lambda_1\Leftrightarrow f(x)=\lambda_1 x$. \pn
c) $\langle x,f(x)\rangle=\lambda_n\Leftrightarrow f(x)=\lambda_n x$. 

\exo [Level=2,Fight=2,Learn=1,Field=\FormesQuadratiques,Type=\Exercices,Origin=] wt. 
Soient $A\in\sc Gl_n(\ob R)$ et $B:=\null^{\hbox{t}}AA$. Pour $X\in\sc M_{n,1}(\ob R)$, on pose 
$q(X)=\null^{\hbox{t}X}BX$. Prouver que $q$ est une forme quadratique, définie positive. 

\exo [Level=2,Fight=2,Learn=1,Field=\FormesQuadratiques,Type=\Colles,Origin=] wu. 
Prouver que $q(M):=\det M$ définit une forme quadratique sur l'espace $E:=\sc M_2(\ob R)$. 
Matrice de $q$ dans la base $I=\pmatrix{1&0\cr0&1\cr}$, 
$J=\pmatrix{0&1\cr1&0\cr}$, 
$K=\pmatrix{0&-1\cr1&0\cr}$, $L=\pmatrix{1&0\cr0&-1\cr}$ ?

\exo [Level=2,Fight=3,Learn=3,Field=\EndomorphismesSymétriques,Type=\Exercices,Origin=\MP] wv. 
Soit $v$ un vecteur unitaire d'un espace euclidien $E$. 
Pour tout nombre réel $\alpha$, on note $\phi_\alpha$ 
l'application $\phi_\alpha:x\mapsto x+\alpha\langle x,v\rangle v$. \pn
a) Pour $\alpha\in\ob R$, prouver que $\phi_\alpha$ est un endomorphisme symétrique. \pn
b) Prouver que $\{\phi_\alpha:\alpha\in\ob R\}$ est stable par composition et commutatif pour $\circ$. \pn
c) Prouver que les valeurs propres de $\phi_\alpha$ sont $1$ et $1+\alpha$. 
Déterminer les espaces propres associés. \pn
d) si $\alpha\neq-1$, prouver que $\phi_\alpha$ est inversible. Quel est la nature de $\phi_{-1}$ ?\pn
e) Déterminer $\alpha$ pour que $\phi_\alpha$ soit une isométrie. Quelle est la nature de $\phi_{-2}$ ?

\exo [Level=2,Fight=0,Learn=0,Field=\MatricesSymétriques,Type=\Exercices,Origin=,Solution={$P:=\pmatrix{{-1\F\sqrt2}&{1\F\sqrt6}&{1\F\sqrt3}\cr{1\F\sqrt2}&{1\F\sqrt6}&{1\F\sqrt3}\cr0&{-2\F\sqrt6}&{1\F\sqrt3}}$ et $D:=\pmatrix{-1&0&0\cr0&-1&0\cr0&0&2\cr}$.}] ww. 
Diagonaliser à l'aide d'une matrice orthogonale la 
matrice $\pmatrix{0&1&1\cr1&0&1\cr1&1&0\cr}$

\exo [Level=2,Fight=1,Learn=0,Field=\MatricesSymétriques,Type=\Exercices,Origin=] wx. 
Pour $A\in\sc M_n(\ob R)$ matrice symétrique vérifiant $A^3+A^2+A=0$, 
prouver que $A=0$.  

\exo [Level=2,Fight=0,Learn=0,Field=\FormesQuadratiques,Type=\Exercices,Origin=] wy. 
Soit $q$ la forme quadratique sur $\ob R^3$ définie par 
$q(x,y,z)=x^2+y^2+z^2+xy+xz+yz$. Diagonaliser $q$ dans une base orthonormée. 

\exo [Level=2,Fight=0,Learn=0,Field=\FormesQuadratiques,Type=\Exercices,Origin=] wz. 
Pour $(x_1,\cdots,x_n)\in\ob R^n$, on pose $q(x_1,\cdots,x_n):=\sum_{1\le i<j\le 20}x_ix_j$. \pn
a) Prouver que $q$ est une forme quadratique. \pn
b) Diagonaliser $q$ dans une base orthogonale pour le produit scalaire canonique de $\ob R^{20}$. 

\exo [Level=2,Fight=0,Learn=0,Field=\FormesQuadratiques,Type=\Exercices,Origin=] xa. 
Réduire la forme bilinéaire $\phi(x,y,z)=x^2+y^2+z^2-2xy-2yz$. 

\exo [Level=2,Fight=0,Learn=0,Field=\MatricesSymétriques,Type=\Exercices,Origin=] xb. 
Soit $A\in M_n(\ob R)$ une matrice symétrique. \pn
a) Prouver qu'il existe une matrice symétrique $B\in\sc M_n(\ob R)$ telle que $A=B^3$. \pn
b) lorsque $A$ est positive, prouver qu'il existe 
une matrice symétrique $C\in\sc M_n(\ob R)$ telle que $A=C^2$. \pn

\exo [Level=2,Fight=3,Learn=3,Field=\MatricesSymétriques,Type=\Exercices,Origin=\MP] xc. 
Soit $A\in\sc M_n(\ob R)$ une matrice symétrique, 
de valeurs propres $\lambda_1,\cdots,\lambda_n$. \pn
En calculant $\mbox{tr}(\NULL^{\mbox{t}}AA)$ de deux fa\c cons, prouver que 
$$
\sum_{i=1}^n\sum_{j=1}^na_{i,j}^2=\sum_{k=1}^n\lambda_k^2
$$

\exo [Level=2,Fight=3,Learn=1,Field=\FormesQuadratiques,Type=\Others,Origin=] xd. 
Soit $E$ un espace euclidien et $f,g$ deux endomorphismes symétriques de $E$. 
Prouver l'équivalence des trois propositions suivantes : \pn
i) $f\circ g$ est symétrique. \pn
$\!\!$ii) $f\circ g=g\circ f$. \pn
$\!\!\!\!$iii) Il existe une base orthonormée diagonalisant simultanément $f$ et $g$. 

\exo [Level=2,Fight=3,Learn=2,Field=\FormesQuadratiques,Type=\Exercices,Origin=,Indication=
On pourra utiliser au moment opportun le résultat de l'exercice \eqrefn{labelexoPTxb}.] xe. 
Soit $A\in\sc M_n(\ob R)$ une matrice symétrique définie positive. 
Prouver que $q(X)=\null^{\hbox{t}}XAX$ est une forme quadratique sur $E=\sc M_{n,1}(\ob R)$ 
associée à une forme bilinéaire, définie positive. \pn

\exo [Level=2,Fight=0,Learn=0,Field=\FormesQuadratiques,Type=\Exercices,Origin=] xf. 
On définit une forme quadratique sur $E=\ob R^2$ 
en posant $q(x,y)=x^2+xy+2y^2$. \pn
a) Trouver la matrice $A$ de $q$ dans la base canonique de $\ob R^2$. \pn
b) Diagonaliser $A$ dans une base orthonormale $B$. \pn
c) Exprimer $q(x,y)$ en fonction des coordonnées $(x', y')$ du vecteur $x$ dans la base  $B$. \pn
d) Propriétés géométriques de l'ensemble $\{(x,y):q(x,y)=1\}$ ? 

\exo [Level=2,Fight=0,Learn=0,Field=\MatricesSymétriques,Type=\Exercices,Origin=] xg. 
Diagonaliser à l'aide d'une matrice orthogonale la matrice 
$\pmatrix{7&2&-2\cr2&4&-1\cr-2&-1&4\cr}$. 

\exo [Level=2,Fight=0,Learn=0,Field=\FormesQuadratiques,Type=\Exercices,Origin=] xh. 
Réduire la forme quadratique $q(x,y)=x^2-3xy+2y^2$. 

\exo [Level=2,Fight=0,Learn=0,Field=\FormesQuadratiques,Type=\Exercices,Origin=] xi. 
Réduire la forme quadratique  $q(x,y,z)=x^2+3y^2+4z^2+2xy+4yz$. 

\exo [Level=1,Fight=2,Learn=2,Field=\TrigonométrieHyperbolique,Type=\Cours,Origin=\Lakedaemon] xj. 
Pour $x\in\ob R$, on pose $\ds\sh x:={\e^x-\e^{-x}\F2}$ (sinus hyperbolique). \pn
a) Prouver que $\sh$ est une bijection continue de $\ob R$ dans $\ob R$. \pn
b) Notant $\hbox{Argsh}$ la bijection réciproque de la fonction $\sh$, 
exprimer $\hbox{argsh}\ u$ à l'aide des fonctions élémentaires ($\exp$, 
$\ln$, $\sin$, $\cos$...)
 
\exo [Level=1,Fight=2,Learn=2,Field=\TrigonométrieHyperbolique,Type=\Cours,Origin=\Lakedaemon] xk. 
Pour $x\in\ob R$, on pose $\ds\ch x:={\e^x+\e^{-x}\F2}$ (cosinus hyperbolique). \pn
a) Prouver que $\ch$ est une bijection continue de $\Q[0,+\infty\W[$ dans $\Q[1,+\infty\W[$. \pn
b) Notant $\hbox{Argch}$ la bijection réciproque de la fonction $\ch$, 
exprimer $\hbox{Argch}\ u$ à l'aide des fonctions élémentaires ($\exp$, 
$\ln$, $\sin$, $\cos$...)

\exo [Level=1,Fight=2,Learn=2,Field=\TrigonométrieHyperbolique,Type=\Cours,Origin=\Lakedaemon]  xl. 
Pour $x\in\ob R$, on pose $\ds\th x:={\e^x-\e^{-x}\F\e^x+\e^{-x}}$ (tangente hyperbolique). \pn
a) Prouver que $\th$ est une bijection continue de $\ob R$ dans $\Q]-1,1\W[$. \pn
b) Notant $\hbox{Argth}$ la bijection réciproque de la fonction $\th$, 
exprimer $\hbox{Argth}\ u$ à l'aide des fonctions élémentaires ($\exp$, 
$\ln$, $\sin$, $\cos$...)

\exo [Level=1,Fight=1,Learn=1,Field=\DéveloppementsLimités,Type=\Exercices,Origin=\MP]  xm. 
Calculer un développement limité en $+\infty$ à l'ordre $4$ de 
$u_n=\ds\arctan\Q(\sqrt{1-\cos(1/n)\F1+\cos(1/n)}\W)$. 

\exo [Level=1,Fight=1,Learn=1,Field=\EquationsDifférentiellesAVariablesSéparables,Type=\Exercices,Origin=] xn. 
a) Ensemble de définition de $f(x)=\ln\Q(x+\sqrt{1+x^2}\W)$ et calcul de sa dérivée. 
Résoudre l'équation différentielle $y'\sqrt{1+x^2}=\sqrt{1+y^2}$ sur $\ob R$.

\exo [Level=1,Fight=0,Learn=0,Field=\EquationsDifférentiellesLinéairesDuPremierOrdre,Type=\Exercices,Origin=] xo. 
Résoudre l'équation différentielle $2y'-3y=x$ sur $\ob R$. 

\exo [Level=1,Fight=0,Learn=0,Field=\EquationsDifférentiellesLinéairesDuPremierOrdre,Type=\Exercices,Origin=] xp. 
Résoudre le problème de Cauchy $y(0)=1$  et $iy'+y=\sin(x)$ sur $\ob R$. 

\exo [Level=1,Fight=0,Learn=0,Field=\EquationsDifférentiellesLinéairesDuPremierOrdre,Type=\Exercices,Origin=]xq. 
Résoudre l'équation différentielle $\cos(x)y'+\sin(x)y=\tan(x)$ sur $\ds\Q]-{\pi\F2},{\pi\F2}\W[$. 

\exo [Level=1,Fight=0,Learn=0,Field=\EquationsDifférentiellesLinéairesDuPremierOrdre,Type=\Exercices,Origin=] xr. 
Résoudre l'équation différentielle $(x^2-1)y'+xy=x^3-x$ sur $\ob R\ssm\{-1,1\}$ puis sur $\ob R$. 

\exo [Level=1,Fight=0,Learn=0,Field=\EquationsDifférentiellesLinéairesDuPremierOrdre,Type=\Exercices,Origin=] xs. 
Résoudre l'équation différentielle $(1-x^2)y'-2xy=x^2$ sur $\ob R\ssm\{-1,1\}$ 
puis sur $\ob R$. 

\exo [Level=1,Fight=0,Learn=0,Field=\EquationsDifférentiellesLinéairesDuPremierOrdre,Type=\Exercices,Origin=] xt. 
Résoudre l'équation différentielle $y'-3\tan x=-\cos^2x$ sur $\ob R$. 

\exo [Level=1,Fight=0,Learn=0,Field=\EquationsDifférentiellesLinéairesDuPremierOrdre,Type=\Exercices,Origin=] xu. 
Résoudre l'équation différentielle $y'\sin x=2y\cos x$ sur $\Q]0,\pi\W[$ 
puis sur $\ob R$. 

\exo [Level=1,Fight=0,Learn=0,Field=\EquationsDifférentiellesLinéairesDuPremierOrdre,Type=\Exercices,Origin=] xv. 
Résoudre l'équation différentielle $2x\b(1+\sqrt x\b)y''+\b(2\sqrt x+1\b)y=0$. 

\exo [Level=1,Fight=0,Learn=0,Field=\EquationsDifférentiellesLinéairesDuPremierOrdre,Type=\Exercices,Origin=] xw. 
Résoudre l'équation différentielle $y'(3x^2-2x)=y(6x-2)$.

\exo [Level=1,Fight=0,Learn=0,Field=\EquationsDifférentiellesLinéairesDuPremierOrdre,Type=\Exercices,Origin=] xx. 
Résoudre l'équation différentielle $|x|y'-y=x^2$ sur $\ob R^+$, sur $\ob R^-$ puis sur $\ob R$.

\exo [Level=1,Fight=0,Learn=0,Field=\EquationsDifférentiellesLinéairesDuPremierOrdre,Type=\Exercices,Origin=] xy. 
Résoudre l'équation différentielle $y'+y\cos x=\cos x$.

\exo [Level=1,Fight=0,Learn=0,Field=\EquationsDifférentiellesLinéairesDuPremierOrdre,Type=\Exercices,Origin=] xz. 
Résoudre l'équation différentielle $y'-y\tan x={1\F\cos^3 x}$.

\exo [Level=1,Fight=0,Learn=0,Field=\EquationsDifférentiellesLinéairesDuPremierOrdre,Type=\Exercices,Origin=] ya. 
Résoudre l'équation différentielle $3x^3\ln^2x=y'x\ln x-y$.

\exo [Level=2,Fight=0,Learn=0,Field=\EquationsDifférentiellesLinéairesDuSecondOrdre,Type=\Exercices,Origin=] yb. 
Résoudre l'équation différentielle $y''+y=\cos x$ sur $\ob R$.

\exo [Level=2,Fight=0,Learn=0,Field=\EquationsDifférentiellesLinéairesDuSecondOrdre,Type=\Exercices,Origin=] yc. 
Résoudre l'équation différentielle $y''-2y'+y=x^2\sin x$ sur $\ob R$.

\exo [Level=2,Fight=0,Learn=0,Field=\EquationsDifférentiellesLinéairesDuSecondOrdre,Type=\Exercices,Origin=] yd. 
Résoudre l'équation différentielle $y''-2y'+y=\e^x$ sur $\ob R$.

\exo [Level=2,Fight=0,Learn=0,Field=\EquationsDifférentiellesLinéairesDuSecondOrdre,Type=\Exercices,Origin=] ye. 
Résoudre l'équation différentielle $y''+y=\e^{-ix}$ sur $\ob R$.

\exo [Level=2,Fight=0,Learn=0,Field=\EquationsDifférentiellesLinéairesDuSecondOrdre,Type=\Exercices,Origin=] yf. 
Résoudre le problème de Cauchy $y(0)=1$, $y'(0)=2$ et $y''+y'+y=\cos(x)\e^x$ sur $\ob R$.

\exo [Level=2,Fight=0,Learn=0,Field=\EquationsDifférentiellesLinéairesDuSecondOrdre,Type=\Exercices,Origin=] yg. 
Résoudre l'équation différentielle $y''-3y'+2y=x^3$ sur $\ob R$.

\exo [Level=2,Fight=1,Learn=0,Field=\EquationsDifférentiellesLinéairesDuSecondOrdre,Type=\Exercices,Origin=] yh. 
Résoudre l'équation différentielle $y''+6y'+9y={\e^{-3x}\F\sqrt{1+x^2}}$ sur $\ob R$.

\exo [Level=2,Fight=0,Learn=0,Field=\EquationsDifférentiellesLinéairesDuSecondOrdre,Type=\Exercices,Origin=] yi. 
Résoudre l'équation différentielle $y''+y'+y=x^2\e^{x/2}$ sur $\ob R$.

\exo [Level=2,Fight=1,Learn=0,Field=\EquationsDifférentiellesLinéairesDuSecondOrdre,Type=\Exercices,Origin=] yj. 
Soit $\theta\in\Q]0,\pi/2\W[$. 
Résoudre l'équation différentielle 
$y''\cos^2\theta-y'\sin(2\theta)+y=x(\cos\theta)^2\e^{x\tan\theta}$. 
Déterminer la solution $\varphi$ qui s'annule ainsi 
que sa dérivèe en $0$. Calculer les primitives de $\varphi$. 

\exo [Level=2,Fight=0,Learn=0,Field=\SériesEntières,Type=\Exercices,Origin=,Indication={$f(t)={x\F1-x}$}] yk. 
Nous cherchons à résoudre $x^2(1-x)y''-x(1+x)y'+y=0$ sur $\ob R\ssm\{0,1\}$. \pn
a) Chercher une solution $f(x)=\sum_{n=0}^\infty a_nx^n$ développable en série entière sur $\Q]-R,R\W[$ avec $R>0$. 
(determiner la suite $a_n$ en fonction de $a_1$ puis en déduire $f$). \pn
b) sur quel intervalle la fonction trouvée est t-elle solution ? \pn
c) Procéder à une variation de la constante pour trouver les autres solutions. \pn
d) Existe t'il des solutions sur $\ob R$ ?

\exo [Level=2,Fight=0,Learn=0,Field=\EquationsDifférentiellesLinéairesDuSecondOrdre,Type=\Others,Origin=] yl. 
Résoudre l'équation différentielle $(6x^3+20x^2-2x)y''-(9x^2+10x-1)y'+(1+9x)y=0$ sur $\ob R$.

\exo [Level=2,Fight=0,Learn=0,Field=\EquationsDifférentiellesLinéairesDuSecondOrdre,Type=\Colles,Origin=] ym. 
Résoudre l'équation différentielle $(1-\cos 4x)y''+2y'\sin(4x)-8y=0$ sur 
$\Q]0,\pi/2\W[$ en sachant qu'elle admet deux solutions $u$ et $v$ vérifiant $uv=1$.

\exo [Level=2,Fight=0,Learn=0,Field=\SériesEntières,Type=\Exercices,Origin=] yn. 
Résoudre l'équation différentielle $(x^2-1)y''-12y=0$ sur $\ob R$.

\exo [Level=2,Fight=0,Learn=0,Field=\EquationsDifférentiellesLinéairesDuSecondOrdre,Type=\Exercices,Origin=] yo. 
Résoudre l'équation différentielle $4xy''+2y'+y=0$.

\exo [Level=2,Fight=1,Learn=1,Field=\EquationsDifférentiellesLinéairesDuSecondOrdre,Type=\Exercices,Origin=] yp. 
Résoudre l'équation différentielle $(x^2-1)y''+xy'-y=0$.

\exo [Level=2,Fight=1,Learn=0,Field=\EquationsDifférentiellesLinéairesDuSecondOrdre,Type=\Exercices,Origin=] yq. 
a) Chercher les solutions communes à $(H)\quad (1+\cos x)y''+\sin x y'+y=0$ 
et $(G)\quad y''+y=0$. \pn
b) Chercher les solutions communnes à $(E)\quad (1+\cos x)y''+\sin x y'+y=1$ et $(G)$. \pn
c) Résoudre $(E)$. 

\exo [Level=2,Fight=2,Learn=2,Field=\EquationsDifférentiellesLinéairesDuSecondOrdre,Type=\Exercices,Origin=,Indication={Procéder au changement de variable $x=\tan t$.}] yr. 
Résoudre l'équation différentielle $(1+x^2)^2y''+2x(1+x^2)y'+my=0$. \pn 


\exo [Level=1,Fight=1,Learn=1,Field=\EquationsDifférentiellesAVariablesSéparables,Type=\Exercices,Origin=] ys. 
Résoudre l'équation différentielle $y'\sqrt{1-x^2}=\sqrt{1-y^2}$ sur $\Q]-1,1\W[$.

\exo [Level=1,Fight=1,Learn=1,Field=\EquationsDifférentiellesAVariablesSéparables,Type=\Exercices,Origin=] yt. 
Résoudre l'équation différentielle $2x^2y'+y^2=1$ sur $\ob R$.

\exo [Level=1,Fight=2,Learn=2,Field=\EquationsDifférentiellesLinéairesDuPremierOrdre,Type=\Exercices,Origin=,Indication={Procéder au changement de fonction inconnue $y=xz$.}] yu. 
Résoudre l'équation différentielle $(x-y)+xy'=0$ sur $\ob R$. 


\exo [Level=1,Fight=2,Learn=2,Field=\EquationsDifférentiellesLinéairesDuPremierOrdre,Type=\Exercices,Origin=,Indication={Procéder au changement de fonction inconnue $y=xz$.}] yv. 
Résoudre l'équation différentielle $x^2y'=y^2-xy+x^2$. 


\exo [Level=1,Fight=2,Learn=2,Field=\EquationsDifférentiellesLinéairesDuPremierOrdre,Type=\Exercices,Origin=,Indication={Procéder au changement de fonction inconnue $y=xz$.}] yw. 
Résoudre l'équation différentielle $xy'-y=\sqrt{y^2+x^2}$. 


\exo [Level=1,Fight=2,Learn=2,Field=\EquationsDifférentiellesLinéairesDuPremierOrdre,Type=\Exercices,Origin=,Indication={Procéder au changement de fonction inconnue $z=y^{-2}$.}] yx. 
Résoudre l'équation différentielle de Bernouilli 
$2y'+{xy\F x^2-1}-xy^3=0$. 


\exo [Level=1,Fight=2,Learn=2,Field=\EquationsDifférentiellesLinéairesDuPremierOrdre,Type=\Exercices,Origin=,Indication={Procéder au changement de fonction inconnue $z=y^{-3}$.}] yy. 
Résoudre l'équation différentielle de Bernouilli 
$5y'-y\sin x+y^4\sin x=0$. 


% Redondant % yz

\exo [Level=2,Fight=2,Learn=2,Field=\EquationsDifférentiellesLinéairesDuSecondOrdre,Type=\Exercices,Origin=,Indication={Procéder au changement de variable $x=\e^t$.}] za. 
Résoudre l'équation différentielle d'Euler $x^2y''+3xy'+y=\ln x$. 


\exo [Level=1,Fight=3,Learn=1,Field=\EquationsDifférentiellesLinéairesDuPremierOrdre,Type=\Exercices,Origin=] zb. 
Soit $a:\ob R\to\ob R$ une fonction continue et $T$-périodique. Pour chaque solution $y$ de $y'+a(x)y=0$ sur $\ob R$, 
démontrer qu'il existe un unique $\alpha\in\ob R$ tel que $x\mapsto y(x)\e^{-\alpha x}$ 
soit périodique.  

\exo [Level=2,Fight=2,Learn=1,Field=\SystèmesDifférentiels,Type=\Exercices,Origin=] zc. 
Résoudre l'équation différentielle $y'''+y''-y'-y=x\e^{-x}$ sur $\ob R$. 

\exo [Level=1,Fight=3,Learn=2,Field=\EquationsDifférentiellesLinéairesDuPremierOrdre,Type=\Exercices,Origin=] zd. 
Soit $f\in\sc C^1(\ob R,\ob R)$ telle que l'application $\phi$ définie par 
$\phi(x)=2xf'(x)-f(x)$ sur $\ob R$ soit paire. Prouver que $f$ est paire. 

\exo [Level=2,Fight=2,Learn=1,Field=\EquationsDifférentiellesLinéairesDuSecondOrdre,Type=\Exercices,Origin=] ze. 
Pour $x>0$, on pose $\ds v(x)={1\F4x}$. Déterminer toutes les fonctions 
$f\in\sc C^1\Q(\Q]0,\infty\W[\W)$ telles que $f'(x)=f\b(v(x)\b)$ pour $x>0$. (on pourra déterminer $v\circ v$). 

\exo [Level=1,Fight=0,Learn=0,Field=\EquationsDifférentiellesLinéairesDuPremierOrdre,Type=\Exercices,Origin=] zf. 
Résoudre sur $\Q]-\pi/2,\pi/2\W[$ le problème différentiel 
$\Q\{\eqalign{&y'\cos x+y\sin x=1\cr y(0)=2\cr}\W.$

\exo [Level=1,Fight=0,Learn=0,Field=\EquationsDifférentiellesLinéairesDuPremierOrdre,Type=\Exercices,Origin=] zg. 
Résoudre l'équation différentielle $y'-2y=\sin(2x)\e^x$. 

\exo [Level=2,Fight=0,Learn=0,Field=\EquationsDifférentiellesLinéairesDuSecondOrdre,Type=\Exercices,Origin=] zh. 
Résoudre sur $\ob R$ l'équation différentielle $y''-3y'+2y=\sh(2x)$ 
et en déterminer les solutions vérifiant $y(0)=1$ et $y'(0)=-1$. 

\exo [Level=2,Fight=1,Learn=0,Field=\EquationsDifférentiellesLinéairesDuSecondOrdre,Type=\Exercices,Origin=] zi. 
Résoudre sur $\ob R$ l'équation différentielle $y''+4y'+4y={\e^{-2x\F1+x^2}}$. 

\exo [Level=2,Fight=1,Learn=0,Field=\EquationsDifférentiellesLinéairesDuSecondOrdre,Type=\Exercices,Origin=] zj. 
Résoudre sur $\Q]0,\infty\W[$ et sur $\Q]-\infty,0\W[$ l'équation différentielle $x^2y''+axy'+by=0$ dans les cas suivants :\pn 
a) $a=2$ et $b=-6$. \pn
b) $a=-1$ et $b=1$. \pn
c) $a=7$ et $b=25$. 

\exo [Level=2,Fight=2,Learn=2,Field=\EquationsDifférentiellesLinéairesDuSecondOrdre,Type=\Exercices,Origin=] zk. 
On considère l'équation différentielle $(E)\quad(2x+1)y''+(4x-2)y'-8y=0$. \pn
a) Déterminer les polynômes solutions de $(E)$ sur $\ob R$. \pn
b) Déterminer les solutions de $(E)$ de la forme $x\mapsto \e^{\alpha x}$ sur $\ob R$. \pn
c) Résoudre $(E)$ sur un intervalle ne contenant pas $-1/2$. 

\exo [Level=2,Fight=1,Learn=0,Field=\EquationsDifférentiellesLinéairesDuSecondOrdre,Type=\Exercices,Origin=] zl. 
Résoudre l'équation différentielle $(1+x)y''-2y'+(1-x)y=x\e^x$ 
sur un intervalle ou $1+x$ ne s'annule pas.  

\exo [Level=2,Fight=2,Learn=2,Field=\EquationsDifférentiellesLinéairesDuSecondOrdre,Type=\Exercices,Origin=] zm. 
Résoudre $(1-x^2)y''-xy'+y=0$ sur $\Q]-1,1\W[$ via le changement de variable $x=\sin t$. 

\exo [Level=2,Fight=0,Learn=0,Field=\SériesEntières,Type=\Exercices,Origin=] zn. 
On considère l'équation différentielle $(E)\quad 4xy''+2y'-y=0$. \pn
a) Déterminer les solutions de $(E)$ développables en série entière. \pn
b) Ensemble solution de $(E)$. 

\exo [Level=1,Fight=0,Learn=0,Field=\EquationsDifférentiellesAVariablesSéparables,Type=\Exercices,Origin=] zo. 
Résoudre l'équation différentielle $(E)\quad y'=\e^{x+y}$. 
Préciser la solution de $(E)$ vérifiant $y(0)=0$. 

\exo [Level=2,Fight=1,Learn=1,Field=\SystèmesDifférentiels,Type=\Exercices,Origin=] zp. 
Montrer sans résoudre le système 
que les trajectoires du système différentiel 
$$
\leqno{S}\Q\{\eqalign{x'&=4x-3y+2z\cr y'&=6x-5y+4z\cr z'&=4x-4y+4z\cr}\W.
$$
sont planes. 

\exo [Level=2,Fight=1,Learn=1,Field=\SystèmesDifférentiels,Type=\Exercices,Origin=] zq. 
Résoudre sur $\ob R$ le système différentiel 
$\ds\Q\{\eqalign{x'=x-y-z+t\cr y'=-x+y-z+t\cr z'=-x-y+z+t\cr}\W.$

\exo [Level=2,Fight=2,Learn=1,Field=\SystèmesDifférentiels,Type=\Exercices,Origin=] zr. 
On considère le système différentiel $S:\quad\Q\{\eqalign{x'=x(1-x)\cr
y'=y(x-1)}\W.$. \pn
Pour toute solution $X=(x,y)$ de $\sc C^1(\ob R,\ob R^2)$ de ce système, 
on considère la trajectoire associée, c'est à dire l'arc paramétré par le couple $(x,y)$. 
\pn
1) Pour $X=(x,y)\in\sc C^1(\ob R,\ob R^2)$ et $a\in\ob R$, on pose $X_a=(x_a,y_a)$ ou $X_a(t)=X(a+t)$ pour $t\in\ob R$. \pn
Montrer que $X$ est solution de $S$ sur $\ob R$ si, et seulement si, 
$X_a$ est solution de $S$ sur $\ob R$ pour $a\in\ob R$. \pn
Montrer alors que toutes les trajectoires associées aux $X_a$ ont le même support. \pn
2) Déterminer les points du plan qui sont des points stationnaires de trajectoires 
du système (les points critiques du système) des ``points d'équilibres''. \pn
3) Montrer que les trajectoires sont des courbes intégrales de l'équation différentielle
$$
w(x,y):=y(x-1)\d x+x(y-1)\d y=0\leqno{(E)}
$$
vérifier que $w$ n'est pas une forme différentielle exacte. \pn
Déterminer une fonction $\varphi$ de classe $\sc C^1$ sur un intervalle $I$ 
à déterminer 
telle que la forme différentielle $w_1(x,y)=\varphi(xy)w(x,y)$ soit exacte sur des ouverts de $\ob R^2$ à déterminer. \pn
4) Déterminer des équations des supports des trajectoires du système $(S)$ correspondant à l'ouvert $U_1=(\ob R_+^*)^2$. 
Représenter celle qui passe par le point $(1,2)$. 


\exo [Level=2,Fight=1,Learn=1,Field=\SystèmesDifférentiels,Type=\Exercices,Origin=] zs. 
On considère l'équation différentielle $(E)\quad (x+2y)y'=-\sqrt{1-y^2}$. \pn
Montrer que les trajectoires du système différentiel autonome
$$
\Q\{\eqalign{x'=x+2y\cr
y'=-\sqrt{1-y^2}\cr}
\W.
$$
sont courbes intégrales de cette équation différentielle. \pn
2) Résoudre ce système et exprimer $x$ en fonction de $y$ pour ces solutions. 

\exo [Level=2,Fight=0,Learn=0,Field=\SystèmesDifférentiels,Type=\Exercices,Origin=] zt. 
Résoudre pour $x(0)=-1$, $y(0)=2$ et $z(0)=0$ 
le problème différentiel suivant 
$$
\leqno{(S)}\Q\{\eqalign{x'=5x+y-z\cr y'=2x+4y-2z\cr z'=x-y+3z\cr}\W.
$$

\exo [Level=2,Fight=1,Learn=1,Field=\SystèmesDifférentiels,Type=\Exercices,Origin=] zu. 
Vérifier que les arcs $\gamma_a$ paramétrés par une solution 
$(x(t),y(t))$ du système différentiel 
$$
\Q\{\eqalign{x'=x\cr y'=3x+2y}\W.
$$
vérifiant $x(a)=1$ et $y(a)=2$ sont contenus dans une parabole. 

\exo [Level=2,Fight=0,Learn=0,Field=\SystèmesDifférentiels,Type=\Exercices,Origin=] zv. 
Déterminer les couples $(x,y)$ de fonctions de classe $\sc C^1$ solutions sur $\ob R$ 
du système différentiel  
$$
(S)\qquad\Q\{\eqalign{x'=3x-2y\cr y'=2x-y\cr}\W.
$$

\exo [Level=2,Fight=0,Learn=0,Field=\SystèmesDifférentiels,Type=\Exercices,Origin=] zw. 
Déterminer les couples $(x,y)$ de fonctions de classe $\sc C^1$ solutions sur $\ob R$ 
du système différentiel  
$$
(S)\qquad\Q\{\eqalign{x'=3x-5y\cr y'=x-y}\W.
$$

\exo [Level=2,Fight=0,Learn=0,Field=\SystèmesDifférentiels,Type=\Exercices,Origin=] zx. 
Déterminer les couples $(x,y)$ de fonctions de classe $\sc C^1$ solutions sur $\ob R$ 
du système différentiel  
$$
(S)\qquad\Q\{\eqalign{x'=-2x+4y+2\cr y'=-3x+5y+\e^t}\W.
$$

\exo [Level=2,Fight=0,Learn=0,Field=\SystèmesDifférentiels,Type=\Exercices,Origin=] zy. 
Déterminer le couple $(x,y)$ de fonctions de classe $\sc C^1$ vérifiant $x(0)=1=-y(0)$ 
et solutions sur $\ob R$ du système différentiel  
$$
(S)\qquad\Q\{\eqalign{x'=-2x+4y\cr y'=-3x+5y}\W.
$$

\exo [Level=2,Fight=2,Learn=1,Field=\SystèmesDifférentiels,Type=\Exercices,Origin=]  zz. 
On cherche les courbes intégrales de l'équation différentielle
$$
(x+y^2)y'=y\leqno{(E)}
$$
que l'on, écrit sous la forme $y\d x-(x+y^2)\d y=0$. \pn
1) vérifier que $w:(x,y)\mapsto y\d x-(x+y^2)\d y$ n'est la différentielle d'aucune fonction de classe $\sc C^2$. 
\pn
2) Déterminer la différentielle de la fonction $\varphi(x,y)={x\F y}$ 
définie sur $U=\ob R\times\ob R^*$. \pn
3) En déduire une fonction $V$ de classe $\sc C^1$ sur $U$ telle que 
pour tout $(x,y)\in U$ on ait 
$$
w(x,y)=0\Leftrightarrow \d V_{(x,y)}=0.
$$ 
4) Déterminer les courbes intégrales de $(E)$. Donner par exemple celle qui passe par $(1,1)$. 

\exo [Level=2,Fight=0,Learn=0,Field=\Enveloppes,Type=\Exercices,Origin=,Indication={choisir un bon repère et paramétrer.}] aaa. 
Dans le plan $\sc P$ euclidien, 
on se donne une droite fixe $D$ et un point fixe $O\notin D$. \pn Pour~$P\in D$, 
on note $\Delta_P$ la droite perpendiculaire à $(OP)$ en $P$. \pn
Enveloppe de la famille de droite $(\Delta_P)_{P\in D}$ ? 


\exo [Level=2,Fight=0,Learn=0,Field=\Enveloppes,Type=\Exercices,Origin=] aab. 
Soit $\sc H$ la courbe du plan d'équation cartésienne $xy=1$ et soient $A$ et $B$ deux points de cette courbe 
tels que l'abscisse du point $A$ soit deux fois celle du point $b$. Déterminer l'enveloppe de la droite $(AB)$. 

\exo [Level=2,Fight=0,Learn=0,Field=\Enveloppes,Type=\Exercices,Origin=] aac. 
a) Déterminer l'enveloppe de la famille de droites 
d'équation $D_t: x\sin t-y\cos t=\sin^2 t$. \pn
b) Symétries de la courbe, tableau de variation de $x$, $y$, 
étude des points singuliers...

\exo [Level=1,Fight=4,Learn=3,Field=\ThéorèmeDeRolle,Type=\Exercices,Origin=\MP] aad. 
Soient $n\ge1$ et $f\in\sc C^n\b([0,1]\b)$ tels que $f(k/n)=0$ pour $0\le k\le n$. \pn
a) prouver qu'il existe $c\in\Q]0,1\W[$ tel que $f^{(n)}(c)=0$. \pn
b) S'il existe un entier $\ell\ge1$ tel que $f\in\sc C^{n+\ell}\b([0,1]\b)$ et tel que $f^{(k)}(0)=0$ 
pour $0\le k\le \ell$, \pn prouver qu'il existe $d\in\Q]0,1\W[$ tel que $f^{(n+\ell)}(d)=0$. 


\exo [Level=1,Fight=2,Learn=2,Field=\ThéorèmeDeRolle,Type=\Exercices,Origin=\MP] aae. 
Soit $f\in\sc C\b(\Q[0,+\infty\W[,\ob R\b)$ telle que $\ds\lim_{x\to\infty}f(x)=f(0)$. 
Prouver qu'il existe $c\in\Q]0,\infty\W[$ tel que $f'(c)=0$. 

\exo [Level=2,Fight=1,Learn=0,Field=\Enveloppes,Type=\Exercices,Origin=,Indication={utiliser les symétries et paramétrer.}] aaf. 
Soient deux droites secantes $D,\Delta$ du plan $\sc P$, 
$O$ leur intersection et $\sc A>0$. Déterminer l'enveloppe des droites $(A,B)$ 
pour lesquelles $A\in D$, $B\in\Delta$ et l'aire du triangle $OAB$ vaut $\sc A$ 
et en donner une équation cartésienne. 


\exo [Level=2,Fight=2,Learn=2,Field=\Courbure|\RepèreDeFrenet,Type=\Exercices,Origin=] aag. 
Pour $t\in\ob R$, on note $\Gamma$ la courbe paramétrée par 
$x(t):=t-\sin t$ et $y(t):=1-\cos t$ pour $t\in\ob R$. \pn
a) Étudier et tracer $\Gamma$. \pn
b) Déterminer le repère de Frenet et la courbure en tout point de $\Gamma$. \pn
c) Calculer la longueur d'une arche et l'aire délimitée par une arche de $\Gamma$ 
et l'axe $(Ox)$. 

\exo [Level=2,Fight=1,Learn=1,Field=\RepèreDeFrenet,Type=\Exercices,Origin=] aah. 
Soit $\sc E$ l'ellipse d'équation cartésienne $x^2+5y^2=5$ 
dans $\ob R^2$. \pn
a) Déterminer l'ensemble des centres $C$ des cercles $\sc C$ passant par $O=(0,0)$ 
et tangents à $\sc E$. \pn
b) Etude de la courbe obtenue. 

\exo [Level=2,Fight=0,Learn=0,Field=\AbscisseCurviligne,Type=\Exercices,Origin=] aai. 
On note $\Gamma$ la courbe d'équation cartésienne dans $\Q]0,\infty\W[^3$ 
$$
\Q\{\eqalign{xy=1\cr z=2\ln x}\W. 
$$
Calculer l'abscisse curviligne en tout point de $\Gamma$ en prenant  
pour origine le point $M$ d'abscisse~$1$. 

\exo [Level=2,Fight=1,Learn=0,Field=\Courbure|\AbscisseCurviligne,Type=\Exercices,Origin=] aaj. 
Déterminer les coordonnées du centre de courbure en chaque point 
de l'arc paramétré en polaire par $\rho(\theta)=\e^\theta$ pour $\theta\in\ob R$. 
Quelle est la longueur  de la courbe pour $\theta\in\Q]-\infty,0\W]$ ? 

\exo [Level=2,Fight=1,Learn=0,Field=\Courbure|\AbscisseCurviligne,Type=\Exercices,Origin=] aak. 
Déterminer les coordonnées du centre de courbure en chaque point 
de l'arc paramétré en polaire par $\rho(\theta)=1+\cos\theta$ pour $\theta\in\ob R$. 
Quelle est la longueur  de la courbe pour $\theta\in[0,2\pi]$ ? 

\exo [Level=2,Fight=1,Learn=0,Field=\Coniques|GéométriePlane,Type=\Exercices,Origin=] aal. 
Déterminer l'ensemble des points d'une plaque carrée de coté $a$ 
qui sont à égale distance du bord et du centre de la plaque. 

\exo [Level=2,Fight=1,Learn=1,Field=\RepèreDeFrenet|\AbscisseCurviligne,Type=\Exercices,Origin=] aam. 
On considère l'arc paramétré par $M_t=(\cos 2t, \sqrt 2\sin 2t, \cos 2t)$ pour $0\le t\le \pi/2$. \pn
a) Longueur $\ell$ de l'arc ?\pn
b) Trouver le repère de Frenet $(\vec T,\vec N,\vec B)$ en un point $M_t$. \pn
c) Quel est le plan osculateur en $s=\ell/4$. \pn
d) Soit $p$ le point $(a,0,-a)$. Calculer la distance entre $p$ et 
la droite tangente à l'arc au point $M_{t}$. 
 
\exo [Level=2,Fight=1,Learn=2,Field=\Développantes,Type=\Exercices,Origin=] aan. Trouver le lieu des centres de courbure 
de l'ellipse $\ds{x^2\F a^2}+{y^2\F b^2}=1$. 

\exo [Level=2,Fight=1,Learn=0,Field=\Enveloppes,Type=\Exercices,Origin=]  aao. 
Soient $\sc D_1$ et $\sc D_2$ deux droites orthogonales 
se coupant en un point $O$. Pour~$\ell>0$, 
déterminer l'enveloppe des droites $(AB)$ lorsque $\sc A\in\sc D_1$ et $B\in\sc D_2$ vérifient $OA+OB=\ell$.

\exo [Level=2,Fight=1,Learn=0,Field=\Enveloppes,Type=\Exercices,Origin=] aap. 
Soient $\sc D_1$ et $\sc D_2$ deux droites orthogonales 
se coupant en un point $O$. 
Pour~$\ell>0$, 
déterminer l'enveloppe des droites $(A,B)$ lorsque $\sc A\in\sc D_1$ et $B\in\sc D_2$ vérifient $OA^2+OB^2=\ell^2$.
 
\exo [Level=2,Fight=1,Learn=0,Field=\Enveloppes,Type=\Exercices,Origin=] aaq. 
Soient $A,B$ deux points du plan, $\sc C$ le cercle de diamètre $[AB]$, $\sc T_A$ 
la tangente à $\sc C$ en $A$ et $\sc T_B$ la tangente à $B$ en $B$. 
Pour chaque point $M$ de $\sc C$, on note respectivement $C$ 
l'intersection de $(AM)$ et $\sc T_B$ et $D$ l'intersection de $(BM)$ et $\sc T_A$. 
Déterminer l'enveloppe des droites $(CD)$ ainsi formées. 

\exo [Level=2,Fight=1,Learn=0,Field=\Enveloppes,Type=\Exercices,Origin=] aar. 
Soient $A$, $B$ deux points distincts du plan et $\ell>\|AB\|$. 
Déterminer l'enveloppe des droites $\sc D$ telles que 
$\rm d(A,\sc D)^2+\rm d(B,\sc D)^2=\ell$. 

\exo [Level=2,Fight=2,Learn=1,Field=\Enveloppes,Type=\Exercices,Origin=] aas. 
Soient $\sc E$ une ellipse de grand axe $a$, de petit axe $b$ et de centre $O$. \pn
a) trouver un paramétrage simple $t\mapsto M(t)$ de $\sc E$ et trouver une relation entre $t$ et $t'$ pour que 
$\vec{OM(t)}$ et $\vec{OM'(t)}$ soient orthogonaux. \pn
b) Déterminer l'enveloppe des droites $\b(M(t)M(t')\b)$ vérifiant la condition de la question a)

\exo [Level=2,Fight=2,Learn=1,Field=\Enveloppes,Type=\Exercices,Origin=] aat. 
Soit $P$ la parabole d'équation réduite $y^2=2px$ 
dans un repère orthonormé $(O,\vec i,\vec j)$, soit $c<0$ 
et soit $\Delta$ la droite d'équation $x=c$. 
Pour $t\in\ob R$, on note $M(t)$ l'unique point de $P$ d'ordonnée $t$. \pn
a) Trouver une équation de la droite $\sc D_t$ passant par $M(t)$ qui coupe $P$ 
en un autre point $M(t_1)$ tel que les tangentes à $P$ en $M(t)$ et $M(t_1)$ se coupent sur $\Delta$. 
\pn
b) Quelle est l'enveloppe des droites $\sc D_t$ ainsi obtenues ? 

\exo [Level=2,Fight=2,Learn=1,Field=\RepèreDeFrenet,Type=\Exercices,Origin=] aau. 
Soit $\sc C$ une courbe d'équation $y=f(x)$ 
dans le repère orthonormé $(O,\vec i,\vec j)$, $f:I\to \ob R$ 
étant de classe $\sc C^2$. Pour $M\in\sc C$, on note $H$ la projection orthogonale de $M$ sur $Oy$ 
et $N$ l'intersection de la normale à $\sc C$ avec $(Oy)$. \pn
a) Trouver les fonctions $f$ pour lesquelles la mesure algèbrique $\ol{HN}$ reste égale à $k\in\ob R$ lorsque $M$ décrit $\sc C$. \pn
b) Préciser la nature des courbes $\sc C$ ainsi obtenues. 

\exo [Level=2,Fight=2,Learn=1,Field=\Hélices|\Enveloppes,Type=\Exercices,Origin=] aav. 
Soit $\sc C$ la courbe paramétrée par 
$$
\Q\{\eqalign{x(t)&=t\cos(\ln t)\cr y(t)&=t\sin(\ln t)\cr z(t)&=t}\W. \qquad(t>0).\leqno{(\sc C)}
$$
a) cette courbe est-elle une hélice ? si oui, préciser son axe et son angle. \pn
b) Rectifier cette courbe et déterminer la courbe décrite par son centre de courbure. \pn
c) Calculer la torsion en un point de paramètre $t$. \pn
d) Déterminer l'intersection $\Delta_t$ du plan $Oxy$ et du plan osculateur au point de paramètre $t$. \pn
e) Déterminer l'enveloppe de la famille des $\Delta_t$. 

\exo [Level=2,Fight=2,Learn=1,Field=\Hélices,Type=\Exercices,Origin=] aaw. 
Soit $\sc C$ la courbe de $\ob R^3$ paramètrée par 
$$
\Q\{\eqalign{x(t)=\cos(t)^2\cos(2t)\cr
y(t)=\cos(t)^2\sin(2t)\cr
z(t)=2\sin(t)\cr}\W.\leqno{(\sc C)}
$$
a) Quelles sont les coordonnées cylindriques du point de paramètre $t$ ? \pn
b) Déterminer en ce point la courbure et la torsion. \pn
c) La courbe $\sc C$ est-elle une hélice ? Si oui, quel est son axe et son angle 
(entre la tangente et la direction fixe) ?

\exo [Level=2,Fight=0,Learn=0,Field=\EquationsDifférentiellesLinéairesDuSecondOrdre,Type=\Maple,Origin=] aax. 
On considère l'équation différentielle $y''-2y'+y=x\e^{-x}\cos x$. \pn
a) Résoudre cette équation différentielle {\it utiliser \rm dsolve}. \pn
b) Determiner la solution vérifiant les conditions initiales $y(0)=0$ et $y'(0)=1$. 
\pn
c) Vérifier que la solution $f$ obtenue est bien solution de l'équation. 
\pn
d) tracer le graphe de $f$ sur $[-3,2]$. 

\exo [Level=2,Fight=0,Learn=0,Field=\EquationsDifférentiellesLinéairesDuSecondOrdre,Type=\Maple,Origin=] aay. 
Soit $E$ l'équation différentielle $|x|y'+(x-1)y=x^3$. \pn
a) Résoudre l'équation différentielle $E$ sur $\Q]-\infty,0\W[$ et sur $\Q]0,\infty\W[$.  \pn
b) Trouver les fonctions $f\in\sc C^1(\ob R)$ vérifiant $E$ sur $\ob R$. \pn
c) Tracer les courbes de diverses solutions sur $\Q[-2,-0.1\W[$ sur le même graphique. \pn
d) Même question sur $[0.1,2]$ et  sur $[-1,1]$.... 


\exo [Level=2,Fight=2,Learn=2,Field=\Enveloppes,Type=\Maple,Origin=] aaz. 
a) Programmer une fonction 
$Enveloppe[equation\_\ ,parametre\_\ ,inconnues\_\ ]$ 
retournant le paramètrage d'une enveloppe de famille de droites. \pn
{\it On rappelle que l'enveloppe de la famille $\Delta_t:a(t)x+b(t)y=c(t)$ 
est ``l'unique solution $(x,y)$ du système formé des deux équations $\Delta_t$ et 
$\ds {\partial \F\partial t}\Delta_t$'', $t$ étant le paramètre et $(x,y)$ les inconnues...}
\pn
b) Application : Tracer le graphe de l'enveloppe de la famille de droites 
$$
x\cos t+y\sin t=\sin 3t.
$$
c) Graphe et points stationnaires (singuliers) de l'enveloppe de droites
$$
x\sin t-y\cos^2t+\sin^2t=0.
$$

\exo [Level=2,Fight=1,Learn=1,Field=\Développées,Type=\Maple,Origin=] aba. 
Soit $f(t)=\b(x(t),y(t)\b)$ la courbe paramètrée définie par  
$$
\Q\{
\eqalign{x(t)={\cos t\F\ch t+\cos t}\cr
y(t)={\sin t\F\ch t+\cos t}\cr}
\W.
$$ 
a) Calculer $\ds\vec T:={f'(t)\F\|f'(t)\|}$ et en déduire l'unique vecteur $\vec N$ tel que $(\vec T,\vec N)$ forme une base orthonormée directe. \pn
b) Calculer la courbure $\gamma$ au point $t$ à l'aide de l'une des formules 
$$
{\d \vec T\F \d t}=\|f'(t)\|\gamma\vec N\qquad\hbox{ou}\qquad 
\gamma={\det\b(f'(t),f''(t)\b)\F\|f'(t)\|^3}\qquad\hbox{ou}\qquad \gamma={f''(t).\vec N\F\|f'(t)\|^2}. 
$$ 
c) en déduire le rayon de courbure $r_c:=1/\gamma$ au point $t$. \pn
d) On note $M$ le point de coordonnées $x(t)$ et $y(t)$ et on appelle  centre de courbure 
en $t$ le point $C_c$ défini par 
$$
\vec{MC_c}=r_c\vec N.
$$ 
Tracer sur le même graphique la courbe $t\mapsto f(t)$ et la courbe $t\mapsto C_c(t)$. \pn

\exo [Level=2,Fight=1,Learn=1,Field=\FormesQuadratiques,Type=\Maple,Origin=] abb. 
Réduire dans une base orthonormale (diagonaliser...) la forme quadratique
$$
q(x,y,z)=x^2-2xy+2xz+2yz
$$

\exo [Level=2,Fight=1,Learn=1,Field=\SystèmesDifférentiels,Type=\Maple,Origin=] abc. 
Résoudre le système
$$\Q\{\eqalign{
x''+\omega^2x-k\omega^2y=0\cr
y''+\omega y^2-k\omega^2x=0
}\W.
$$
pour la condition initiale $x(0)=x'(0)=y(0)=0$ et $y'(0)=1$ 
et tracer les graphes pour $k=1$ et $\omega=1$. 

\exo [Level=1,Fight=1,Learn=1,Field=\Fonctions,Type=\Colles,Origin=] abd. 
Étudier la fonction de $\ds f:x\mapsto\arctan\sqrt{1-x\F1+x}$ 
et en déduire une expression simplifiée. 

\exo [Level=1,Fight=1,Learn=1,Field=\Fonctions,Type=\Colles,Origin=] abe. 
Montrer que l'application $\ds x\mapsto {1\F x}-{1\F\sin x}$ est prolongeable 
en une fonction de classe $\sc C^1$ sur $\Q]-\pi,\pi\W[$. 

\exo [Level=1,Fight=2,Learn=1,Field=\Fonctions,Type=\Colles,Origin=] abf. 
En étudiant une fonction, trouver tous les couples $(x,y)\in(\ob N^*)^2$ tels que $x^y=y^x$. 

\exo [Level=1,Fight=1,Learn=1,Field=\Fonctions,Type=\Colles,Origin=] abg. 
Étudier les fonctions $\ds x\mapsto\arcsin{2\sqrt x\F1+x}$ et $\ds x\mapsto 2\arctan\sqrt x$. 

\exo [Level=1,Fight=1,Learn=1,Field=\Fonctions,Type=\Colles,Origin=] abh. 
On pose $\ds \hbox{Argth}(u)={1\F2}\ln{1+u\F1-u}$ pour $-1<u<1$ et 
$\ds \hbox{Argch}\ u=\ln\Q(u+\sqrt{u^2-1}\W)$ pour $u\ge1$. \pn
Étudier les fonctions $\ds f:x\mapsto\hbox{Argth}\Q({3x+x^3\F1+3x^2}\W)$ 
et $\ds g:x\mapsto\hbox{Argch}\b(4x^3-3x\b)$. 

\exo [Level=1,Fight=1,Learn=1,Field=\Fonctions,Type=\Colles,Origin=] abi. 
On pose $\ds \hbox{th}\ x={\e^x-\e^{-x}\F\e^x+\e^{-x}}$ pour $x\in\ob R$ 
et $\ds \hbox{Argth}(u)={1\F2}\ln\Q({1+u\F1-u}\W)$ pour $-1<u<1$. \pn Étudier la fonction 
$\ds f:x\mapsto \hbox{Argth}\Q({1+3\th x\F3+\th x}\W)$. 

\exo [Level=2,Fight=3,Learn=2,Field=\Développées,Type=\Exercices,Origin=]  abj. 
On dit qu'une courbe $\sc C$ est une cardioïde si elle admet 
dans un repère orthonormé une équation polaire du type $r=a(1+\cos \theta)$ ou $a>0$. 
\pn
a) Montrer que la développée d'une cardioïde est une cardioïde. \pn
b) Quelle transformation géométrique simple permet de passer de la courbe à sa développée ?\pn
c) Qu'en déduisez vous pour les développantes d'une cardioïde ? 

\exo [Level=2,Fight=2,Learn=1,Field=\Roulement,Type=\Exercices,Origin=] abk. 
Un cercle de rayon $R$ roule sans glisser sur 
une parabole d'équation $y=2px^2$. 
Trajectoire d'un point du cercle ?

\exo [Level=2,Fight=1,Learn=1,Field=\Développantes,Type=\Exercices,Origin=] abl. 
Soit $\sc C$ la courbe paramétrée par $$
\Q\{\eqalign{x&=3t^2\cr y&=2t^3\cr}\W.\qquad(t\in\ob R).
$$
Trouver une représentation paramétrique de la développante de $\sc C$ 
qui passe par $A$ de coordonnées $(a,b)$. Y a t'il des points $A$ interdits ?

\exo [Level=1,Fight=1,Learn=2,Field=\CourbesParamétréesPolaires,Type=\Exercices,Origin=] abm. 
Étudier et tracer la courbe d'équation polaire 
$$
r=\tan\theta+\tan{\theta\F2}
$$ 
en précisant l'allure  locale en chaque point singulier, les branches infinies 
et les éventuels points d'inflexions ainsi que les points doubles. 

\exo [Level=1,Fight=1,Learn=2,Field=\CourbesParamétréesPolaires,Type=\Exercices,Origin=] abn. 
Étudier et tracer la courbe d'équation polaire 
$$
r={\cos(\theta/2)\F\sqrt{1-\sin\theta}}
$$ 
en précisant l'allure  locale en chaque point singulier, les branches infinies 
et les éventuels points d'inflexions ainsi que les points doubles. 

\exo [Level=1,Fight=1,Learn=2,Field=\CourbesParamétréesPolaires,Type=\Exercices,Origin=] abo. 
Étudier et tracer la courbe d'équation polaire 
$$
r={\cos(2\theta)\F2\cos\theta-1}
$$ 
en précisant l'allure  locale en chaque point singulier, les branches infinies 
et les éventuels points d'inflexions ainsi que les points doubles. 

\exo [Level=1,Fight=1,Learn=2,Field=\CourbesParamétréesPolaires,Type=\Exercices,Origin=] abp. 
Étudier et tracer la courbe d'équation polaire 
$$
r=\sqrt{1-\cos(2\theta)}+\sqrt{1-\sin(2\theta)}
$$ 
en précisant l'allure  locale en chaque point singulier, les branches infinies 
et les éventuels points d'inflexions ainsi que les points doubles. 

\exo [Level=1,Fight=1,Learn=2,Field=\CourbesParamétréesPolaires,Type=\Exercices,Origin=] abq. 
Étudier et tracer la courbe d'équation polaire 
$$
r={1+\cos\theta\F\sin^2\theta}
$$ 
en précisant l'allure  locale en chaque point singulier, les branches infinies 
et les éventuels points d'inflexions ainsi que les points doubles. 


\exo [Level=1,Fight=0,Learn=1,Field=\CourbesParamétréesCartésiennes,Type=\Exercices,Origin=] abr. 
Étudier et tracer la courbe d'équation 
$$
\Q\{\eqalign{
x&={t^2+1\F t^2-1}\cr
y&={t^2\F t+1}
}\W.
$$ 
en précisant l'allure  locale en chaque point singulier, les branches infinies 
et les éventuels points d'inflexions ainsi que les points doubles. 

\exo [Level=1,Fight=0,Learn=1,Field=\CourbesParamétréesCartésiennes,Type=\Exercices,Origin=] abs. 
Étudier et tracer la courbe d'équation 
$$
\Q\{\eqalign{
x&={1+\cos t\F 1-\cos t-\sin t}\cr
y&={1\F\cos t}
}\W.
$$ 
en précisant l'allure  locale en chaque point singulier, les branches infinies 
et les éventuels points d'inflexions ainsi que les points doubles. 

\exo [Level=2,Fight=1,Learn=1,Field=\RepèreDeFrenet,Type=\Exercices,Origin=] abt. 
Soit $\sc E$ une ellipse de centre $O$ et soient deux points $M_1 $et $M_2$ sur $\sc E$. 
\pn
a) prouver que $(OM_1)$ est parallèle à la tangente à $\sc E$ en $M_2\Longleftrightarrow$ 
$(OM_2)$ est parallèle à la tangente à $\sc E$ en $M_1$\pn
b) En supposant que $M_1$ et $M_2$ soient dans cette disposition, 
calculer l'aire du triangle $OM_1M_2$. 

\exo [Level=2,Fight=1,Learn=1,Field=\RepèreDeFrenet,Type=\Exercices,Origin=] abu. 
Démontrer qu'on définit une courbe plane en posant 
$$
\Q\{\eqalign{
x&={a+bt\F1+t^2}\cr 
y&=tx,\cr 
z&={1-t^2\F1+t^2}\cr
}\W.
$$
Nature ? 

\exo [Level=2,Fight=3,Learn=2,Field=\RepèreDeFrenet|\Enveloppes,Type=\Exercices,Origin=\MP] abv. 
Démontrer que les plans d'équation $x\cos t+y\sin t+z\ch t=a\sh t\quad(t\in\ob R)$ 
sont les plans osculateurs à un arc $\Gamma$ que l'on determinera. 

\exo [Level=2,Fight=3,Learn=2,Field=\Roulement|\Développées,Type=\Exercices,Origin=\MP] abw. 
On dit qu'une courbe $\sc C$ est une cycloïde si 
c'est la trajectoire d'un point d'un cercle roulant sans glisser sur une droite. 
\pn
a) Trouver l'équation vérifiée par les cycloïdes. 
b) Déterminer la développée d'une cycloïde. \pn
b) Quelle transformation géométrique simple permet de passer de la courbe à sa développée ?\pn
c) Qu'en déduisez vous pour les développantes d'une cycloïde ? 

\exo [Level=2,Fight=1,Learn=1,Field=\FormesQuadratiques,Type=\Exercices,Origin=] abx. 
Nature et éléments caractéristiques selon $\lambda$ 
de l'ensemble d'équation cartésienne 
$$ 
x^2-4xy+4y^2+6x-3y=\lambda
$$

\exo [Level=2,Fight=1,Learn=1,Field=\FormesQuadratiques,Type=\Exercices,Origin=] aby. 
Nature et éléments caractéristiques selon $\lambda$ 
de l'ensemble d'équation cartésienne 
$$ 
x^2-3xy+2y^2+2x-y=\lambda
$$

\exo [Level=2,Fight=1,Learn=1,Field=\FormesQuadratiques,Type=\Exercices,Origin=] abz. 
Nature et éléments caractéristiques selon $\lambda$ 
de l'ensemble d'équation cartésienne 
$$ 
2x^2-3xy+2y^2-x-y=\lambda
$$

%%%%%%%%%%%% après le crash

\exo [Level=1,Fight=1,Learn=1,Field=\GéométrieSpatiale,Type=\Maple,Origin=] aca. 
Trouver les nombres réels $m$ pour lesquels 
$$
(\sc D)\qquad\Q\{\eqalign{mx-z-m^2&=0\cr y+2z+m-3&=0}\W.\qquad\hbox{et}
\qquad\Q\{\eqalign{2mx+z+2m^2-1&=0\cr 2y+z+2m-6&=0}\W.
$$
sont deux droites coplanaires.  

\exo [Level=1,Fight=1,Learn=1,Field=\GéométrieSpatiale,Type=\Maple,Origin=] acb. 
Trouver le nombre de sphères tangentes au plan d'équation $z=0$ 
et passant par $A(1,2,1)$, $B(3,1,2)$ et $C(2,2,2)$. \pn
Même chose avec $A'(1,-2,1)$, $B'(-3,1,2)$ et $C'(2,2,2)$.

\exo [Level=1,Fight=1,Learn=1,Field=\CourbesParamétréesCartésiennes,Type=\Maple,Origin=] acc. 
Tracer la courbe paramétrée par 
$$\Q\{
\eqalign{x&=t^3-4t\cr
y&=2t^2-3} \W.\qquad(t\in\ob R)
$$
et calculer l'angle des tangentes au point double. Trouver une équation cartésienne de cette courbe,  
en utilisant par exemple la commande ```Eliminate[equations, variable(s) à éliminer]''.
 
\exo [Level=2,Fight=1,Learn=1,Field=\GéométrieSpatiale,Type=\Maple,Origin=] acd. 
Soit $\Gamma$ l'intersection de la sphère $S$ de centre $O(0,0,0)$ et de rayon $2$ avec le cylindre de révolution de rayon $1$ 
et d'axe $\sc D\b((0,1,0),\vec{k}\b)$. \pn
a) Paramétrer $\Gamma$. \pn
b) Tracer $\Gamma$.\pn 
c) Longueur de $\Gamma$ ?

\exo [Level=2,Fight=1,Learn=1,Field=\MatricesOrthogonales,Type=\Maple,Origin=] ace. 
Matrice de rotation vectorielle d'angle $\pi/3$ et d'axe $\Delta$ passant par $0$ et orienté par $\vec u=\vec i-\vec j+\vec k$ dans la base $(\vec i,\vec j,\vec k)$ ?\pn
En déduire les coordonnées de l'image $M'(x',y',z')$ du point $M(x,y,z)$ par la rotation d'axe $\Delta'$  passant par $A(1,1,1)$ 
et orienté par $\vec u$. 

\exo [Level=1,Fight=1,Learn=1,Field=\GéométrieSpatiale,Type=\Maple,Origin=,Indication={On pourra tester si les points $(x,y,z)\in\ob Z^3$ vérifiant $-20\le x,y,z\le 20$ appartiennent à la sphère à l'aide d'une triple boucle emboitée}] acf. 
a) Centre et rayon de la sphère passant par $(1,2,3)$, $(-3,4,5)$, $(3,-4,5)$~et~$(-3,4,-5)$. \pn
b) Trouver tous les points à coordonnées entières de cette sphère. 

\exo [Level=2,Fight=0,Learn=0,Field=\Quadriques,Type=\Maple,Origin=] acg. 
Nature et éléments caractéristiques de la quadrique d'équation $$
-x-z^2+2y+xy+x^2+5z-yz=6.
$$ 
Même question pour 
$$
-10x+2y+4xy-10z+4yz=6.
$$
\par

\exo [Level=2,Fight=2,Learn=1,Field=\RepèreDeFrenet,Type=\Exercices,Origin=\MP] ach. 
a) Etude de l'astro\"ide de paramétrage
$$
\Q\{\eqalign{x(t)&=a\cos^3t\cr y(t)&=a\sin^3(t)}\W.\qquad(t\in\ob R)
$$
b) Déterminer la longueur de l'astroïde. \pn 
c) Rayon de courbure ? \pn
d) Déterminer la courbe orthoptique de l'astro\" ide, c'est à dire l'ensemble des points du plan 
par lesquels passent deux tangentes à l'astroïde, orthogonales entre elles. 

\exo [Level=2,Fight=2,Learn=1,Field=\RepèreDeFrenet,Type=\Exercices,Origin=\MP] aci. 
a) Etude de la cardioïde d'équation polaire $\rho(\theta)=a(1+\cos\theta)$. \pn
b) Déterminer la longueur de la cardioïde. \pn
c) Rayon de courbure ?\pn
d) Déterminer la podaire du point $O$ par rapport au cercle de rayon $a$ et de centre $C(a,0)$, 
c'est à dire le lieu des projections orthogonales de $O$ sur les tangentes au cercle. 

\exo [Level=2,Fight=1,Learn=1,Field=\FonctionsDéfiniesParUneIntégrale,Type=\Exercices,Origin=] acj. 
Étudier la fonction $\ds x\mapsto \int_0^\infty\min\Q\{x,{1\F1+\e^t}\W\}\d t$. 

\exo [Level=2,Fight=2,Learn=1,Field=\Séries,Type=\Exercices,Origin=] ack. 
Nature de la série $\ds\sum_{n=1}^\infty\int_0^{\pi/n}{\sin x\F1+x}\d x$ ?

\exo [Level=2,Fight=2,Learn=1,Field=\FonctionsDéfiniesParUneIntégrale,Type=\Exercices,Origin=] acl. 
Déterminer la limite de la suite $\ds u_n=\int_0^1{\sin (n t)\F 1+t^2}\d t$ lorsque $n\to\infty$. 

\exo [Level=2,Fight=1,Learn=1,Field=\Volumes,Type=\Exercices,Origin=] acm. 
Déterminer le volume de $D:=\{(x,y,z)\in\ob R^3:x^2+y^2\le az\hbox{ et }x+y+z\le a\}$. 

\exo [Level=2,Fight=1,Learn=1,Field=\Volumes,Type=\Exercices,Origin=] acn. 
Déterminer le volume de l'intersection de deux cylindres de rayon $R$ 
et d'axes $(Ox)$ et $(Oy)$. 

\exo [Level=2,Fight=1,Learn=1,Field=\Aires,Type=\Exercices,Origin=] aco. 
Aire du domaine plan délimité par les paraboles $y=x^2$ et $y=2x^2$ d'une part 
et les hyperboles $y={2\F x}$ et $y={5\F x}$ d'autre part. 

\exo [Level=1,Fight=2,Learn=1,Field=\EspacesPréHilbertiens,Type=\Exercices,Origin=] acp. 
Soit $E$ l'espace des fonctions $f:[0,1]\to \ob R$ de classe $\sc C^2$ telles que $f(0)=0=f(1)$ muni du produit scalaire usuel 
$\langle f,g\rangle:=\int_0^1f(t)g(t)\d t$. \pn
a) On pose $(f|g):=\langle-f'',g\rangle$. Prouver que $(.|.)$ définit un produit scalaire sur $E$.  \pn
b) L'application $(f,g)\mapsto(f|g)+\langle f,ag\rangle$ définit elle un produit scalaire sur $E$ ?\pn
c) Etant donnés $(\alpha,\beta)\in\ob R^2$ et $a:[0,1]\to\Q]0,\infty\W[$ et $b:[0,1]\to\ob R$ 
deux fonctions continues, prouver que le problème de``tir''
$$
\Q\{\eqalign{&y''(t)-a(t)y(t)=b(t)\qquad(0\le t\le 1)\cr&y(0)=\alpha\cr&y(1)=\beta}
\W.
$$
possède au plus une solution dans $E$. 

\exo [Level=2,Fight=1,Learn=0,Field=\MatricesOrthogonales,Type=\Exercices,Origin=] acq. 
Déterminer $(a,b)\in\ob R^2$ pour que la matrice $\pmatrix{a&b&b\cr b&a&b\cr b&b&a}$ soit orthogonale. 

\exo [Level=2,Fight=0,Learn=0,Field=\IntégralesGénéralisées,Type=\Exercices,Origin=] acr. 
Nature de $\ds\int_0^\infty{\root3\of{x+1}-\root3\of x\F\sqrt x}\d x$ ?

\exo [Level=2,Fight=1,Learn=1,Field=\Orthonormalisation,Type=\Cours,Origin=\Lakedaemon] acs. 
Soit $n\in\ob N$ et $E:=\ob R_n[X]$. On pose $\langle P,Q\rangle:=P(-1)Q(-1)+P(0)Q(0)+P(1)Q(1)$. \pn
a) Pour quelles valeurs de $n$ définit-on ainsi un produit scalaire sur $E$ ? On note $n_0$ la plus grande.\pn
b) Dorénavant $n=n_0$. Déterminer une base orthonormale pour ce produit scalaire. \pn
c) Déterminer la distance de $1$ à $\hbox{Vect}(X,X^2)$. 
  
\exo [Level=2,Fight=2,Learn=1,Field=\ValeursPropres,Type=\Exercices,Origin=\MP] act. 
Soient $a_1,\cdots,a_n\in\ob R$ et $b_1,\cdots,b_{n-1}\in\ob R^*$. Prouver que 
$A=\pmatrix{
a_1&b_1&0&\cdots&0\cr
b_1&a_2&b_2&\ddots&\vdots\cr
0&b_2&\ddots&\ddots&0\cr
\ldots&&\ddots&\ddots &b_{n-1}\cr
0&\cdots&\cdots&b_{n-1}&a_n
}$\vskip-3em\noindent 
possède $n$ valeurs propres réelles distinctes deux à deux. \vskip3em

\exo [Level=2,Fight=2,Learn=2,Field=\SériesEntières,Type=\Exercices,Origin=] acu. 
Résoudre l'équation différentielle 
$$
(x+x^2)f''(x)+(3x+1)f'(x)+f(x)=0\qquad\b(x\in\Q]0,\infty\W[\b)
$$
en cherchant une solution développable en série entière. 

\exo [Level=2,Fight=0,Learn=0,Field=\SystèmesDifférentiels,Type=\Exercices,Origin=] acv. 
Résoudre sur $\ob R$ le système différentiel 
$$
\Q\{\eqalign{x'=-x+y+z-1\cr y'=x-y+z-1\cr z'=x+y-z-1}\W.
$$

\exo [Level=2,Fight=1,Learn=0,Field=\EquationsDifférentiellesLinéairesDuSecondOrdre,Type=\Exercices,Origin=] acw. 
Résoudre l'équation différentielle 
$$
x(x^2+1)f''(x)-2(x^2+1)f'(x)+2xf(x)=0
$$
en cherchant d'abord une solution polynômiale. 

\exo [Level=2,Fight=1,Learn=0,Field=\Hélices,Type=\Exercices,Origin=] acx. 
Montrer que la courbe paramétrée par
$$
\Q\{\eqalign{x(t)=t-t^3/3\cr
y(t)=t^2\cr
z(t)=t+t^3/3}
\W.\qquad(t\in\ob R)
$$
est une hélice. Déterminer le trièdre de Frenêt, la courbure et la torsion au point de paramètre $t$. 

\exo [Level=2,Fight=1,Learn=0,Field=\Courbure,Type=\Exercices,Origin=] acy. 
Pour $a>0$, déterminer le rayon de courbure de la courbe d'équation paramétrique 
$$
\Q\{\eqalign{x=a(2\cos t+\cos (2t))\cr
y=a(2\sin t-\sin(2t))}
\W.
$$

\exo [Level=2,Fight=1,Learn=1,Field=\SystèmesDifférentiels,Type=\Exercices,Origin=] acz. 
Résoudre l'équation différentielle $y'''-6y''+11y'-6y=0$ 
en la transformant en système différentiel d'ordre $1$

\exo [Level=2,Fight=2,Learn=1,Field=\RepèreDeFrenet,Type=\Exercices,Origin=] ada. 
a) Déterminer l'ensemble des centres $C$ des cercles $\sc C$ passant par $O=(0,0)$ et tangent à l'ellipse 
d'équation cartésienne $x^2+5y^2=5$ dans $\ob R^2$. \pn
b) Etude de la courbe obtenue. 

\exo [Level=2,Fight=2,Learn=2,Field=\FonctionsDéfiniesParUneIntégrale,Type=\Exercices,Origin=] adb. 
Pour $x\in\ob R$, on pose $\ds f(x)=\int_0^\infty\e^{-t^2}\cos(xt)\d t$. \pn
a) Montrer que $f$ est définie sur $\ob R$ et qu'elle est paire. \pn
b) Montrer que $f$ est de classe $\sc C^1$ sur $\ob R$ et exprimer $f'$ à l'aide d'une intégrale impropre. \pn
c) Montrer que $f$ est solution de l'équation différentielle $2y'+xy=0$ sur $\ob R$. \pn
En déduire une expression de $f$ simple. On pourra utiliser que $\ds\int_0^\infty\e^{-t^2}\d t={\sqrt\pi\F2}$. 

\exo [Level=2,Fight=1,Learn=1,Field=\FonctionsDéfiniesParUneIntégrale,Type=\Exercices,Origin=] adc. 
Prouver que l'application $\ds x\mapsto\int_0^\infty\e^{-xt}\th t\d t$ est de classe $\sc C^\infty$ sur $\Q]0,\infty\W[$. 

\exo [Level=2,Fight=2,Learn=2,Field=\FonctionsDéfiniesParUneIntégrale,Type=\Exercices,Origin=] add. 
a) Pour $0<a<b$, montrer que 
$$
\int_a^b{\e^{-xt}\F t}\d t=\ln{b\F a}+\int_0^x{\e^{-bu}-\e^{-au}\F u}\d u.
$$
b) En déduire la convergence et la valeur de l'intégrale $\ds\int_0^\infty{\e^{-bu}-\e^{-au}\F u}\d u$. 

\exo [Level=2,Fight=1,Learn=1,Field=\FonctionsDéfiniesParUneIntégrale,Type=\Exercices,Origin=] ade. 
Pour $x\in\ob R^+$, on pose $\ds g(x)=\int_0^1{\e^{-x^2(1+t^2)}\F1+t^2}\d t$ et $\ds h(x)=\int_0^x\e^{-t^2}\d t$. \pn
a) Montrer que $g$ et $h$ sont de classe $\sc C^1$ sur $\ob R^+$ et calculer leurs dérivées. \pn
b) Vérifier que $g+h^2$ est une constante sur $\ob R^+$ que l'on determinera. \pn
c) Vérifier que $0\le g(x)\le \e^{-x^2}$ pour $x\ge0$. En déduire la convergence et la valeur de $\ds\int_0^\infty\e^{-t^2}\d t$. 

\exo [Level=2,Fight=1,Learn=1,Field=\FonctionsDéfiniesParUneIntégrale,Type=\Exercices,Origin=] adf. 
Soit $f:\ob R\to\ob R$ une application continue.  \pn
a) Prouver que $\ds F:x\mapsto {1\F x}\int_0^xf(t)\d t$ est prolongeable 
en une fonction $\tilde F$ continue sur  $\ob R$. \pn
b) Lorsque $f$ est de classe $\sc C^1$ sur $\ob R$, prouver que $\tilde F$ est de classe $\sc C^1$ sur $\ob R$. 

\exo [Level=2,Fight=1,Learn=1,Field=\FonctionsDéfiniesParUneIntégrale,Type=\Exercices,Origin=] adg. 
Soit $f:\ob R\to\ob R$ une application continue telle que la limite $\ds \lim_{x\to\infty}f(t)=\ell$ existe. \pn 
Étudier la limite de $\ds F(x)={1\F x}\int_0^xf(t)\d t$ lorsque $x\to+\infty$. 
 
\exo [Level=2,Fight=2,Learn=2,Field=\Intégrales,Type=\Exercices,Origin=\MP] adh. 
Soient $a<b$ et $f:[a,b]\to\ob R$ une fonction continue telle que $\ds\Q|\int_a^bf(t)\d t\W|=\int_a^b|f(t)|\d t$. \pn
Prouver que $\ds f(t)\ge0\ \,(a\le t\le b)$ ou $\ds f(t)\le 0\ \,(a\le t\le b)$.  Et avec $\ob C$ à la place de $\ob R$ ?

\exo [Level=2,Fight=3,Learn=2,Field=\Surfaces,Type=\Exercices,Origin=\MP] adi.  
On appelle {\bf Contour apparent conique} d'une surface $S\subset\ob R^3$ 
vue d'un point $A\in\ob R^3$ l'ensemble des points $M\in S$ tels que le plan tangent à $S$ en $M$ contienne la droite $(MA)$. \pn
a) Déterminer le contour apparent de l'ellipsoïde $2x^2+y^2+2z^2=2$ vu du point $(0,1,1)$. \smallskip\noindent
b) Equation paramétrique/cartésienne du cône circonscrit à $S$ de sommet $A$. (l'ensemble des droites passant par $A$ et tangentes à $S$ 
(dans un plan tangent de $S$ en un point $M$)). 

\exo [Level=2,Fight=3,Learn=2,Field=\Surfaces,Type=\Exercices,Origin=\MP] adj. 
On appelle  {\bf Contour apparent cylindrique} d'une surface $S\subset\ob R^3$ 
de direction $\vec u\neq\vec 0$ l'ensemble des points $M\in S$ tels que le plan tangent à $S$ en $M$ 
contienne la droite $\sc D(M,\vec u)$. \smallskip\noindent
a) Déterminer le contour apparent cylindrique de l'ellipsoïde $x^2+2y^2+2z^2=2$ 
selon $\vec u=(1,1,1)$. \pn
b) Equation paramétrique/cartésienne du cylindre de direction $\vec u$ 
circonscrit à $S$. (l'ensemble des droites de direction  $\vec u$ tangentes à $S$ 
(dans un plan tangent de $S$ en un point $M$)). 

\exo [Level=2,Fight=3,Learn=2,Field=\Surfaces,Type=\Exercices,Origin=\MP] adk. 
On appelle {\bf Ligne de niveau} de la surface $S$ relativement à la direction $\vec u$ 
toute courbe obtenue par intersection de $S$ avec un plan de vecteur normal $\vec u$. \pn
Déterminer les courbes de niveau selon la direction $\vec u=\vec i+\vec j+\vec k$ des surfaces : \pn
a)  $S_1: x^2+2xz-yz=0$ \qquad b) $S_2:x^2+y^2+2z^2=2$\qquad c) $S_3: x^2-y^2=z$ \pn
d) $S_4:x^2+y^2-z^2=-1$ \qquad e) $S_5:x^2+xy+y^2=1$.

\exo [Level=2,Fight=3,Learn=2,Field=\Surfaces,Type=\Exercices,Origin=\MP] adl. 
On appelle {\bf Ligne de plus grande pente} de la surface $S$ selon la direction $\vec k$ 
toute courbe $\Gamma$ tracée sur $S$ telle qu'en chacun de ses points $M$, la tangente à $\Gamma$ en $M$ soit incluse dans le plan $(M,\vec k, \vec n)$ 
où $\vec n$ est la normale à $S$ en $M$. \pn
Pour $\theta\in\ob R$, vérifier que les courbes $t\mapsto(\cos\theta\cos t,\sin\theta\cos t,\sin t)$ 
sont des lignes de plus grande pente  de la sphère de centre $O$ et de rayon $1$ dans la direction $\vec k$. 

\exo [Level=2,Fight=0,Learn=0,Field=\Surfaces,Type=\Exercices,Origin=] adm. 
a) Equation du cylindre $C$ de révolution de rayon $R$ et d'axe 
$$
(D)\qquad\Q\{\eqalign{x=z+2\cr y=z+1}\W.
$$
b) Condition nécessaire et suffisante sur $R$ pour que $Oz$ soit tangent à $C$. 

\exo [Level=2,Fight=1,Learn=1,Field=\Surfaces,Type=\Exercices,Origin=] adn. 
Condition nécessaire et suffisante sur $\lambda$ pour que $S: x^2-2\lambda y z=0$ soit de révolution. Axe ?

\exo [Level=2,Fight=0,Learn=0,Field=\Surfaces,Type=\Exercices,Origin=] ado. 
Equation du cône de sommet $(0,a,0)$ s'appuyant sur 
$$
\Q\{\eqalign{x=a\cos\theta\sin\theta\cr 
y=a\sin^2\theta\cr
z=a\cos \theta}
\W.
$$

\exo [Level=2,Fight=3,Learn=2,Field=\Surfaces,Type=\Exercices,Origin=\MP] adp. 
Equation du cylindre $C$ de génératrice dirigées par $\vec v=\vec i+\vec j+\vec k$ et circonscrit à la surface d'équation $x^2+y^2-2z=0$. 

\exo [Level=2,Fight=0,Learn=1,Field=\Enveloppes,Type=\Exercices,Origin=] adq. 
Enveloppe de la famille de droite $\{D_\theta\}_{\theta\in\ob R}$ passant par $M_\theta=(\cos\theta,\sin\theta)$ et dirigée par $u_\theta=(\cos 2\theta,\sin2\theta)$. 

\exo [Origin=,Level=2,Fight=1,Learn=1,Type=\Exercices,Field=\Développantes] adr. 
Développantes de la courbe paramétrée par 
$$
\Q\{\eqalign{x=\sh^2t\cr y=2\sh^2t}\W.
$$
Reconnaitre celle qui passe par le point $A=(-2,0)$. 

\exo [Level=2,Fight=0,Learn=1,Field=\SommesDeRiemann,Type=\Exercices,Origin=X] ads. 
Soit $f:[0,1]\to\ob R$ de classe $\sc C^1$. Déterminer la limite
$$
\lim_{n\to\infty}{1\F n}\sum_{0\le k<n}f\Q({k\F n}\W)f'\Q({k+1\F n}\W).
$$

\exo [Level=2,Fight=0,Learn=1,Field=\SommesDeRiemann,Type=\Exercices,Origin=\MP] adt. 
Soit $f:[0,1]\to\ob R$ une fonction de classe $\sc C^3$. \pn
a) Déterminer la limite $\ell$ de la suite $\ds u_n={1\F n}\sum_{1\le k\le n}f\Q({2k-1\F2n}\W)$. \pn
b) Equivalent de la suite $n^2(u_n-\ell)$ lorsque $n$ tends vers $+\infty$. 
 
\exo [Level=2,Fight=1,Learn=1,Field=\Intégrales,Type=\Exercices,Origin=\MP] adu. 
Soit $f:[a,b]\to [f(a),f(b)]$,  une bijection croissante, de classe $\sc C^1$ sur $[a,b]$. 
Trouver une expression simple pour  
$$
\int_a^bf(t)\d t+\int_{f(a)}^{f(b)}f^{-1}(t)\d t
$$
Interpretation géométrique ?

\exo [Level=1,Fight=2,Learn=2,Field=\SommesDeRiemann,Type=\Exercices,Origin=\MP] adv. 
Soit $f:\ob R\to\ob R$ une application de classe $\sc C^1$ telle que $f(0)=0$. Déterminer la limite
$$
\lim_{n\to\infty}\sum_{0\le k<n}f\Q({n\F k^2+n^2}\W).
$$

\exo [Level=1,Fight=2,Learn=2,Field=\Intégrales,Type=\Exercices,Origin=\MP] adw. 
 Déterminer $\ds\lim_{n\to\infty}\int_a^b|\sin (nt)|\d t$. 

\exo [Level=1,Fight=1,Learn=2,Field=\SommesDeRiemann,Type=\Exercices,Origin=\MP] adx. 
Soit $f:[0,1]\to\ob R$ une fonction de classe $\sc C^2$. \pn
a) Déterminer la limite $\ell$ de la suite $\ds u_n={1\F n}\sum_{0\le k< n}f\Q({k+1\F n}\W)$. \pn
b) Equivalent de la suite $n(u_n-\ell)$ lorsque $n$ tends vers $+\infty$. 
 
\exo [Level=1,Fight=2,Learn=1,Field=\Intégrales,Type=\Exercices,Origin=\MP]  ady. 
Déterminer les fonctions $f:\ob R\to\ob R$ continues telles que 
$$
f(x)+\int_0^x(x-t)f(t)\d t=0\qquad(x\in\ob R).
$$

\exo [Level=1,Fight=2,Learn=2,Field=\Intégrales,Type=\Exercices,Origin=]  adz. 
Pour $n\in\ob N$, on pose $I_n=\int_0^{\pi/2}\sin^nt\d t$. \pn
a) Etablir une relation de récurrence entre $I_{n+2}$ et $I_n$. \pn
b) En déduire une expression de $I_n$ ne faisant intervenir que des factorielles. 

\exo [Level=1,Fight=0,Learn=0,Field=\Intégrales,Type=\Exercices,Origin=]  aea. 
Calculer $\ds I=\int_0^{2\pi}{\d t\F2+\cos t}$. 

\exo [Level=1,Fight=1,Learn=0,Field=\Intégrales,Type=\Exercices,Origin=] aeb. 
Calculer $\ds \int_0^1\arcsin\Q({2t\F1+t^2}\W)\d t$. 

\exo [Level=1,Fight=2,Learn=1,Field=\Intégrales,Type=\Exercices,Origin=] aec. 
Calculer $\ds \int_0^1{x^2\F 1+x^2}\arctan x\d x$ puis $\ds\int_0^1x\arctan^2x\d x$. 

\exo [Level=1,Fight=1,Learn=0,Field=\Intégrales,Type=\Exercices,Origin=] aed. 
Calculer $\ds \int_1^2{\ln(1+t)\F t^2}\d t$. 

\exo [Level=1,Fight=0,Learn=0,Field=\Intégrales,Type=\Exercices,Origin=] aee. 
Calculer $\ds \int_0^{\pi/4}\cos^4x\sin^2x\d x$. 

\exo [Level=1,Fight=2,Learn=2,Field=\Fonctions,Type=\Cours,Origin=] aef. 
a) Prouver que la fonction $\ch$ est une bijection de $\Q[0,+\infty\W[$ dans $\Q[0,+\infty\W[$ 
et déterminer sa bijection réciproque $\ch^{-1}$ et sa dérivée. \pn
b) Prouver que la fonction $\sh$ est une bijection de $\ob R$ dans $\ob R$ et 
déterminer sa bijection réciproque $\sh^{-1}$ et sa dérivée.  \pn
c) En déduire les primitives $\ds \int{\d x\F\sqrt{x^2+2x+\lambda}}$ selon la valeur de $\lambda\in\ob R$. 

\exo [Level=1,Fight=0,Learn=0,Field=\TrigonométrieHyperbolique,Type=\Exercices,Origin=] aeg. 
Pour $(a,b,c,x)\in\ob R^4$, calculer les sommes suivantes : 
$$
C(a,b)=\sum_{p=0}^n\ch(a+pb), \qquad\ds S(a,b)=\sum_{p=0}^n\sh(a+pb)\quad\hbox{et}\quad
\ds \sum_{p=0}^np\ch(px).
$$ 

\exo [Level=2,Fight=3,Learn=2,Field=\Déterminant,Type=\Exercices,Origin=\MP,Indication={faire $L_k-L_{k-1}\to L_k$ pour $2\le k\le n$},Solution={$det A_p=\det A_{p-1}=\cdots=\det A_0=1$}] aeh. 
Pour $p\ge1$, calculer le déterminant de la matrice $A_p:=\pmatrix{
\Q({n\atop 0}\W)&\Q({n\atop 1}\W)&\ldots&\Q({n\atop p}\W)\cr
\Q({n+1\atop 0}\W)&\Q({n+1\atop 1}\W)&\ldots&\Q({n+1\atop p}\W)\cr
\vdots&\vdots&\ldots&\vdots\cr
\Q({n+p\atop 0}\W)&\Q({n+p\atop 1}\W)&\ldots&\Q({n+p\atop p}\W)\cr
}$. 


\exo [Level=2,Fight=1,Learn=1,Field=\VecteursPropres,Type=\Exercices,Origin=] aei. 
Soit $V$ un vecteur de $\ob R^3$. Trouver  les valeurs propres (réelles et complexes) et les vecteurs propres 
de l'endomorphisme $u$ de $\ob R^3$ défini par $u(W)=V\wedge W$ pour $W\in\ob R^3$. 

\exo [Level=2,Fight=2,Learn=2,Field=\Normes,Type=\Exercices,Origin=\MP] aej. 
On pose $\|x\|_\infty:=\max_{1\le i\le n}\{|x_i|\}$ pour $x=(x_1,\cdots,x_n)$ dans $\ob R^n$. Pour chaque norme $N$ de $\ob R^n$, 
prouver qu'il existe $\alpha>0$ tel que  
$$
N(x)\le\alpha\|x\|_\infty\qquad(x\in\ob R^n).
$$

\exo [Level=2,Fight=1,Learn=1,Field=\IntégralesMultiples,Type=\Exercices,Origin=\MP] aek. 
Calculer $\ds \int\int_D{x^3+y^3\F\exp(xy)}\d x\d y$ pour $D:=\{(x,y):y^2\le 2px\hbox{ et }x^2\le 2py\}$. \pn
On pourra poser $x=u^2v$ et $y=uv^2$. 

\exo [Level=2,Fight=1,Learn=1,Field=\PotentielsVecteurs,Type=\Exercices,Origin=\MP] ael. 
Déterminer $f:\ob R\to\ob R$ de classe $\sc C^1$ pour que $\vec V(x,y,z):=\b(1-x^2,f(y),z(2x-y)\b)$ soit 
un champ de rotationnels. 
Trouver alors un potentiel vecteur. 

\exo [Level=2,Fight=1,Learn=1,Field=\ChampsDeVecteurs,Type=\Exercices,Origin=\MP] aem. 
Calculer le flux du champ $\vec V(x,y,z):=(x,y,-z)$ au travers de 
la surface 
$$
\Sigma:=\{(x,y,z)\in\ob R^2\times\ob R^+:x^2+y^2+z^2=1\}.
$$ 

\exo [Level=2,Fight=1,Learn=1,Field=\ChampsDeVecteurs,Type=\Exercices,Origin=\MP] aen. 
Calculer le flux du champ vectoriel $F(x,y,z)=(x^3,y^3,z^3)$ traversant la sphère de centre $O$ et de rayon $1$.  

\exo [Level=2,Fight=1,Learn=1,Field=\MatricesOrthogonales,Type=\Exercices,Origin=\MP] aeo. 
Soient $M=\pmatrix{1&-1&1\cr0&2&1\cr1&1&-1\cr}$, $Q_1:=\null^{\hbox{t}}MM$ et  $Q_2:=M^{\hbox{t}}M$. \pn
Prouver qu'il existe $R\in\sc SO(3)$ 
telle que $Q_2=\null^{\hbox{t}}RQ_1R$ et la déterminer. 

\exo [Level=2,Fight=2,Learn=1,Field=\SériesEntières,Type=\Exercices,Origin=\MP] aep. 
Résoudre $x(x-1)y''+3xy'+y=0$ en cherchant d'abbord une solution développable en série entière. 

\exo [Level=2,Fight=1,Learn=1,Field=\Extrema,Type=\Exercices,Origin=] aeq. 
Extremums de $\sin x+\sin y+\cos(x+y)$ sur $\Q]0,\pi\W[^2$ ?

\exo [Level=2,Fight=1,Learn=1,Field=\EquationsAuxDérivéesPartielles,Type=\Exercices,Origin=,Indication={Procéder au changement de variable $u=xy$ et $v=x/y$.}] aer. 
Trouver les fonctions $f:\ob R^2\to\ob R$ de classe $\sc C^1$ sur $\Q]0,\infty\W[^2$ telles que 
$$
x{\partial f\F\partial x}(x,y)+y{\partial f\F\partial y}(x,y)=2xy\qquad(x>0,y>0).
$$


\exo [Level=2,Fight=1,Learn=0,Field=\Développantes|\Développées,Type=\Exercices,Origin=] aes. 
Développée et développantes de la chainette $y=\ch x\quad(x\in\ob R)$. 

\exo [Level=2,Fight=1,Learn=0,Field=\Développées,Type=\Exercices,Origin=] aet. 
Développée de 
$$
\Q\{\eqalign{x=2(1+\cos^2t)\sin t\cr y=\sin t\sin(2t)}\W.
$$

\exo [Level=1,Fight=2,Learn=2,Field=\Matrices,Type=\Cours,Origin=] aeu. 
Soient $(a,b)\in\ob R^2$ et $A:=\pmatrix{a&b&0&0\cr 0&a&b&0\cr 0&0&a&b\cr0&0&0&a}$. 
Pour $n\in\ob N$, calculer $A^n$. 

\exo [Level=1,Fight=1,Learn=1,Field=\Matrices,Type=\Exercices,Origin=]  aev. 
Soient $A=\pmatrix{a&b\cr c&d\cr}$ et $E:=\sc M_2(\ob R)$. \pn
a) Prouver que $\Phi:M\mapsto MA-AM$ est-elle un endomorphisme de $E$. \pn
b) Matrice de $\Phi$ dans la base canonique de $E$ ?

\exo [Level=1,Fight=1,Learn=1,Field=\Matrices|\Anneaux,Type=\Exercices,Origin=] aew. 
Quel est la structure de  $E$ l'ensemble des matrices de la forme 
$M(a,b,c)=\pmatrix{a+c&b&-c\cr b&a+2c&-b\cr-c&-b&a+c}$ 
si on le munit des opérations usuelles $+, \cdots, \times$ ?

\exo [Level=1,Fight=2,Learn=2,Field=\Trigonalisation,Type=\Exercices,Origin=\MP] aex. 
Calculer les puissances de $\pmatrix{1&1\cr1&0}$. 

\exo [Level=1,Fight=1,Learn=1,Field=\Matrices,Type=\Exercices,Origin=] aey. 
Soient $n\ge m\ge 1$ des entiers. 
Résoudre pour $A\in\sc M_{n,m}(\ob R)$ et $B\in\sc M_{m,n}(\ob R)$ l'équation $AB=I_n$. 

\exo [Level=1,Fight=3,Learn=2,Field=\DimensionFinie,Type=\Exercices,Origin=]  aez. 
Soient $E$ un $\ob R$-espace vectoriel de dimension finie et $f\in\sc L(E)$. Prouver que 
$f^2=0\Longleftrightarrow$
$$
\exists (g,h)\in\sc L(E)^2\hbox{ tel que }\Q\{\eqalign{g\circ h=f\cr h\circ g=0}\W.
$$

\exo [Level=1,Fight=2,Learn=2,Field=\DimensionFinie,Type=\Exercices,Origin=] afa. 
Soient $(M,N)\in\sc M_n(\ob R)^2$ deux matrices de projecteurs telles que $I_n-M-N\in\sc Gl_n(\ob R)$. Prouver que les matrices 
$M, N, MN, NM, MNM$ et  $NMN$ ont le même rang.  

\exo [Level=1,Fight=3,Learn=2,Field=\DimensionFinie,Type=\Exercices,Origin=] afb. 
Soient $n,p\in\ob N^*$ et $A\in\sc M_{n,p}(\ob R)$. \pn
a) On suppose que $\hbox{rg}(A)=n$. Montrer qu'il existe $B\in\sc M_{p,n}(\ob R)$ telle que $AB=\hbox{I}_n$. \pn
b) On suppose que $\hbox{rg}(A)=p$. Montrer qu'il existe $C\in\sc M_{p,n}(\ob R)$ telle que $CA=\hbox{I}_p$. 

\exo [Level=1,Fight=1,Learn=1,Field=\Polynômes,Type=\Exercices,Origin=] afc. 
Résoudre dans $\ob C^*$ le système
$\ds
\Q\{\eqalign{x+y+z=3\cr
{1\F x}+{1\F y}+{1\F z}={1\F3}\cr
x^2+y^2+z^2=17\cr
}\W.
$

\exo [Level=1,Fight=2,Learn=2,Field=\SommesDeRiemann,Type=\Exercices,Origin=] afd. 
Calculer les limites $\ds\lim_{n\to\infty}\sum_{k=1}^n\tan\Q({k\F n^2}\W)$ 
et $\ds\lim_{n\to\infty}\sum_{k=1}^n\tan\Q({k\F n^2+k^2}\W)$

\exo [Level=1,Fight=1,Learn=1,Field=\Fonctions,Type=\Exercices,Origin=] afe. 
On note $f:\ob R^2\to\ob R^2$ l'application définie par
$$
f(x,y)=\Q({x\F1+|y|},{y\F1+|x|}\W).
$$
Déterminer $f(\ob R^2)$, montrer que $f$ est bijective et exprimer sa bijection réciproque.  

\exo [Level=2,Fight=2,Learn=1,Field=\Séries,Type=\Exercices,Origin=] aff. 
Déterminer la nature de la série de terme général $\ds u_n=\prod_{k=2}^n\Q({1\F\sqrt k}-1\W)$. 

\exo [Level=2,Fight=3,Learn=2,Field=\Polynômes,Type=\Exercices,Origin=\MP] afg. 
Soient $(\alpha,\beta)\in\ob R^2$ tel que $0<\beta\le 1<\alpha$. Pour $n\ge1$, on note
$$
R_n=\sum_{k=n+1}^\infty{1\F k^\alpha}\qquad S_n=\sum_{k=1}^n{1\F k^\beta}, \qquad u_n:={R_n\F S_n}\quad\hbox{et}\quad v_n=(-1)^nu_n
$$
a) Trouver des équivalents simples de $R_n$ et $S_n$ lorsque $n$ tends vers l'infini. \pn
b) étudier la nature de la série de terme général $u_n$. \pn
c) Montrer que la série de terme général $v_n$ converge. 

\exo [Level=1,Fight=1,Learn=1,Field=\FractionsRationnelles,Type=\Exercices,Origin=] afh. 
Décomposer en éléments simples dans $\ob C[X]$ les trois fractions rationnelles suivantes : 
$$
F_n:={1\F X^n-1}, \qquad G_n:={X^{n-1}\F(X^n-1)^2}\quad\hbox{et}\quad H_n:={1\F(X^n-1)^2}. 
$$

\exo [Level=2,Fight=1,Learn=0,Field=\Orthonormalisation,Type=\Maple,Origin=] afi. 
Pour $E:=\ob R_{20}[X]$. On pose $\langle P,Q\rangle:=\sum_{k=0}^{20}P(k)Q(k)$. \pn
a) Prouver que c'est un produit scalaire dans $E$. \pn
b) Déterminer une base orthonormale pour ce produit scalaire. \pn
c) Déterminer la distance de $1+X^{10}-X^{20}$ à $\hbox{Vect}(X,\cdots, X^{19})$. 

\exo [Level=1,Fight=2,Learn=2,Field=\Polynômes,Type=\Exercices,Origin=] afj. 
On pose $H_0=1$, $H_1=2X$ et 
$$
H_n=2X H_{n-1}-H_{n-2}\qquad(2\le n\le 20).
$$
a) $H_n$ Polynôme ? si oui, degré ?\pn
b) Que vaut $H_n(\cos \theta)$ ?\pn
c) Calculer $\langle H_i,H_j\rangle$ pour 
$$
\langle P,Q\rangle=\int_{-1}^1\sqrt{1-x^2}P(x)Q(x)\d x
$$

\exo [Level=1,Fight=2,Learn=2,Field=\Polynômes,Type=\Exercices,Origin=]  afk. 
Pour $n\ni\ob N$, déterminer l'existence d'un polynome $P_n$ et d'un seul vérifiant 
$$
P_n(X+1)+P_n(X)=2X^n.
$$
Déterminer une relation entre $P_n'$ et $P_{n-1}$. \pn
Expliciter $P_n$ pour $0\le n\le 10$. 

\exo [Level=1,Fight=3,Learn=2,Field=\DimensionFinie,Type=\Exercices,Origin=\MP] afl. 
Déterminer $\{x,y,z\}\in[-1,1]^3$ et $\{a,b,c\}\in\ob R^3$ tels que la formule 
$$
\int_{-1}^1{P(t)\d t\F\sqrt{1-t^2}}=aP(x)+bP(y)+cP(z)\qquad(P\in\ob R_n[X])
$$ 
soit vraie pour le plus grand $n$ possible. 

\exo [Level=1,Fight=0,Learn=0,Field=\Anneaux,Type=\Exercices,Origin=\MP] afm. 
Dans $\ob Z^2$ on défini la loi de composition $\star$ par 
$$
(a,b)\star(c,d)=(ac,d+(-1)^db)
$$
Étudier les propriétés de cette loi : associativité, commutativité, élément neutre et éléments symétrisables. 

\exo [Level=2,Fight=1,Learn=1,Field=\Diagonalisation,Type=\Exercices,Origin=]  afn. 
Démontrer que les matrices $\pmatrix{1&2&3\cr3&1&2\cr2&3&1\cr}$ et $\pmatrix{1&3&2\cr2&1&3\cr3&2&1\cr}$ sont semblables

\exo [Level=2,Fight=2,Learn=2,Field=\Diagonalisation,Type=\Maple,Origin=] afo. 
Ecrire une fonction permettant de calculer la puissance $n^{\hbox{\sevenrm ième}}$ d'une matrice diagonalisable. 

\exo [Level=2,Fight=2,Learn=1,Field=\Trigonalisation,Type=\Exercices,Origin=\MP]  afp. 
Trigonaliser $M=\pmatrix{0&1&0\cr0&0&1\cr1&-3&3\cr}$ et calculer sa puissance $n^{\hbox{\sevenrm ième}}$. 

\exo [Level=2,Fight=2,Learn=2,Field=\Trigonalisation,Type=\Exercices,Origin=] afq. 
Trois suites de nombres vérifient 
$$
\eqalign{10 a_{n+1}=3b_n+7c_n\cr
10 b_{n+1}=2a_n+8c_n\cr 10c_{n+1}=4a_n+6b_n}\qquad(n\in\ob N).
$$
Étudier la convergence des trois suites. 

\exo [Level=2,Fight=2,Learn=2,Field=\SériesEntières,Type=\Exercices,Origin=] afr. 
Développer la fonction 
$$
f(x)=\e^{-x^2}\int_0^x\e^{t^2}\d t
$$
en série entière au voisinage de $0$ et calculer $f(1)$ à $10^{-3}$ près. 

\exo [Level=2,Fight=1,Learn=1,Field=\Orthonormalisation,Type=\Exercices,Origin=] afs. 
Déterminer le minimum 
$$
\min_{(c,d)\in\ob R^2}\int_a^b\Q(f(x)-c-d x\W)^2\d x
$$
pour $f(x)=x^2$ et $[a,b]=[0,1]$ puis pour $f(x)=\sin x$ et $[a,b]=[-\pi/2,\pi/2]$. 

\exo [Level=2,Fight=1,Learn=1,Field=\GéométrieSpatiale,Type=\Exercices,Origin=] aft. 
On considère les droites $D_1$ et $D_2$ définies par 
$$
D_1:\qquad\Q\{\eqalign{x+2y+3z=6\cr x-y=1}\W.\qquad D_2:\qquad\Q\{\eqalign{2x+y+z=1\cr x-2y+3z=2}\W.
$$
Vérifier que ces deux droites sont non-coplanaires et étudier la perpendiculaire communne aux deux droites. \pn 
Quelle est la distance entre ces deux droites ?

\exo [Level=2,Fight=1,Learn=1,Field=\FonctionsDePlusieursVariables,Type=\Exercices,Origin=]  afu. 
Déterminer $f\in\sc C^2\b(\Q]-1,1\W[,\ob R\b)$ pour que 
$$
g(x,y)=f\Q({\cos x\F\ch y}\W)
$$
ait un laplacien nul. 

\exo [Level=2,Fight=1,Learn=1,Field=\Extrema,Type=\Exercices,Origin=]  afv. 
Parmi les triangles dont le périmètre est fixé, quels sont ceux dont l'aire est maximale ?

\exo [Level=2,Fight=1,Learn=1,Field=\FonctionsDePlusieursVariables,Type=\Exercices,Origin=,Indication={On pourra utiliser le changement de variable défini par
$$
\Q\{
\eqalign{u&=x^{-1}+y^{-1}\cr
v&=x+y}
\W.
$$}] afw. 
Résoudre l'équation suivante sur l'ouvert $U: 0<y<x$. 
$$
x^2{\partial f\F\partial x}-y^2{\partial f\F\partial y}=(y^2-x^2)\sin(x+y)
$$


\exo [Level=2,Fight=2,Learn=1,Field=\Surfaces,Type=\Exercices,Origin=] afx. 
Soit $S$ la surface d'équation $z=x^2(x-y)$. Étudier les droites tracées sur $S$. 

\exo [Level=2,Fight=1,Learn=1,Field=\IntégralesMultiples,Type=\Exercices,Origin=] afy. 
Calculer le moment d'inertie du solide 
$$
\Delta:\qquad\Q\{\eqalign{1\le z\le 2\cr z(x^2+y^2)\le2x}\W.
$$
par rapport à l'axe $(Oz)$. 

\exo [Level=2,Fight=3,Learn=2,Field=\Surfaces,Type=\Exercices,Origin=] afz. 
On considère la nappe 
$$
\Q\{\eqalign{x=u\cos t\cr y=u\sin t\cr z=t}\W.\qquad(u,t)\in\ob R^2
$$
1) Montrer que la nappe est réglée. Est-elle développable ?\pn
2) Calculer l'aire du morceau de surface défini par $0\le u\le 1$ et $0\le t\le \pi/2$. 

\exo [Level=2,Fight=2,Learn=1,Field=\Surfaces,Type=\Exercices,Origin=] aga. 
Déterminer une équation cartésienne et la nature de la surface engendrée 
par la rotation de la droite $D$ autour de $\Delta$ avec 
$$
D:\qquad\Q\{\eqalign{y=x+1\cr z=0}\W.\qquad\Delta:\qquad\Q\{\eqalign{x=z\cr y=0}\W.
$$

\exo [Level=1,Fight=2,Learn=2,Field=\Anneaux,Type=\Exercices,Origin=] agb. 
Pour $(x,y)\in\ob C^2$, on pose $x\dag y=x+y-1/2$ et $x\star y=a+b-2ab$. \pn
Prouver que $\ob R$ muni des lois $\dag$ et $\star$ est un corps commutatif. 

\exo [Level=2,Fight=1,Learn=1,Field=\FonctionsDéfiniesParUneIntégrale,Type=\Exercices,Origin=] agc. 
Soient $h>0$ et $f\in\sc C(\ob R,\ob R)$. On pose $\ds g(x):={1\F 2h}\int_{x-h}^{x+h}f(t)\d t$ pour $x\in\ob R$. \pn
a) Pour $x\in\ob R$, montrer que $\ds g(x)={1\F 2}\int_{-1}^1f(x+hu)\d u$. \pn
b) En déduire que, si $f$ est croissante (resp. convexe), alors $g$ est croissante (resp. convexe). 

\exo [Level=2,Fight=1,Learn=1,Field=\FonctionsDéfiniesParUneIntégrale,Type=\Exercices,Origin=] agd. 
Pour $n\ge1$, on pose $\ds a_n:=\int_0^n\sqrt{1+(1-x/n)^n}\d x$. \pn 
Montrer que $a_n\sim n$ lorsque $n\to\infty$. 

\exo [Level=2,Fight=2,Learn=2,Field=\Intégration,Type=\Exercices,Origin=] age. 
Soit $f:\Q]-1,1\W[\to\ob R$ une application continue telle que 
$$
\forall x\in\Q]-1,1\W[, \qquad f(x)=1+\int_0^xf(t)^2\d t. 
$$
a) Montrer que $f$ ne s'annule pas sur $\Q]-1,1\W[$. \pn
b) En déduire $f$. 

\exo [Level=2,Fight=0,Learn=0,Field=\IntégralesGénéralisées,Type=\Exercices,Origin=] agf. 
Existence et calcul de $\ds f(y):=\int_{-\infty}^\infty{\d x\F(x^2+y^2+1)^{3/2}}$. 

\exo [Level=2,Fight=1,Learn=1,Field=\FonctionsDéfiniesParUneIntégrale,Type=\Exercices,Origin=] agg. 
Pour $(a,b)\in\Q]0,\infty\W[$, montrer que 
$$
\int_0^1(1-x^a)^{1/b}\d x=\int_0^1(1-x^b)^{1/a}\d x.
$$

\exo [Level=2,Fight=1,Learn=0,Field=\SériesNumériques,Type=\Exercices,Origin=] agh. 
Convergence et somme de $\ds\sum_{n=1}^\infty{2n+1\F n^2(n+2)^2}$ 
sachant que $\ds\sum_{n=1}^\infty{1\F n^2}={\pi^2\F 6}$. 

\exo [Level=2,Fight=2,Learn=1,Field=\SériesNumériques,Type=\Exercices,Origin=] agi.  
La suite de Fibonacci est la suite $(\phi_n)_{n\in\ob N}$ défine par
$$
\phi_0=0, \qquad \phi_1=1\quad\hbox{et}\quad \phi_{n+2}=\phi_n+\phi_{n+1}\qquad(n\in\ob N).
$$
Convergence et somme des séries $\ds\sum_{n\ge 2}{1\F\phi_{n-1}\phi_{n+1}}$ 
et $\ds\sum_{n=1}^\infty{\phi_{n-1}\F\phi_n\phi_{n+1}}$. 

\exo [Level=2,Fight=0,Learn=0,Field=\SériesEntières,Type=\Exercices,Origin=] agj. 
Soit $(a_n)_{n\in\ob N}$ une suite de nombres complexes et soient $R$, $R'$ et $R''$ les rayons de convergence respectifs 
des séries entières $\sum_{n=0}^\infty a_nz^n$, $\sum_{n=0}^\infty \Re e(a_n)z^n$ et $\sum_{n=0}^\infty \Im m(a_n)z^n$. \pn
Prouver que $R=\min(R',R'')$. 

\exo [Level=2,Fight=1,Learn=0,Field=\FonctionsDéfiniesParUneIntégrale,Type=\Exercices,Origin=] agk. 
Ensemble de définition de la fonction $f$ définie par 
$$
f(x)=\int_0^\pi{t\d t\F1-x\sin t}
$$
Développer $f$ en série entière au voisinage de $0$. 

\exo [Level=2,Fight=3,Learn=2,Field=\Fonctions,Type=\Exercices,Origin= \MP] agl. 
Soit $f:\ob R\to\ob R$ une application vérifiant $f(x)\le x$ pour $x\in\ob R$ et  
$$
f(x+y)\le f(x)+f(y)\qquad(x,y)\in\ob R^2.
$$
Prouver que $f=\hbox{Id}_{\ob R}$. 

\exo [Level=2,Fight=1,Learn=0,Field=\Surfaces,Type=\Exercices,Origin=] agm. 
Nature de l'ensemble des points equidistants de deux droites $\sc D$ et $\sc D'$. 

\exo [Level=2,Fight=2,Learn=1,Field=\Surfaces,Type=\Exercices,Origin=\MP] agn. 
Droites tracées sur $x^3+y^3+z^3=1$ ?

\exo [Level=2,Fight=3,Learn=2,Field=\Surfaces,Type=\Exercices,Origin=\MP] ago. 
Soit $a>0$. On note $\sc D$ et $\Delta$ les deux droites données par 
$$
D:\qquad\Q\{\eqalign{x=0\cr y=a}\W.\qquad\hbox{et}\qquad\Delta:\qquad\Q\{\eqalign{x=0\cr y=z}\W.
$$ 
a) Equation du cône $\Sigma$ de revolution d'axe $\Delta$ et contenant $Oz$ ?\pn
b) Equation de la surface engendrée par la perpendiculaire commune à $D$ et à une génératrice variable de $\Sigma$ ?

\exo [Level=2,Fight=2,Learn=1,Field=\Surfaces,Type=\Exercices,Origin=\MP] agp. %%% doute pour le =1
Montrer que la surface d'équation 
$$
(x^2-yz)^2+(y^2-zx)^2+(z^2-xy)^2=1
$$
est une surface de révolution dont on donnera axe et méridienne. 

\exo [Level=2,Fight=2,Learn=1,Field=\Surfaces,Type=\Exercices,Origin=] agq. 
Soit $a>0$. Donner un paramétrage et une équation cartésienne de la surface constituée par l'ensemble des cercles tangents en $O$ à $Oz$ et rencontrant la droite d'équation 
$$
D:\qquad\Q\{\eqalign{x=2a\cr z=0}\W.
$$

\exo [Level=2,Fight=2,Learn=1,Field=\Surfaces,Type=\Exercices,Origin=] agr. 
Pour $a,b>0$,  équation cartésienne de la surface engendrée par les droites s'appuyant sur 
$$
P:\qquad\Q\{\eqalign{y^2=ax\cr z=0}\W.,\qquad D:\qquad\Q\{\eqalign{x=0\cr z=b}\W.\quad\hbox{et}\quad D:qquad
\Q\{\eqalign{y=0\cr z=-b\cr}\W.
$$

\exo [Level=2,Fight=3,Learn=1,Field=\Surfaces,Type=\Exercices,Origin=\MP] ags. 
Soit $a>0$. Déterminer les arcs tracés sur le cylindre $D$ d'équation $x^2+y^2=a^2$, de classe $\sc C^2$ biréguliers 
tels que la binormale en tout point soit coplanaire  ave cla droite $D:\{x=a,y=0\}$. 

\exo [Level=2,Fight=3,Learn=1,Field=\Surfaces|\Enveloppes,Type=\Exercices,Origin=\MP] agt. 
Montrer que les plans d'équation $P_t:x\cos t+y\sin t+z\ch t=a\sh t$ sont les plans osculateurs d'un arc birégulier $\Gamma$ 
que l'on déterminera

\exo [Level=2,Fight=1,Learn=1,Field=\GéométrieSpatiale,Type=\Exercices,Origin=] agu. 
Soit $P$ le plan d'équation $x-y+2z=1$ et $D$ la droite d'équation $\{x=z+2,y=-z+1\}$. \pn
Equation de $D'$ symétrique de $D$ par rapport à $P$ ? \pn
Equation de $P'$ symétrique de $P$ par rapport à $D$ ?

\exo [Level=2,Fight=2,Learn=1,Field=\Surfaces,Type=\Exercices,Origin=\MP] agv. 
Un cercle de rayon $R$ varie en restant tangent aux trois plans d'un trièdre trirectangle. 
Déterminer le lieu du centre de ce cercle.  

\exo [Level=2,Fight=1,Learn=1,Field=\Surfaces,Type=\Exercices,Origin=] agw. 
Soit $S$ une surface régulière définie par $\vec{OM}=\vec F(u,v)$. 
A chaque point $M$ de $S$, on associe un point $M'$ à $M$ 
en posant 
$$
\vec{MM'}=a{{\partial F\F\partial u}\wedge{\partial F\F\partial v}\F\|{\partial F\F\partial u}\wedge{\partial F\F\partial v}\|}.
$$
Lorsque $M$ décrit $S$, le point $M'$ décrit une surface $M'$. 
Montrer que les plants tangents à $S$ en $M$ et à $S'$ en $M'$ sont parallèles. 

\exo [Level=2,Fight=0,Learn=0,Field=\Surfaces,Type=\Exercices,Origin=] agx. 
Démontrer que la surface d'équation 
$$
{1\F y-z}+{1\F z-x}+{1\F x-y}=1
$$ 
est un cylindre dont on donnera une section droite. 

\exo [Level=2,Fight=1,Learn=1,Field=\Surfaces,Type=\Exercices,Origin=] agy. 
Ensemble des points équidistants d'un plan et d'une droite ?

\exo [Level=2,Fight=2,Learn=1,Field=\Surfaces,Type=\Exercices,Origin=] agz. 
Intersection d'un tore de centre $O$ et d'un plan passant par $O$ et tangent au tore. 

\exo [Level=1,Fight=2,Learn=2,Field=\Anneaux,Type=\Exercices,Origin=] aha. 
Soit $\omega$ une racine de $X^2+bX+c=0$ avec $b^2-4c<0$ et soit $\ob Z[\omega]$ l'ensemble 
$$
\ob Z[\omega]=\{z=p+q\omega:(p,q)\in\ob Z^2\}
$$
a) Démontrer que $\ob Z[\omega]$ est un sous-anneau de $\ob C$ et que $\ob Z[\omega]=\ob Z[\ol{\omega}]$. \pn
b) Pour $z\in\ob Z[\omega]$ démontrer qu'il existe un unique coupe $(p,q)\in\ob Z^2$ tel que $z=p+q\omega$. \pn
c) Dans le cas $b=0$ et $c=1$, déterminer les éléments inversibles de $\ob Z[\omega]$. 

\exo [Level=1,Fight=1,Learn=1,Field=\Polynômes,Type=\Exercices,Origin=] aia. 
Déterminer $n\ge2$ tel que $(X+1)^n+X^n-1$ admette au moins une racine double. 

\exo [Level=1,Fight=1,Learn=1,Type=\Maple,Field=\Orthonormalisation,Origin=\BanquePT] aib. 
Dans $\ob R_6[X]$, on pose
$$
\forall(P,Q)\in E^2, \qquad \langle P, Q\rangle :=\int_0^\infty P(t)Q(t)\e^{-t}\d t
$$
On notera $\left\Vert . \right\Vert$ la norme associée. \pn 
1. Vérifier qu'il s'agit bien d'un produit scalaire. Construire une base orthonormale du sous-espace $\ob R_5[X]$. \pn
2. Pour $H:=X^6+X+1$ déterminer $P$, élément de $\ob R_5[X]$, tel que $\left\Vert H-P\right\Vert$ soit minimale.



\exo [Level=1,Fight=2,Learn=1,Field=\SommesDeRiemann,Type=\Exercices,Origin=X] aic. 
Soient $f,g:[0,1]\to\ob R$ de classe $\sc C^1$. Déterminer la limite
$$
\lim_{n\to\infty}{1\F n}\sum_{1\le k\le n}f\Q({k\F n}\W)g\Q({k-1\F n}\W)
$$

\exo [Level=1,Fight=2,Learn=1,Field=\Dérivation,Type=\Exercices,Origin=] aid. 
$f:\ob R\to\ob R$ application continue en $0$ telle que 
$$
\lim_{x\to0}{f(2x)-f(x)\F x}=L.
$$ 
Démontrer que $f$ est dérivable en $0$. 

\exo [Level=2,Fight=3,Learn=1,Field=\MatricesOrthogonales,Type=\Exercices,Origin=] aie. 
Soit $A(t)$ une matrice orthogonale d'ordre impaire pour tout $t\in \ob R$ dont les coefficients sont de classe $\sc C^1$. 
Prouver que $A'(t)$ n'est pas inversible pour $t\in\ob R$. 

\exo [Level=2,Fight=2,Learn=1,Field=\FonctionsDéfiniesParUneIntégrale,Type=\Exercices,Origin=] aif. 
pour $n\ge0$, on pose $I_n:=\ds\int_0^1{\d t\F1+t^n}$. \pn
a) Prouver que $I_n$ converge ves $1$. \pn
b) Démontrer que $I_n-1$ est équivalent à $\ds-{\ln2\F n}$

\exo [Level=2,Fight=2,Learn=1,Field=\FonctionsDéfiniesParUneIntégrale,Type=\Exercices,Origin=] aig. 
Soit $f:\ob R\to\ob R$ telle que $\int_{-\infty}^\infty f(t)\d t$ converge. 
Pour $x\in\ob R$, on pose $F(x):=\ds \int_{x-1}^{x+1}f(t)\d t$. \pn
Convergence et calcul de $\int_{-\infty}^\infty F(x)\d x$ ? 

\exo [Level=2,Fight=2,Learn=2,Field=\SériesNumériques,Type=\Exercices,Origin=] aih. 
On pose $u_1=-1$ et pour $n\ge2$, on pose
$$
u_n:=(n-1/2)\ln\Q({n\F n-1}\W)-1
$$
a) Démontrer que la série de terme général $(u_n)$ converge. \pn
b) Calculer sa somme pour montrer que $\ds A_n={n!\F n^{n+1/2}\e^{-n}}$ a une limite $L$. \pn
c) Déterminer $L$ sachant que $\ds{1.3.5\cdots(2n-1)\F2.4.6\cdots 2n}$ est équivalent à $1/\sqrt{\pi n}$. \pn
d) En déduire la formule de Stirling. 


\exo [Level=1,Fight=1,Learn=1,Field=\DimensionFinie,Type=\Exercices,Origin=]  aii. 
Résoudre et discuter selon les paramètres le système  $\ds
\Q\{\eqalign{
\lambda x+y+z+t&=1\cr
x+\lambda y+z+t&=a\cr
x+y+\lambda z+t&=a^2\cr
x+y+z+\lambda t&=a^3
}\W.$

 

\exo [Level=2,Fight=1,Learn=1,Field=\SystèmesDifférentiels,Type=\Exercices,Origin=] aij. 
Résoudre sur $\ob R$ le système différentiel 
$$ 
\Q\{\eqalign{
x''(t)=&3x(t)+y(t)+\ch (2t)
\cr
y''(t)=&2x(t)+2y(t)
}\W.
$$

\exo [Level=2,Fight=0,Learn=0,Field=\SystèmesDifférentiels,Type=\Exercices,Origin=] aik.  
Résoudre sur $\Q]0,\infty\W[$ le système différentiel 
$$
\Q\{\eqalign{
x'(t)
=&x(t)+{y(t)\F t}-{z(t)\F t}
\cr
y'(t)=&{x(t)\F t}+y(t)+{z(t)\F t}
\cr
z'(t)=&-{x(t)\F t}+{y(t)\F t}+z(t)
}\W.
$$


\exo [Level=2,Fight=2,Learn=2,Field=\EquationsAuxDérivéesPartielles,Type=\Exercices,Origin=] ail.  
Soit $f:\ob R^3\ssm\{(0,0,0)\}\to\ob R$ une application de classe $\sc C^2$ 
vérifiant 
$$ 
\Delta f(x,y,z):={\partial^2f\F\partial x^2}(x,y,z)
+{\partial^2f\F\partial y^2}(x,y,z){\partial^2f\F\partial y^2}(x,y,z)=0\qquad\b((x,y,z)\neq(0,0,0)\b). 
$$
On suppose qu'il existe une application $g:\Q]0,\infty\W[\to\ob R$ telle que 
$$
f(x,y,z)=g\Q(\sqrt{x^2+y^2+z^2}\W)\qquad\b((x,y,z)\neq(0,0,0)\b).
$$
a) Exprimer $\Delta f(x,y,z)$ en fonction des dérivées $g$ au point $r:=\sqrt{x^2+y^2+z^2}$. 
\smallskip\noindent
b) En déduire une équation différentielle satisfaite par $g$
\smallskip\noindent
c) En déduire $f$. 

\exo [Level=1,Fight=0,Learn=0,Origin=\Lakedaemon,Type=\Cours,Field=\EspacesVectoriels]  aim. 
Montrer que les fonctions $f:\ob R^+\to\ob R$ dérivables en $x=1$ forment un $\ob R$-EV. 

\exo [Origin=,Level=1,Fight=2,Learn=2,Type=\Exercices,Field=\Polynômes|\EspacesVectoriels|\DimensionFinie] ain. 
Pour $n\ge0$, prouver que l'on définit un isomorphisme  en posant 
$$
\eqalign{\Phi:\ob R_n[X]&\to\ob R^{n+1}\cr P&\mapsto\big(P(0), P(1), \cdots, P(n)\big)\cr}
$$

\exo [Origin=,Level=1,Fight=1,Learn=1,Type=\TravauxDirigés,Field=\EspacesVectoriels] aio. 
Soit $E$ un $\ob K$-EV et $f$ un endomorphisme de $E$. Prouver que
$$
\ker f=\ker f^2\qquad \ssi\qquad \IM f\cap\ker f=\{0\}.
$$


\exo [Origin=,Level=1,Fight=1,Learn=1,Type=\TravauxDirigés,Field=\EspacesVectoriels|\DimensionFinie] aip. 
Montrer que $\ob R^4=\mbox{Vect}(a,b,c,d)$ pour 
$$
a=(1,1,1,0), \qquad b=(1,1,0,1), \qquad c=(1,0,1,1), \quad \hbox{et}\quad d=(0,1,1,1).
$$

\exo [Origin=,Level=1,Fight=2,Learn=2,Type=\TravauxDirigés,Field=\EspacesVectoriels|\DimensionFinie] aiq. 
Pour chaque polynôme $P$ de $\ob R[X]$, on note $\Phi(P)$ le polynôme définit par 
$$
\Phi(P)=P(X+1)-P(X)\qquad\qquad (\underline{\mbox{\bf substitutions}})
$$
a) Prouver que $\Phi:\ob R[X]\to\ob R[X]$ est un endomorphisme, non-injectif. \pn
b) Étudier la restriction $\tilde{\Phi}$ de l'endomorphisme $\Phi$ donnée par 
$$
\eqalign{\tilde{\Phi}:\ob R_{n+1}[X]&\to\ob R_n[X]\cr P&\mapsto \Phi(p)}.
$$\
c) En déduire que $\Phi:\ob R[X]\to\ob R[X]$ est surjective. 

\exo [Level=1,Fight=0,Learn=0,Field=\EspacesVectoriels,Type=\Cours,Origin=\Lakedaemon]  air. 
Soient $\sc A:=\{a_1, \cdots, a_n\}$ et $\sc B:=\{b_1, \cdots, b_n\}$ deux familles de vecteurs tels que 
$$
a_1=b_1, \qquad a_2=b_1+b_2, \qquad \sbox{ et }\qquad a_n=b_1+\cdots+b_n.
$$
Prouver que la famille $\sc A$  est libre $\Longleftrightarrow$ la famille $\sc B$ est  libre. 
 
\exo [Level=1,Fight=0,Learn=0,Type=\Maple,Field=\Coniques,Origin=\BanquePT] ais. 
Soit, dans un plan muni d'un repère orthonormé $(O,\vec i,\vec j)$, la conique $\Gamma$ d'équation $x^2+3xy+2y^2=1$. 
Déterminez une équation réduite de $\Gamma$ et en déduire ses principales caractéristiques.
La représenter.


\exo [Origin=,Level=1,Fight=3,Learn=4,Type=\Problèmes,Field=\EspacesVectoriels,Indication={1d) calculer l'image par $\Delta$ des espaces $\ob R_n[X]$.\medskip\noindent 5) Calculer $H_k(a)$ selon 3 cas.}] ait. 
1. Pour chaque polynôme $P\in\ob R[X]$, on pose 
$$
\Delta(P):=P(X+1)-P(X).
$$
a) Prouver que $\Delta$ définit un endomorphisme de l'espace $\ob R[X]$. 
\pn 
b) Déterminer le noyau de $\Delta$. 
\pn
c) Déterminer l'image de $\Delta$. 
\pn
d) Mêmes questions appliquées à $\Delta^k$ pour $k\ge2$. 
\pn
2. Démontrer que la suite $\{H_k\}_{k\in\ob N}$ définie par $H_0:=1$ et 
$$
H_k:={X(X-1)\cdots(X-k+1)\F k!}\qquad(k\ge1).
$$ 
est une base de l'espace $\ob R[X]$.
\medskip
\noindent
3. Calculer $\Delta(H_k)$ et en déduire $\Delta^n(H_k)$ pour $(k,n)\in\ob N^2$. 
\medskip
\noindent
4. Pour chaque polynôme $P$ de degré $n$, prouver que 
$$
\ds P=\sum_{0\le k\le n}\Delta^k(P)(0)H_k.
$$
5. Pour $k\in\ob N$ et $a\in\ob Z$, montrer que $H_k(a)\in\ob Z$. 
\medskip
\noindent
6.Soit $P\in\ob R[X]$ un polynôme. Prouvez l'équivalence des propositions suivantes : 
\medskip
\noindent
\ i) Pour chaque entier $a\in\ob Z$, $P(a)$ est un entier relatif. 
\pn
ii) Il existe $(\lambda_0,\cdots,\lambda_n)\in\ob Z^{n+1}$ tel que $P=\sum_{k=0}^n\lambda_kH_k$. 
\medskip
\noindent
7. Etant donné un entier $n\in\ob N^*$ fixé, on note respectivement $\delta$ et $d$ les restrictions à l'espace $\ob R_n[X]$ des applications $\Delta$ et $P\mapsto P'$. 
\medskip
\noindent
a) Démontrer que $\ds\delta=\sum_{k=1}^n\ds{d^k\F k!}$. 
\smallskip
\noindent
b) Démontrer que $\ds d=\sum_{k=1}^n\ds{(-1)^{k-1}\F k}\delta^k$. 
\medskip
\noindent
8. Pour $k\in\ob N^*$ et $P\in\ob R[X]$, démontrer que 
$$
\Delta^k(XP)=X\Delta^k(P)+k\Delta^k(P)+k\Delta^{k-1}(P).
$$
9. Pour chaque couple $(p,q)$ d'entiers naturels, on pose $a_{p,q}:=\Delta^q\b(X^p\b)(0)$. 
\pn
a) Remarquer que $a_{p,q}=0$ lorsque $0\le p<q$. 
\pn
b) Calculer $a_{p,p}$ et $a_{p,0}$ pour $p\in\ob N$. 
\pn
c) Démontrer que $a_{p,q}=qa_{p-1,q}+qa_{p-1,q-1}$ lorsque $p\ge1$ et $q\ge1$. 
\pn
d) En déduire les valeurs de $a_{p,q}$ pour $(p,q)\in\{0,\cdots,5\}^2$ et les classer dans un tableau analogue au triangle de Pascal.  
\medskip
\noindent
10. Soit $Q:=\sum_{k=0}^n\lambda_kH_k$ un polynôme. Déterminer par ses coordonnés sur la base $\{H_k\}_{k\in\ob N}$ l'unique polynôme $P$ vérifiant $\Delta(P)=Q$ et $P(0)=0$. 
\medskip
\noindent
11. Résoudre le problème précédent pour $Q_1=X$, $Q_2=X^2$, $Q_3=X^3$ et $Q_4=X^4$. 
\medskip
\noindent
12. En déduire une formule polynomiale pour les sommes $\sum\limits_{1\le k\le n}k$, $\sum\limits_{1\le k\le n}k^2$, $\sum\limits_{1\le k\le n}k^3$ et $\sum\limits_{1\le k\le n}k^4$. 

\exo [Level=2,Fight=1,Learn=1,Field=\IntégralesGénéralisées,Type=\Exercices,Origin=\Fac,Indication={Faire un DL en $0$ et majorer en $+\infty$},Solution={L'intégrale converge.}] aiu. 
Convergence de $\ds\int_0^\infty{\e^{-x}\F\sqrt x}\d x$. 

\exo [Level=2,Fight=-1,Learn=0,Field=\IntégralesGénéralisées,Type=\Exercices,Origin=\Fac,Indication={Minorer par $1$},Solution={L'intégrale diverge}] aiv. 
Convergence de $\int_1^\infty x^x\d x$. 

\exo [Level=2,Fight=1,Learn=1,Field=\Déterminant,Type=\Exercices,Origin=\MP,Indication={développer par rapport à la première colonne puis procéder par récurrence.},Notion={développement par rapport à une colonne|récurrence|$n$-linéarité}] aiw. 
Pour $n\ge2$ et $(x, y)\in\ob R^2$,  calculer le déterminant 
$$
D_n:=\Q|\matrix{
x+y&xy&\cdots&\ob O\cr
1&\ddots&\ddots&\cr
&\ddots&\ddots&xy\cr
\ob O&&1&x+y\cr
}\W|
$$

\exo [Level=2,Fight=1,Learn=0,Field=\IntégralesGénéralisées,Type=\Exercices,Origin=\Fac,Indication={\item{En 0 : }majorer en module par ${\sqrt x\F \ln (1+x)}$ puis trouver un équivalent. \item{en $+\infty$ : }trouver un équivalent.},Solution={L'intégrale converge}] aix. 
Convergence de $\ds\int _0^\infty{\sqrt x\sin({1\F x})\F \ln (1+x)}\d x$. 

\exo [Level=2,Fight=0,Learn=0,Field=\IntégralesGénéralisées,Type=\Exercices,Origin=\Fac,Indication={Utiliser une primitive du cours ou procéder au changement de variable $x=\ch t$.},Solution={$I:=\ln\Q(2+\sqrt 3\W)$.}] aiy. 
Nature et calcul de $I:=\ds\int_1^2{1\F\sqrt {x^2-1}}\d x$.  

\exo [Level=2,Fight=3,Learn=1,Field=\Déterminant,Type=\Exercices,Origin=,Indication={Décomposer la matrice $A$ comme un produit $A=BC$ de matrices dont le déterminant est facile à calculer.},Notion={binôme de Newton}] aiz. 
Pour $n\ge2$ et  $(\alpha_1,\cdots,\alpha_n,\beta_1,\cdots,\beta_n)\in\Q]0,\infty\W[^{2n}$,  
calculer le déterminant de la matrice~$A:=\Q((\alpha_i+\beta_j)^{n-1}\W)_{1\le i,j\le n}$.

\exo [Level=2,Fight=1,Learn=1,Field=\Diagonalisation,Type=\Exercices,Origin=,Solution={$A=PDP^{-1}$ avec $P:=\pmatrix{2&0&1\cr1&1&0\cr3&-2&2}$,  $D:=\pmatrix{-2&0&0\cr0&1&0\cr0&0&1\cr}$ et $P^{-1}:=\pmatrix{-2&2&1\cr2&-1&-1\cr5&-4&-2}$, d'ou $A^k=\pmatrix{-4(-2)^k+5&(-2)^k-4&2(-2)^k-2\cr-2(-2)^k+2&2(-2)^k-1&(-2)^k-1\cr-6(-2)^k+6&6(-2)^k-6&3(-2)^k-2}$}] aja. 
Diagonaliser la matrice $A=\pmatrix{13&-12&-6\cr6&-5&-3\cr18&-18&-8}$ puis en déduire $A^k$ pour $k\ge1$. 

\exo [Level=2,Fight=1,Learn=1,Field=\Diagonalisation,Type=\Exercices,Origin=] ajb. 
Diagonaliser la matrice $A=\pmatrix{1}_{1\le i\le n\atop1\le j\le n}$ qui ne contient que des $1$. \pn 
Pour $k\ge2$, comment calculer $A^k$ en faisant le moins de calculs possible ?  

\exo [Level=2,Fight=0,Learn=0,Field=\Diagonalisation,Type=\TravauxDirigés,Origin=,Solution={$P:=\pmatrix{0&1&0&-1\cr-1&1&-1&1\cr0&0&1&1\cr1&0&0&1}$ et $D:=\pmatrix{-2&0&0&0\cr0&2&0&0\cr0&0&2&0\cr0&0&0&-2\cr}$.}] ajc. 
Diagonaliser la matrice $\pmatrix{1&1&1&1\cr1&1&-1&-1\cr1&-1&1&-1\cr1&-1&-1&1}$. 

\exo [Level=2,Fight=1,Learn=0,Field=\Diagonalisation,Type=\TravauxDirigés,Origin=,Solution={$P:=\pmatrix{-1&-1&-1&-1\cr0&1&-1&-1\cr1&0&0&1\cr1&0&1&0}$ et $D:=\pmatrix{3&0&0&0\cr0&3&0&0\cr0&0&-3&0\cr0&0&0&-3\cr}$}] ajd. 
Diagonaliser la matrice $\pmatrix{-1&-4&-2&-2\cr-4&-1&-2&-2\cr2&2&1&4\cr2&2&4&1}$. 

\exo [Level=2,Fight=3,Learn=1,Field=\Déterminant,Type=\Colles,Origin=] aje.  
Pour $n\ge2$ et $\alpha_1,\cdots,\alpha_n,\beta_1,\cdots,\beta_n$ dans $\Q]0,\infty\W[$,  
calculer le déterminant (de Cauchy) de la matrice $A:=\Q({1\F \alpha_i+\beta_j}\W)_{1\le i,j\le n}$.

\exo [Level=2,Fight=0,Learn=0,Field=\Trigonalisation,Type=\TravauxDirigés,Origin=,Solution={$P=\pmatrix{-1&2&0\cr0&0&3\cr1&-5&1}$ et $T=\pmatrix{-2&0&0\cr0&1&-1\cr0&0&1}$}] ajf. 
Trigonaliser la matrice $\pmatrix{-4&0&-2\cr0&1&0\cr5&1&3}$

\exo [Level=2,Fight=0,Learn=0,Field=\Trigonalisation,Type=\Colles,Origin=,Solution={$P=\pmatrix{1&0&1\cr0&1&0\cr1&1&0}$ et $T=\pmatrix{2&0&0\cr0&4&1\cr0&0&4}$}] ajg. 
Trigonaliser la matrice $\pmatrix{4&2&-2\cr1&5&-1\cr1&3&1}$

\exo [Level=2,Fight=1,Learn=1,Field=\Trigonalisation,Type=\Colles,Origin=,Solution={$P=\pmatrix{1&1&1&1\cr1&0&-1&0\cr1&0&1&0\cr0&-1&0&1}$ et $T=\pmatrix{1&-3&0&0\cr0&1&0&0\cr0&0&-1&3\cr0&0&0&-1}$}] ajh. 
Trigonaliser la matrice $\pmatrix{0&1&0&2\cr-3&0&4&0\cr0&1&0&3\cr-1&0&1&0}$

\exo [Level=2,Fight=2,Learn=1,Field=\Trigonalisation,Type=\Exercices,Origin=,Solution={%
$P=\pmatrix{1&1&1&0&0\cr1&0&0&1&1\cr1&1&0&1&0\cr-2&-3&0&0&0\cr-2&5&0&0&0}$ et $T=\pmatrix{-1&8&0&0&0\cr0&-1&0&0&0\cr0&0&1&0&-1\cr0&0&0&1&-1\cr0&-10&0&0&1}$}] aji. 
Trigonaliser la matrice $\pmatrix{1&-1&1&0&1\cr0&0&1&1&0\cr0&-1&2&0&1\cr0&0&0&1&-2\cr0&0&0&2&-3}$

\exo [Level=2,Fight=1,Learn=0,Field=\Trigonalisation,Type=\Exercices,Origin=,Solution={$P=\pmatrix{1&0&1\cr1&1&0\cr0&1&1}$ et $T=\pmatrix{2&0&0\cr0&0&-4\cr0&0&0}$}] ajj. 
Trigonaliser la matrice $\pmatrix{1&1&-1\cr-1&3&-3\cr-2&2&-2}$

\exo [Level=2,Fight=1,Learn=1,Field=\Diagonalisation,Type=\Exercices,Origin=] ajk. 
Condition nécéssaire et suffisante sur $a$ pour que $A$ soit diagonalisable ? \pn
Quand $A$ est diagonalisable, trouver $P$ telle que $P^{-1}AP$ soit diagonale (le vérifier). 
$$
A:=\pmatrix{1&\ds a+{1\F a}&\ds a^2+{1\F a^2}\cr\ds a+{1\F a}&1&\ds a+{1\F a}\cr\ds a^2+{1\F a^2}&\ds a+{1\F a}&1}.
$$

\exo [Level=2,Fight=2,Learn=1,Field=\MatricesOrthogonales,Type=\Exercices,Origin=]ajl. 
a) Condition nécéssaire et suffisante sur $u$, $v$ et $w$ pour que la matrice $A$ suivante soit diagonalisable ? 
$$
A:=-{2\F3}\pmatrix{-{1\F 2}&{v\F u}&{w\F u}\cr {u\F v}&-{1\F2}&{w\F v}\cr {u\F w}&{v\F w}&-{1\F2}}.
$$
b) Trouver $u$, $v$ et $w$ pour que $A$ soit orthogonale.  


\exo [Level=2,Fight=1,Learn=1,Field=\ValeursPropres,Type=\Exercices,Origin=] ajm. 
Condition nécéssaire et suffisante sur $a$ pour que la matrice $M(a)$ ait toutes ses valeurs propres strictements positives, avec 
$$
M(a)=\pmatrix{a&2&1&1&1&1\cr2&a&2&1&1&1\cr1&2&a&2&1&1\cr1&1&2&a&2&1\cr1&1&1&2&a&2\cr1&1&1&1&2&a}
$$

\exo [Level=2,Fight=1,Learn=1,Field=\Diagonalisation,Type=\Exercices,Origin=] ajn. 
Montrer que la matrice $A$ est diagonalisable dans une base indépendante de $a$, $b$, $c$ puis montrer que $A$ est inversible et calculer $A^n$, avec 
$$
A=\pmatrix{a & b & c\cr b&a+c&b\cr c&b&a}.
$$

\exo [Level=2,Fight=1,Learn=1,Field=\VecteursPropres,Type=\Exercices,Origin=] ajo. 
étudier suivant les valeurs du paramètre $a$ les valeurs propres et la dimension des sous-espaces propres de la matrice
$$
\pmatrix{10-a&-a+7&1\cr-8+a&-5+a&-1\cr-6+a&a-7&3}
$$

\exo [Level=2,Fight=3,Learn=1,Field=\Trigonalisation,Type=\Others,Origin=,Solution={$P=\pmatrix{0&1&1&0&1\cr0&0&0&1&0\cr1&0&0&0&0\cr1&0&1&0&0\cr0&1&0&0&0}$ et $T=\pmatrix{2&0&0&0&0\cr0&3&0&0&0\cr0&0&1&1&-1\cr0&0&0&1&1\cr0&0&0&0&1}$}] ajp. Trigonaliser la matrice 
$$
A=\pmatrix{0&1&-1&1&3\cr1&1&1&-1&-1\cr0&0&2&0&0\cr-1&1&0&2&1\cr0&0&0&0&3}
$$

\exo [Level=2,Fight=2,Learn=1,Field=\EquationsAuxDérivéesPartielles,Type=\Exercices,Origin=] ajq.  
A l'aide du changement de variables $u={x+y\F2}$ et $v={x-y\F2}$, trouver les solutions de classe $\sc C^2$ sur $\ob R^2$ de l'équation des cordes vibrantes 
$$
{\partial^2 f\F\partial x^2}={\partial^2f\F\partial y^2}.
$$

\exo [Level=2,Fight=2,Learn=1,Field=\EquationsAuxDérivéesPartielles,Type=\Exercices,Origin=] ajr. 
Résoudre à l'aide des coordonnées polaires l'équation aux dérivées partielles 
$$
x{\partial f\F\partial x}(x,y)+y{\partial f\F\partial y}(x,y)=\sqrt{x^2+y^2}
$$

\exo [Level=2,Fight=2,Learn=1,Field=\EquationsAuxDérivéesPartielles,Type=\Exercices,Origin=] ajs. 
Résoudre à l'aide du changement de variable $x=u$, $y=uv$ l'équation 
$$
x^2{\partial^2f\F\partial x^2}+2xy{\partial^2f\F\partial x\partial y}+y^2{\partial^2f\F\partial y^2}=0.
$$


\exo [Level=2,Fight=0,Learn=0,Field=\FonctionsDePlusieursVariables,Type=\Exercices,Origin=] ajt. 
Soit $f\in\sc C^2(\ob R^3,\ob R)$. On pose $F(x,y,z)=f(x-y,y-z,z-x)$. Calculer
$$
{\partial F\F\partial x}+{\partial F\F\partial y}+{\partial F\F\partial z}.
$$

\exo [Level=2,Fight=0,Learn=0,Field=\FonctionsDePlusieursVariables,Type=\Exercices,Origin=] aju. 
Soit $f:\ob R^2\to\ob R$ une fonction de classe $\sc C^2$. On pose $g(x,y):=f(x^2-y^2,2xy)$. Calculer $\Delta(g)$ en fonction de $\Delta(f)$. 


\exo [Level=2,Fight=2,Learn=1,Field=\EquationsAuxDérivéesPartielles,Type=\Exercices,Origin=] ajv. 
Déterminer les fonctions $f\in\sc C^1(\ob R^2,\ob R)$ vérifiant 
$$
\forall (x,y)\in\ob R^2, \qquad {\partial f\F\partial x}-{\partial f\F\partial y}=0
$$
On pourra effectuer le changement de variables $u=x+y$ et $v=x-y$. 

\exo [Level=2,Fight=1,Learn=1,Field=\Extrema,Type=\Exercices,Origin=] ajw. 
Chercher les extremums sur $\ob R^2$ de l'application $f:(x,y)\mapsto x^4+y^4-4xy$. 


\exo [Level=2,Fight=1,Learn=1,Field=\Extrema,Type=\Exercices,Origin=] ajx. 
Chercher les extremums sur $\ob R^2$ de l'application $f:(x,y)\mapsto(x-y)\e^{xy}$. 


\exo [Level=2,Fight=1,Learn=1,Field=\Extrema,Type=\Exercices,Origin=] ajy. 
Chercher les extremums sur $\ob R^2$ de l'application $f:(x,y)\mapsto x\e^y+y\e^x$. 


\exo [Level=2,Fight=1,Learn=1,Field=\Extrema,Type=\Exercices,Origin=] ajz. 
Chercher les extremums sur $\ob R^2$ de l'application $f:(x,y)\mapsto\e^{x\sin y}$. 


\exo [Level=2,Fight=1,Learn=1,Field=\Extrema,Type=\Exercices,Origin=] aka. 
Chercher les extremums sur $\ob R^2$ de l'application $f:(x,y)\mapsto x^3+y^3$. 


\exo [Level=2,Fight=1,Learn=1,Field=\Extrema,Type=\Exercices,Origin=] akb. 
Déterminer les extremums sur $[0,1]^2$ de l'application 
$$
f:(x,y)\mapsto xy(1-x^2-y^2).
$$ 

\exo [Level=2,Fight=1,Learn=1,Field=\Extrema,Type=\Exercices,Origin=] akc. 
Trouvez les points critiques et déterminer si ce sont des extrema locaux de 
$$
f(x,y,z)=\cos(2x)\sin(y)+z^2
$$


\exo [Level=2,Fight=1,Learn=1,Field=\Extrema,Type=\Exercices,Origin=] akd. 
Trouvez les points critiques et déterminer si ce sont des extrema locaux de l'application 
$$
f(x,y)=\sin(x)+y^2-2y+1
$$

\exo [Origin=\Lakedaemon,Level=2,Fight=2,Learn=1,Type=\Exercices,Field=\Diagonalisation] ake. 
Trouver toutes les matrices $M\in\sc M_2(\ob R)$ vérifiant $M^2=\mbox I_2$. 

\exo [Origin=\Lakedaemon,Level=2,Fight=1,Learn=1,Type=\TravauxDirigés,Field=\Diagonalisation] akf. 
Exprimer la matrice $A^n$ en fonction de $A=\pmatrix{1&1\cr1&0\cr}$,  de $\mbox I_2$ et de $n\ge1$.

\exo [Origin=\Lakedaemon,Level=2,Fight=1,Learn=1,Type=\Exercices,Field=\SystèmesDifférentiels] akg. 
Résoudre le système différentiel
$$
\Q\{\eqalign{
x'(t)&=x(t)+y(t)+1\cr
y'(t)&=x(t)-y(t)+\e^t
}\W.\qquad (t\in\ob R).
$$

\exo [Level=2,Fight=1,Learn=1,Type=\Cours,Field=\ValeursPropres,Origin=\Lakedaemon] akh. 
Trouvez toutes les valeurs propres de l'endomorphisme $u:f\mapsto f''$ de l'espace $\sc C^\infty(\ob R,\ob R)$. 

\exo [Level=2,Fight=1,Learn=1,Type=\Cours,Field=\ValeursPropres,Origin=\Lakedaemon,Solution={$1$ est valeur propre de multiplicité $3$ et $-1$ est valeur propre de multiplicité $1$.}] aki. 
Trouvez toutes les valeurs propres de l'endomorphisme $u:M\mapsto\NULL^tM$ de l'espace $\sc M_2(\ob R)$. 

\exo [Level=2,Fight=1,Learn=1,Type=\Exercices,Field=\EquationsDifférentiellesLinéairesDuSecondOrdre,Origin=\Lakedaemon] akj. 
Résoudre $y''-2y'+y=\cos t$ sur $\ob R$. 

\exo [Origin=\Lakedaemon,Level=2,Fight=0,Learn=0,Type=\Exercices,Field=\EquationsDifférentiellesLinéairesDuSecondOrdre] akk. 
Courbes intégrales de $y'=y+1$  et de $y''+y=0$. 

\exo [Origin=\Lakedaemon,Level=2,Fight=1,Learn=1,Type=\TravauxDirigés,Field=\SystèmesDifférentiels] akl. 
Trouver les trajectoires du système
$$
\Q\{\eqalign{x'(t)=2x(t)+y(t)\cr
y'(t)=2y(t)+x(t)\cr}\W. 
$$

\exo [Origin=\Lakedaemon,Level=2,Fight=2,Learn=1,Type=\Exercices,Field=\Orthonormalisation] akm. 
Dans l'espace $\sc C\b([-1,1],\ob R\b)$ muni du 
produit scalaire $\langle f,g\rangle=\int_{-1}^1f(t)g(t)\d t$, orthonormaliser 
la famille constituée des fonctions polynômes 
$$
p_0:t\mapsto1, \qquad p_1:t\mapsto t, \qquad p_2:t\mapsto t^2, \qquad p_3:t\mapsto t^3\quad\sbox{et}\quad p_4:t\mapsto t^4.
$$ 

%% Redondant % akn. %

\exo [Origin=\Lakedaemon,Level=2,Fight=1,Learn=1,Type=\TravauxDirigés,Field=\Orthonormalisation] ako. 
Trouver une formule pour la symétrie orthogonale $s(x,y,z)$ et la projection orthogonale $p(x,y,z)$ (vectorielles) de $\ob R^3$ sur le plan d'équation $x+y+z=0$. 

\exo [Origin=\Lakedaemon,Level=2,Fight=1,Learn=1,Type=\Exercices,Field=\Orthonormalisation] akp. 
Pour $E=\sc C\b([-1,1]\b)$ muni du produit scalaire 
$$
\langle f,g\rangle=\int_{-1}^1f(t)g(t)\d t,
$$ 
calculer la distance de l'application $t\mapsto \sh t$ à l'espace des fonctions polynômes de degré inférieur à $2$. 

\exo [Origin=\Lakedaemon,Level=2,Fight=1,Learn=1,Type=\Cours,Field=\Courbes] akq. 
Étudier l'arc paramétré $\Phi:t\mapsto(\cos^3t,\sin^3t)$ pour $I=\ob R$. 

\exo [Origin=\Lakedaemon,Level=1,Fight=0,Learn=0,Type=\Cours,Field=\Suites] akr. 
Prouver que la suite de terme général $u_n:=({1\F 3\e^n+2}, {2n\F n+1}, 1)$ converge vers $(0,2,1)$ dans $\ob R^3$. 

\exo [Origin=\Lakedaemon,Level=1,Fight=0,Learn=0,Type=\Cours,Field=\Suites] aks. 
Prouver que la suite de terme général $u_n:=({5n^2-3\F n^2+1},1+\e^{-n})$ converge vers $(5,1)$. 

\exo [Origin=\Lakedaemon,Level=2,Fight=1,Learn=1,Type=\Exercices,Field=\FonctionsDePlusieursVariables] aks. 
Prolonger par continuité en $(0,0)$ pour l'application $f$ définie par 
$$
\forall (x,y)\neq(0,0), \qquad f(x,y)={xy^2\F x^2+y^2}. 
$$

\exo [Origin=\Lakedaemon,Level=2,Fight=0,Learn=0,Type=\Cours,Field=\Fonctions] akt. 
Prouver que la fonction $f:x\mapsto \Q[{1\F x}\W]$ est continue par morceaux sur l'intervalle $\Q]0, \infty\W[$ mais ne l'est pas sur le segment $[0, 1]$, si l'on pose $f(0):=0$.  

\exo [Origin=\Lakedaemon,Level=2,Fight=0,Learn=0,Type=\Cours,Field=\Fonctions] aku. 
Prouver que la fonction $\pi$-périodique $f$ uniquement déterminée par $f(x):=x\ \,(0\le x\le \pi)$ est de classe $\sc C^\infty$ par morceaux.  

\exo [Origin=\Lakedaemon,Level=2,Fight=1,Learn=1,Type=\Exercices,Field=\IntégralesGénéralisées] akv. 
Prouver que $\ds \int_1^\infty\Q({1\F t}-{1\F t+1}\W)\d t=\int_1^2{\d t\F t}=\ln(2)$ sans écrire d'horreur. 

\exo [Origin=\Lakedaemon,Level=2,Fight=1,Learn=1,Type=\Exercices,Field=\IntégralesGénéralisées] akw. 
Calculer $I:=\ds\int_0^{2\pi}\sqrt{1+\cos t}\d t$ et montrer que $I\neq 0$. 

\exo [Origin=\Lakedaemon,Level=2,Fight=2,Learn=1,Type=\Exercices,Field=\DéveloppementsLimités|\IntégralesGénéralisées] akx. 
Déterminer un développement asymptotique à l'ordre $2$ de $f(x)=\arcsin(x)$ en $x=1^-$. 

\exo [Origin=\Lakedaemon,Level=2,Fight=1,Learn=1,Type=\Exercices,Field=\SériesEntières] aky. 
Calculer le rayon de convergence de $\sum\limits_{n=0}^\infty\e^nz^{n^2}$. 

\exo [Origin=\Lakedaemon,Level=2,Fight=1,Learn=1,Type=\Exercices,Field=\SériesEntières] akz. 
Calculer le rayon de convergence de $\sum\limits_{n=0}^\infty n^{(-1)^n}z^n$. 

\exo [Origin=\Lakedaemon,Level=2,Fight=1,Learn=1,Type=\TravauxDirigés,Field=\SériesEntières] ala. 
Calculer le rayon de convergence de $\sum\limits_{n=0}^\infty\e^{\sin(n)}z^n$. 

\exo [Origin=\Lakedaemon,Level=2,Fight=1,Learn=1,Type=\Cours,Field=\SériesEntières] alb. 
Calculer le rayon de convergence de $\sum\limits_{n=0}^\infty\cos(n)z^n$. 

\exo [Origin=Fac,Level=2,Fight=1,Learn=0,Type=\Cours,Field=\FormesMultilinéaires] alc. 
Prouver que l'on définit une forme trilinéaire $\phi:\ob R^3\times\ob R^3\times\ob R^3\to\ob R$ sur $\ob R^3$ en posant   
$$
\phi\Q(
\pmatrix{x_1\cr x_2\cr x_3\cr}, 
\pmatrix{y_1\cr y_2\cr y_3\cr}, 
\pmatrix{z_1\cr z_2\cr z_3\cr}\W)
:=x_1y_1z_1 +x_2y_2z_2 +x_3y_3z_3
$$

\exo [Origin=Fac,Level=2,Fight=1,Learn=1,Type=\Cours,Field=\FormesMultilinéaires] ald. 
Montrer que l'ensemble des formes bilinéaires sur $\ob R^2$ est un espace vectoriel. En donner une base.

\exo [Origin=\Lakedaemon,Level=2,Fight=1,Learn=1,Type=\Cours,Field=\FormesMultilinéaires] ale. 
Donner toutes les formes bilinéaires alternées sur $\ob R^2$. 

\exo [Origin=Fac,Level=2,Fight=1,Learn=1,Type=\Cours,Field=\FormesMultilinéaires] alf. 
Donner toutes les formes trilinéaires alternées sur $\ob R^2$. Plus généralement, que dire des formes $m$-linéaires alternées sur un espace de dimension $n$ lorsque $ m>n$ ? 

\exo [Origin=\Lakedaemon,Level=2,Fight=1,Learn=1,Type=\Cours,Field=\FormesMultilinéaires] alg. 
Prouver que l'on définit une forme trilinéaire $f$, alternée sur $\ob R^3$, en posant 
$$
\forall (\vec u,\vec v,\vec w)\in\ob R^3, \qquad f(\vec u,\vec v,\vec w):=(u\wedge v).\vec w.
$$

\exo [Origin=Fac,Level=2,Fight=1,Learn=1,Type=\Cours,Field=\Déterminant] alh. 
Les nombres $119$, $153$ et $289$ sont divisibles par 17. Montrer sans le calculer que $\Q|\matrix{
1&1&9\cr 
1&5&3\cr 
2&8&9\cr
}\W|$ l'est aussi. 

\exo [Origin=\Quercia,Level=2,Fight=1,Learn=1,Type=\Cours,Field=\Déterminant,Indication={essayer avec $A = \pmatrix{
1&\lambda\cr
\lambda&1\cr
}$ et $B = \pmatrix{
0&\lambda\cr
-\lambda&0\cr}$.}] ali. 
Montrer que  $\det(A^2 + B^2)\ge 0$ pour des matrices $A$ et $B$ de $\sc M_n(\ob R)$ qui commuttent. \pn
2) Chercher deux matrices $A$ et $B$ ne commutant pas telles que $\det(A^2 + B^2) < 0$. \pn


\exo [Origin=,Level=2,Fight=1,Learn=1,Type=\Exercices,Field=\Diagonalisation] alj.
Diagonaliser la matrice $A:=\pmatrix{\ob O&&&1\cr
&&\LD@addots&\cr
&\LD@addots&&\cr
1&&&\ob O\cr}$. 

\exo [Origin=,Level=2,Fight=1,Learn=1,Type=\Exercices,Field=\PolynômesCaractéristiques] alk.
Calculer le polynôme caractéristique $P_A$ de 
la matrice $A:=\pmatrix{0&1&&\ob O\cr
&\ddots&\ddots&\cr
\ob O&&0&1\cr
a_0&a_1&\ldots&a_{n-1}
\cr
}$ \vskip-1em \noindent pour $(a_0,\cdots,a_{n-1})\in\ob C^n$. \vskip1em

\exo [Origin=,Level=2,Fight=1,Learn=1,Type=\Exercices,Field=\VecteursPropres] all.
Soient $n\in\ob N^*$ et $(a,b)\in\ob R^2$. 
Pour $P\in\ob R_n[X]$, on pose $u(P):=-a(X-b)P'-nP$. \pn
a) Prouver que $u$ est un endomorphisme de $\ob R_n[X]$. \pn 
b) Déterminer les éléments propres de $u$. 

\exo [Origin=,Level=1,Fight=1,Learn=1,Type=\Exercices,Field=\SystèmesLinéaires] alm.
Enoncer une condition nécessaire et suffisante 
sur $\lambda\in\ob R$ 
pour qu'il existe au moins une~solution du système
$$
\Q\{\eqalign{%
\lambda x+y+z&=1
\cr
x+(\lambda-1)y+z&=\lambda
\cr
x+2y+\lambda z&=1\cr}\W.
$$

\exo [Origin=,Level=2,Fight=1,Learn=2,Type=\TravauxDirigés,Field=\Matrices] aln.
Soient $P\in\ob C[X]$ et $A\in\sc M_n(\ob C)$ tels que $P(0)\neq0$ et $P(A)=0$. 
\pn
Prouver que $A$ est inversible. 

\exo [Origin=,Level=2,Fight=0,Learn=0,Type=\TravauxDirigés,Field=\Diagonalisation,Solution={$P:=\pmatrix{0&-1&1\cr1&0&1\cr0&1&-2}$ et $D:=\pmatrix{1&0&0\cr0&1&0\cr0&0&0\cr}$}] alo. 
Diagonaliser, si c'est possible, la matrice $A:=\pmatrix{2&0&1\cr1&1&1\cr-2&0&-1\cr}$. 

\exo [Origin=,Level=2,Fight=1,Learn=1,Type=\Exercices,Field=\Diagonalisation,Indication={Diagonaliser !}] alp.  
On définit par récurrence les suites $u$, $v$ et $w$ par $u_0:=1$, $v_0:=0$, $w_0:=1$ et 
$$
\Q\{\eqalign{
12 u_{n+1}&=9u_n\hfill+3w_n
\cr
12 v_{n+1}&=3u_n+8v_n+3w_n
\cr
12 w_{n+1}&=\hfill4v_n+6w_n
\cr
}\W.\qquad(n\in\ob N). 
$$
Déterminer la limite de ces trois suites. 

\exo [Origin=,Level=2,Fight=1,Learn=0,Type=\Exercices,Field=\Trigonalisation,Origin=,Solution={$P=\pmatrix{0&1&-1&0\cr2&2&-1&0\cr0&0&0&1\cr1&0&1&0}$ et $T=\pmatrix{-1&0&0&0\cr0&-1&0&0\cr0&0&2&-1\cr0&0&0&2}$}] alq.  
Trigonaliser sur le corps des réels la matrice 
$A:=\pmatrix{
-7&3&1&-6
\cr
-6&2&1&-6
\cr
0&0&2&0
\cr
6&-3&-1&5
\cr}
$. 

\exo [Origin=,Level=1,Fight=2,Learn=2,Type=\Cours,Field=\Matrices] alr.
Montrer qu'il n'existe pas de matrice $A\in\sc M_3(\ob R)$ telle que $A^2+I_3=0$. 

\exo [Origin=,Level=1,Fight=3,Learn=2,Type=\Exercices,Field=\Matrices,Indication={b) on rappelle que les $\{E_{i,j}\}_{1\le i,j\le n}$ forment une base de $\sc M_n(\ob R)$.}] als.
Pour $(i,j)\in\{1,\cdots,n\}$, on note $E_{i,j}$ 
la matrice de $\sc M_n(\ob R)$ dont tous les coefficients sont nuls 
sauf celui situé à la ligne $i$ et à la colonne $j$ qui est égal à $1$. 
\pn
a) Exprimer le produit $E_{i,j}E_{k,\ell}$ en fonction de $E_{i,\ell}$ 
et du symbôle de Kronecker 
$$
\delta_{j,k}:=\Q\{\eqalign{ 
1 \hbox{ si} j=k,
\cr
0 \hbox{ si }j\neq k
\cr}\W.
$$ 
b) Soit $\Phi$ une forme linéaire sur $\sc M_n(\ob R)$ vérifiant 
$$
\Phi(AB)=\Phi(BA)\qquad\b((A,B)\in\sc M_n(\ob R)\b). 
$$
Démontrer qu'il existe $\lambda\in\ob K$ tel que $\Phi(M)=\lambda\rm{tr}\ M$ pour $M\in\sc M_n(\ob R)$. 


\exo [Origin=,Level=2,Fight=1,Learn=0,Type=\Exercices,Field=\Trigonalisation,Solution={$P=\pmatrix{0&0&-1&1\cr-1&0&1&0\cr1&0&0&0\cr0&1&0&0}$ et $T=\pmatrix{-1&0&0&0\cr0&-1&0&0\cr0&0&1&-1\cr0&0&0&1}$}] alt.
Trigonaliser sur le corps des réels la matrice
$A:=\pmatrix{
2&1&1&0
\cr
-1&0&1&0
\cr
0&0&-1&0
\cr
0&0&0&-1
\cr}
$. 

\exo [Level=2,Fight=1,Learn=0,Field=\IntégralesGénéralisées,Type=\Exercices,Origin=\Fac,Indication={procéder au changement de variable $t=x^6$.},Solution={$I:={\pi\F12}$}] alu. 
Convergence de $\ds I:=\int _0^\infty{x^5\d x\F x^{12}+1}$. 

\exo [Origin=,Level=2,Fight=1,Learn=1,Type=\TravauxDirigés,Field=\Diagonalisation|\RécurrencesLinéaires] alv.
Soit $P:=X^2-sX+p$ un polynôme 
admettant $2$ racines distinctes $\lambda_1$, $\lambda_2$ et soit $\{u_n\}_{n\in\ob N}$ 
une suite vérifiant 
$$
u_{n+2}:=su_{n+1}-pu_n\qquad(n\ge0). 
$$
Déterminer la suite $\{u_n\}_{n\in\ob N}$ 
en fonction de $\lambda_1$, $\lambda_2$, $n$, $u_0$ et $u_1$. 

\exo [Origin=,Level=2,Fight=2,Learn=2,Type=\TravauxDirigés,Field=\Continuité] alw.
Soient $p,q$ deux entiers supérieurs à $1$ et $f:\ob R^p\to\ob R^q$. \pn
1) Montrer que $f\in\sc C(\ob R^p,\ob R^q)\Longleftrightarrow f^{-1}(O)$ 
est un ouvert de $\ob R^p$ pour chaque ouvert $O$ de $\ob R^q$. \pn
2) En déduire que $f\in\sc C(\ob R^p,\ob R^q)\Longleftrightarrow f^{-1}(F)$ 
est un fermé de $\ob R^p$ pour chaque fermé $O$ de $\ob R^q$. \pn
3) Prouver que $\sc Gl_n(\ob R)$ est ouvert dans $\sc M_n(\ob R)$. \pn
4) De même, prouver que l'ensemble des matrices symétriques est fermé. 


\exo [Origin=,Level=2,Fight=1,Learn=1,Type=\Exercices,Field=\Continuité] alx.
Montrer que l'application $(x,y)\mapsto\ds{\sin x+\sin y\F x+y}$ 
est bornée sur $\{(x,y)\in\ob R^2:x+y\neq0\}$. 

\exo [Level=2,Fight=0,Learn=0,Field=\Séries,Type=\Colles,Origin=\MP] aly. 
Nature de la série $\sum_{n=1}^\infty{2^n n!\F n^n}$.

\exo [Level=2,Fight=1,Learn=0,Field=\Séries,Type=\Colles,Origin=\MP] alz. 
Nature de la série $\sum_{n=1}^\infty\ds \Q({\ln n\F\ln (n+1)}\W)^{n^2}$.

\exo [Level=2,Fight=2,Learn=2,Field=\Séries,Type=\Colles,Origin=\MP] ama.
Nature de la série $\ds\sum_{n=1}^\infty\sin\b(\pi\sqrt{n^2+2n}\b)$.

\exo [Level=2,Fight=2,Learn=2,Field=\Séries,Type=\Colles,Origin=\MP] amb.
Nature de la série $\ds\sum_{n=1}^\infty{(-1)^n\F n^{2/3}+\cos n}$.

\exo [Level=2,Fight=2,Learn=2,Field=\Séries,Type=\Colles,Origin=\MP] amc.
Nature de la série $\ds\sum_{n=2}^\infty(-1)^n\int_n^\infty{\d x\F x^{\alpha+1}+1}$ lorsque $\alpha>0$.

\exo [Level=2,Fight=0,Learn=0,Field=\SériesEntières,Type=\Exercices,Origin=] amd. 
Rayon de convergence $R$ de la série entière $\sum_{n\ge0}\ds{\sh n\F\ch^2 n}x^n$. 

\exo [Level=2,Fight=2,Learn=1,Field=\SériesEntières,Type=\Exercices,Origin=] ame. 
Rayon de convergence $R$ de la série entière $\sum_{n\ge1}\b(\cos(1/n)\b)^{n^\alpha}x^n$. 

\exo[Level=2,Fight=2,Learn=1,Field=\Déterminant,Type=\Colles,Origin=\MP]  amf. 
Pour $n\ge2$ et $(x_1,\cdots, x_n)\in\ob R^n$,  
Calculer le déterminant de la matrice $A:=\Q(\sin(x_i+x_j)\W)_{1\le i,j\le n}$. 

\exo [Level=2,Fight=1,Learn=0,Field=\IntégralesGénéralisées,Type=\Exercices,Origin=\Fac,Indication={Multiplier en haut et en bas par $\sh x$, utiliser que $\sh^2x=\ch^2x-1$,  procéder au changement de variable $u=\ch x$, puis primitiver.},Solution={$\ds I:=\ln{\e-1\F\e+1}$.}] amg. 
Nature et calcul de $\ds I:=\int _0^\infty{\d x\F\sh(x)}$. 

\exo [Origin=\Lakedaemon,Level=1,Fight=3,Learn=2,Type=\Exercices,Field=\Matrices] amh. 
Exprimer la matrice $A^n$ en fonction de $A=\pmatrix{1&1\cr1&0\cr}$,  de $\mbox I_2$ et de $n\ge1$.

\exo [Origin=\Fac,Level=2,Fight=1,Learn=1,Type=\Exercices,Field=\VecteursPropres] ami. 
Soit $ m \in\ob R$ et $ A_m\in M_3(\ob R)$ la matrice $\pmatrix{m & 1 & 1 \cr 1 & m & 1 \cr 1 & 1 & m\cr}$.\pn
a. Calculer les valeurs propres de $A_m$ et une base de vecteurs propres.\pn
b. Déterminer suivant les valeurs de $m$ le rang de $A_m$. Déterminer lorsque cela est possible $A_m^{-1}$. 
c. Lorsque $A_m$ n'est pas inversible, déterminer le noyau et l'image de $A_m$. 

\exo [Origin=\Fac,Level=2,Fight=1,Learn=1,Type=\Exercices,Field=\Diagonalisation] amj. 
a. Soient $f$ et $g$ deux endomorphismes d'un espace vectoriel $E$ de dimension $n$ sur $\ob K=\ob R$ ou $\ob C$, ayant chacun $ n$ valeurs propres distinctes dans $\ob K$. \pn
Montrer que
$$
f \circ g = g \circ f \ssi   f\hbox{ et } g \hbox{ ont les mêmes valeurs propres}.
$$
b. Supposons maintenant que $ K=\ob C$ et que $f\circ g =g\circ f$. Montrer que tout sous-espace propre de $ f$ est stable par $g$. \pn
{\it  Remarque : On peut montrer par récurrence sur $n$ qu'il existe un vecteur propre commun à $f$ et $g$. On admettra ce résultat.}\pn
c. Considérons $f$ et $g$ deux endomorphismes de $\ob R^3 $ dont les matrices dans la base canonique sont respectivement
$$
M = \pmatrix{1 & 0 & 0 \cr 0 & 0 & -1 \cr 0 & 1 & 2 \cr} \hbox{ et }     N = \pmatrix{0 & 1 & 1 \cr -1 & 1 & -1 \cr  1 & 1 & 3 \cr}.
$$
Vérifier que $f \circ g=g \circ f$ et déterminer les sous-espaces propres de $M$ et $N$. \pn
Déterminer une base de $\ob R^3 $ dans laquelle les matrices de $f$ et $g$ sont diagonales.

\exo [Origin=\Fac,Level=2,Fight=1,Learn=1,Type=\Exercices,Field=\PolynômesCaractéristiques] amk. 
Soient $u$ et $ v$ deux endomorphismes d'un espace vectoriel $E$ de dimension finie. Montrer que $ u\circ v$ et $ v\circ u$ ont les mêmes valeurs propres. 

\exo [Origin=\Fac,Level=2,Fight=1,Learn=1,Type=\Exercices,Field=\ValeursPropres] aml. 
Soient $A$ et $B$ deux matrices de $\ob M_n(\ob R)$ telles que
$$
A=AB-BA.
$$
Le but de cet exercice est de montrer que $A$ est nilpotente, c'est-à-dire qu'il existe $k\in\ob N$ tel que $A^k=0$. \pn
On note $ E$ l'espace vectoriel $\sc M_n(\ob R)$ et l'on considère l'application :    
$$
\eqalign{
\psi : E & \rightarrow  E\cr 
M &\mapsto MB-BM}
$$
a. Montrer que $ \psi$ est un endomorphisme de $E$ dans $ E$. \pn
b. Montrer que $\psi(A^k)=kA^k$ pour $k\in\ob N$. \pn 
c. On suppose que $A^{k}\neq0$ pour $k\in\ob N$. Montrer que $\psi$ a une infinité de valeurs propres. \pn
d. Conclure.

\exo [Origin=\Fac,Level=2,Fight=1,Learn=1,Type=\Exercices,Field=\VecteursPropres] amm. 
On considère l'application suivante :
$$
\eqalign{
f: \ob R_n[X] &\rightarrow\ob R_{n}[X] \cr 
P &\mapsto (X^2-1)P'-(nX+a)P. \cr}
$$
Vérifier que cette application est bien définie. Puis, déterminer ses valeurs propres et leurs espaces propres associés. 

\exo [Origin=\Quercia,Level=2,Fight=1,Learn=0,Type=\Exercices,Field=\ValeursPropres,Sol=$0$ et les racines de $6X^2 -6nX -n(n-1)(2n-1)=0$] amn. 
Chercher les valeurs propres de la matrice $\pmatrix{
0      &\dots  &0      &1      \cr
\vdots &       &\vdots &\vdots \cr
0      &\dots  &0      &n-1    \cr
1      &\dots  &n-1    &n      \cr 
}$.

\exo [Origin=\Quercia,Level=2,Fight=1,Learn=0,Type=\Exercices,Field=\ValeursPropres,Solution={Les valeurs propres sont $\sin\alpha+\sin2\alpha$, $-\sin\alpha$ et $-\sin2\alpha$.}] amo. 
Chercher les valeurs propres de la matrice $\pmatrix{
0            &\sin\alpha &\sin2\alpha \cr
\sin\alpha   &0          &\sin2\alpha \cr
\sin2\alpha  &\sin\alpha &0
}$.

\exo [Origin=\CCP,Level=2,Fight=2,Learn=1,Type=\Cours,Field=\ValeursPropres,Solution={Le spectre de $T$ est $\Q]-1,1\W]$, la fonction $x\mapsto \lambda^{[x]}$ est un vecteur propre pour la valeur propre $\lambda$.}] amp. 
Soit $E$ l'espace vectoriel des fonctions $f:\ob R^+\to\ob R$ admettant une limite finie en $+\infty$. Pour chaque fonction $f\in E$, on pose
$$
\forall x\ge0, \qquad T(f)(x):=f(x+1).
$$
a. Montrer que l'opérateur $T:f\mapsto T(f)$ définit  un endomorphisme de $E$. \pn
b. Déterminer les valeurs propres de l'opérateur $T$.


\exo [Origin=\Quercia,Level=2,Fight=2,Learn=1,Type=\Cours,Field=\VecteursPropres,Solution={$\lambda\in\Q]0,1\W[$ de vecteur propre associé $f(x) = Cx^{1/\lambda-1}$}] amq. 
Pour chaque fonction $f$ de l'espace $E=\sc C\big(\Q[0,+\infty\W[,\ob R\big)$, on pose 
$$
\forall x\ge0, \qquad T(f)(x) = \Q\{\eqalign{
&{1\F x}\int_0^x f(t)\d t\hbox{ si }x>0\cr
&f(0)\hbox{ si }x=0.}
\W.
$$
a. Montrer que l'opérateur $T:f\mapsto T(f)$ définit  un endomorphisme de $E$. \pn
b. Déterminer les valeurs propres et les vecteurs propres de l'opérateur $T$.

\exo [Origin=\Fac,Level=2,Fight=0,Learn=0,Type=\TravauxDirigés,Field=\Diagonalisation,Annee=2007,Solution={$P:=\pmatrix{1&-1&1\cr0&1&2\cr1&0&1}$ et $D:=\pmatrix{3&0&0\cr0&3&0\cr0&0&5\cr}$}] amr. 
Diagonaliser la matrice $A:=\pmatrix{ 4&1&-1\cr 2&5&-2\cr 1&1&2}$.  

\exo [Origin=\Lakedaemon,Level=2,Fight=0,Learn=0,Type=\TravauxDirigés,Field=\Diagonalisation,Annee=2007,Solution={$P:=\pmatrix{1&1&1\cr-1&1&0\cr1&0&1}$ et $D:=\pmatrix{-1&0&0\cr0&0&0\cr0&0&2\cr}$}] ams. 
Diagonaliser la matrice $A:=\pmatrix{ -3&3&5\cr1&-1&-1\cr-3&3&5}$.  

\exo [Origin=\Fac,Level=2,Fight=-1,Learn=0,Type=\TravauxDirigés,Field=\Diagonalisation,Solution={$P:=\pmatrix{1&-2\cr1&1}$ et $D:=\pmatrix{5&0\cr0&-1}$}] amt. 
Diagonaliser la matrice $A=\pmatrix{1&4\cr 2&3}$. 

\exo [Origin=\Fac,Level=2,Fight=0,Learn=0,Type=\Exercices,Field=\Diagonalisation] amu. 
Trouver les valeurs propres de $A=\pmatrix{1&4\cr 2&3}$ et les sous-espaces propres correspondants. 
En déduire une matrice inversible $P$ telle que $ P^{-1}AP$ soit diagonale. 

\exo [Level=2,Fight=0,Learn=0,Field=\SériesEntières,Type=\Exercices,Origin=] amv. 
Déterminer le rayon de convergence de la  série entière $\ds\sum_{n\ge0}n^{3/2}\e^nz^{n^3}$.

\exo [Level=2,Fight=0,Learn=0,Field=\SériesEntières,Type=\Exercices,Origin=] amw. 
Déterminer le rayon de convergence de la  série entière $\ds\sum_{n\ge0}{(n!)^2\F(2n)!}z^n$.

\exo [Level=2,Fight=0,Learn=0,Field=\SériesEntières,Type=\Exercices,Origin=] amx. 
Déterminer le rayon de convergence de la  série entière $\ds\sum_{n\ge0}n!z^{n^2}$. 

\exo [Level=2,Fight=1,Learn=0,Field=\SériesEntières,Type=\Exercices,Origin=] amy. 
Calculer la somme de la série entière $\ds\sum_{n\ge0}x^n/(n+2)$.

\exo [Level=2,Fight=1,Learn=0,Field=\SériesEntières,Type=\Exercices,Origin=] amz. 
Calculer la somme de la série entière $\ds\sum_{n\ge1}{x^n\F n(n+1)}$.

\exo [Level=2,Fight=1,Learn=0,Field=\SériesEntières,Type=\Exercices,Origin=] ana. 
Calculer la somme de la série entière $\ds\sum_{n\ge0}{n-x\F n!}x^n$.

\exo [Level=2,Fight=1,Learn=0,Field=\SériesEntières,Type=\Exercices,Origin=] anb. 
Calculer la somme de la série entière $\ds\sum_{n\ge0}{3+(-1)^n\F 2^{n+1}}x^n$.

\exo [Level=2,Fight=1,Learn=0,Field=\SériesEntières,Type=\Exercices,Origin=] anc. 
Calculer la somme de la série entière $\ds\sum_{n\ge0}(n^2+3n+1)x^n$.

\exo [Level=2,Fight=1,Learn=0,Field=\SériesEntières,Type=\Exercices,Origin=] and. 
Calculer la somme de la série entière $\ds\sum_{n\ge0}{x^{4n}\F 4n+1}$.

\exo [Level=2,Fight=1,Learn=0,Field=\SériesEntières,Type=\Exercices,Origin=] ane. 
Calculer la somme de la série entière $\ds\sum_{n\ge0}{x^n\F4n^2-1}$. 

\exo [Level=2,Fight=2,Learn=2,Field=\FonctionsDePlusieursVariables,Type=\TravauxDirigés,Origin=] anf. 
Existence et calcul de la limite $\ds \lim_{(x,y)\to(0,0)\atop x>0}{x^2y^2\F x^3+y^4}$.

\exo [Level=2,Fight=2,Learn=2,Field=\FonctionsDePlusieursVariables,Type=\TravauxDirigés,Origin=] ang. 
Existence et calcul de la limite $\ds \lim_{(x,y)\to(0,0)\atop x>0}{x^2y^2\F x^3+2y^6}$. 

\exo [Level=2,Fight=2,Learn=2,Field=\FonctionsDePlusieursVariables,Type=\TravauxDirigés,Origin=] anh. 
Étudier si la fonction  $f(x,y):=\ds{x^2y^2\F x^3+y^4}$ est bornée sur $\Q]0,\infty\W[^2$.
 
\exo [Level=2,Fight=2,Learn=2,Field=\FonctionsDePlusieursVariables,Type=\TravauxDirigés,Origin=] ani. 
Étudier si la fonction  $f(x,y):=\ds{x^2y^2\F x^3+y^6}$ est bornée sur $\Q]0,\infty\W[^2$. 

\exo [Level=2,Fight=2,Learn=2,Field=\FonctionsDePlusieursVariables,Type=\TravauxDirigés,Origin=] anj. 
Étudier si la fonction  $f(x,y):=\ds{xy\e^{x+y}\F\e^{x^2+y^2}-1}$ est bornée sur $\Q]0,\infty\W[^2$. 

\exo [Level=2,Fight=1,Learn=1,Type=\Exercices,Field=\FonctionsDePlusieursVariables,Origin=] ank. 
Étudier la différentiabilité en $(0,0)$ de la fonction définie par 
$$
f(x,y):=\Q\{\eqalign{
&{x^3-3x^2y^2+xy^2\F x^2+y^2}\hbox{ pour }(x,y)\neq(0,0)\cr
&0\hbox{ sinon}
}\W.
$$

\exo [Level=2,Fight=1,Learn=1,Type=\Exercices,Field=\FonctionsDePlusieursVariables,Origin=] anl. 
Étudier la différentiabilité en $(0,0)$ de la fonction définie par 
$$
f(x,y):=\Q\{\eqalign{
&{xy\F\sqrt x^2+y^2}\hbox{ pour }(x,y)\neq(0,0)\cr
&0\hbox{ sinon}
}\W.
$$

\exo [Level=2,Fight=1,Learn=1,Type=\Exercices,Field=\FonctionsDePlusieursVariables,Origin=] anm. 
Étudier la différentiabilité en $(0,0)$ de la fonction définie par 
$$
f(x,y):=\Q\{\eqalign{
&{\sin(x+y)\F x+y}\hbox{ si }x+y\neq0\cr
&1\hbox{ sinon}
}\W.
$$

\exo [Level=2,Fight=0,Learn=0,Type=\Cours,Field=\Trigonalisation,Origin=\Lakedaemon,Solution={$P=\pmatrix{1&1\cr1&0}$ et $T=\pmatrix{2&1\cr0&2}$}] ann. 
Trigonaliser sur $\ob R$ la matrice $\pmatrix{3&-1\cr1&1\cr}$. 

\exo [Level=2,Fight=2,Learn=2,Type=\Problèmes,Field=\Diagonalisation,Origin=Maroc06] ano. 
On note $\sc B=\{\vec i,\vec j,\vec k\}$ la base canonique de $\ob R^3$ et l'on considère les matrices carrées suivantes 
$$
A:=\pmatrix{
5&5&-14\cr
6&6&-16\cr
5&5&-14
}
\qquad\hbox{et}\qquad
B:=\pmatrix{
8&4&-16\cr
0&4&-8\cr
4&4&-12
}
$$
1. Déterminer les valeurs propres de $A$ ainsi que les sous-espaces propres associés. 
\medskip\noindent
2. Vérifier que les vecteurs $\vec I:=(1,2,1)$, $\vec J:=(1,-1,0)$ et $\vec K:=(1,1,1)$ forment une base $\sc C$~de~$\ob R^3$. 
\medskip\noindent
3. Justifier que la matrice de passage $P$ de la base $\sc B$ à la base $\sc C$ est inversible et calculer $P^{-1}$. 
\medskip\noindent
4. Calculer la matrice produit $D:=P^{-1}AP$. 
\medskip\noindent
5. On cherche les matrices $M$, réelles d'ordre $3$, telles que $AM=MA$. \pn
a. Soit $M$ une telle matrice et soit $N:=P^{-1}MP$. Montrer que 
$$
AM=MA\ssi ND=DN.
$$
b. Déterminer toutes les matrices $N$, réelles d'ordre $3$, telles que $ND=DN$. 
\pn
c. En déduire l'ensemble des matrices $M$, réellles d'ordre $3$, telles que $AM=MA$. 
\medskip\noindent
6. Existe-t-il une matrice $Q$, réelle d'ordre $3$, telle que $Q^2=A$ ?
\medskip\noindent
7. Soit $(X_n)_{n\ge0}$ la suite des matrices de trois lignes et une colonne, définie par 
$$
X_0:=\pmatrix{1\cr0\cr1},\qquad X_1:=\pmatrix{0\cr-1\cr1}\quad\hbox{et}\quad X_{n+2}=AX_{n+1}+BX_n\qquad (n\ge0).
$$ 
Pour tout entier $n\ge0$, on note $Y_n=P^{-1}X_n$ et on note $Y_n=\pmatrix{u_n\cr v_n\cr w_n}$. \pn
a. Calculer la matrice produit $D_1=P^{-1}BP$. \pn
b. Calculer $Y_0$ et $Y_1$. \pn
c. Pour $n\ge0$, montrer que $Y_{n+2}=DY_{n+1}+D_1Y_n$. 
\pn
d. Pour $n\ge0$, vérifier que 
$$
\Q\{\eqalign{
u_{n+2}&=u_{n+1}\cr
v_{n+2}&=4v_n\cr
w_{n+2}&=-4w_{n+1}-4w_n
}\W.
$$
puis donner les expressions explicites de $u_n$, $v_n$ et $w_n$ en fonction de $n$. 
\pn
e. Pour $n\ge0$, donner l'expression explicite de $X_n$ en fonction de $n$. 
\medskip\noindent
8. On considère trois fonctions dérivables $u$, $v$, $w$ de $\ob R$ dans $\ob R$ vérifiant le système
$$
\Q\{\eqalign{
 u'(t)&=8u(t)+4v(t)-16w(t)\cr
v'(t)&=4v(t)-8w(t)\cr
w'(t)&=4u(t)+4v(t)-12w(t)
}\W.
\qquad (t\in\ob R)\leqno{(1)}
$$
a. Pour $t\in\ob R$, justifier l'existence d'un unique triplet $\big(x(t), y(t), z(t)\big)$ de nombres réels tel que 
$$
\big(u(t), v(t), w(t)\big)=x(t)\vec I+y(t)\vec J+z(t)\vec K. 
$$
b. Exprimer les applications $x$, $y$, $z$ en fonction de $u$, $v$, $w$. En déduire qu'elles sont dérivables~sur~$\ob R$.
\pn
c. Montrer que le système $(1)$ équivaut au système
$$
\Q\{\eqalign{
x'(t)&=0\cr
y'(t)&=4y(t)\cr
z'(t)&=-4z(t)
}\W.\qquad (t\in\ob R).\leqno{(2)}
$$
d. On suppose que $u(0)=1$ et que $v(0)=w(0)=0$, calculer alors $x(0)$, $y(0)$ et $z(0)$. 
\pn
e. Résoudre le système $(2)$ avec les conditions initiales $x(0)$, $y(0)$ et $z(0)$ trouvées à la question d. 
\pn
f. En déduire la solution de $(1)$ vérifiant les conditions initiales $u(0)=1$ et $v(0)=w(0)=0$. 

\exo [Level=1,Fight=2,Learn=2,Type=\Problèmes,Field=\Intégration,Origin=Maroc06] anp. 
1. Soient $\alpha$ et $\beta$ deux réels positifs ou nuls et $g_{\alpha,\beta}$ la fonction définie sur l'intervalle $\Q]0,1\W[$ par 
$$
g_{\alpha,\beta}(t)=t^\alpha(1-t)^\beta.
$$
a. Montrer que $g_{\alpha,\beta}$ peut se prolonger en une fonction continue à droite en $0$ et à gauche en $1$, que l'on notera encore $g_{\alpha, \beta}$. Préciser $g_{\alpha,\beta}(0)$ et $g_{\alpha,\beta}(1)$ selon les valeurs de $\alpha$ et $\beta$. 
\medskip\noindent
Dans la suite, on pose
$$
I(\alpha,\beta):=\int_0^1g_{\alpha,\beta}(t)\d t=\int_0^1t^\alpha(1-t)^\beta\d t. 
$$
b. Calculer $I(\alpha,0)$. \pn
c. Comparer $I(\alpha,\beta)$ et $I(\beta,\alpha)$. \medskip\noindent
d. Trouver une relation entre $I(\alpha+1,\beta)$ et $I(\alpha,\beta+1)$. \medskip\noindent
e. En déduire soigneusement que l'on a 
$$
\forall n\in\ob N, \qquad I(\alpha,n)={n!\F (\alpha+1)(\alpha+2)\cdots(\alpha+n+1)}.
$$
2. Pour $a>0$, on note $f_a$ la fonction définie par 
$$
f_a(x)=x\ln\Q(1-{a\F x}\W).
$$
a. Préciser le domaine de définition de la fonction $f_a$. 
\medskip\noindent
b. Si $x$ et $a$ sont deux réels vérifiant $0<a<x$, montrer que 
$$
{a\F x}\le\ln(x)-\ln(x-a)\le{a\F x-a}.
$$
c. En déduire les variations de la restriction de $f_a$ à l'intervalle $\Q]a,+\infty\W[$ {\it (on fera un tableau de variations)}
et préciser la nature des branches infinies de sa courbe qu'on notera $\sc C_a$. 
\medskip
\noindent
d. Donner l'allure des courbes $\sc C_1$, $\sc C_2$ et $\sc C_3$ sur un même graphique. 
\medskip\noindent
e. Soit $a>0$. On considère la suite $(y_n)_{n>a}$ définie par  
$$
\forall n>a, \qquad y_n:=\Q(1-{a\F n}\W)^n. 
$$
Préciser le sens de variation et la limite de cette suite. 
\medskip\noindent
3. Pour $x\ge0$ et $n\in\ob N^*$, on pose 
$$
 F_n(x):=\int_0^n\Q(1-{u\F n}\W)^nu^x\d u.
$$
a. Montrer que $F_n(x)=n^{x+1}I(x,n)$. \medskip\noindent
b. Soit $x\ge0$. En utilisant les résultats de la question 2, montrer que la suite $\big(F_n(x)\big)_{n\ge1}$ est croissante. 
\medskip\noindent
c. Soit $x\ge0$. \pn
i. Trouver la limite en $+\infty$ de la fonction $u\mapsto u^{x+2}\e^{-u}$ et en déduire l'existence d'un réel strictement positif $A$ tel que 
$$
\forall u\ge A, \qquad \e^{-u}\le {1\F u^{x+2}}.
$$
ii. Pour $n\ge1$, en déduire la majoration
$$
F_n(x)\le{1\F A}+\int_0^A\e^{-u}u^x\d u. 
$$
iii. Montrer alors que la suite $\big(F_n(x)\big)_{n\ge1}$ est convergente et que sa limite notée $F(x)$ vérifie la relation fonctionnelle
$$
F(x+1)=(x+1)F(x). 
$$

\exo [Level=2,Fight=1,Learn=1,Type=\Exercices,Field=\Diagonalisation,Origin=\Lakedaemon,Indication={c) Utiliser la seconde caractérisation des matrices diagonalisables}] anq. 
Soit $k\ge2$ et soit $M$ une matrice carrée. Dans cet exercice, on étudie la propriété suivante : 
\medskip
$$
\hbox{si $A:=M^k$ est inversible et diagonalisable dans $\ob C$, alors $M$ est diagonalisable dans $\ob C$.}  \leqno{(\sc P)}
$$
\medskip\noindent
a. En utilisant $M:=\pmatrix{0&1\cr0&0}$, pouver que la propriété $\sc P$ est fausse si l'on enlève l'hypothèse que $A$ est inversible. \pn
b. Pour $(\theta,\varphi)\in\ob R^2$, prouver que $M(\theta)M(\varphi)=M(\theta+\varphi)$ où l'on a posé 
$$
M(\theta):=\pmatrix{\cos\theta&-\sin\theta\cr\sin\theta&\cos\theta}\qquad (\theta\in\ob R).
$$ 
A l'aide de la matrice $M=M({\pi\F k})$, en déduire que $\sc P$ est fausse, si l'on remplace "$\ob C$" par "$\ob R$".  \pn
c. Etablir la propriété $\sc P$. 

\exo [Level=1,Fight=1,Learn=1,Type=\Exercices,Field=\Matrices,Origin=\Lakedaemon] anr. 
Soit $A$ une matrice carrée de $\sc M_n(\ob C)$. \pn
1.  Pour $k\in\ob N$, prouver que les matrices $A$ et $B:=A^k$ commutent (i.e. que $AB=BA$). \pn
2. Soit $B$ une matrice de $\sc M_n(\ob C)$ qui commute avec $A$. \pn
a. Pour $k\in\ob N$, prouver que $A^k$ commute avec $B$. \pn
b. Pour $P\in\ob C[X]$, en déduire que $P(A)$ commute avec $B$. \pn
c. Pour $(P,Q)\in\ob C[X]^2$, en déduire que $P(A)$ commute avec $Q(B)$. 

\exo [Level=2,Fight=0,Learn=1,Type=\TravauxDirigés,Field=\VecteursPropres,Origin=\Lakedaemon] ans. 
Soit $u$ et $v$ deux endomorphismes d'un espace vectoriel $E$, qui commuttent (i.e. tels que $u\circ v=v\circ u$). \pn
Pour chaque valeur propre $\lambda$ de $u$, montrer que l'espace propre $E_\lambda:=\ker(u-\lambda\hbox{Id}_E)$ 
est stable par $v$, i.e. que 
$$
v(E_\lambda)\subset E_\lambda. 
$$

\exo [Level=2,Fight=1,Learn=1,Type=\TravauxDirigés,Field=\Diagonalisation,Origin=\Lakedaemon,Indication={2) Utiliser la seconde caractérisation des matrices diagonalisables.}] ant. 
Soit $u$ un endomorphisme de $E$ et soit $F$ un sous-espace vectoriel de $E$ stable par $u$ (tel que $u(F)\subset F$). \pn
1. Prouver que la restriction de l'endomorphisme $u$ à l'espace $F$ définit un endomorphisme $v:F\to F$. \pn
2. Lorsque $u$ est diagonalisable sur $\ob K$, montrer que $v$ est diagonalisable sur $\ob K$. \pn

\exo [Level=2,Fight=2,Learn=1,Field=\IntégralesGénéralisées,Type=\Exercices,Origin=\Fac,Indication={\item{En $0$ : }Utiliser que $\ds {\e^t\F t}\Sim_0{1\F t}$ et le théorème d'intégration des équivalents\item{En $+\infty$ : }intégrer deux fois par parties et utiliser la majoration  $\ds 0\le\int_1^x{\e^t\F t^3}\d t\le {\e^x\F x^2}\quad(x\to+\infty)$.},Solution={$\ds I(x)\Sim_0\ln(x)$ et $\ds I(x)\Sim_{+\infty}{\e^x\F x}$.}] anu. 
Donner un équivalent en $0$ et en $+\infty$ de $\ds I(x):=\int_1^x{\e^t\F t}\d t$. 

\exo [Level=2,Fight=2,Learn=1,Field=\IntégralesGénéralisées,Type=\Exercices,Origin=\Fac,Indication={\item{En $0$ : }trouver un  équivalent\item{En $+\infty$ : } pour $\alpha\ge 0$, intégrer par partie et utiliser la convergence absolue. \pn Pour $\alpha<0$, prouver que $\ds\lim_{n\to+\infty}\int_0^{2n\pi}{\sin(t)\F t^\alpha}\d t=\ds\lim_{n\to+\infty}\sum_{0\le k<n}\int_0^\pi\Q({\sin(t)\F (t+2k\pi)^\alpha}-{\sin(t)\F (t+2k\pi+\pi)^\alpha}\W)\d t=+\infty$.},Solution={l'intégrale $I$ converge si, et seulement si $0<\alpha<2$.}] anv. 
Pour $\alpha\in\ob R$, étudier la nature de $\ds I:=\int_0^\infty{\sin t\F t^\alpha}\d t$. 

\exo [Level=2,Fight=0,Learn=1,Type=\Exercices,Field=\Continuité,Origin=] anw. 
Étudier l'existence de la limite  $\ds \lim_{(x,y)\to(0,0)\atop(x,y)\neq(0,0)}{\sin x\sin y\F xy}$. 

\exo [Level=1,Fight=0,Learn=0,Type=\Exercices,Field=\DéveloppementsLimités,Origin=] anx. 
Calculer un développement limité à l'ordre $2$ de $f(x)=\arctan\sqrt x$ en $1$. 

\exo [Level=1,Fight=0,Learn=0,Type=\Exercices,Field=\DéveloppementsLimités,Origin=] any. 
Calculer un développement limité à l'ordre $2$ de $f(x)=\ds{1\F\e^x-1}-{1\F x}$ en $0$. 

\exo [Level=1,Fight=0,Learn=0,Type=\Exercices,Field=\Equivalents,Origin=] anz. 
Déterminer la limite (ou un équivalent simple en cas de limite nulle ou infinie) pour $f(x)=(\sin x-1)\e^{\tan x}$ en $\ds{\pi\F2}$. 

\exo [Level=1,Fight=0,Learn=0,Type=\Exercices,Field=\Equivalents,Origin=] aoa. 
Déterminer la limite (ou un équivalent simple en cas de limite nulle ou infinie) pour $f(x)=(\sin x)^{\sh x}-(\sh x)^{\sin x}$ en $0^+$. 

\exo [Level=1,Fight=0,Learn=0,Type=\Exercices,Field=\Equivalents,Origin=] aob. 
Déterminer la limite (ou un équivalent simple en cas de limite nulle ou infinie) pour $f(x)=(x+1)^{(x+1)/x}-(x-1)^{(x-1)/x}$ en $+\infty$. 

\exo [Level=1,Fight=0,Learn=0,Type=\Exercices,Field=\Courbes,Origin=] aoc. 
Construire la courbe d'équation $\ds y=(x^2-1)\ln{x+1\F x}$. 

\exo [Level=1,Fight=0,Learn=0,Type=\Exercices,Field=\Courbes,Origin=] aod. 
Construire la courbe d'équation $\ds y=\Q(x+2-{1\F x}\W)\arctan x$. 

\exo [Level=1,Fight=1,Learn=0,Type=\Exercices,Field=\DéveloppementsLimités,Origin=] aoe. 
Effectuer un développement limité au voisinage de $0$ pour  $f(x)={\ln(1+x)\F1+x}$ à l'ordre $4$.

\exo [Level=1,Fight=1,Learn=0,Type=\Exercices,Field=\DéveloppementsLimités,Origin=] aof. 
Effectuer un développement limité au voisinage de $0$ pour $f(x)=\ln{\sin x\F x}$ à l'ordre $5$.

\exo [Level=1,Fight=1,Learn=0,Type=\Exercices,Field=\DéveloppementsLimités,Origin=] aog. 
Effectuer un développement limité au voisinage de $0$ pour $f(x)=(1+x)^x$ à l'ordre $5$. 

\exo [Level=1,Fight=1,Learn=0,Type=\Exercices,Field=\DéveloppementsLimités,Origin=] aoh. 
Effectuer un développement limité au voisinage de $0$ pour  $f(x)=\exp{x\F\tan x}$ à l'ordre $5$.

\exo [Level=1,Fight=1,Learn=0,Type=\Exercices,Field=\DéveloppementsLimités,Origin=] aoi. 
Effectuer un développement limité au voisinage de $0$ pour $f(x)={x\F\sin x}$ à l'ordre $5$. 

\exo [Level=1,Fight=1,Learn=0,Type=\Exercices,Field=\DéveloppementsLimités,Origin=] aoj. 
Effectuer un développement limité au voisinage de $0$ pour  $f(x)=(\tan x)^2$ à l'ordre $7$. 

\exo [Level=1,Fight=1,Learn=0,Type=\Exercices,Field=\DéveloppementsLimités,Origin=] aok. 
Effectuer un développement limité au voisinage de $0$ pour  $f(x)=\e^{x\sin x}$ à l'ordre $6$. 

\exo [Level=1,Fight=1,Learn=0,Type=\Exercices,Field=\DéveloppementsLimités,Origin=] aol. 
Effectuer un développement limité au voisinage de $0$ pour  $f(x)=(1+x)^{1/F x}$ à l'ordre $3$. 

\exo [Level=1,Fight=1,Learn=0,Type=\Exercices,Field=\DéveloppementsLimités,Origin=] aom. 
Effectuer un développement limité au voisinage de $0$ pour $f(x)=\sqrt{1+\e^{2x}}$ à l'ordre $3$. 

\exo [Level=2,Fight=0,Learn=1,Type=\TravauxDirigés,Field=\Dérivation,Origin=] aon. 
Déterminer si l'on définit une fonction de classe $\sc C^1$ sur $\ob R^2$ en posant $f(0,0):=0$ et 
$$
f(x,y)=\ds{xy\F x^2+y^2}\qquad  (x,y)\neq(0,0).
$$

\exo [Level=2,Fight=0,Learn=1,Type=\TravauxDirigés,Field=\Dérivation,Origin=] aoo. 
Déterminer si l'on définit une fonction de classe $\sc C^1$ sur $\ob R^2$ en posant $f(0,0):=0$ et 
$$
f(x,y):={x^3+xy^2+y^4\F x^2+y^2}\qquad(x,y)\neq(0,0).
$$

\exo [Level=2,Fight=0,Learn=1,Type=\TravauxDirigés,Field=\Dérivation,Origin=] aop. 
Déterminer si l'on définit une fonction de classe $\sc C^1$ sur $\ob R^2$ en posant $f(x,x):=2x$ si $x\in\ob R$ et 
$$
f(x,y):={\sin(y^2-x^2)\F y-x}\qquad(y\neq x).
$$

\exo [Level=2,Fight=0,Learn=0,Type=\TravauxDirigés,Field=\SystèmesDifférentiels,Origin=\Quercia,Solution={$
\Q\{\eqalign{
x&=2\alpha e^t + (2\gamma t + 2\beta-\gamma) e^{2t}\cr
y&=(\gamma t + \beta) e^{2t}\cr
z&=\alpha e^t + (\gamma t + \beta) e^{2t}}\W.$}] aoq. 
Résoudre sur $\ob R$ le système différentiel du premier ordre 
$$
\Q\{\eqalign{
x' &= 2y + 2z\cr 
y' &= -x + 2y + 2z\cr 
z'& = -x + y  + 3z.}\W.
$$

\exo [Level=2,Fight=1,Learn=0,Type=\TravauxDirigés,Field=\SystèmesDifférentiels,Origin=\Quercia,Solution={$
\Q\{\eqalign{
x&= {-3\cos t-13\sin t\F25} + (at+b)e^{2t}\cr
y&= {-4\cos t- 3\sin t\F25} + (at+a+b)e^{2t}}
\W.$}] aor. 
Résoudre sur $\ob R$ le système différentiel du premier ordre 
$$
\Q\{\eqalign{
x'&= x+y+ \sin t\cr 
y'&=-x+3y.
}\W.$$

\exo [Level=2,Fight=0,Learn=0,Type=\TravauxDirigés,Field=\EspacesPréHilbertiens,Origin=] aos. 
Dans $E$ espace vectoriel muni d'un produit scalaire, 
établir le théorème de Pythagore, c'est à dire que 
$$
\forall (x,y)\in E^2, \qquad x\perp y\ssi\|x+y\|^2=\|x\|^2+\|y\|^2.
$$

\exo [Level=2,Fight=0,Learn=2,Type=\TravauxDirigés,Field=\EspacesPréHilbertiens,Origin=] aot. 
Soit $E$ un espace vectoriel muni d'un produit scalaire et soit $F\subset E$ un espace vectoriel de dimension $n\ge1$, muni d'une base orthonormée $\{e_1, \cdots, e_n\}$. \pn
a) Montrer que l'on définit un projecteur de $E$, d'image $F$, en posant 
$$
\forall x\in E, \qquad p(x):=\sum_{k=1}^n\langle x,e_k\rangle e_k.
$$
b) Pour $x\in E$, prouver que $x-p(x)$ est orthogonal à $F$, i.e. que $x-p(x)\perp y$ pour chaque $y\in F$. \pn 
Le vecteur $p(x)$ est alors appelé la projection orthogonale de $x$ sur $F$. \pn
c) Montrer que la distance entre un vecteur $x\in E$ et l'espace $F$ est 
$$
d(x,F):=\inf_{y\in F}d(x,y)=d\big(x,p(x)\big)=\|x-p(x)\|, 
$$
i.e. montrer que la distance $d(x,y)$ atteint un minimum en $y=p(x)$ lorsque $y$ varie dans $F$. \pn
d) Expliquer pourquoi un sous-espace vectoriel $F\subset E$ de dimension $n$ admet toujours au moins une base orthonormée $\{e_1, \cdots, e_n\}$. Conclusion ?

 
\exo [Level=2,Fight=1,Learn=1,Type=\TravauxDirigés,Field=\SériesDeFourier,Origin=\Lakedaemon] aou. 
a) Trouver six relations entre les nombres 
$$
A=\sum_{n=1}^\infty{1\F n^2},  \qquad B=\sum_{n=1}^\infty{(-1)^n\F n^2}, \qquad C=\sum_{n=1\atop n\hbox{ \sevenrm pair}}^\infty{1\F n^2}\quad \hbox{et}\qquad D=\sum_{n=1\atop n\hbox{ \sevenrm impair}}^\infty{1\F n^2}
$$
i.e. trouver toutes les relations exprimant l'un deux en fonction d'un autre de ces nombres. 
\medskip\noindent
b) En utilisant Parseval, on a montré que $\ds\sum_{n=1}^\infty{1\F n^2}={\pi^2\F 6}$. En déduire $B$, $C$ et $D$. 

\exo [Level=1,Fight=1,Learn=1,Type=\Maple,Field=\GéométriePlane,Origin=\BanquePT] aov. 
Dans $\ob C$ résoudre l'équation suivante :     $$ 
z^{2}+\ol z+iz=0.
$$
Montrer que les points dont les affixes sont solutions forment un triangle rectangle.


\exo [Level=1,Fight=1,Learn=1,Type=\Maple,Field=\Polynômes,Origin=\BanquePT] aow. 
On considère le polynôme réel $P:=X^4+X^3+aX^2+\sqrt{2}X+b$. \pn
1. Déterminer $a$ et $b$ pour que $1+i$ soit racine de $P$. Calculer alors tous les zéros de $P$. \pn
2. Factoriser $P$ en facteurs irréductibles dans $\ob R[X]$ et dans $\ob C[X]$. 

\exo [Level=1,Fight=1,Learn=1,Type=\Maple,Field=\Matrices,Origin=\BanquePT] aox. 
Déterminer les matrices $B$ telles que $AB=BA$ pour $
A:=\pmatrix{1 & 2 & 3 \cr 0 & 1 & 2 \cr 0 & 0 & 1\cr}$.

\exo [Level=1,Fight=1,Learn=1,Type=\Maple,Field=\SystèmesLinéaires,Origin=\BanquePT] aow. 
Résoudre, en discutant suivant le paramètre réel $m$, le système linéaire suivant:
$$
\Q\{\eqalign{
mx + y  + mt&=1 \cr
x+m^2y+z+mt&=m\cr
x+my+t&=1\cr
my+z+m^2t=m^2}\W.
$$


\exo [Level=1,Fight=0,Learn=0,Type=\Maple,Field=\GéométriePlane,Origin=\BanquePT] aoy. 
On donne la droite $\sc D$ d'équation 
$$
 \Q\{\eqalign{ 
x+2y-5z-1&=0 \cr 
10x-7y+3z+2=0 }\W.
$$
dans un repère orthonormé et le point $A$ de coordonnées $( 1,2,3)$. \pn
1. Calculer les coordonnées du projeté orthogonal de $A$ sur la droite $\sc D$. 
2. Calculer la distance de $A$ à la droite $\sc D$. 



\exo [Level=1,Fight=0,Learn=0,Type=\Maple,Field=\CourbesParamétréesCartésiennes,Origin=\BanquePT] aoz. 
Soit la courbe de représentation paramétrique
$$
\Q\{\eqalign{
x&={u^3\F u^2-9}\cr
y&={u( u-2)\F u-3}}
\W.
$$
Représenter cette courbe. Préciser les asymptotes, les points doubles et les points d'inflexion.

\exo [Level=1,Fight=0,Learn=0,Type=\Maple,Field=\GéométrieSpatiale,Origin=\BanquePT] apa. 
Dans un repère orthonormé on considère les points  
$$
A( 0,0,0), \qquad  B(1,0, 0), \qquad C( 0,1,0)\hbox{et}\quad  D( 1,1,1).
$$
Trouver le centre et le rayon de la sphère inscrite dans le tétraèdre $ ABCD$, c'est-à-dire tangente à chaque face du tétraèdre et intérieure à celui-ci.

\exo [Level=1,Fight=0,Learn=0,Type=\Maple,Field=\GéométrieSpatiale,Origin=\BanquePT] apb. 
Soit $ f$ la fonction $( a,b) \longmapsto \int_0^\infty \Q( x^2+ax+b\W) ^2e^{-2x}\d x$.
En quel point $(a,b)$ la fonction $f$ atteint-elle son minimum ? Le calculer.


\exo [Level=1,Fight=0,Learn=0,Type=\Maple,Field=\EquationsDifférentielles,Origin=\BanquePT] apc. 
1. Résoudre l'équation différentielle
$$
xy'+{3x+4\F 2x+2}y={x\F\sqrt{x+1}}.
$$
2. Existe-t-il des solutions sur $\Q]-1,+\infty\W[$ ? Si oui, tracer le graphe de telles solutions.

\exo [Level=1,Fight=0,Learn=0,Type=\Maple,Field=\SériesNumériques,Origin=\BanquePT] apd. 
Trouver tous les polynômes $P\in\ob R[X]$ tels que la série de terme général
$$
u_n=(n^7+3n^6)^{1/7}-P(n)^{1/3}
$$
soit convergente.

\exo [Level=1,Fight=0,Learn=0,Type=\Maple,Field=\SériesDeFourier,Origin=\BanquePT] ape. 
Soit $f:\ob R\rightarrow \ob R$ la fonction de période $2\pi$ définie par
$$
\forall t\in \Q[-\pi,\pi\W[ ,\qquad f(t):=t^4-kt^2.
$$
1. Déterminer $k\in\ob R$ tel que les coefficients de Fourier de $f$ soient les plus simples possibles. \pn
2. Développer alors $f$ en série de Fourier et en déduire les sommes 
$$
\sum\limits_{n=1}^\infty{1\F n^4},\qquad 
\sum\limits_{p=0}^\infty{1\F\Q( 2p+1\W) ^4} \quad\hbox{et}
\sum\limits_{n=1}^\infty{1\F n^8}.
$$

\exo [Level=1,Fight=2,Learn=2,Type=\Maple,Field=\Programmation,Origin=\BanquePT] apf. 
Etant donné un entier naturel $a$, on appelle diviseur propre de $a$ tout diviseur de $a$ différent de $a$. \pn
Deux entiers naturels différents de $0$ sont dits amiables si chacun d'eux est égal à la somme des diviseurs propres de l'autre. \pn
Ecrire un programme qui détermine tous les couples d'entiers amiables inférieurs ou égaux à $1500$. 

\exo [Level=1,Fight=2,Learn=2,Type=\Maple,Field=\Programmation,Origin=\BanquePT] apg. 
Montrer que l'on définit une suite convergente en posant 
$$
u_n:=\sum\limits_{k=1}^n{1\F k^2}\qquad (n\ge1).
$$.  
Ecrire une procédure d'arguments $p$ et $e$ déterminant le plus petit entier $N$ tel que $\Q\vert u_{N+p}-u_N\W\vert<e$. 

\exo [Origin=,Level=2,Fight=2,Learn=2,Type=\Exercices,Field=\Déterminant,Indication={b) Remarquer que l'ensemble $\{\mu\in\ob C^*:\mu A+I_n\in\sc Gl_n(\ob C)\}$ est non vide.}] apg. 
Soient $(A,B)\in\sc M_n(\ob C)^2$. \pn
a) Lorsque $A$ est inversible, prouver que $\det(AB-\lambda I_n)=\det(BA-\lambda I_n)$ pour $\lambda\in\ob C$. \pn
b) Prouver que cette identité est encore vraie lorsque $A$ n'est pas inversible. 

\exo [Level=2,Fight=1,Learn=0,Field=\IntégralesGénéralisées,Type=\Exercices,Origin=\Fac,Indication={Pour la convergence, trouver une limite en $0$.  Pour le calcul, intégrer par partie puis primitiver une fraction rationnelle.},Solution={$\ds I={\pi\F2}-\ln 2$}] aph. 
Convergence et calcul de $\ds I:=\int_0^1{\ln (1+t^2)\F t^2}\d t$. 

\exo [Level=2,Fight=1,Learn=0,Field=\IntégralesGénéralisées,Type=\Exercices,Origin=\Fac,Indication={Intégrer par partie, en dérivant $\ln(1+1/x^2)$, puis primitiver une fraction rationnelle.},Solution={$\ds I=\ln 2-{\pi\F2}$}] api. 
Convergence et calcul de $\ds I:=\int_0^1\ln \Q(1+{1\F t^2}\W)\d t$. 

\exo [Level=2,Fight=1,Learn=0,Field=\IntégralesGénéralisées,Type=\Exercices,Origin=\Fac,Indication={Intégrer par partie},Solution={L'intégrale $I$ diverge pour $n\in\{0,1\}$. Pour $n\ge2$, on a $\ds I={1\F (n-1)^2}$.}] apj. 
Convergence et calcul de $\ds I:=\int_1^\infty{\ln t\F t^n}\d t$. 

\exo [Level=2,Fight=2,Learn=1,Field=\IntégralesGénéralisées,Type=\Exercices,Origin=\Fac,Indication={\item{En $0$ : }trouver une limite.\item{En $+\infty$ : }Trouver un développement asymptotique de $\sin(x)\sin\Q({1\F x}\W)$ avec un reste $o(1/x^2)$, intégrer par partie puis utiliser la convergence absolue},Solution={L'intégrale est semi-convergente}] apk. 
Nature de l'intégrale $\ds\int_0^\infty\sin(x)\sin\Q({1\F x}\W)\d x$. 

\exo [Level=2,Fight=0,Learn=0,Field=\IntégralesGénéralisées,Type=\Exercices,Origin=\Fac,Indication={Minorer pr ${1\F \e x}$},Solution={L'intégrale diverge}] apl. 
Nature de l'intégrale $\ds\int_0^\infty{\e^{\sin x}\F x}\d x$.

\exo [Level=2,Fight=0,Learn=0,Field=\IntégralesGénéralisées,Type=\Exercices,Origin=\Fac,Indication={Trouver un développement asymptotique de ${\sin x\F \sqrt x+\sin x}$ en $+\infty$ avec un reste $o(1/x)$.},Solution={L'intégrale diverge}] apm. 
Nature de l'intégrale $\ds\int_2^\infty{\sin x\F \sqrt x+\sin x}\d x$.

\exo [Level=2,Fight=0,Learn=0,Field=\IntégralesGénéralisées,Type=\Exercices,Origin=\Fac,Indication={Procéder au changement de variable $x=\e^u$ puis utiliser que $\cos(u)={\e^u+\e^{-u}\F2}$.},Solution={$\ds I:={1\F 2}.$}] apn. 
Nature et calcul de l'intégrale $\ds\int_0^1\cos(\ln x)\d x$.

\exo [Level=2,Fight=0,Learn=0,Field=\IntégralesGénéralisées,Type=\Exercices,Origin=\Fac,Indication={Procéder au changement de variable $x=\ln(u)$, intégrer par partie puis utiliser la convergence absolue.},Solution={l'intégrale converge.}] apo. 
Nature de l'intégrale $\ds\int_0^\infty\cos\Q(\e^x\W)\d x$.

\exo [Level=2,Fight=4,Learn=1,Field=\IntégralesGénéralisées,Type=\Exercices,Origin=\Fac,Indication={Procéder au changement de variable $x={1\F u}$, puis utiliser Chasles pour écrire que $$
I=\int_1^\infty{u-[u]\F u^2}\d u=\sum_{k\ge1}\int_0^1{u\F (u+k)^2}\d u.
$$ 
Enfin on sommera en utilisant les sommes telescopiques},Solution={$\ds I=\sum_{k\ge1}\Q(\ln{k+1\F k}-{1\F k+1}\W)=1-\gamma$, la constante d'Euler $\gamma$ étant définie par $\ds\gamma:= \lim_{n\to\infty}\Q(\sum_{k=1}^n{1\F k}-\ln n\W)$.}] app. 
Nature et calcul de l'intégrale $\ds\int_0^1\Q({1\F x}-\Q[{1\F x}\W]\W)\d x$.

\exo [Level=2,Fight=3,Learn=1,Field=\IntégralesGénéralisées,Type=\Exercices,Origin=\Fac,Indication={Intégrer par partie pour montrer que $\ds I=2a^2\int_0^\infty{\ln x\F x^2+a^2}\d x-\pi a$ puis procéder aux changements de variables $x=au$ puis $u={1\F x}$.},Solution={$\ds I=-\pi a+\pi\ln(a)$.}] apq. 
Pour $a>0$, nature et calcul de l'intégrale $\ds\int_0^\infty\ln(x)\ln\Q(1+{a^2\F x^2}\W)\d x$.

\exo [Level=2,Fight=2,Learn=1,Field=\IntégralesGénéralisées,Type=\Exercices,Origin=\Fac,Indication={Procéder au changement de variable $x=u^n$ puis intégrer par partie $n-1$ fois l'exponentielle.},Solution={$I=n!$.}] apr. 
Nature et calcul de l'intégrale $\ds I:=\int_0^\infty\e^{-x^{1/n}}\d x$.

\exo [Level=2,Fight=1,Learn=0,Field=\IntégralesGénéralisées,Type=\Exercices,Origin=\Fac,Indication={Procéder au changement de variable $x=\e^u$, décomposer la fraction rationnelle en éléments simples puis primitiver.},Solution={$I={\ln\Q(3+2\sqrt2\W)\F 2\sqrt2}$.}] aps. 
Nature et calcul de l'intégrale $\ds I:=\int_0^\infty{1\F1+\ch(x)^2}\d x$.

\exo [Level=2,Fight=1,Learn=1,Field=\IntégralesGénéralisées,Type=\Exercices,Origin=\Fac,Indication={Multiplier en haut et en bas par $\ch x$, utiliser que $1+\sh^2x=\ch^2x$,  procéder au changement de variable $u=\sh x$, puis primitiver.},Solution={$I=\pi$.}] apt. 
Nature et calcul de l'intégrale $\ds I:=\int_0^\infty{1\F\ch(x)}\d x$.


\exo [Level=1,Fight=0,Learn=0,Field=\Programmation,Type=\Maple,Origin=\Lakedaemon] apu. 
Donner une valeur décimale à 10 chiffres significatifs des nombres  
$$
a_1:={1-{\pi+3\F \e+4}\F 1-{4\F1+\pi}}\qquad a_2:={4\F1+{1^2\F2+{3^2\F2+{5^2\F2+{7^2\F2+9^2}}}}}\qquad 
a_3:={\sqrt{\sqrt2+\sqrt3}\F 1-(\sqrt 3+\sqrt2)(\sqrt5-\sqrt2)}
$$

\exo [Level=1,Fight=2,Learn=1,Type=\Colles,Field=\Coniques,Origin=\Quercia,Solution={Soit $O'$ ce centre.
         Les triangles $MPQ$ et $MAB$ sont semblables, donc $O'$ est l'image de
         $O$ par l'homothétie de centre $M$ qui transforme $A$ en $P$. \pn
         Soit $(A'B')$ la symétrique de $(AB)$ par rapport à $O$.
         D'après l'homothétie,
         $${O'M\F \d(O',\Delta)} ={OM\F \d(O,(AB))} = (cste)
           = {OM-O'M\F\d(O,(AB))-\d(O',\Delta)}
           = {OO'\F\d(O',(A'B'))}.
	   $$
         Donc $O'$ décrit une partie d'une conique de foyer $O$ et de directrice
         $(A'B')$.}] apv. 
Soit $\sc C$ un cercle de centre $O$, et $A,B$ deux points distincts
de $\sc C$.
Soit $\Delta$ le diamètre parallèle à $(AB)$.\pn
Pour $M \in {\cal C}$, on note $P,Q$ les intersections de $(MA)$ et $(MB)$ avec
$\Delta$. Chercher le lieu du centre du cercle circonscrit à $MPQ$.

\exo [Level=1,Fight=2,Learn=1,Type=\Colles,Field=\GéométrieSpatiale,Origin=\Quercia,Solution={a) $\Omega:(1,1,1)$, $R = \sqrt5$. \pn
	b) $(ABC) : x+y+z = 6$.\quad
             $(ABD) : 4x-2y+z = 3$.\quad
             $(ACD) : x+4y-2z = 3$.\quad
             $(BCD) : 7x+y-5z = 12$.
	c) $$I:(a,b,c) \ssi \eqalign{ 6-a-b-c     &= r\sqrt3    \cr
                                    4a-2b+c-3   &= r\sqrt{21} \cr
                                    a+4b-2c-3   &= r\sqrt{21} \cr
                                    12-7a-b+5c  &= r\sqrt{75} \cr }
              \ssi \eqalign{ 2a &= 9 - 2\sqrt 7\cr
                          2b &= 6 - \sqrt7  \cr
                          2c &= 3           \cr
                          2r &=  \sqrt{21}-2\sqrt3.\cr }$$
}] apw.
étant  donnés les points $A:(1,2,3)$, $B:(2,3,1)$, $C:(3,1,2)$ et $D:(1,0,-1)$ : \pn
a) chercher le centre $\Omega$ et le rayon $R$ de la sphère circonscrite à $ABCD$.
b) Chercher les équations cartésiennes des plans $(ABC)$, $(ABD)$, $(ACD)$, $(BCD)$. \pn 
c)Chercher le centre et le rayon de la sphère inscrite dans le tétraèdre $ABCD$.

\exo [Level=2,Fight=2,Learn=1,Type=\Colles,Field=\GéométrieSpatiale,Origin=\Quercia,Indication={Introduire la perpendiculaire commune à $D$ et $D'$.},
Solution={$d=4\sqrt{19}$ avec $H:(-2/19, 28/19, 35/19)$ et $K:(-6/19, 40/19, 23/19)$ les points de la perpendiculaire commune appartenant respectivement à $D$ et $D'$.}] apx.
Calculer la distance $d$ entre les droites
$D  : \cases {x +2y - z = 1  \cr 2x - y +2z = 2 \cr}$ et
$D' : \cases {x + y + z = 3  \cr  x - y +2z = 0.\cr}$.

\exo  [Level=2,Fight=3,Learn=1,Type=\Oraux,Field=\Topologie,Origin=\Mines] apy. 
Montrer que $N:(x,y)\mapsto\int_0^1|x+ty|\d t$ définit une norme sur $\ob R^2$ et représenter la boule unité. 

\exo [Level=2,Fight=3,Learn=1,Type=\Oraux,Field=\Suites,Origin=\CCP] apz. 
a. Étudier la suite réelle définie par $u_0:=1$ et 
$$
u_{n+1}:=u_n{1+2u_n\F 1+3u_n}\qquad(n\ge0).
$$
b. Trouver un équivalent de la suite $u_n$. 

\exo [Level=2,Fight=2,Learn=1,Type=\Oraux,Field=\Fonctions,Origin=\CCP] aqa. 
Étudier la fonction $f:x\mapsto x\Q|1+{1\F x}\W|^{x+1}$. 

\exo [Level=2,Fight=3,Learn=1,Type=\Oraux,Field=\Intégrales,Origin=\Mines,Solution={$I={\pi\F4}+{1\F2}\ln\Q(\sqrt2+1\W)-{1\F\sqrt2}$.}] aqb.
Calculer $I:=\ds\int_0^1{\d t\F \sqrt{1+t^2}+\sqrt{1-t^2}}$. 

\exo [Level=2,Fight=3,Learn=1,Type=\Oraux,Field=\Intégrales,Origin=\Mines,Solution={$I=4\sqrt2-\pi-2$.}] aqc.
Calculer $I:=\ds\int_{-1}^1{\d t\F \sqrt{1+t}+\sqrt{1-t}+2}$. 

\exo [Level=2,Fight=2,Learn=1,Type=\Oraux,Field=\Intégrales,Origin=\Centrale,Solution={$I(x)=0$.}] aqd.
Calculer $I(x):=\ds\int_x^{{\pi\F2}-x}\sin(t)\cos(t)\ln\Q({\sin t\F\cos t}\W)\d t$. 

\exo [Level=2,Fight=2,Learn=1,Type=\Oraux,Field=\IntégralesGénéralisées,Origin=\Mines,Solution={$I=-2\pi$.}] aqe.
Convergence et calcul de l'intégrale $\ds I:=\int_0^1{\ln x\F\sqrt x(1-x)^{3/2}}\d x$. 

\exo [Level=2,Fight=2,Learn=1,Type=\Oraux,Field=\IntégralesGénéralisées,Origin=\Centrale,Solution={$I=1$.}] aqf.
Convergence et calcul de l'intégrale $\ds I:=\int_0^{+\infty}\Q[\int_x^{+\infty}{\sin t\F t\d t}\W]\d x$. 

\exo [Level=2,Fight=2,Learn=1,Type=\Oraux,Field=\IntégralesGénéralisées,Origin=X] aqg.
Étudier la convergence de l'intégrale $I(\alpha,\beta):=\int_0^1|1-t^\alpha|^\beta\d t$ en fonction des nombres réels non nuls $\alpha$ et $\beta$.

\exo [Level=2,Fight=2,Learn=1,Type=\Oraux,Field=\IntégralesGénéralisées,Origin=\Mines,Solution={$I=\ln 2$.}] aqh.
Convergence et calcul de l'intégrale $\ds I:=\int_0^{\pi\F2}\cos(x)\ln(\tan x)\d x$. 

\exo [Level=2,Fight=3,Learn=1,Type=\Oraux,Field=\IntégralesGénéralisées,Origin=\Centrale,Solution={$\ell=0$.}] aqf.
Soit $f:]0,1]\to\ob R$ une application décroissante telle que l'intégrale $\int_0^1f(t)\d t$ converge. 
Déterminer la limite $\ds \ell:=\lim\limits_{x\to0^+}xf(x)$. 

\exo [Level=2,Fight=2,Learn=1,Type=\Oraux,Field=\IntégralesGénéralisées,Origin=\Mines,Solution={$I=\ln 3-\ln 2$.}] aqh.
Convergence et calcul de l'intégrale $\ds I:=\int_0^{+\infty}{\th(3x)-\th(2x)\F x}\d x$. 

\exo [Level=2,Fight=0,Learn=0,Type=\TravauxDirigés,Field=\PolynômesCaractéristiques,Origin=\Lakedaemon,Solution=] aqi.
Calculer et factoriser le polynôme caractéristique de $A:=\pmatrix{2&0&1\cr1&1&1\cr-2&0&-1\cr}$. 

\exo [Level=2,Fight=1,Learn=0,Type=\TravauxDirigés,Field=\PolynômesCaractéristiques,Origin=\Lakedaemon,Solution=] aqj.
Calculer et factoriser le plolynôme caractéristique de $A:=\pmatrix{
-7&3&1&-6
\cr
-6&2&1&-6
\cr
0&0&2&0
\cr
6&-3&-1&5
\cr}
$. 

\exo [Level=2,Fight=2,Learn=2,Type=\TravauxDirigés,Field=\IntégralesGénéralisées,Origin=,Solution=] aql.
On pose $\ds A(x):=\int_0^\infty{\sin^2(tx)\F t^2}\d t$ et $\ds F(x):=\int_0^\infty{\sin^2(tx)\F t^2(1+t^2)}\d t$. 
\medskip
\noindent 1) Démontrer que $A$ et $F$ sont définies sur $\ob R$. 
\medskip
\noindent 2) Donner une expression de $A(x)$ en fonction de $x$ et de $A(1)$. 
\medskip
\noindent 3) En étudiant $F(x)-A(x)$, donner un équivalent de $F$ en $+\infty$ et en $-\infty$. 
\medskip
\noindent 4) Montrer rexpectivement que $F$ est continue, de classe $\sc C^1$, de classe $\sc C^2$ sur $\ob R$ et 
exprimer les dérivées $F'(x)$ et $F''(x)$ à l'aide d'intégrales impropres. 
\medskip
\noindent 5) Exprimer $F(x)-A(x)$ en fonction de $F''(x)$. 
\medskip
\noindent 6) En déduire que $F$ est solution d'une équation différentielle 
du type $y''-4y=a|x|+b$ où $a$ et $b$ sont des constantes réelles qu'on déterminera. 
\medskip
\noindent 7) Déterminer $F(x)$ au moyen des fonctions élémentaires et du nombre $A(1)$. 
\medskip
\noindent 8) En déduire la valeur de $\ds \int_0^\infty{\sin^2(t)\F t^2}\d t$. 

\endinput