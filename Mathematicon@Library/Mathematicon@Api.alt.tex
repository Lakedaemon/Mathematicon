\catcode`@=11\relax
\let\EA=\expandafter
\let\CS=\csname
\let\EC=\endcsname
\let\Par\par

%%%%%%%%%%%%%%%%%%%%%%%%%%%%%%%%%%%%%%%%%%%%%%%%%%%%%%%%%%%%%%%%%%%%
%                                                                                                                                                                                      %
%                                                                        Maths : Symbols                                                                                   %
%                                                                                                                                                                                      %
%%%%%%%%%%%%%%%%%%%%%%%%%%%%%%%%%%%%%%%%%%%%%%%%%%%%%%%%%%%%%%%%%%%%
\def\Ker#1{%
	\def\LD@Temp{(}%
	\hbox{Ker}%
	\unless\ifx#1\LD@Temp
		\unless\ifx#1\Q
			\ 
		\fi
	\fi
	#1%
}%
\def\Ima#1{%
	\def\LD@Temp{(}%
	\hbox{Im}%
	\unless\ifx#1\LD@Temp
		\unless\ifx#1\Q
			\ 
		\fi
	\fi
	#1%
}%
\def\Ima{\hbox{Im}}%

\def\Norme#1{%
	\Vert#1\Vert
}%


%%%%%%%%%%%%%%%%%%%%%%%%%%%%%%%%%%%%%%%%%%%%%%%%%%%%%%%%%%%%%%%%%%%%
%                                                                                                                                                                                      %
%                                                                        Index : Concept                                                                                     %
%                                                                                                                                                                                      %
%%%%%%%%%%%%%%%%%%%%%%%%%%%%%%%%%%%%%%%%%%%%%%%%%%%%%%%%%%%%%%%%%%%%

\def\Concept{\@getoptionalarg\@Concept}%
\def\@Concept#1\par{%
\ifHtml\noindent
\ifx\@optionalarg\empty\else\expandafter\sidx\expandafter{\@optionalarg}\fi
\HCode{<div class="concept">}\ignorespaces#1\HCode{</div>}\else\ifdim\lastskip<\medskipamount\removelastskip\medskip\fi
                           \vskip0pt plus.01\vsize\penalty-100\vskip0pt plus-.01\vsize
                            \noindent\ifx\@optionalarg\empty\else\expandafter\sidx\expandafter{\@optionalarg}\fi$\underline{\hbox{\it \ignorespaces#1}}$\medskip
\fi
}




%%%%%%%%%%%%%%%%%%%%%%%%%%%%%%%%%%%%%%%%%%%%%%%%%%%%%%%%%%%%%%%%%%%%
%                                                                                                                                                                                      %
%                                                                                      Equations  numérotées                                                           %
%                                                                                                                                                                                      %
%%%%%%%%%%%%%%%%%%%%%%%%%%%%%%%%%%%%%%%%%%%%%%%%%%%%%%%%%%%%%%%%%%%%



\def\Equation{\@getoptionalarg\@LEquation}
\def\@LEquation#1$$#2$${\ifx\@optionalarg\empty\def\tempol{\the\eqnumber}\advance\eqnumber by1\else\edef\tempol{\@optionalarg}\fi
\def\argoL{#1}\trim\argoL\ifx\argoL\empty\else\definexref{\argoL}{\tempol}{eq}\fi
\ifHtml
$$\displaystyle\eqalignno{&\displaystyle #2&\mbox{(\bf\tempol)}}$$\else
$$#2\leqno{\hbox{(\tempol)}}$$\fi}




%%%%%%%%%%%%%%%%%%%%%%%%%%%%%%%%%%%%%%%%%%%%%%%%%%%%%%%%%%%%%%%%%%%%
%                                                                                                                                                                                      %
%                                                        Définitions, propriétés, théorèmes, lemmes                                                        %
%                                                                                                                                                                                      %
%%%%%%%%%%%%%%%%%%%%%%%%%%%%%%%%%%%%%%%%%%%%%%%%%%%%%%%%%%%%%%%%%%%%


% Pour inverser l'ordre Hypothèses-propriétés. 


\newif\ifInverted\Invertedfalse


% La macro qui se charge du formatage en couleur Html/TeX - Normal/Inversé
\newdimen\OLdimen


%\DefineRGBcolor F0F9E3=VLGreen.
%\DefineRGBcolor E5F9D1=LGreen.
%\DefineRGBcolor 5DA93B=Green.

\def\ColorFrame{\@getoptionalarg\@LColorFrame}%
\def\@LColorFrame#1#2#3{%
	\noindent
	\ifHtml % Html
		\ifInverted % Inverted
			\EndP
			\HCode{<div class="ICF#1">}\vbox{\ignorespaces#2\EndP}\HCode{</div>}%
			\ifx\@optionalarg\empty
			\else
				\HCode{<div class="centerline"><span class="ICFHyp#1">}\@optionalarg\HCode{</span></div>}%
			\fi
			\Invertedfalse
		\else % Inverted false
			\EndP
			\ifx\@optionalarg\empty\else
				\HCode{<div class="centerline"><span class="CFHyp#1">}\@optionalarg\HCode{</span></div>}%
			\fi
			\HCode{<div class="CF#1">}\vbox{\noindent\ignorespaces#2\EndP}%
			\HCode{</div>}%
		\fi
	\else  % Html false
		\tikzstyle{body} = [draw=Color#1,shading=axis,shading angle=45,top color=ColorVL#1,bottom color=ColorT#1,middle color=ColorL#1,thick,rectangle, rounded corners, inner xsep=5pt, inner ysep=10pt]%
		\tikzstyle{head}= [draw=Color#1,fill=ColorVL#1,thick,rectangle,rounded corners,inner xsep=3pt,inner ysep=3pt]%
		\ifInverted
			\ifx\@optionalarg\empty
				\OLdimen=\hsize\advance\OLdimen by -15.3pt%
				\tikzpicture
					\node[text width=\OLdimen,body] {\noindent #2} ;
				\endtikzpicture
				\PAR
			\else
				\OLdimen=\hsize\advance\OLdimen by -15.3pt
				\tikzpicture
					\node[text width=\OLdimen,body] (a) {\noindent #2} ;
					\node [head] at (a.south) {\noindent\@optionalarg} ;
					#3
				\endtikzpicture
				\PAR
			\fi
			\vskip15pt
			\Invertedfalse
		\else % Inverted false
			\ifx\@optionalarg\empty
				\bigskip\noindent
				\OLdimen=\hsize\advance\OLdimen by -15.3pt
				\tikzpicture
					\node[text width=\OLdimen,body] {\noindent #2} ; 
				\endtikzpicture
				\PAR
			\else
				\OLdimen=\hsize\advance\OLdimen by -15.3pt
				\tikzpicture
					\node[text width=\OLdimen,body] (a) {\noindent #2} ;
					\node [head] at (a.north)  {\noindent\@optionalarg} ;
					#3
				\endtikzpicture
				\PAR
				\vskip9pt
			\fi
		\fi
	\fi
}%


% Définition
\def\Definition{\@getoptionalarg\@LDefinition}%
\def\@LDefinition#1\par{%
	\ifx\@optionalarg\empty
		\@LColorFrame{Green}{\ignorespaces#1}{}%
	\else
		\edef\temp{[\@optionalarg]}%
		\EA\ColorFrame\temp{Green}{\ignorespaces#1}{}%
	\fi
}%

% Propriété

\def\Propriete{\@getoptionalarg\@LPropriete}%
\def\@LPropriete#1\par{%
	\ifx\@optionalarg\empty
		\@LColorFrame{Orange}{\ignorespaces#1}{}%
	\else
		\edef\temp{[\@optionalarg]}%
		\EA\ColorFrame\temp{Orange}{\ignorespaces#1}{}%
	\fi
}%

% Théorème

\def\Theoreme{\@getoptionalarg\@LTheoreme}%
\def\@LTheoreme#1\par{%
	\ifx\@optionalarg\empty
		\@LColorFrame{Red}{\ignorespaces#1}{}%
	\else
		\edef\temp{[\@optionalarg]}%
		\EA\ColorFrame\temp{Red}{\ignorespaces#1}{}%
	\fi
}%


% Nocolor

\def\Assertion{\@getoptionalarg\@LAssertion}%
\def\@LAssertion#1\par{%
	\ifx\@optionalarg\empty
		\@LColorFrame{Black}{\ignorespaces#1}{}%
	\else
		\edef\temp{[\@optionalarg]}%
		\EA\ColorFrame\temp{Black}{\ignorespaces#1}{}%
	\fi
}%

%%%%%%%%%%%%%%%%%%%%%%%%%%%%%%%%%%%%%%%%%%%%%%%%%%%%%%%%%%%%%%%%%%%%
%                                                                                                                                                                                      %
%                                      Démonstrations, Exemples, Applications, Exercices, Remarques                                       %
%                                                                                                                                                                                      %
%%%%%%%%%%%%%%%%%%%%%%%%%%%%%%%%%%%%%%%%%%%%%%%%%%%%%%%%%%%%%%%%%%%%


\long\def\Demonstration. #1\CQFD{\noindent{\eightpts D\'emonstration. #1}\bigskip}


\long\def\Rappels. #1\par{\noindent{\eightpts{\it Rappels. }#1}\medskip}
\long\def\Exemples : #1\par{\noindent{\eightpts{\it Exemples : }#1}\medskip}
\long\def\Application : #1\par{\noindent{\eightpts{\it Application : }#1}\medskip}
\long\def\Exercice. #1\par{\noindent{\eightpts{\bf Exercice.} #1}\bigskip}
\def\Remarque{\noindent{\it Remarque}}
\def\Remarques{\noindent{\it Remarques}}
\def\Notation{\noindent{\it Notation}}

\def\Exemple{\@getoptionalarg\LD@Maths@Exemple}
\newcount\LD@Maths@Exemple@Counter\LD@Maths@Exemple@Counter=0\relax
\long\def\LD@Maths@Exemple : #1\par{%
	\noindent
	{%
		\eightpts
		\ifx\@optionalarg\LD@empty
			{\it Exemple : }%
		\else
			\global\advance\LD@Maths@Exemple@Counter by1\relax
			\definexref{\@optionalarg}{\the\LD@Maths@Exemple@Counter}{ex}%
			{\it Exemple \@optionalarg : }%
		\fi
		#1%
	}%
	\medskip
}%


\def\CenterLine{%
	\Par
	\centerline
}%

\newcount\LD@Maths@Picture@Counter\LD@Maths@Picture@Counter=0\relax
\def\Figure{\@getoptionalarg\LD@Maths@Figure}%
\def\LD@Maths@Figure#1{%
	\LD@Option@Process
	\global\advance\LD@Maths@Picture@Counter by1\relax
	\definexref{#1}{\the\LD@Maths@Picture@Counter}{fig}%
	\ifcsname LD@Option@@\endcsname
		\ifcsname LD@Option@@Index\endcsname
			\EA\EA\EA\sidx\EA\EA\EA{\EA\LD@Option@@Index\LD@Option@@}%
		\fi
		Figure \the\LD@Maths@Picture@Counter. \LD@Option@@.
	\else
		\ifcsname LD@Option@@Index\endcsname
			\EA\sidx\EA{\LD@Option@@Index}%
		\fi
		Figure \the\LD@Maths@Picture@Counter.
	\fi
}%

\newdimen\LD@Dimen@A
\newbox\LD@Box@A
\def\LD@Margin@H{5pt}%
\def\Incrust\par#1\par#2\par{%
	\setbox\LD@Box@A=\hbox{#2}%
	{%
		\LD@Dimen@A=\hsize\relax
		\advance\LD@Dimen@A by-\wd\LD@Box@A\relax
		\moveright\LD@Dimen@A\hbox{\smash{\copy\LD@Box@A}}%
		\advance\LD@Dimen@A by-\LD@Margin@H
		\parshape 1 0pt \LD@Dimen@A
		#1\par
	}%
}%



\def\LDFrame{\@getoptionalarg\LD@Frame}%
\def\LD@Frame#1#2#3{%
	\noindent
	\ifHtml % Html
		\ifInverted % Inverted
			\EndP
			\HCode{<div class="ICF#1">}\vbox{\ignorespaces#2\EndP}\HCode{</div>}%
			\ifx\@optionalarg\empty
			\else
				\HCode{<div class="centerline"><span class="ICFHyp#1">}\@optionalarg\HCode{</span></div>}%
			\fi
			\Invertedfalse
		\else % Inverted false
			\EndP
			\ifx\@optionalarg\empty\else
				\HCode{<div class="centerline"><span class="CFHyp#1">}\@optionalarg\HCode{</span></div>}%
			\fi
			\HCode{<div class="CF#1">}\vbox{\noindent\ignorespaces#2\EndP}%
			\HCode{</div>}%
		\fi
	\else  % Html false
		\tikzstyle{body} = [draw=Color#1, shading=axis,shading angle=45,top color=ColorVL#1,bottom color=ColorT#1,middle color=ColorL#1,thick,rectangle, inner xsep=5pt, inner ysep=10pt]%
		\tikzstyle{head}= [draw=Color#1,fill=ColorVL#1,thick,rectangle,rounded corners,inner xsep=3pt,inner ysep=3pt]%
		\ifInverted
			\ifx\@optionalarg\empty
				\OLdimen=\hsize\advance\OLdimen by -15.3pt%
				\tikzpicture
					\node[text width=\OLdimen,body] {\noindent #2} ;
				\endtikzpicture
				\PAR
			\else
				\OLdimen=\hsize\advance\OLdimen by -15.3pt
				\tikzpicture
					\node[text width=\OLdimen,body] (a) {\noindent #2} ;
					\node [head] at (a.south) {\noindent\@optionalarg} ;
					#3
				\endtikzpicture
				\PAR
			\fi
			\vskip15pt
			\Invertedfalse
		\else % Inverted false
			\OLdimen=\hsize\relax
			\advance\OLdimen by -15.3pt%
			\ifx\@optionalarg\empty
				\bigskip\noindent
				\tikzpicture
					\pgfdeclarelayer{background}
					\pgfsetlayers{background,main}
					\node[text width=\OLdimen] {\noindent #2} ; 
					\pgfonlayer{background}
						\draw [body] (current bounding box.south west) rectangle (current bounding box.north east) ;
					\endpgfonlayer
					#3
				\endtikzpicture
				\PAR
			\else
				\tikzpicture
					\pgfdeclarelayer{background}
					\pgfsetlayers{background,main}
					\node[text width=\OLdimen] (a) {\noindent #2} ; 
					\pgfonlayer{background}
						\draw [body] (current bounding box.south west) rectangle (current bounding box.north east) ;
					\endpgfonlayer
					\node [head] at (a.north)  {\noindent\@optionalarg} ;
					#3
				\endtikzpicture
				\PAR
				\vskip9pt
			\fi
		\fi
	\fi
}%


\def\LD@Frame{\@getoptionalarg\LD@@Frame}%
\def\LD@@Frame#1{%
	\LD@Option@Process
	\LD@Dimen@A=\hsize\relax
	\advance\LD@Dimen@A by -15.3pt\relax
	\noindent
	\tikzpicture
		\pgfdeclarelayer{background}
		\pgfsetlayers{background,main}
		\node[text width=\LD@Dimen@A] (LD@Frame@Node@A) {\noindent\ignorespaces #1} ; 
		\ifcsname LD@Option@@Title\endcsname
			\node [yshift=2pt] (LD@Frame@Node@B) at (LD@Frame@Node@A.north) {};
		\fi
		\pgfonlayer{background}
			\draw [body] (current bounding box.south west) rectangle (current bounding box.north east) ;
		\endpgfonlayer
		\ifcsname LD@Option@@Title\endcsname
			\node [head] at (LD@Frame@Node@B.north)  {\noindent\LD@Option@@Title} ;
		\fi
		\LD@Frame@Code@After
	\endtikzpicture
	\Par
	\vskip9pt
}%
%%%%%%%%%%%%%%%%%%%%%%%%%%%%%%%%%%%%%%%%%%%%%%%%%%%%%%%%%%%%%%%%%%
%															%
%							Methode                                                                                  %
%															%
%%%%%%%%%%%%%%%%%%%%%%%%%%%%%%%%%%%%%%%%%%%%%%%%%%%%%%%%%%%%%%%%%%
\def\Methode{\@getoptionalarg\LD@Maths@Methode}%
\def\LD@Maths@Methode#1\par{%
	\LD@Option@Process
	\def\LD@Frame@Code@After{}%
	\ifcsname LD@Option@@Index\endcsname
		\def\LD@Temp{M\'ethode!Methode@}%
		\EA\EA\EA\sidx\EA\EA\EA{\EA\LD@Temp\LD@Option@@Index}%
	\fi
	\tikzstyle{body} = [snake=saw,draw=ColorGreen, shading=axis,shading angle=45,top color=ColorVLGreen,bottom color=ColorTGreen,middle color=ColorLGreen,thick,rectangle, inner xsep=5pt, inner ysep=10pt]%
	\tikzstyle{head}= [draw=ColorRed,fill=ColorVLRed,thick,rectangle,rounded corners,inner xsep=3pt,inner ysep=3pt]%
	\ifcsname LD@Option@@\endcsname
		{\eightpts\rm
		\edef\LD@Temp{[Title=\LD@Option@@]}%
		\EA\LD@Frame\LD@Temp{#1}%
		}%
	\else
		{\eightpts\rm
		\EA\LD@Frame{#1}%
		}%
	\fi
}%

%%%%%%%%%%%%%%%%%%%%%%%%%%%%%%%%%%%%%%%%%%%%%%%%%%%%%%%%%%%%%%%%%%
%															%
%							Conseil							%
%															%
%%%%%%%%%%%%%%%%%%%%%%%%%%%%%%%%%%%%%%%%%%%%%%%%%%%%%%%%%%%%%%%%%%
\def\Conseil : #1\par{{%
	\eightpts
	\noindent
	\llap{%
		$\underline{\mbox{Conseil}}$ : 
	}%
	#1\medskip
}}%
\def\Rappel : #1\par{{%
	\eightpts
	\noindent
	\llap{%
		$\underline{\mbox{Rappel}}$ : 
	}%
	#1\medskip
}}%
%%%%%%%%%%%%%%%%%%%%%%%%%%%%%%%%%%%%%%%%%%%%%%%%%%%%%%%%%%%%%%%%%%
%															%
%						Level									%
%															%
%%%%%%%%%%%%%%%%%%%%%%%%%%%%%%%%%%%%%%%%%%%%%%%%%%%%%%%%%%%%%%%%%%

% 0 	Bac
% 1	Sup
% 2 	Spé
% 3	Licence
% 4	Maitrise
% 5	DEA+

%%%%%%%%%%%%%%%%%%%%%%%%%%%%%%%%%%%%%%%%%%%%%%%%%%%%%%%%%%%%%%%%%%
%															%
%						Fight									%
%															%
%%%%%%%%%%%%%%%%%%%%%%%%%%%%%%%%%%%%%%%%%%%%%%%%%%%%%%%%%%%%%%%%%%

%				PT			PT*/MP		MP*
% -1 (évident)			Trop facile
% 0 (application du cours)	Facile			Trop facile
% 1 (adapter le cours)		Facile/Moyen		Facile			Trop facile
% 2 (Ensam)			Moyen/dur		Facile/Moyen		Facile
% 3 (Centrale/Mines)		Dur/Baleze		Moyen/dur		Facile/Moyen
% 4 (Polytechnique)		Baleze/Infernal	Dur/Baleze		Moyen/dur
% 5 (Ens Ulm)			Infernal/Impossible	Baleze/Infernal	Dur/Baleze

%%%%%%%%%%%%%%%%%%%%%%%%%%%%%%%%%%%%%%%%%%%%%%%%%%%%%%%%%%%%%%%%%%
%															%
%						Learn									%
%															%
%%%%%%%%%%%%%%%%%%%%%%%%%%%%%%%%%%%%%%%%%%%%%%%%%%%%%%%%%%%%%%%%%%

% -1 (Perte de temps)		
% 0 (Entrainement)		0 idée
% 1 (Formateur)		1+ idée nouvelle
% 2 (Classique)		2+ idées nouvelles
% 3 (Culte)			3+ idées nouvelles
%%%%%%%%%%%%%%%%%%%%%%%%%%%%%%%%%%%%%%%%%%%%%%%%%%%%%%%%%%%%%%%%%%
%															%
%						Field: classement par thèmes					%
%															%
%%%%%%%%%%%%%%%%%%%%%%%%%%%%%%%%%%%%%%%%%%%%%%%%%%%%%%%%%%%%%%%%%%

\newtoks\LD@String@Toks@A
\newtoks\LD@String@Toks@B
\unless\ifcsname LD@Exo@Theme@List\endcsname
\def\LD@String@Append#1#2{%
	\LD@String@Toks@A={#1}%
	\LD@String@Toks@B=\EA{#2}%
	\edef#2{\the\LD@String@Toks@B\the\LD@String@Toks@A}%
}%
\def\LD@Exo@Theme@Add#1#2{%
	\EA\def\CS #1\EC{#2}%
	\ifcsname LD@Exo@Theme@List\endcsname
		\EA\LD@String@Append\EA{\CS #1\EC}\LD@Exo@Theme@List
	\else
		\EA\def\EA\LD@Exo@Theme@List\EA{\CS #1\EC}%
	\fi
}%

%% Fondements

\LD@Exo@Theme@Add{Récurrences}{Récurrences}%
\LD@Exo@Theme@Add{Applications}{Applications}%


%% Géométrie

\LD@Exo@Theme@Add{GéométriePlane}{Géométrie plane}%
\LD@Exo@Theme@Add{GéométrieSpatiale}{Géométrie de l'espace}%
\LD@Exo@Theme@Add{Courbes}{Courbes}%
\LD@Exo@Theme@Add{Coniques}{Coniques}%
\LD@Exo@Theme@Add{AbscisseCurviligne}{Abscisse Curviligne}%
\LD@Exo@Theme@Add{RepèreDeFrenet}{Repère de Frenet}%
\LD@Exo@Theme@Add{Courbure}{Courbure}%
\LD@Exo@Theme@Add{Torsion}{Torsion}%
\LD@Exo@Theme@Add{Hélices}{Hélices}%
\LD@Exo@Theme@Add{Enveloppes}{Enveloppes}%
\LD@Exo@Theme@Add{Développées}{Développées}%
\LD@Exo@Theme@Add{Développantes}{Développantes}%
\LD@Exo@Theme@Add{Roulement}{Roulement}%
\LD@Exo@Theme@Add{Surfaces}{Surfaces}%
\LD@Exo@Theme@Add{Quadriques}{Quadriques}%

%% Algèbre

\LD@Exo@Theme@Add{NombresComplexes}{Nombres Complexes}%
\LD@Exo@Theme@Add{Trigonométrie}{Trigonométrie}%
\LD@Exo@Theme@Add{NombresEntiers}{Nombres entiers}%
\LD@Exo@Theme@Add{Arithmétique}{Arithmétique}%
\LD@Exo@Theme@Add{Groupes}{Groupes}%
\LD@Exo@Theme@Add{Anneaux}{Anneaux}%
\LD@Exo@Theme@Add{Polynômes}{Polynômes}%
\LD@Exo@Theme@Add{FractionsRationnelles}{Fractions rationnelles}%
\LD@Exo@Theme@Add{EspacesVectoriels}{Espaces vectoriels}%
\LD@Exo@Theme@Add{DimensionFinie}{Dimension finie}%
\LD@Exo@Theme@Add{Rang}{Rang}%
\LD@Exo@Theme@Add{SystèmesLinéaires}{Systèmes linéaires}%
\LD@Exo@Theme@Add{RécurrencesLinéaires}{Récurrences linéaires}%
\LD@Exo@Theme@Add{Matrices}{Matrices}%
\LD@Exo@Theme@Add{SystèmesLinéaires}{Systèmes linéaires}%
\LD@Exo@Theme@Add{FormesMultilinéaires}{Formes multilinéaires}%
\LD@Exo@Theme@Add{Déterminant}{Déterminant}%
\LD@Exo@Theme@Add{ValeursPropres}{Valeurs propres}%
\LD@Exo@Theme@Add{VecteursPropres}{Vecteurs propres}%
\LD@Exo@Theme@Add{PolynômesCaractéristiques}{Polynômes caractéristiques}%
\LD@Exo@Theme@Add{Diagonalisation}{Diagonalisation}%
\LD@Exo@Theme@Add{Trigonalisation}{Trigonalisation}%
\LD@Exo@Theme@Add{Réduction}{Réduction}%
\LD@Exo@Theme@Add{SystèmesDifférentiels}{Systèmes différentiels}%

\LD@Exo@Theme@Add{EspacesPréHilbertiens}{Espaces pré-Hilbertiens}%
\LD@Exo@Theme@Add{InégalitéDeCauchySchwarz}{Inégalité de Cauchy-Schwarz}%
\LD@Exo@Theme@Add{Orthonormalisation}{Orthonormalisation}%
\LD@Exo@Theme@Add{EndomorphismesSymétriques}{Endomorphismes symétriques}%
\LD@Exo@Theme@Add{MatricesSymétriques}{Matrices symétriques}%
\LD@Exo@Theme@Add{FormesQuadratiques}{Formes quadratiques}%
\LD@Exo@Theme@Add{EndomorphismesOrthogonaux}{Endomorphismes orthogonaux}%
\LD@Exo@Theme@Add{MatricesOrthogonales}{Matrices orthogonales}%

%% Analyse

\LD@Exo@Theme@Add{Fonctions}{Fonctions}%
\LD@Exo@Theme@Add{TrigonométrieHyperbolique}{Trigonométrie hyperbolique}%
\LD@Exo@Theme@Add{Convexite}{Convexité}%
\LD@Exo@Theme@Add{CourbesParamétréesCartésiennes}{Courbes paramétrées Cartésiennes}%
\LD@Exo@Theme@Add{CourbesParamétréesPolaires}{Courbes paramétrées Polaires}%

%% Topologie

\LD@Exo@Theme@Add{Normes}{Normes}%
\LD@Exo@Theme@Add{Suites}{Suites}%
\LD@Exo@Theme@Add{Topologie}{Topologie}%
\LD@Exo@Theme@Add{Connexité}{Connexité}%
\LD@Exo@Theme@Add{Compacité}{Compacité}%

\LD@Exo@Theme@Add{FonctionsDePlusieursVariables}{Fonctions de plusieurs variables}%
\LD@Exo@Theme@Add{Continuité}{Continuité}%
\LD@Exo@Theme@Add{Dérivation}{Dérivation}%
\LD@Exo@Theme@Add{Extrema}{Extrema}%
\LD@Exo@Theme@Add{EquationsAuxDérivéesPartielles}{Equation aux dérivées partielles}%

\LD@Exo@Theme@Add{ThéorèmeDeRolle}{Théorème De Rolle}%
\LD@Exo@Theme@Add{DéveloppementsLimités}{Développement limités}%
\LD@Exo@Theme@Add{Limites}{Limites}%
\LD@Exo@Theme@Add{Equivalents}{Equivalents}%
\LD@Exo@Theme@Add{EquationsDifférentielles}{Equations différentielles}%
\LD@Exo@Theme@Add{EquationsDifférentiellesAVariablesSéparables}{Equations différentielles à variables séparables}%
\LD@Exo@Theme@Add{EquationsDifférentiellesLinéairesDuPremierOrdre}{Equations différentielles linéaires du premier ordre}%
\LD@Exo@Theme@Add{EquationsDifférentiellesLinéairesDuSecondOrdre}{Equations différentielles linéaires du second ordre}%
\LD@Exo@Theme@Add{Primitives}{Primitives}%
\LD@Exo@Theme@Add{Intégrales}{Intégrales}%
\LD@Exo@Theme@Add{SommesDeRiemann}{Sommes de Riemann}%
\LD@Exo@Theme@Add{Intégration}{Intégration}%
\LD@Exo@Theme@Add{IntégralesGénéralisées}{Intégrales généralisées}%
\LD@Exo@Theme@Add{FonctionsDéfiniesParUneIntégrale}{Fonctions définies par une intégrale}%
\LD@Exo@Theme@Add{IntégralesMultiples}{Intégrales multiples}%
\LD@Exo@Theme@Add{IntégralesCurvilignes}{Intégrales curvilignes}%
\LD@Exo@Theme@Add{Aires}{Aires}%
\LD@Exo@Theme@Add{Volumes}{Volumes}%

\LD@Exo@Theme@Add{Séries}{Séries}%
\LD@Exo@Theme@Add{SériesNumériques}{Séries numériques}%
\LD@Exo@Theme@Add{SériesDeFonctions}{Séries de fonctions}%
\LD@Exo@Theme@Add{SériesEntières}{Séries entières}%
\LD@Exo@Theme@Add{SériesDeFourier}{Séries de Fourier}%


\LD@Exo@Theme@Add{ChampsDeVecteurs}{ChampsDeVecteurs}%
\LD@Exo@Theme@Add{PotentielsScalaires}{Potentiel Scalaire}%
\LD@Exo@Theme@Add{PotentielsVecteurs}{Potentiel Vecteur}%

% Algorythmique

\LD@Exo@Theme@Add{Programmation}{Programmation}%
\fi
%%%%%%%%%%%%%%%%%%%%%%%%%%%%%%%%%%%%%%%%%%%%%%%%%%%%%%%%%%%%%%%%%%
%															%
%						Type									%
%															%
%%%%%%%%%%%%%%%%%%%%%%%%%%%%%%%%%%%%%%%%%%%%%%%%%%%%%%%%%%%%%%%%%%
% Cours -> illustration du cours, classiques. 
% TD -> simples, applications, entrainement
% Exo -> plus dur, classiques
% Colle
\def\Colles{Colles}%
\def\Cours{Cours}%
\def\Exercices{Exercices}%
\def\TravauxDirigés{Travaux dirigés}%
\def\Problèmes{Problèmes}%
\def\Others{Autres}%
\def\Mathematica{Mathematica}%
\def\Maple{Maple}%
% Autre -> vide ?
%%%%%%%%%%%%%%%%%%%%%%%%%%%%%%%%%%%%%%%%%%%%%%%%%%%%%%%%%%%%%%%%%%
%															%
%						Origin									%
%															%
%%%%%%%%%%%%%%%%%%%%%%%%%%%%%%%%%%%%%%%%%%%%%%%%%%%%%%%%%%%%%%%%%%
\def\MP{MP}%
\def\Capaces{$\kappa$}%
\def\Lakedaemon{$\Lambda$}% Λ 
\def\Mercier{MP*}%
\def\Quercia{MP*}%
\def\BanquePT{B. PT}%
\def\CCP{CCP}%
\def\Fac{Fac}%
%%%%%%%%%%%%%%%%%%%%%%%%%%%%%%%%%%%%%%%%%%%%%%%%%%%%%%%%%%%%%%%%%%
%															%
%						Exo									%
%															%
%%%%%%%%%%%%%%%%%%%%%%%%%%%%%%%%%%%%%%%%%%%%%%%%%%%%%%%%%%%%%%%%%%

\newcount\LD@Exo@Count\LD@Exo@Count=0\relax
\def\sol{\@getoptionalarg\LD@Maths@Sol}%
\def\LD@Maths@Sol#1. #2\par{%
	\LD@Option@Process
	\def\LD@Trash{Sol={#2}}% 
	\def\LD@Temp{#1}%
	\trim\LD@Temp
	\EA\let\CS DataSol\LD@Temp\EC\LD@Trash
}%
\def\Solution{\@getoptionalarg\LD@Maths@Solution}%
\def\LD@Maths@Solution#1{%
	\def\LD@Maths@Label@Internal{#1}%
	\LD@Option@Process
	\ifcsname DataSol#1\endcsname
		\LD@Data@Def{Sol}{Sol#1}\LD@Sol@@Sol
		\LD@Maths@Solution@Display
	\fi
}%
\def\LD@Solution@Display@Code{}%
\def\LD@Maths@Solution@Text{%
	{\bf Corrig\'e de l'exercice \refn{labelexo\LD@Maths@Label@Internal}.}\PAR
}%
\def\LD@Maths@Solution@Display{%
	{\LD@Solution@Display@Code
	\noindent
	\LD@Maths@Solution@Text 
	\ignorespaces\LD@Sol@@Sol
	}%
}%
\def\exo{\@getoptionalarg\LD@Maths@Exo}%
\def\LD@Maths@Exo#1. #2\par{%
	\def\LD@Option@@Origin{}%
	\def\LD@Option@@Level{}%
	\def\LD@Option@@Fight{}%
	\def\LD@Option@@Learn{}%
	\def\LD@Option@@Field{}%
	\def\LD@Option@@Type{}%
	\def\LD@Option@@Solution{}%
	\LD@Option@Process
	\edef\LD@Temp{Level={\LD@Option@@Level} Fight={\LD@Option@@Fight} Learn={\LD@Option@@Learn} Field=}%
	\EA\LD@String@Append\EA{\EA{\LD@Option@@Field} Type=}\LD@Temp
	\EA\LD@String@Append\EA{\EA{\LD@Option@@Type} Origin=}\LD@Temp
	\EA\LD@String@Append\EA{\EA{\LD@Option@@Origin} Sol=}\LD@Temp
	\EA\LD@String@Append\EA{\EA{\LD@Option@@Solution} Exo=}\LD@Temp
	\LD@String@Append{{#2}}\LD@Temp
	\def\LD@Label{#1}%
	\trim\LD@Label
	\EA\let\CS Data\LD@Exo@Prefix\LD@Label\EC\LD@Temp
	\LD@Loop@For\LD@Theme=\LD@Option@@Field\WithSeparator |\Do{%
		\unless\ifcsname LD@Exo@Theme@@\LD@Option@@Level\endcsname
			\EA\edef\CS LD@Exo@Theme@@\LD@Option@@Level\EC{\LD@Theme ,}%
		\else
			\edef\LD@Temp{\LD@Theme}%
			\EA\let\EA\LD@Trash\CS LD@Exo@Theme@@\LD@Option@@Level\EC
			\edef\LD@Test{,\LD@Trash ,\LD@Theme,}%
			\EA\LD@String@Split\EA\LD@Test\EA{\EA,\LD@Temp ,}\LD@Trash\LD@Test
			\ifx\LD@Test\LD@Empty
				\EA\let\EA\LD@Trash\CS LD@Exo@Theme@@\LD@Option@@Level\EC
				\EA\edef\CS LD@Exo@Theme@@\LD@Option@@Level\EC{\LD@Trash ,\LD@Temp}%	
			\fi
		\fi	
		\unless\ifcsname LD@Exo@Theme@@\LD@Option@@Level @@\LD@Theme @@\LD@Option@@Type\endcsname
			\EA\edef\CS LD@Exo@Theme@@\LD@Option@@Level @@\LD@Theme @@\LD@Option@@Type\EC{\LD@Exo@Prefix\LD@Label}%
		\else
			\EA\let\EA\LD@Trash\CS LD@Exo@Theme@@\LD@Option@@Level @@\LD@Theme @@\LD@Option@@Type\EC
			\EA\edef\CS LD@Exo@Theme@@\LD@Option@@Level @@\LD@Theme @@\LD@Option@@Type\EC{\LD@Trash,\LD@Exo@Prefix\LD@Label}%	
		\fi
	}%
}%
\def\Exercice{\@getoptionalarg\LD@Maths@Exercice}%
\def\LD@Maths@Exercice#1{%
	\def\LD@Option@@Label{#1}%
	\LD@Option@Process
	\ifcsname Data#1\endcsname
		\LD@Data@Def{Exo}{#1}\LD@Exo@@Exo
		\LD@Data@Def{Level}{#1}\LD@Exo@@Level
		\LD@Data@Def{Fight}{#1}\LD@Exo@@Fight
		\LD@Data@Def{Learn}{#1}\LD@Exo@@Learn
		\LD@Data@Def{Field}{#1}\LD@Exo@@Field
		\LD@Data@Def{Type}{#1}\LD@Exo@@Type
		\LD@Data@Def{Origin}{#1}\LD@Exo@@Origin
		\LD@Data@Def{Sol}{#1}\LD@Exo@@Solution
		\global\advance\LD@Exo@Count by1\relax % \global
		\definexref{labelexo#1}{\the\LD@Exo@Count}{exolabel}%
		\LD@Maths@Exercice@Display
	\fi
}%
\def\LD@Maths@Exercice@Text{\underline{\bf Exercice}}%
\def\LD@Exercice@Display@Code{}%
\def\LD@Exercice@Display@Code@Post{}%
\def\LD@Maths@Exercice@Display{%
	{\LD@Exercice@Display@Code
	\noindent
	\smash{%
		\unless\ifx\LD@Exo@@Origin\LD@Empty
			\def\LD@Temp{}%%%%%%%%%%%%%%%%%%%%%%%%%%    don't display these 
			\EA\EA\EA\def\EA\EA\EA\LD@Test\EA\EA\EA{\EA\LD@Temp\LD@Exo@@Origin}%
			\EA\LD@String@Split\EA\LD@Test\EA{\LD@Exo@@Origin}\LD@Trash\LD@Test
			\ifx\LD@Test\LD@Empty
				\raise2pt\llap{\fivepts(\LD@Exo@@Origin)\quad}%
			\fi
		\fi
		\vbox{%
			\fivepts
			\llap{%
				\LD@Space
				\def\LD@Temp{0}%
				\unless\ifx\LD@Exo@@Fight\LD@Temp
					\LD@Exo@@Fight
				\fi
			}%
			\llap{%
				\LD@Space
				\def\LD@Temp{0}%
				\unless\ifx\LD@Exo@@Learn\LD@Temp
					\LD@Exo@@Learn
				\fi
			}%
		}%
	}%
	$\LD@Maths@Exercice@Text
	\unless\ifx\LD@Exo@@Level\LD@Empty
			\ifcase\LD@Exo@@Level
				^{\flat\flat}%
			\or 
				^{\flat}%
			\or 
			\or
				^{\sharp}%
			\or
				^{\sharp\sharp}%
			\or
				^{\sharp\sharp\sharp}%
			\or
			\else
			\fi
	\fi
	$ \the\LD@Exo@Count. \ignorespaces\LD@Exo@@Exo}%
	\LD@Exercice@Display@Code@Post
}%
\newcount\LD@Count@Temp
\def\LD@Exercice@Display@Code{\eightpts}%
\def\LD@Display#1{%
	\LD@Count@Temp=#1\relax
	\ifcase\LD@Count@Temp
	\or
	Math. Sup.
	\or
	Math. Sp\'e
	\else
	\fi
}%
\newcount\LD@Exo@Total\LD@Exo@Total=0\relax
\def\LD@Exo@Label@Show{%
	\def\LD@Exo@Label@@Show{Y}%
}%
\def\LD@Exo@Label@Hide{%
	\def\LD@Exo@Label@@Show{N}%
}%
\LD@Exo@Label@Hide
\def\LD@Exo@Theme@Display#1#2#3{%
	{\LD@Loop@For\LD@Theme=#2\WithSeparator\Do{%
		\goodbreak\bigskip
		\centerline{\LD@Font@Arial \LD@Theme}%
		\medskip
		{\LD@Loop@For\LD@Exo@Type=#3\WithSeparator ,\Do{%
			\ifcsname LD@Exo@Theme@@#1@@\LD@Theme @@\LD@Exo@Type\endcsname
				%\medskip\noindent$\underline{\hbox{\LD@Font@Arial\LD@Exo@Type}}$\medskip\noindent
				\EA\let\EA\LD@Exo@List\CS LD@Exo@Theme@@#1@@\LD@Theme @@\LD@Exo@Type\EC
				{\LD@Loop@For\LD@Exo@Label=\LD@Exo@List\WithSeparator ,\Do{%
					\if Y\LD@Exo@Label@@Show
						\noindent\llap{$\underline{\hbox{{\eightpts \LD@Exo@Label}}}$\qquad}%
					\fi
					\EA\Exercice\EA{\LD@Exo@Label}\medskip\noindent
					\global\advance\LD@Exo@Total by1\relax
				}}%
			\fi
		}}%
	}}%

}%
\catcode`@=12\relax