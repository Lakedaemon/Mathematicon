\catcode`@=11\relax
%\def\Api{Mathematicon@Api}%
\input LD@Header.tex
\input LD@Library.tex
\input LD@Typesetting.tex
\input LD@Exercices.tex
\tenrm

\DefineRGBcolor F0F9E3=VLGreen.
\DefineRGBcolor E5F9D1=LGreen.
\DefineRGBcolor DAF9BE=TGreen.
\DefineRGBcolor 5DA93B=Green.
\DefineRGBcolor F6DCCA=VLRed.
\DefineRGBcolor F6D4BD=LRed.
\DefineRGBcolor DAF9BE=TRed.
\DefineRGBcolor B5F9A1=TTRed.
\DefineRGBcolor F6B080=Red.
\DefineRGBcolor F9F5E3=VLOrange.
\DefineRGBcolor F9F5D0=LOrange.
\DefineRGBcolor DAF9BE=TOrange.
\DefineRGBcolor B5F9A1=TTOrange.
\DefineRGBcolor D7A93B=Orange.
\DefineRGBcolor EEEEEE=VLBlack.
\DefineRGBcolor DDDDDD=LBlack.
\DefineRGBcolor CCCCCC=TBlack.
\DefineRGBcolor B5F9A1=TTBlack.
\DefineRGBcolor 000000=Black.

%\DefineRGBcolor 000000=Green.
%\definecolor{ColorVLGreen}{rgb}{1,1,1}%
%\definecolor{ColorLGreen}{rgb}{1,1,1}%
%\definecolor{ColorTGreen}{rgb}{1,1,1}%
%\expandafter\definecolor\temp
%\DefineRGBcolor 000000=Red.
%\definecolor{ColorVLRed}{rgb}{1,1,1}%
%\definecolor{ColorLRed}{rgb}{1,1,1}%
%\definecolor{ColorTRed}{rgb}{1,1,1}%
%\input Exercices.tex

\newcount\LD@Count@Temp


\newif\ifLD@Inferno@Master@\LD@Inferno@Master@false
\LD@Exo@Label@Hide

\LD@Colors@Hide

\def\LD@Exercice@Display@Code{}%%\LD@Option@@Label\qquad\eightpts}%
\gdef\LD@Exercice@Solution@List{}%
\gdef\LD@Exercice@Indication@List{}%
\gdef\LD@Exercice@Notion@List{}%
\def\LD@Exercice@Display@Code@Post{%
	\ifcsname LD@Exo@@Solution\endcsname
		\unless\ifx\LD@Exo@@Solution\LD@Empty
			\ifLD@Inferno@Master@
				\pn{\eightpts Solution : \eightpts \LD@Exo@@Solution}%
			\else
				\ifx\LD@Exercice@Solution@List\LD@Empty
					\EA\gdef\EA\LD@Exercice@Solution@List\EA{\LD@Option@@Label}%
				\else
					\EA\EA\EA\gdef\EA\EA\EA\LD@Exercice@Solution@List\EA\EA\EA{\EA\LD@Exercice@Solution@List\EA ,\LD@Option@@Label}%
				\fi
			\fi
		\fi
	\fi
	\ifcsname LD@Exo@@Notion\endcsname
		\unless\ifx\LD@Exo@@Notion\LD@Empty
			\ifLD@Inferno@Master@
				\pn{\eightpts Notions intervenant dans la solution : \eightpts \LD@Exo@@Notion}%
			\else
				\ifx\LD@Exercice@Notion@List\LD@Empty
					\EA\gdef\EA\LD@Exercice@Notion@List\EA{\LD@Option@@Label}%
				\else
					\EA\EA\EA\gdef\EA\EA\EA\LD@Exercice@Notion@List\EA\EA\EA{\EA\LD@Exercice@Notion@List\EA ,\LD@Option@@Label}%
				\fi
			\fi	
		\fi
	\fi
	\ifcsname LD@Exo@@Indication\endcsname
		\unless\ifx\LD@Exo@@Indication\LD@Empty
			\ifLD@Inferno@Master@
				\pn{\eightpts Indication : \eightpts \LD@Exo@@Indication}%	
			\else
				\ifx\LD@Exercice@Indication@List\LD@Empty
					\EA\gdef\EA\LD@Exercice@Indication@List\EA{\LD@Option@@Label}%
					\else
					\EA\EA\EA\gdef\EA\EA\EA\LD@Exercice@Indication@List\EA\EA\EA{\EA\LD@Exercice@Indication@List\EA ,\LD@Option@@Label}%
				\fi
			\fi
		\fi
	\fi
	\medskip\penalty-100
}%
\def\LD@Display#1{%
	\LD@Count@Temp=#1\relax
	\ifcase\LD@Count@Temp
	\or
	Math. Sup.
	\or
	Math. Spé
	\else
	\fi
}%

\def\LD@Exercice@Display@Code{}%%\LD@Option@@Label\qquad\eightpts}%
\gdef\LD@Exercice@Solution@List{}%
\gdef\LD@Exercice@Indication@List{}%
\gdef\LD@Exercice@Notion@List{}%
\def\LD@Exercice@Display@Code@Post{%
	\ifcsname LD@Exo@@Solution\endcsname
		\unless\ifx\LD@Exo@@Solution\LD@Empty
			\ifLD@Inferno@Master@
				\pn{\eightpts Solution : \eightpts \LD@Exo@@Solution}%
			\else
				\ifx\LD@Exercice@Solution@List\LD@Empty
					\EA\gdef\EA\LD@Exercice@Solution@List\EA{\LD@Option@@Label}%
				\else
					\EA\EA\EA\gdef\EA\EA\EA\LD@Exercice@Solution@List\EA\EA\EA{\EA\LD@Exercice@Solution@List\EA ,\LD@Option@@Label}%
				\fi
			\fi
		\fi
	\fi
	\ifcsname LD@Exo@@Notion\endcsname
		\unless\ifx\LD@Exo@@Notion\LD@Empty
			\ifLD@Inferno@Master@
				\pn{\eightpts Notions intervenant dans la solution : \eightpts \LD@Exo@@Notion}%
			\else
				\ifx\LD@Exercice@Notion@List\LD@Empty
					\EA\gdef\EA\LD@Exercice@Notion@List\EA{\LD@Option@@Label}%
				\else
					\EA\EA\EA\gdef\EA\EA\EA\LD@Exercice@Notion@List\EA\EA\EA{\EA\LD@Exercice@Notion@List\EA ,\LD@Option@@Label}%
				\fi
			\fi	
		\fi
	\fi
	\ifcsname LD@Exo@@Indication\endcsname
		\unless\ifx\LD@Exo@@Indication\LD@Empty
			\ifLD@Inferno@Master@
				\pn{\eightpts Indication : \eightpts \LD@Exo@@Indication}%	
			\else
				\ifx\LD@Exercice@Indication@List\LD@Empty
					\EA\gdef\EA\LD@Exercice@Indication@List\EA{\LD@Option@@Label}%
					\else
					\EA\EA\EA\gdef\EA\EA\EA\LD@Exercice@Indication@List\EA\EA\EA{\EA\LD@Exercice@Indication@List\EA ,\LD@Option@@Label}%
				\fi
			\fi
		\fi
	\fi
	\medskip\penalty-100
}%


\def\LD@Exo@Sol@Display{%
	{\EA\LD@Loop@For\EA\LD@Exo@Label\EA=\csname LD@Exercice@Solution@List\endcsname\WithSeparator ,\Do{%
		\LD@Data@Def{Sol}\LD@Exo@Label\LD@Temp
		\noindent{\eightpts \ref{labelexo\LD@Exo@Label}. \LD@Temp}
		\medskip
	}}%

}%
\def\LD@Exo@Notion@Display{%
	{\EA\LD@Loop@For\EA\LD@Exo@Label\EA=\csname LD@Exercice@Notion@List\endcsname\WithSeparator ,\Do{%
		\LD@Data@Def{Notion}\LD@Exo@Label\LD@Temp
		\noindent\pn{\eightpts \ref{labelexo\LD@Exo@Label} : \LD@Temp}%
		\medskip
	}}%
}%


\def\LD@Exo@Indication@Display{%
	{\EA\LD@Loop@For\EA\LD@Exo@Label\EA=\csname LD@Exercice@Indication@List\endcsname\WithSeparator ,\Do{%
		\LD@Data@Def{Ind}\LD@Exo@Label\LD@Temp
		\noindent\pn{\eightpts \ref{labelexo\LD@Exo@Label}. \LD@Temp}%
		\medskip
	}}%

}%




\catcode`@=11\relax

\vglue-10mm\rightline{PT\hfill TD 1 :  Déterminant\hfill}
\bigskip
\bigskip




\centerline{Opérations élémentaires\strut}
\hrule\medskip

\Exercice{PTSIuo}
\medskip
\Exercice{PTSIuq}
\medskip
\Exercice{PTSIur}
\medskip
\Exercice{PTSIup}
\medskip
\Exercice{PTSIun}
\medskip
\Exercice{PTSIus}
\medskip
\Exercice{PTSIut}

\hrule
\centerline{Déterminant par blocs\strut}
\hrule\medskip

\overfullrule0pt

\Methode [$A\in\sc M_p(\ob K)${,} $B\in\sc M_q(\ob K)${,} $C\in\sc M_{p,q}(\ob K)$]
$$
\det\pmatrix{A&C\cr 0& B}=\det(A)\times \det(B)
$$

\medskip
\Exercice{PTfb}

\centerline{Développement par rapport à une ligne ou une colonne\strut}
\hrule\medskip

\hrule
\centerline{Développement par rapport à une colonne\strut}
\hrule\medskip

\Methode [$A$ matrice carrée de taille $n\ge1$]
En développant la matrice $A=(a_{i,j})_{1\le i\le n\atop1\le j\le n}$ par rapport à la colonne $j\in\{1, \cdots n\}$, nous obtenons que 
$$
\det A=\sum_{1\le i\le n}(-1)^{i+j}a_{i,j}\det \tilde A_{i, j}, 
$$
le symbôle $\tilde A_{i, j}$ désignant la matrice $A$ privée de sa $i^\ieme$ ligne et de sa $j^\ieme$ colonne. 


\hrule
\centerline{Développement par rapport à une ligne\strut}
\hrule\medskip


\Methode [$A$ matrice carrée de taille $n\ge1$]
En développant la matrice $A$ par rapport à la ligne $i\in\{1, \cdots n\}$, nous obtenons que 
$$
\det A=\sum_{1\le j\le n}(-1)^{i+j}a_{i,j}\det \tilde A_{i, j}, 
$$
le symbôle $\tilde A_{i, j}$ désignant la matrice $A$ privée de sa $i^\ieme$ ligne et de sa $j^\ieme$ colonne. 
\bigskip

\medskip
\Exercice{PTfi}
\medskip
\Exercice{PTaiw}

	\Chapter Indications, Indications.

	\LD@Exo@Indication@Display

	%\Chapter Notions, Notions.

	%\LD@Exo@Notion@Display

	\Chapter Solutions, Solutions.

	\LD@Exo@Sol@Display

\bye









