\catcode`@=11\relax

\input LD@Maths@TD.tex

\vglue-10mm\rightline{PT\hfill TD 1 :  Déterminant\hfill}
\bigskip
\bigskip




\centerline{Opérations élémentaires\strut}
\hrule\medskip

\Exercice{PTSIuo}
\medskip
\Exercice{PTSIuq}
\medskip
\Exercice{PTSIur}
\medskip
\Exercice{PTSIup}
\medskip
\Exercice{PTSIun}
\medskip
\Exercice{PTSIus}
\medskip
\Exercice{PTSIut}

\hrule
\centerline{Déterminant par blocs\strut}
\hrule\medskip

\overfullrule0pt

\Methode [$A\in\sc M_p(\ob K)${,} $B\in\sc M_q(\ob K)${,} $C\in\sc M_{p,q}(\ob K)$]
$$
\det\pmatrix{A&C\cr 0& B}=\det(A)\times \det(B)
$$

\medskip
\Exercice{PTfb}

\centerline{Développement par rapport à une ligne ou une colonne\strut}
\hrule\medskip

\hrule
\centerline{Développement par rapport à une colonne\strut}
\hrule\medskip

\Methode [$A$ matrice carrée de taille $n\ge1$]
En développant la matrice $A=(a_{i,j})_{1\le i\le n\atop1\le j\le n}$ par rapport à la colonne $j\in\{1, \cdots n\}$, nous obtenons que 
$$
\det A=\sum_{1\le i\le n}(-1)^{i+j}a_{i,j}\det \tilde A_{i, j}, 
$$
le symbôle $\tilde A_{i, j}$ désignant la matrice $A$ privée de sa $i^\ieme$ ligne et de sa $j^\ieme$ colonne. 


\hrule
\centerline{Développement par rapport à une ligne\strut}
\hrule\medskip


\Methode [$A$ matrice carrée de taille $n\ge1$]
En développant la matrice $A$ par rapport à la ligne $i\in\{1, \cdots n\}$, nous obtenons que 
$$
\det A=\sum_{1\le j\le n}(-1)^{i+j}a_{i,j}\det \tilde A_{i, j}, 
$$
le symbôle $\tilde A_{i, j}$ désignant la matrice $A$ privée de sa $i^\ieme$ ligne et de sa $j^\ieme$ colonne. 
\bigskip

\medskip
\Exercice{PTfi}
\medskip
\Exercice{PTaiw}

	\Chapter Indications, Indications.

	\LD@Exo@Indication@Display

	%\Chapter Notions, Notions.

	%\LD@Exo@Notion@Display

	\Chapter Solutions, Solutions.

	\LD@Exo@Sol@Display

\bye









