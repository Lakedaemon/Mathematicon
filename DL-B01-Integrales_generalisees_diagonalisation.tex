\catcode`@=11\relax
\input LD@Header.tex
\input LD@Library.tex

\def\LD@Maths@Exercice@Display{\ignorespaces\LD@Exo@@Exo}%

%%%%%%%%%%%%%%%%%%%%%%%%%%%%%%%%%%%%%%%%%%%%%%%%%%%%%%%%%%%%%%%%%%
%															%
%						Probleme 00 : Révisions de sup					%
%															%
%%%%%%%%%%%%%%%%%%%%%%%%%%%%%%%%%%%%%%%%%%%%%%%%%%%%%%%%%%%%%%%%%%

\vglue-10mm\rightline{Sp\'e PT\hfill mini DL 2 (pour le 17/10) : pr\'eparation au ds.\hfill\date}%
\bigskip
\bigskip
\vfill
\noindent
$E$ d\'esigne l'espace vectoriel r\'eel des fonctions continues sur $\ob R$ \`a valeurs dans $\ob R$. 
\par\noindent
$E_1$ est l'ensemble des fonctions $f$ de $E$ telles que l'int\'egrale g\'en\'eralis\'ee
$$
\int_x^{+\infty}f(t)\e^{-t}\d t
$$
converge pour chaque $x\in\ob R$. 
\bigskip\noindent
1. Montrer que $E_1$ est un sous-espace vectoriel de $E$. 
\bigskip\noindent
2.  {Etude de quelques exemples : }
\smallskip\noindent
a. Montrer que l'int\'egrale $\int_x^{+\infty}\e^{-t}\d t$ converge pour chaque $x\in\ob R$. 
Plus g\'en\'eralement, \'etant donn\'e $n\in\ob N$, montrer que $\int_x^{+\infty}t^n\e^{-t}\d t$ converge pour $x\in\ob R$.
\medskip\noindent
b. L'int\'egrale g\'en\'eralis\'ee $\int_x^{+\infty}\cos(t)\e^{-t}\d t$ existe-t-elle pour chaque $x\in\ob R$ ?
\medskip\noindent
c. En d\'eduire des exemples de fonctions appartenant \`a $E_1$. 
\bigskip\noindent
3. On pose 
$$
\eqalign{\varphi:E_1&\to E\cr f&\mapsto F}\quad\mbox{ avec }\quad F(x):=\e^x\int_x^{+\infty}f(t)\e^{-t}\d t\qquad(x\in\ob R).
$$
a. Montrer que $F$ est de classe $\sc C^1$ sur $\ob R$ et que $F=F'+f$. 
\medskip\noindent
b. Montrer que $\varphi$ est une application lin\'eaire. 
\medskip\noindent
c. L'application $\varphi$ est-elle injective ?
\medskip\noindent
d. Lorsque $f\in E_1$ est un vecteur propre de $\varphi$, montrer que $f$ est solution d'une \'equation diff\'erentielle lin\'eaire du premier ordre \`a coefficients constants. 
D\'eterminer les valeurs propres et les vecteurs propres~de~$\varphi$. 
\medskip\noindent
e. Soit $f\in E_1$ une fonction born\'ee et de classe $\sc C^1$ sur $\ob R$. Montrer que $f'\in E_1$ et que $\varphi(f')=\varphi(f)'$. 
\bigskip\noindent
4. Soient $n\in\ob N$ et $E_2$  le $\ob R$-espace vectoriel des polyn\^omes de degr\'e inf\'erieur ou \'egal \`a $n$. 
\smallskip\noindent
a. Montrer que $E_2\subset E_1$. 
\medskip\noindent
b. Soit $\psi$ la restriction de $\varphi$ \`a $E_2$ (au d\'epart et \`a l'arriv\'ee).
\smallskip\noindent
i) Montrer que $\psi$ est un endomorphisme de $E_2$.\pn
ii) Est-ce un automorphisme de $E_2$ ? \pn
iii) $\psi$ est-il diagonalisable ?
\medskip\noindent
c. Soit $f\in E_2$ une fonction pour laquelle il existe un nombre r\'eel $a$ tel que 
$$
f(x)\ge 0\qquad(x\ge a).
$$
Prouver que l'on a 
$$
\sum_{k=0}^nf^{(k)}(x)\ge0\qquad(x\ge a).
$$
\vfill
\bye