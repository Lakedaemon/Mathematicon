\catcode`@=11\relax
%\def\Api{Mathematicon@Api}%
\input LD@Header.tex
\input LD@Library.tex
\input LD@Exercices.tex
\input LD@Typesetting.tex

\DefineRGBcolor F0F9E3=VLGreen.
\DefineRGBcolor E5F9D1=LGreen.
\DefineRGBcolor DAF9BE=TGreen.
\DefineRGBcolor 5DA93B=Green.
\DefineRGBcolor F6DCCA=VLRed.
\DefineRGBcolor F6D4BD=LRed.
\DefineRGBcolor DAF9BE=TRed.
\DefineRGBcolor B5F9A1=TTRed.
\DefineRGBcolor F6B080=Red.
\DefineRGBcolor F9F5E3=VLOrange.
\DefineRGBcolor F9F5D0=LOrange.
\DefineRGBcolor DAF9BE=TOrange.
\DefineRGBcolor B5F9A1=TTOrange.
\DefineRGBcolor D7A93B=Orange.
\DefineRGBcolor EEEEEE=VLBlack.
\DefineRGBcolor DDDDDD=LBlack.
\DefineRGBcolor CCCCCC=TBlack.
\DefineRGBcolor B5F9A1=TTBlack.
\DefineRGBcolor 000000=Black.

%\DefineRGBcolor 000000=Green.
%\definecolor{ColorVLGreen}{rgb}{1,1,1}%
%\definecolor{ColorLGreen}{rgb}{1,1,1}%
%\definecolor{ColorTGreen}{rgb}{1,1,1}%
%\expandafter\definecolor\temp
%\DefineRGBcolor 000000=Red.
%\definecolor{ColorVLRed}{rgb}{1,1,1}%
%\definecolor{ColorLRed}{rgb}{1,1,1}%
%\definecolor{ColorTRed}{rgb}{1,1,1}%

\catcode`@=11\relax


%%%%%%%%%%%%%%%%%%%%%%%%%%%%%%%%%%%%%%%%%%%%%%%%%%%%%%%%%%%%%%%%%%
%															%
%						TD 03 : Trigonalisation						%
%															%
%%%%%%%%%%%%%%%%%%%%%%%%%%%%%%%%%%%%%%%%%%%%%%%%%%%%%%%%%%%%%%%%%%
\def\Maple#1{{\it #1}}%
\vglue-10mm\rightline{PT\hfill Maple 1 :  Marices, diagonalisation, trigonalisation\hfill}
\bigskip
\bigskip
Dans ce TD Maple, vous aurez besoin des commandes \Maple{eigenvalues} (valeurs propres en allemand), \Maple{eigenvectors} (vecteurs propres en allemand), \Maple{det} (determinant), \Maple{inverse} (inversion de matrices), ainsi que d'autres commandes utiles en alg\`ebre lin\'eaire, qui se trouvent dans la librairie "linalg". Avant de pouvoir les utiliser, vous devrez donc "faire apprendre son cours d'alg\`ebre lin\'eaire'' \`a Maple en tappant
$$
\Maple{>with(linalg);}
$$
Maple, c'est trop cool : si seulement il suffisait de tapper les \'el\`eves pour qu'ils connaissent leur cours...soupir...
\bigskip
\Exercice{PTSIto}%
\bigskip
\Exercice{PTSItp}%
\smallskip
\Exercice{PTko}%
\smallskip
\Exercice{PTkl}%
\bigskip
\Exercice{PTkn}%
\bigskip

\Exercice{PTafo}%
\bigskip
\Exercice{PTSIqm}%
\smallskip
\Exercice{PTaox}%
\smallskip
\Exercice{PTSIsv}%
\bigskip
\Exercice{PTkk}%
\smallskip

\Exercice{PTkp}%
\bigskip
\Exercice{PTkm}%
\bye









