%\def\Variables{MathsVariables}%
\catcode`@=11\relax
\def\Api{Mathematicon@Api}%
%%%% Newif
%\def\@firstofone#1{#1}

\newif\ifexonumber
%%%% Switches
\exonumberfalse
\catcode`@=11\relax
\input LD@Header.tex
\input LD@Library.tex
\input LD@Typesetting.tex
\input LD@Exercices.tex




\long\def\Touche#1{\tikz\node[draw,baseline]{\eightpts #1};}
\long\def\touche#1{\tikz\node[draw,baseline]{\sevenpts #1};}
\def\UnderBrace#1_#2{{\underbrace{#1}_{\hbox{\sevenpts #2}}}}


\def\red{}
\def\black{}
\def\blue{}

\Chapter Maple1, Bases de Maple. 
\bigskip


Maple est un langage de manipulation symbolique, muni d'une larges biblioth\`eque de fonctions permettant 
de r\'ealiser des calculs num\'eriques ou formels.  
\bigskip
L'apprentissage de Maple figure au programme des classes de PTSI/PT et certaines \'ecoles d'ing\'enieurs (ENSAM,...) \'evaluent \`a l'oral  
la capacit\'e des candidats au concour  \`a utiliser l'outil informatique pour les math\'ematiques.  
\bigskip


\Section Maple1Interface, Interface. 

\Concept  La ligne de commande

Lorsqu'il ne calcule pas, Maple attend. L'utilisateur peut alors lui demander d'executer une expression Maple
en la tapant  apr\`es le prompt ``$>$'' de la ligne de commande\footnote{1}{Mapple affiche le caract\`ere $>$ pour vous signifier qu'il attend vos ordres} 
puis en appuyant sur la touche ``Entr\'ee''. 
$$
\UnderBrace{\red >}_{prompt} \hbox{\red \bf Expression Maple}\ \Touche{Entr\'ee}
$$

\Concept  Ligne de commande suppl\'ementaire

Pour obtenir une ligne de commande suppl\'ementaire, appuyer sur l'ic\^one \touche{[$>$} du menu. 

\Concept  Ligne de texte

Pour transformer une ligne de commande en ligne de texte, c'est \`a dire une ligne non executable sans prompt, que vous pouvez utiliser pour \'ecrire des commentaires, appuyer sur la touche \touche{T} du menu

\Concept  Passer une ligne

Pour passer une ligne, appuyer simultan\'ement sur les touches \touche{Maj}et \touche{Entr\'ee}.  
Ceci est tr\`es utile pour formater de mani\`ere lisible une s\'equence de commandes complexes. 
\medskip

\Concept  Aide

Maple dispose d'une rubrique d'aide (en anglais) � laquelle on acc\`ede : \pn 
soit par le menu {\bf Help$\backslash$Topic Search} pour une recherche par th\`emes, \pn 
soit par le menu {\bf Help$\backslash$Full Text Search}   pour une recherche par mots cl\'es. 
\medskip

L'aide donne pour chaque commande la syntaxe \`a utiliser, ses param\`etres, une d\'escriptio d\'etaill\'ee 
de son fonctionnement et de son utilisation qu'elle illustre par quelques exemples. 
\medskip

A partir du prompt, il est \'egalement possible d'acc\'eder \`a la rubrique d'aide d'une commande dont on connait la syntaxe. 
Pour cela, apr\`es le prompt, il suffit de taper un point d'exclamation "?" le nom de la commande puis d'appuyer sur la touche ``Entr\'ee''. 
$$
\red>\  \Touche{?\ }\  \hbox{\bf Commande}\ \Touche{\black Entr\'ee}
$$

\Concept  Variantes selon les claviers

Sur certains claviers, \pn 
la touche \touche{Maj} est la touche \touche{Shift} ou \touche{$\Uparrow$} ; \pn
la touche  \touche{Entr\'ee} est la touche \touche{Return} ou \touche{{$\tenpts \hookleftarrow$}}. 




\Section Maple1Calculatrice, Calculatrice. 

Maple est un formidable outil. Qu'on puisse s'en servir pour effectuer des calculs simples, \`a l'instar d'une calculatrice, est la moindre des choses. 
Dans ce qui suit, nous expliquons comment proc\'eder. 
\bigskip

\Concept  Syntaxe d'une expression Maple. 

Mapple execute les commandes suivies d'un {\it point-virgule} \touche{;\ } en affichant leur r\'esultat. 
Mapple execute les commandes suivies de {\it deux-points} \touche{:\ } sans afficher leur r\'esultat.
\bigskip

Une expression Maple comporte une ou plusieurs commandes, chacune d'entre elles 
\'etant suivie par  un {\it point-virgule} \touche {;\ } ou {\it deux-points} \touche{:\ }. 
$$
\eqalign{
&\hskip-12em\red{>\UnderBrace{\red 7*8}_{ordre $1$}\ \Touche{:\ }\  \UnderBrace{\red 1332/2}_{ordre $2$}\ \Touche{;\ }\ 
\UnderBrace{\red \log(3)}_{ordre $3$}\ \Touche{;\ }\ \UnderBrace{\red \exp(7)}_{ordre $4$}\ \Touche{:\ }}
\cr
&\hskip10em{\blue666}
\cr
&\hskip10em{\blue\ln(3)}
}
$$


\Concept  Constantes 

Pour les constantes c\'el\`ebres, Maple utilise une syntaxe tr\`es particuli\`ere faisant intervenir des caract\`eres minuscules et parfois majuscules. 

$$\vbox{
\offinterlineskip
\halign{
\vrule#&\quad\hfil#\hfil\strut\quad&\vrule#&\quad\hfil#\hfil\quad&\vrule#&\quad\hfil#\hfil\quad&\vrule#\tabskip=0pt\cr
\noalign{\hrule}\cr
&Math\'ematiques&&Maple&&Description&\cr
\noalign{\hrule}\cr
&$\pi$&&Pi&&le nombre $\pi=3.14159265\ldots$&\cr
\noalign{\hrule}\cr
&$\pi$&&pi&&la lettre grecque $\pi$&\cr
\noalign{\hrule}\cr
&$i$&&I&&le nombre complexe $i$ racine de $-1$&\cr
\noalign{\hrule}\cr
&$\e$&&E&&la base $\e$ du logarithme n\'ep\'erien&\cr
\noalign{\hrule}\cr
&$\infty$&&infinity&&le symbole infini&\cr
\noalign{\hrule}\cr
&$\gamma$&&gamma&&la constante d'Euler $\gamma=0.577215\ldots$ &
\cr
\noalign{\hrule}\cr
}}
$$

\noindent{\it Attention :} ne pas utiliser les mots r\'eserv\'es ``E'', ``I'' ou ``gamma'' comme nom de variable !
\medskip



\Concept  Op\'erateurs \'el\'ementaires

$$\vbox{
\offinterlineskip
\halign{
\vrule#&\quad#\hfil\strut\quad&\vrule#&\quad\hfil#\hfil\quad&\vrule#\tabskip=0pt\cr
\noalign{\hrule}\cr
&Math\'ematiques&&Maple&\cr
\noalign{\hrule}\cr
&Addition&&+&\cr
\noalign{\hrule}\cr
&Soustration&&-&\cr
\noalign{\hrule}\cr
&Multiplication&&*&\cr
\noalign{\hrule}\cr
&Division&&/&\cr
\noalign{\hrule}\cr
&Factorielle&&!&\cr
\noalign{\hrule}\cr
}}
$$

\Concept  Puissances

$$\vbox{
\offinterlineskip
\halign{
\vrule#&\quad#\hfil\strut\quad&\vrule#&\quad\hfil#\hfil\quad&\vrule#&\quad#\hfil\quad&\vrule#&\quad#\hfil\quad&\vrule#\tabskip=0pt\cr
\noalign{\hrule}\cr
&Math\'ematiques&&Maple&&Exemple Maple&&Exemple Math\'ematiques&\cr
\noalign{\hrule}\cr
&\hfil Puissance&&$\hat{\quad}$&&\hfil$2\hat{\quad}3$&&\hfil$2^{3\strut}$&\cr
\noalign{\hrule}\cr
}}
$$


\Concept  Racine, exponentielle, logarithmes, puissances

$$\vbox{
\offinterlineskip
\halign{
\vrule#&$\quad\vcenter to1.5em{}$\hfil#\hfil\strut\quad&\vrule#&\quad\hfil#\hfil\quad&\vrule#\tabskip=0pt\cr
\noalign{\hrule}\cr
&Math\'ematiques&&Maple&\cr
\noalign{\hrule}\cr
&$\sqrt x$&&sqrt(x)&\cr
\noalign{\hrule}\cr
&$\exp(x)$ ou $\e^x$&&exp(x)&\cr
\noalign{\hrule}\cr
&$\ln(x)$ ou $\log(x)$&&ln(x) ou log(x)&\cr
\noalign{\hrule}\cr
&$\log_{10}(x)$&&log10(x)&\cr
\noalign{\hrule}\cr
&$\log_b(x)$&&log[b](x)&\cr
\noalign{\hrule}\cr
}}$$

\Concept  Fonctions trigonometriques et hyperboliques

$$\vbox{
\offinterlineskip
\halign{
\vrule#&\quad\hfil#\hfil\strut\quad&\vrule#&\quad\hfil#\hfil\quad&\vrule#\tabskip=0pt\cr
\noalign{\hrule}\cr
&Math\'ematiques&&Maple&\cr
\noalign{\hrule}\cr
&$\cos(x)$&&cos(x)&\cr
\noalign{\hrule}\cr
&$\sin(x)$&&sin(x)&\cr
\noalign{\hrule}\cr
&$\tan(x)$&&tan(x)&\cr
\noalign{\hrule}\cr
&$\cot(x)$&&cot(x)&\cr
\noalign{\hrule}\cr
}}\qquad
\vbox{
\offinterlineskip
\halign{
\vrule#&\quad\hfil#\hfil\strut\quad&\vrule#&\quad\hfil#\hfil\quad&\vrule#\tabskip=0pt\cr
\noalign{\hrule}\cr
&Math\'ematiques&&Maple&\cr
\noalign{\hrule}\cr
&$\ch(x)$&&cosh(x)&\cr
\noalign{\hrule}\cr
&$\sh(x)$&&sinh(x)&\cr
\noalign{\hrule}\cr
&$\th(x)$&&tanh(x)&\cr
\noalign{\hrule}\cr
&$\coth(x)$&&cotanh(x)&\cr
\noalign{\hrule}\cr
}}
$$

\Concept  Fonctions trigonometriques et hyperboliques r\'eciproques

$$\vbox{
\offinterlineskip
\halign{
\vrule#&\quad\hfil#\hfil\strut\quad&\vrule#&\quad\hfil#\hfil\quad&\vrule#\tabskip=0pt\cr
\noalign{\hrule}\cr
&Math\'ematiques&&Maple&\cr
\noalign{\hrule}\cr
&$\arccos(x)$&&arccos(x)&\cr
\noalign{\hrule}\cr
&$\arcsin(x)$&&arcsin(x)&\cr
\noalign{\hrule}\cr
&$\arctan(x)$&&arctan(x)&\cr
\noalign{\hrule}\cr
&peu utilis\'ee&&arccot(x)&\cr
\noalign{\hrule}\cr
}}\qquad
\vbox{
\offinterlineskip
\halign{
\vrule#&\quad\hfil#\hfil\strut\quad&\vrule#&\quad\hfil#\hfil\quad&\vrule#\tabskip=0pt\cr
\noalign{\hrule}\cr
&Math\'ematiques&&Maple&\cr
\noalign{\hrule}\cr
&$\argch(x)$&&arccosh(x)&\cr
\noalign{\hrule}\cr
&$\argsh(x)$&&arcsinh(x)&\cr
\noalign{\hrule}\cr
&$\argth(x)$&&arctanh(x)&\cr
\noalign{\hrule}\cr
&peu utilis\'ee&&arccotanh(x)&\cr
\noalign{\hrule}\cr
}}
$$

\Concept  Valeur absolue, parties entieres et fractionnaires, arrondi

$$\vbox{
\offinterlineskip
\halign{
\vrule#&\quad\hfil#\hfil\strut\quad&\vrule#&\quad\hfil#\hfil\quad&\vrule#\tabskip=0pt\cr
\noalign{\hrule}\cr
&Math\'ematiques&&Maple&\cr
\noalign{\hrule}\cr
&$|x|$&&abs(x)&\cr
\noalign{\hrule}\cr
&partie enti\`ere de $x$&&floor(x)&\cr
\noalign{\hrule}\cr
&partie fractionnaire de $x$&&frac(x)&\cr
\noalign{\hrule}\cr
&arrondi de $x$ \`a l'entier le plus proche&&round(x)&\cr
\noalign{\hrule}\cr
}}
$$

\Concept  \'Ecriture d\'ecimale d'un nombre 

La commande {\it evalf($x$)} retourne une \'evaluation d\'ecimale \`a $10$ chiffres significatifs du nombre $x$. 
On peut changer le nombre $n$ de chiffres significatifs \`a l'aide de la variable (reserv\'ee) {\it $Digits$}. 
$$
\eqalign{
&{\red >Digits:=20;}
\cr
&\hskip 10em{\blue Digits:=20}
\cr
&{\red >evalf(4/3);}
\cr 
&\hskip10em\UnderBrace{\blue 1.3333333333333333333}_{$20$ chiffres significatifs}
}
$$
On peut \'egalement utiliser la syntaxe {\it evalf($x,n$)} qui retourne une \'evaluation d\'ecimale avec $n$ chiffres significatifs du nombre $x$. 
$$
\eqalign{
&{\red >evalf(Pi,6);}
\cr 
&\hskip10em\UnderBrace{\blue 3.14159}_{$6$ chiffres significatifs}
}
$$


\Section Maple1Variables, Variables. 

\Concept  Variables

Comme tous les langages, Maple utilise des variables. Le nom d'une variable doit commencer par une lettre, comporter moins de 500 caract\`eres et peut comporter des lettres, des chiffres ou le symbole soulign\'e \touche{\_\strut}. 
\bigskip

{\bf Attention \`a la casse !} Maple fait la diff\'erence entre les majuscules et les minuscules pour les noms de commandes et de variables. 
Il faut \^etre tr\`es vigilant quant \`a l'orthographe de vos commandes, car c'est une source tr\`es fr\'equente d'erreur. 
\bigskip

\Concept  Affectation. 

Pour affecter une valeur \`a une variable, on \'ecrit ``variable:=valeur''
$$
\eqalign{
&\red\hskip-16em>  xa:=2*3;
\cr
&\blue xa:=6
\cr
&\red\hskip-16em>  xb:=5*xa;
\cr
&\blue xb:=30
}
$$

\Concept  Lib\'erer les variables

On peut r\'einitialiser Maple et, ce faisant, lib\'erer toutes les variables utilis\'ees via la commande {\it restart}

$$
\eqalign{
&\hskip-16em\red>  x:=1;
\cr
&\blue x:=1
\cr
&\hskip-16em\red>  restart;
\cr
&\hskip-16em\red>  x;
\cr
&\blue x
}
$$

On peut \'egalement lib\'erer une variable en lui affectant son nom \`a l'aide de la touche apostrophe \touche{'}, c'est \`a dire en tapant 
$
variable:=\touche{'}\ variable\ \touche{'}
$

$$
\eqalign{
&\hskip-16em\red>  x:=1:
\cr
&\hskip-16em\red>  x;
\cr
&\blue 1
\cr
&\hskip-16em\red>  x:='x':
\cr
&\hskip-16em\red>  x;
\cr
&\blue x
}
$$
\eject
\centerline{\fourteenbf Exercices}

\noindent{Exercice 1. }Donner une valeur d\'ecimale \`a 10 chiffres significatifs des nombres  
$$
a_1:={1-{\pi+3\F \e+4}\F 1-{4\F1+\pi}}\qquad a_2:={4\F1+{1^2\F2+{3^2\F2+{5^2\F2+{7^2\F2+9^2}}}}}\qquad 
a_3:={\sqrt{\sqrt2+\sqrt3}\F 1-(\sqrt 3+\sqrt2)(\sqrt5-\sqrt2)}
$$
\medskip
\noindent{Exercice 2. }\'Etant donn\'e le polyn\^ome $P:x\mapsto 5x^4+3x^3-x+3$, 
determiner les 12 premiers chiffres apr\`es la virgule des nombres  
$$
b_1:=P\Q({1\F3}\W), \qquad b_2:=P\Q(-{1\F4}\W)\qquad\hbox{et}\qquad b_3:=P(\pi)
$$
\medskip

\noindent{Exercice 3. }Soit $f$ la fonction $\ds f:x\mapsto {1\F 1+x^2}$. \pn
a) Dessiner le graphique de cette fonction pour $x\in[-5,5]$. {\it on pourra utiliser ``plot''}. \pn
b) Calculer les nombres d\'eriv\'es $f'(0)$ et $f'(1)$. {\it on pourra utiliser ``$D$''}. \pn
c) Calculer $f''(x)$ et dessiner son graphe sur $[-5,5]$. 
\bigskip

 
\noindent{Exercice 4. }Pour $n\in\ob N$ et $x\in\ob R$, calculer {\it (en utilisant sum)} les sommes 
$$
S_1:=\sum_{k=1}^nk, \qquad S_2:=\sum_{k=1}^nk^2\qquad \hbox{et}\qquad S_3:=\sum_{k=0}^{n-1}x^k
$$
b) D\'evelopper les expressions obtenues pour $S_1$ et  $S_2$ {\it avec expand ou simplify}. 
\pn
c) Que vaut $S_3$ pour $x={1\F2}$ et $n=5$ ? et pour $x={1\F3}$ et $n=10$ ? 
\bigskip

\noindent{Exercice 5. }Pour $n\in\ob N$ et $x\in\ob R$, calculer la somme 
$$
T_n:=\sin^3{x\F 3}+3\sin^3{x\F 3^2}+\cdots+3^{n-1}\sin^3{x\F 3^n}.
$$

\medskip
\noindent{Exercice 6. }Tracer la courbe param\'etr\'ee par 
$$
\Q\{\eqalign{x(t)=t^3-4t\cr
y(t):=2t^2-3}\W.\qquad(-4\le t\le 4).
$$
On pourra regarder \`a {\it plot} dans la rubrique {\it parametric} 

\bye