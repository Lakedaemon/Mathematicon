%%%%%     Banque de solutions  PTSI 2005 (commencée le 25/08/2005)
%
\sol PTSIa. 
Le cercle trigonom\'etrique \'etant sym\'etrique par rapport \`a l'axe des ordonn\'ees, nous observons que 
$\cos(\pi-\theta)=-\cos(\theta)$ pour $\theta\in\ob R$ et nous remarquons d'une part que 
$$
A=\cos\Q({\pi\F10}\W)+\cos\Q({4\pi\F10}\W)+\underbrace{\cos\Q({6\pi\F10}\W)}_{-\cos\Q({4\pi\F10}\W)}
+\underbrace{\cos\Q({9\pi\F10}\W)}_{-\cos\Q({\pi\F10}\W)}=0
$$
et d'autre part que  
$$
B=\cos^2\Q({\pi\F10}\W)+\cos^2\Q({4\pi\F10}\W)+\underbrace{\cos^2\Q({6\pi\F10}\W)}_{\cos^2\Q({4\pi\F10}\W)}
+\underbrace{\cos^2\Q({9\pi\F10}\W)}_{\cos^2\Q({\pi\F10}\W)}=2\cos^2\Q({\pi\F10}\W)+2\cos^2\Q({4\pi\F10}\W).
$$
Comme $2\cos^2(\theta)=1+\cos(2\theta)$ pour $\theta\in\ob R$ (formule de duplication), 
en~proc\'edant  comme pr\'ec\'edemment, nous obtenons alors que 
$$
B=1+\cos\Q({2\pi\F10}\W)+1+\underbrace{\cos\Q({8\pi\F10}\W)}_{-\cos\Q({2\pi\F10}\W)}=2.
$$

\sol PTSIb. 
Commen\c cons par d\'eterminer les nombres r\'eels $x$ pour lesquels l'\'equation \eqref{Eqexob} a un sens. \pn
La premi\`ere tangente est d\'efinie si, et seulement si 
$$
2x-{3\pi\F2}\not\equiv{\pi\F 2}\quad[\pi]\ssi 2x\not\equiv2\pi\quad[\pi]\ssi x\not\equiv\pi\quad\Q[{\pi\F2}\W]\ssi x\not\equiv0\quad\Q[{\pi\F2}\W]
$$
La seconde tangente existe si, et seulement si 
$$
{\pi\F4}-x\not\equiv{\pi\F 2}\quad[\pi]\ssi {\pi\F4}-{\pi\F 2}\not\equiv x\quad[\pi]\ssi x\not\equiv-{\pi\F2}\quad[\pi]
$$
En particulier, l'ensemble des nombres r\'eels pour lesquels les deux tangentes sont simultan\'ement d\'efinies  est 
l'ensemble 
$$
E:=\ob R\ssm{\pi\F2}\ob Z.
$$ 
Rappelons que la tangente est une fonction $\pi$-p\'eriodique et que sa restriction $\tan:]-{\pi\F2},{\pi\F2}[\to\ob R$ est une bijection. 
En particulier, les deux tangentes sont \'egales si, et seulement si leurs angles sont \'egaux modulo $\pi$, c'est \`a dire si
$$
2x-{3\pi\F2}\equiv{\pi\F4}-x\quad[\pi]\ssi3x\equiv{7\pi\F4}\quad[\pi]\ssi x\equiv{7\pi\F12}\quad\Q[{\pi\F3}\W].
$$ 
Enfin, comme les nombres ${3\pi\F12}$, ${7\pi\F12}$, ${11\pi\F12}$, ${15\pi\F12}$, ${19\pi\F12}$ et ${23\pi\F12}$ appartiennent tous \`a l'ensemble~$E$, 
l'ensemble des solutions de l'\'equation \eqref{Eqexob} est  
$$
S:=\Q\{{7\pi\F12}+{\pi\F3}k:k\in\ob Z\W\}={7\pi\F12}+{\pi\F3}\ob Z.
$$
\Remarque : \'etant donn\'e un nombre $x\in\sc E$, une variante consiste \`a proc\'eder de la fa\c con suivante :   
$$
\eqalign{
\eqref{Eqexob}&\ssi{\sin\Q(2x-{3\pi\F4}\W)\F\cos\Q(2x-{3\pi\F4}\W)}-{\sin\Q({\pi\F4}-x\W)\F\cos\Q({\pi\F4}-x\W)}=0\cr
&\ssi\underbrace{\sin\Q(2x-{3\pi\F4}\W)\cos\Q({\pi\F4}-x\W)-\cos\Q(2x-{3\pi\F4}\W)\sin\Q({\pi\F4}-x\W)}_{\sin\Q[2x-{3\pi\F4}-({\pi\F4}-x)\W]}=0\cr
&\ssi 2x-{3\pi\F4}\equiv{\pi\F4}-x\quad[\pi].
}
$$

\sol PTSIc. 
Soit $x$ un nombre r\'eel. Comme $\sin(\theta)=\cos\Q(\theta-{\pi\F2}\W)$ pour $\theta\in\ob R$, nous remarquons que 
$$
\eqref{Eqexoc}\ssi\cos\Q(x+{\pi\F4}-{\pi\F2}\W)=\cos\Q(2x-{\pi\F3}\W)\ssi \cos\Q(x-{\pi\F4}\W)=\cos\Q(2x-{\pi\F3}\W). 
$$
Comme deux cosinus sont \'egaux si, et seulement si leurs angles sont \'egaux ou oppos\'es modulo $2\pi$, il suit
$$
\eqref{Eqexoc}\ssi x-{\pi\F 4}=\pm\Q(2x-{\pi\F3}\W)\quad[2\pi]\ssi \Q\{
\eqalign{
&x={\pi\F3}-{\pi\F4}\quad[2\pi]\cr
&\hbox{ou}\cr
&3x={\pi\F3}+{\pi\F4}\quad[2\pi]\cr}\W.
$$
En conclusion, l'ensemble des solutions de l'\'equation \eqref{Eqexoc} est 
$$
S:=\Q\{x\in\ob R:x\equiv{\pi\F12}\quad[2\pi]\ \hbox{ou}\ x\equiv{7\pi\F 36}\quad\Q[{2\pi\F 3}\W]\W\}
$$

\sol PTSIxk. 
\noindent
1a. L'application $f$ est d\'erivable sur $I$, de d\'eriv\'ee
$$
\forall x\in\Q]0,{1\F\sqrt6}\W[, \qquad f'(x)=1-6x^2>1-6\times\Q({1\F\sqrt6}\W)^2=0.
$$
En particulier, $f'$ est strictement positive sur $I$ et, par suite, la fonction $f $est strictement croissante sur $I$. 
Comme $f(0^+)=0$ et comme 
$$
f\Q({1\F\sqrt6}^-\W)={1\F\sqrt6}-2\times\Q({1\F 6}\W)^3={1\F\sqrt6}\times\Q(1-{2\F 6}\W)={2\F 3\sqrt6}<{1\F\sqrt6},
$$
nous en d\'eduisons que $f(I)=\Q]0,{2\F3\sqrt6}\W[\subset I$. \medskip\noindent
1b. Pour $n\ge1$, prouvons par r\'ecurrence la proposition 
$$
u_n\in I.\leqno{(\sc P_n)}
$$
i) $\sc P_1$ est vraie car $u_1:={1\F 10}\in I$. \pn
ii) Soit $n\ge1$ tel que $\sc P_n$ soit vraie. Alors $u_n\in I$, et nous d\'eduisons du r\'esultat de la question pr\'ec\'edente que $u_{n+1}=f(u_n)\in I$. En particulier, $\sc P_{n+1}$ est vraie.\pn
Nous avons montr\'e par r\'ecurrence que $u_n\in I$ et \`a fortiori que $u_n>0$ pour $n\ge1$. En remarquant que 
$$
u_{n+1}-u_n=-2u_n^3<0\qquad (n\ge1), 
$$
nous en d\'eduisons que la suite $u$ est d\'ecroissante. \medskip\noindent
1c. Comme la suite $u$ est monotonne et minor\'ee par $0$, elle converge vers un nombre r\'eel $\ell\ge0$. En passant \`a la limite dans l'identit\'e 
$$
u_{n+1}=u_n-2u_n^3\qquad(n\ge1), 
$$
nous obtenons alors que $\ell=\ell-2\ell^3$. A fortiori, il suit  $-2\ell^3=0$ et $\ell=0$.\pn
En conclusion, la suite $u$ converge vers $0$. 
\medskip\noindent
2a. D'apr\`es la d\'efinition des limites du cours, 
$$
\lim v_n=\ell\ssi\forall\epsilon>0,\exists N\ge0:\forall n\ge N, |v_n-\ell|<\epsilon
$$ 
2b. Soient $p$ et $n$ des entiers tels que $1\le p\le n$. Nous remarquons que
$$ 
M_n-\ell={1\F n}\sum_{1\le k\le n}v_k-\ell={1\F n}\Q(\sum_{1\le k\le n}v_k-n\ell\W)={1\F n}\sum_{1\le k\le n}(v_k-\ell)
$$
et nous d\'eduisons de la premi\`ere in\'egalit\'e triangulaire que 
$$
\eqalign{
|M_n-\ell|&=\Q|{1\F n}\sum_{1\le k\le n}(v_k-\ell)\W|={1\F n}\Q|\sum_{1\le k\le n}(v_k-\ell)\W|
\cr
&\le {1\F n}\sum_{1\le k\le n}|v_k-\ell|={1\F n}\sum_{1\le k\le p}|v_k-\ell|+ {1\F n}\sum_{p< k\le n}|v_k-\ell|.
}
$$ 
En remarquant que $|v_k-\ell|\le\max_{p<m\le n}|v_m-\ell|$ lorsque $p<k\le n$, nous en d\'eduisons que 
$$
|M_n-\ell|\le {1\F n}\sum_{1\le k\le p}|v_k-\ell|+ {1\F n}\sum_{p< k\le n}\max_{p<m\le n}|v_m-\ell|.
$$
Comme il y a exactement $n-p$ entier dans l'intervalle $]p,n]$ et comme $0<(n-p)/n\le 1$, il suit 
$$
|M_n-\ell|\le {1\F n}\sum_{1\le k\le p}|v_k-\ell|+{n-p\F n}\max_{p<m\le n}|v_m-\ell|\le  {1\F n}\sum_{1\le k\le p}|v_k-\ell|+\max_{p<m\le n}|v_m-\ell|.
$$
2c. Supposons que la suite $v$ converge vers $\ell$ et montrons que la suite $M$ converge vers $\ell$. \pn
Fixons $\epsilon>0$ et utilisons la d\'efinition de la convergence de la suite $v$ pour $\epsilon'=\epsilon/2$ : 
comme la suite $v$ converge vers $0$, il existe un entier $N\ge1$ tel que 
$$
\forall n\ge N, \qquad |v_n-\ell|<\epsilon'={\epsilon\F2}.
$$
Nous prenons $p=N$ et nous remarquons alors que
$$
|M_n-\ell|\le  {1\F n}\sum_{1\le k\le p}|v_k-\ell|+{\epsilon\F2}\qquad (n\ge p).
$$
Le nombre $p$ \'etant fix\'e, la quantit\'e $c:=1+\sum_{1\le k\le p}|v_k-\ell|$ est un nombre r\'eel strictement positif. 
Lorsque ${c\F n}<{\epsilon\F 2}$, c'est-\`a-dire lorsque $n>{2c\F\epsilon}$, nous obtenons alors que 
$$
\forall n>{2c\F\epsilon}, \qquad {1\F n}\sum_{1\le k\le p}|v_k-\ell|={c\F n}<{\epsilon\F 2}
$$
et posant $N':=N+{2c\F\epsilon}\ge0$, nous remarquons que la condition $n\ge N'$ induit d'une part que $n\ge N$ mais aussi que $n>{2c\F\epsilon}$ et nous concluons que 
$$
\eqalign{
|M_n-\ell|&\le {1\F n}\sum_{1\le k\le p}|v_k-\ell|+{\epsilon\F2}\cr
&\le{\epsilon\F 2}+{\epsilon\F2}=\epsilon\qquad (n\ge N').
}
$$
En particulier, si la suite $v$ converge vers $\ell$, nous avons montr\'e que 
$$
\forall\epsilon>0, \exists N'\ge0:\forall n\ge N', |M_n-\ell|<\epsilon, 
$$
c'est-\`a-dire que la suite $M$ converge vers $\ell$. \medskip\noindent
3a. Commen\c cons par poser
$$
g(x):={1\F(x-2x^3)^2}-{1\F x^2}={1\F x^2}\Q({1\F(1-2x^2)^2}-1\W).
$$
Comme $1/(1-u)=1+u+o_0(u)$, un petit calcul de d\'eveloppement limit\'e 
(l'arme absolue) donne :
$$
g(x)={1\F x^2}\Q(\b(1+2x^2+o_0(x^2)\b)^2-1\W)={1\F x^2}\Q(1+4x^2+o_0(x^2)-1\W)=4+o_0(1).
$$
En particuilier, nous obtenons que 
$$
\lim_{x\to 0\atop x\neq0}\Q({1\F(x-2x^3)^2}-{1\F x^2}\W)=\lim_{x\to 0\atop x\neq0}g(x)=4. 
$$
Pour chaque $n\ge1$, la relation $u_{n+1}=u_n-2u_n^3$  induit que 
$$
v_n:=v_n:={1\F u_{n+1}^2}-{1\F u_n^2}={1\F (u_n-2u_n^3)^2}-{1\F u_n^2}=g(u_n).
$$
Comme $u$ converge vers $0$ (sans jamais \^etre \'egale \`a $0$) et comme $g$ converge vers $4$ en $0$, une composition de limite donne alors que  
$$
\lim_{n\to\infty}v_n=4. 
$$
3b. Comme la suite $v$ converge vers $4$, il r\'esulte du th\'eor\`eme de Ces\`aro que la suite $M$ d\'efinie par la relation 
$$
M_n:={1\F n}\sum_{1\le k\le n}v_k
$$
converge \'egalement vers $4$. En utilisant l'identit\'e $v_n={1\F u_{n+1}^2}-{1\F u_n^2}$ ainsi que le principe des sommes telescopiques, nous obtenons alors que   
$$
4\sim M_n={1\F n}\sum_{1\le k\le n}\Q({1\F u_{k+1}^2}-{1\F u_k^2}\W)={1\F n}\Q({1\F u_{n+1}^2}-{1\F u_1^2}\W).
$$
Comme $u_{n+1}$ converge vers $0$, le terme $u_1$ est n\'egligeable. De sorte que 
$$
4\sim {1\F nu_{n+1}^2}\sim {1\F(n+1)u_{n+1}^2}.
$$
En passant \`a l'inverse, il vient alors  $(n+1)u_{n+1}^2\sim{1\F4}$ et un d\'ecalage d'indices induit alors que la suite $nu_n^2$ converge vers $1/4$. Comme $u_n>0$ pour $n\ge1$, nous prenons la racine pour en d\'eduire que la suite $u_n\sqrt n$ converge vers ${1/2}$ et nous concluons que  
$$
u_n\sim{1\F 2\sqrt n}.
$$
4a. En substituant, nous obtenons que 
$$
\eqalign{
f(1)&=X(1-1)=0\cr
f(X)&=X\big(X-(X-1)\big)=X\cr
f(X^2)&=X\big(X^2-(X-1)^2\big)=X(2X-1)=2X^2-X.
}
$$
4b. Soit $P\in\ob R_n[X]$ de degr\'e $n$. Alors, il existe des coefficients $(a_k)_{0\le k\le n}$ r\'eels tels que $a_n\neq0$ et 
$$
P=\sum_{0\le k\le n}a_kX^k.
$$
Alors, nous d\'eduisons du bin\^ome de Newton que 
$$
\eqalign{
P(X-1)&=\sum_{0\le k\le n}a_k(X-1)^k=a_n(X-1)^n+\sum_{0\le k<n}a_k(X-1)^k
\cr
&=a_nX^n+a_n\sum_{0\le k<n}\Q({n\atop k}\W)(-1)^{n-k}X^k+\sum_{0\le k<n}a_k(X-1)^k
}
$$
En particulier, nous remarquons que $P=a_nX^n+Q+R$ avec
$$
\eqalign{
Q&:=a_n\sum_{0\le k<n}\Q({n\atop k}\W)(-1)^{n-k}\underbrace{X^k}_{\deg<n},\cr
R&:=\sum_{0\le k<n}a_k\underbrace{(X-1)^k}_{\deg<n},
}
$$ 
qui sont deux polyn\^omes de $\ob R_{n-1}[X]$. A fortiori, le terme de plus haut degr\'e de $P(X-1)$ est $a_nX^n$. \medskip\noindent
4c. Soit $P\in\ob R_n[X]$ tel que $P(X)=P(X-1)$ et soit $Q:=P-P(0)$. Pour $n\ge0$, \'etablissons par r\'ecurrence la propri\'et\'e 
$$
Q(n)=0.\leqno{(\sc P_n)}
$$
i) la propri\'et\'e $\sc P_0$ est vraie car $Q(0)=P(0)-P(0)=0$. \pn
ii) Soit $n\in\ob N$ tel que $\sc P_n$ soit vraie. Alors $Q(n)=0$. En substituant $n+1$ \`a $X$ dans l'\'identit\'e $P(X)=P(X-1)$, nous obtenons alors que $P(n)=P(n+1)$ et nous en d\'eduisons que 
$$
Q(n+1)=P(n+1)-P(0)=P(n)-P(0)=Q(n)=0. 
$$
A fortiori, la proposition $\sc P_{n+1}$ est vraie. Nous avons prouv\'e par r\'ecurrence que tous les nombres entiers sont racines du polyn\^ome $Q$. Comme le seul polyn\^ome admettant plus de racine que son degr\'e est le polyn\^ome nul, nous en d\'eduisons que $P-P(0)=Q=0$ et que $P$ est le polyn\^ome constant $P(0)$. 
\medskip\noindent
5. i) Soit $P\in\ob R_n[X]$. Comme $\ob R[X]$  est stable par substitution, combinaison lin\'eaire et multiplication, nous remarquons que $f(P)$ est un polyn\^ome de $\ob R[X]$ (l'application $f$ est donc bien d\'efinie). Par ailleurs, d'apr\`es le r\'esultat de la question 4b, le polyn\^ome $P(X-1)$ est de terme principal $a_nX^n$. Comme c'est le m\^eme terme principal que $P$, ils se simplifient dans le polyn\^ome $P(X)-P(X-1)$ qui appartient de ce fait \`a $\ob R_{n-1}[X]$.
A fortiori, le polyn\^ome $f(P)$ appartient \`a l'espace $\ob R_n[X]$ et l'on peut donc restreindre l'application $f$ \`a l'arriv\'ee \`a l'espace $\ob R_n[X]$. \pn
ii) $\ob R_n[X]$ est un $\ob R$-espace vectoriel, d'apr\`es le cours. \pn
iii) $f$ est lin\'eaire car, \'etant donn\'es $P$ et $Q$ dans $\ob R_n[X]$ et $\lambda$ et $\mu$ dans $\ob R$, nous avons 
$$
\eqalign{
f(\lambda P+\mu Q)&=X\big(\lambda P+\mu Q-(\lambda P+\mu Q)(X-1)\big)=X\big(\lambda P+\mu Q-\lambda P(X-1)-\mu Q(X-1)\big)\cr
&=\lambda X\big(P-P(X-1)\big)+\mu\big(Q-Q(X-1)\big)=\lambda f(P)+\mu f(Q).
}
$$ 
En conclusion, $f:\ob R_n[X]\to\ob R_n[X]$ est une application lin\'eaire, c'est un endomorphisme de $\ob R_n[X]$. 
\medskip\noindent
6. Nous d\'eduisons du r\'esultat de la question 4c que  
$$
\eqalign{
P\in\ker P&\ssi f(P)=0\ssi X\big(P(X)-P(X-1)\big)=0\ssi P(X)=P(X-1)\cr
&\ssi P \hbox{ est un polyn\^ome constant}\ssi P\in\ob R_0[X].}
$$
A fortiori, le noyau de $f$ est $\ker f=\ob R_0[X]$, qui est de dimension $1$. En appliquant le th\'eor\`eme du rang, nous obtenons alors que 
$$
\dim\IM f=\dim\ob R_n[X]-\dim\Ker f=n+1-1=n.
$$
Remarque : il n'est pas tr\`es difficile d'en d\'eduire que $\IM f=X\ob R_{n-1}[X]$. \medskip\noindent
7a. La base canonique de $\ob R_2[X]$ est $\sc B:=\{1, X, X^2\}$ et la matrice de $f$ dans cette base est 
$$
A:=\hbox{\sc M\rm at}_{\hbox{\sc B}}f=
\pmatrix{
0&0&0\cr
0&1&-1\cr
0&0&2\cr}
$$
7b. L'endomorphisme est diagonalisable car ses valeurs propres 0, 1 et 2 (que l'on peut lire sur la diagonale puisque c'est une matrice triangulaire sup\'erieure) sont distinctes deux \`a deux. \medskip\noindent
8a. Comme le polyn\^ome $\sc P_k$ est de degr\'e $k$ pour $0\le k\le n$, la famille de polyn\^ome $(P_0, P_1, \cdots, P_n)$ est \'echelonn\'es en degr\'e. C'est donc une famille libre. Comme elle comporte $n+1$ \'el\'ements de $\ob R_n[X]$,  qui est un espace vectoriel de dimension $n+1$, c'est aussi une base de $\ob R_n[X]$. \medskip\noindent
8b. Soit $k\ge2$. Alors, en proc\'edant au changement d'indice $\ell+1=m$, nous obtenons que 
$$
\eqalign{
f(P_k)&=X\big(P(X)-P(X-1)\big)=X\prod_{0\le\ell<k}(\ell-X)-X\prod_{0\le\ell<k}\big(\ell-(X-1)\big)
\cr
&=X\prod_{0\le\ell<k}(\ell-X)-X\prod_{0\le\ell<k}\big(\ell+1-X)\big)=X\prod_{0\le\ell<k}(\ell-X)-X\prod_{1\le\ell\le k}(m-X)
\cr}
$$
En factorisant ce qui peut l'\^etre, il vient  
$$
\eqalign{
f(P_k)&=X\prod_{1\le\ell<k}(\ell-X)\times \big(-X-(k-X)\big)=-kX\prod_{1\le\ell<k}(\ell-X)=k\prod_{0\le\ell<k}(\ell-X)\cr
&=kP_k
}
$$
Comme $f(1)=0=0.1$ et comme $f(X)=X=1.X$, nous concluons que 
$$
\forall k\in\ob N, \qquad f(P_k)=kP_k. 
$$
8c. les valeurs propres de $f$ sont les entiers $0, 1, \cdots, n$ d'apr\`es le r\'esultat de la question pr\'ec\'edente. 
Elles sont distinctes deux \`a deux, donc $f$ est diagonalisable (on a diagonalis\'e $f$ en trouvant une base de vecteurs propres de $f$, les polyn\^omes $P_0, \cdots, P_n$).
\medskip\noindent
9. Sur $]0, \infty[$, l'\'equation diff\'erentielle $(E)$ est \'equivalente (on divise par $|x|=x$) \`a 
$$
y'(x)+\Q(1-{1\F x}\W)y(x)=x.
$$
La fonction $f_p:x\mapsto x$ est trivialement solution particuli\`ere de cette \'equation diff\'erentielle lin\'eaire du premier ordre avec second membre (dont le coefficient de y' vaut $1$, \`a coefficients continus sur $]0,\infty[$, car 
$$
\forall x>0, \qquad f_p'(x)+\Q(1-{1\F x}\W)f_p(x)=1+\Q(1-{1\F x}\W)x=x. 
$$
Par ailleurs, la fonction $g_p:x\mapsto x\e^{-x}$ est une solution, ne s'annulant pas, de l'\'equation diff\'erentielle homog\`ene associ\'ee sur $]0,+\infty[$ car 
$$
\forall x>0, \qquad g_p'(x)+\Q(1-{1\F x}\W)g_p(x)=\e^{-x}-x\e^{-x}+\Q(1-{1\F x}\W)x\e^{-x}=0. 
$$
D'apr\`es le cours, $f$ est solutions de l'\'equation $(E)$ sur $]0,+\infty[$ si et seulement s'il existe $\lambda\in\ob R$ tel que 
$$
\forall x>0, \qquad f(x)=x+\lambda x\e^{-x}. 
$$
9b. Supposons qu'il existe une solution $f$ de $E$ sur $\ob R$. Comme $f$ est solution de $(E)$ sur $]0,+\infty[$, il existe une constante $\lambda\in\ob R$  telle que 
$$
\forall x>0, \qquad f(x)=x+\lambda x\e^{-x}. 
$$ 
Comme $f$ est solution de $(E)$ sur $]-\infty,0[$, il existe une constante $\beta\in\ob R$  telle que 
$$
\forall x<0, \qquad f(x)=x+2+{2\F x}+\beta{\e^x\F x}. 
$$ 
Comme $f $est de classe $\sc C^1$ sur $\ob R$ et donc en $0$, on doit forc\'ement avoir $f(0^-)=f(0^+)$ et $f'(0^-)=f'(0^+)$. Plus subtilement, $f$ doit admettre un d\'eveloppement limit\'e  $f(x)=f(0)+f'(0)x+o_0(x)$ \`a l'ordre $1$. Deux petits calculs de d\'eveloppement limit\'e donnent  
$$
\eqalign{
f(x)&=(1+\lambda)x+o_{0^+}(x)\cr
f(x)&=x+2+{2+\beta\big(1+x+x^2/2+o_0(x)\big)\F x}={2+\beta\F x}+2+\beta+\Q(1+{\beta\F 2}\W)x+o_{0^-}(x)
}
$$
Comme ces deux d\'eveloppements limit\'es doivent \^etre \'egaux \`a $f(0)+f'(0)x+o_0(x)$, nous remarquons d'abord que $\beta=-2$ (sinon la limite $f(0^-)$ n'existe pas) et ensuite que $\lambda=-1$ (sinon $f'(0^-)=0\neq f'(0^+)$). A fortiori, la fonction $f$ est n\'ecessairement l'application d\'efinie par
$$
f(x)=\Q\{\eqalign{
&x-x\e^{-x}\qquad \hbox{ si }x>0\cr
&0\qquad \hbox{ si }x=0 \hbox{ (valeur obtenue par continuit\'e)}\cr
& x+2+{2\F x}-{2\e^x\F x}\qquad \hbox{ si }x<0
}\W.\leqno{(*)}
$$ 
R\'eciproquement, cette fonction est trivialement solution de $(E)$ sur $\ob R^*$. Comme elle satisfait le d\'evelopement limit\'e $f(x)=o(x)$, elle est continue et d\'erivable en $0$. De plus, d'apr\`es le th\'eor\`eme de prolongement $\sc C^1$, $f$ est de classe $\sc C^1$ sur $\ob R$ et, comme $f(0)=0$, l'\'equation $(E)$ est v\'erifi\'ee en $x=0$ et donc sur $\ob R$.\pn
En conclusion, il existe une unique solution de $E$ sur $\ob R$, c'est la fonction $f$ d\'efinie par $(*)$. 





\sol PTano. 
\noindent
1. Les valeurs propres de $A$ sont $1$, $0$ et $-4$ et leurs espaces propres respectivement associ\'es sont 
$$
E_1=\hbox{Vect}\Q(\pmatrix{1\cr2\cr1}\W), \qquad  E_0=\hbox{Vect}\Q(\pmatrix{1\cr-1\cr0}\W)
\quad\hbox{et}\quad E_{-4}=\hbox{Vect}\Q(\pmatrix{1\cr1\cr-1}\W).  
$$
En effet, nous remarquons que 
$$
\eqalign{
&A\pmatrix{1\cr2\cr1}=\pmatrix{5+2*5-14\cr 6+2*6-16\cr5+2*5-14}=1.\pmatrix{1\cr2\cr1}.
\cr
&A\pmatrix{1\cr-1\cr0}=\pmatrix{5-5\cr5-6+0\cr5-5+0}=0.\pmatrix{1\cr-1\cr0}, 
\cr
&A\pmatrix{1\cr1\cr-1}=\pmatrix{5+5-14\cr 6+6-16\cr5+5-14}=-4\pmatrix{1\cr1\cr1},
}
$$
En particulier, comme la matrice $A$ admet trois valeurs propres distinctes deux \`a deux, elle est diagonalisable.  
\medskip\noindent
2. Nous remarquons que les matrices des vecteurs $\vec I$, $\vec J$ et $\vec K$ dans la base canonique $\sc B$ sont les vecteurs propres pr\'ec\'edemment trouv\'es
$$
\sc M\hbox{at}_{\sc B}(\vec I)=\pmatrix{1\cr2\cr1}, \qquad \sc M\hbox{at}_{\sc B}(\vec J)=\pmatrix{1\cr-1\cr0}\quad\hbox{et}\quad\sc M\hbox{at}_{\sc B}(\vec K)=\pmatrix{1\cr1\cr1}.  
$$
Comme ces vecteurs propres sont associ\'es \`a des valeurs propres distinctes deux \`a deux, ils forment une famille 
libre. A fortiori, les trois vecteurs $\vec I$, $\vec J$ et $\vec K$ forment une famille libre de trois \'el\'ements de $\ob R^3$, et par cons\'equent une base $\sc C$ de $\ob R^3$. 
\medskip
\noindent
3. La matrice de passage de la base canonique $\sc B$ \`a la base $\sc C:=\{\vec I,\vec J,\vec K\}$ est la matrice
$$
P:=\sc M\hbox{at}_{\sc C,\sc B}(\hbox{Id}_{\ob R^3})=\pmatrix{1&1&1\cr2&-1&1\cr1&0&1}.
$$
Nous remarquons que 
$$
P\times\pmatrix{1&1&-2\cr1&0&-1\cr-1&-1&3}=\pmatrix{1+1-1&1+0-1&-2-1+3\cr2-1-1&2+0-1&1+0-1\cr-1-2+3&-1+1+0&-1-1+3}=\hbox{I}_3.
$$
A fortiori, la matrice $P$ est inversible et son inverse est la matrice 
$$
P^{-1}=\pmatrix{1&1&-2\cr1&0&-1\cr-1&-1&3}.
$$
4. Comme les colonnes de la matrice $P$ sont des vecteurs propres de la matrice $A$ pour les valeurs propres respectives $1$, $0$ et $-4$, il r\'esulte du cours de diagonalisation que 
$$
A=P\pmatrix{1&0&0\cr0&0&0\cr0&0&-4}P^{-1}.
$$
En particulier, nous remarquons que $D$ est la matrice diagonale 
$$
D=P^{-1}AP=\pmatrix{1&0&0\cr0&0&0\cr0&0&-4}.
$$
5a. En multipliant \`a gauche par $P$ et \`a droite par $P^{-1}$, nous d\'eduisons des relations $N=P^{-1}MP$ et $D=P^{-1}AP$ que $PNP^{-1}=M$ et $PDP^{-1}=A$. A fortiori, nous remarquons que 
$$
\eqalign{
AM=MA&\ssi (PDP^{-1})(PNP^{-1})=(PNP^{-1})(PDP^{-1})
\cr
&\ssi PDNP^{-1}=PNDP^{-1}
\cr
&\ssi DN=ND\qquad \hbox{en multipliant \`a gauche par $P^{-1}$ et \`a droite par $P$}
}
$$
\bigskip\noindent
5b. Soit $N:=(a_{i,j})_{1\le i\le 3\atop1\le j\le 3}$ une matrice r\'eelle d'ordre $3$. Alors, on a 
$$
DN=ND\ssi\pmatrix{a_{1,1}&a_{1,2}&a_{1,3}\cr 0&0&0\cr -4a_{3,1}&-4a_{3,2}&-4a_{3,3}}=\pmatrix{a_{1,1}&0&-4a_{1,3}\cr a_{2,1}&0&-4a_{2,3}\cr a_{3,1}&0&-4a_{3,3}}
\ssi\smash{\Q\{\eqalign{
a_{1,2}&=0\cr
a_{1,3}&=0\cr
a_{2,1}&=0\cr
a_{2,3}&=0\cr
a_{3,1}&=0\cr
a_{3,2}&=0\cr
a_{1,1}&\in\ob R\cr
a_{2,2}&\in\ob R\cr
a_{3,3}&\in\ob R.
}\W.}
$$ 
En particulier, les matrices $N$ v\'erifiant $ND=DN$ sont les matrices 
$$
N=\pmatrix{a&0&0\cr0&b&0\cr0&0&c}\qquad (a,b,c)\in\ob R^3. 
$$
Comme $AM=MA\ssi DN=ND$ avec $M=PNP^{-1}$, nous d\'eduisons du r\'esultat de la question pr\'ec\'edente que les matrices $M$ r\'eelles d'ordre $3$ v\'erifiant $AM=MA$ sont les matrices 
$$
M=P\pmatrix{a&0&0\cr0&b&0\cr0&0&c}P^{-1}\qquad (a,b,c)\in\ob R^3,  
$$
c'est-\`a-dire les matrices
$$
M=\pmatrix{a&b&c\cr2a&-b&c\cr a&0&c}P^{-1}=\pmatrix{a+b-c&a-c&-2a-b+3c\cr2a-b-c&2a-c&-2a+b+3c\cr a-c&a-c&-2a-3c}\qquad (a,b,c)\in\ob R^3.  
$$
6. Prouvons par l'absurde qu'il n'existe pas de matrices $Q$ r\'eelle d'ordre $3$ v\'erifiant $Q^2=A$. \pn
Supposons qu'il existe une telle matrice $Q$. Alors, $QA=Q*Q^2=Q^3=Q^2*Q=AQ$. En particulier, la matrice $Q$ est une matrice r\'eelle d'ordre $3$ v\'erifiant la relation $AM=MA$. Il r\'esulte alors du r\'esultat de la question 5b qu'il existe des nombres r\'eels $a$, $b$ et $c$ tels que 
$$
Q=P\pmatrix{a&0&0\cr0&b&0\cr0&0&c}P^{-1}. 
$$
Mais alors, nous remarquons que 
$$
A=Q^2=P\pmatrix{a^2&0&0\cr0&b^2&0\cr0&0&c^2}P^{-1}
$$
et par cons\'equent que la matrice $A$, qui est semblable avec la matrice $\pmatrix{a^2&0&0\cr0&b^2&0\cr0&0&c^2}$, 
admet les nombres r\'eels positifs ou nuls $a^2$, $b^2$ et $c^2$ comme valeurs propres, ce qui est impossible car $-4$ est valeur propre de $A$. Notre supposition de d\'epart est donc absurde. 
A fortiori, nous avons prouv\'e qu'il n'existe pas de matrice  $Q$ r\'eelle d'ordre $3$ v\'erifiant $Q^2=A$.
\medskip\noindent
7a. Un calcul bourrin donne 
$$
\eqalign{
D_1=P^{-1}BP&=\pmatrix{8+0-8&4+4-8&-16-8+24\cr8+0-4&4+0-4&-16+12\cr-8+12&-4-4+12&16+8-36}P\cr
&=
\pmatrix{0&0&0\cr4&0&-4\cr 4& 4&-12}P=\pmatrix{0&0&0\cr0&4&0\cr0&0&-4}
}
$$ 
7b. Un calcul bourrin donne 
$$
\eqalign{
Y_0&=P^{-1}X_0=\pmatrix{1&1&-2\cr1&0&-1\cr-1&-1&3}\pmatrix{1\cr0\cr1}=\pmatrix{-1\cr0\cr2}, 
\cr
Y_1&=P^{-1}X_1=\pmatrix{1&1&-2\cr1&0&-1\cr-1&-1&3}\pmatrix{0\cr-1\cr1}=\pmatrix{-3\cr-1\cr4}. 
}
$$
7c. Pour $n\ge0$, nous multiplions la relation $X_{n+2}=AX_{n+1}+BX_n$ \`a gauche par $P^{-1}$ et nous d\'eduisons des identit\'es $Y_n=P^{-1}X_n$, $A=PDP^{-1}$ et $B=PD_1P^{-1}$ que 
$$
\eqalign{
\forall n\ge0, \qquad Y_{n+2}&=P^{-1}X_{n+2}=P^{-1}AX_{n+1}+P^{-1}BX_n=DP^{-1}X_{n+1}+D_1P^{-1}X_n
\cr
&=DY_{n+1}+D_1Y_n. }
$$
7d. Comme $Y_n=\pmatrix{u_n\cr v_n\cr w_n}$ pour $n\ge0$, il r\'esulte de la relation pr\'ec\'edente que 
$$
\pmatrix{u_{n+2}\cr v_{n+2}\cr w_{n+2}}=\pmatrix{1&0&0\cr0&0&0\cr0&0&-4}\pmatrix{u_{n+1}\cr v_{n+1}\cr w_{n+1}}+\pmatrix{0&0&0\cr0&4&0\cr0&0&-4}\pmatrix{u_n\cr v_n\cr w_n}=\pmatrix{u_{n+1}\cr4v_n\cr-4w_{n+1-4w_n}}.
$$
En particulier, nous obtenons le syst\`eme 
$$
\Q\{
\eqalign{
u_{n+2}&=u_{n+1}\cr
v_{n+2}&=4v_n\cr
w_{n+2}&=-4w_{n+1}-4w_n}
\W.\qquad (n\ge0).
$$
Nous remarquons alors que la suite $(u_n)_{n\ge0}$ est constante \`a partir du rang $1$, i.e. que 
$$
u_0=-1\quad \hbox{et}\quad u_n=u_1=-3 \qquad (n\ge1).
$$
Nous remarquons que les suites $(v_{2n})_{n\ge0}$ et $(v_{2n+1})_{n\ge0}$ des termes de rang pairs et impairs de la suite $v$ sont g\'eom\'etriques de raison $4$ et de terme initial $v_0$ et $v_1$. A fortiori, nous avons
$$
v_{2n}=4^nv_0=0\quad \hbox{et}\quad v_{2n+1}=4^nv_1=-4^n\qquad (n\ge0). 
$$
Enfin, la suite $w$ satisfait la r\'ecurence lin\'eaire $w_{n+2}=-4w_{n+1}-4w_n$, de polyn\^ome caract\'eristique $P=X^2+4X+4=(X+2)^2$ admettant $-2$ comme racine de multiplicit\'e $2$. A fortiori, il existe un unique couple de nombres r\'eels $(\lambda,\mu)$ tel que 
$$
w_n=\lambda (-2)^2+\mu n(-2)^n\qquad (n\ge0).
$$
Nous d\'eduisons alors des conditions initiales $w_0=2$ et $w_1=4$ que $\lambda=2$ et $\mu=-4$, i.e. que 
$$
w_n=2(-2)^2-4n(-2)^n=(2-4n)(-2)^n\qquad (n\ge0).
$$
e. Comme $X_n=PY_n$, nous d\'eduisons du r\'esultat de la question pr\'ec\'edente que 
$$
X_n=\pmatrix{1&1&1\cr2&-1&1\cr1&0&1}\pmatrix{u_n\cr v_n\cr w_n}=\pmatrix{u_n+v_n+w_n\cr2u_n-v_n+w_n\cr u_n+w_n}
$$
Pour me simplifier un peu la vie, je d\'eduis des formules $v_{2n}=0$ et $v_{2n+1}=-4^n$ que 
$$
v_n={(-1)^n-1\F 4}2^n\qquad (n\ge0)
$$
et j'obtiens alors que 
$$
\forall n\ge2, \quad X_n=\pmatrix{
-3+{(-1)^n-1\F 4}2^n+(2-4n)(-2)^n
\cr
-6-{(-1)^n-1\F 4}2^n+(2-4n)(-2)^n
\cr
-3+(2-4n)(-2)^n
}
=\pmatrix{
-3-2^{n-2}+9(-2)^{n-2}-4n(-2)^n
\cr
-6+2^{n-2}+7(-2)^{n-2}-4n(-2)^n
\cr
-3+(2-4n)(-2)^n
}
$$
8a. Comme $\{\vec I,\vec J,\vec K\}$ est une base de $\ob R^3$, pour chaque $t\in\ob R$, il existe un unique triplet de nombres r\'eels $\big(x(t),y(t),z(t)\big)$  tel que 
$$
\big(u(t),v(t),w(t)\big)=x(t)\vec I+y(t)\vec J+y(t)\vec K.
$$
8b. En \'ecrivant la matrice du vecteur $U(t):=\big(u(t),v(t),w(t)\big)$ dans la base canonique $\sc B$ nous obtenons alors que 
$$
U(t)=\pmatrix{u(t)\cr v(t)\cr w(t)}=x(t)\pmatrix{1\cr2\cr1}+y(t)\pmatrix{1\cr-1\cr0}+z(t)\pmatrix{1\cr1\cr1}.  
$$
Nous posons $X(t):=\pmatrix{x(t)\cr y(t)\cr z(t)}$ et nous en d\'eduisons le syst\`eme matriciel 
$$
U(t)=\pmatrix{1&1&1\cr2&-1&1\cr1&0&1}\pmatrix{x(t)\cr y(t)\cr z(t)}=PX(t)\qquad (t\in\ob R).
$$
En inversant ce syst\`eme (la matrice $P$ \'etant inversible), nous obtenons alors que 
$$
\pmatrix{x(t)\cr y(t)\cr z(t)}=X(t)=P^{-1}U(t)=\pmatrix{u(t)+v(t)-2w(t)\cr u(t)-w(t)\cr-u(t)-v(t)+3w(t)}\qquad (t\in\ob R).
$$
Les fonctions $x$, $y$ et $z$ sont d\'erivables sur $\ob R$ car elles sont combinaisons lin\'eaires des fonctions d\'erivables $u$, $v$ et $w$. 
\medskip\noindent
c. En d\'erivant la relation $U(t)=PX(t)$, nous obtenons que $U'(t)=P'X(t)+PX'(t)=0X(t)+PX'(t)=PX'(t)$ pour $t\in\ob R$. A fortiori, nous d\'eduisons de la relation $B=PD_1P^{-1}$ que 
$$
\eqalign{
(1)&\ssi U'(t)=BU(t)\qquad (t\in\ob R),
\cr
&\ssi PX'(t)=PD_1P^{-1}U(t)\qquad (\in\ob R),
\cr
&\ssi X'(t)=D_1P^{-1}U(t)\qquad (t\in\ob R),
\cr
&\ssi X'(t)=D_1X(t)\qquad (t\in\ob R)\ssi (2).
}
$$
d. En reportant $u(0)=1$ et $v(0)=w(0)=0$ pour $t=0$ dans la relation $X(t)=P^{-1}U(t)$, nous obtenons que 
$$
\pmatrix{x(0)\cr y(0)\cr z(0)}=X(0)=P^{-1}U(0)=\pmatrix{1&1&-2\cr1&0&-1\cr-1&-1&3}\pmatrix{1\cr 0\cr 0}=\pmatrix{1\cr1\cr -1}.
$$
A fortiori, nous avons $x(0)=1$, $y(0)=1$ et $z(0)=-1$. 
\medskip\noindent
e. Le syst\`eme $(2)$ avec les conditions initiales $x(0)=1$, $y(0)=1$ et $z(0)=-1$ est un probl\`eme de Cauchy dont l'unique solution est 
$$
\Q\{
\eqalign{
x(t)&=1\cr
y(t)&=\e^{4t}\cr
z(t)&=-\e^{-4t}
}
\W.
\qquad (t\in\ob R).
$$
f. Comme $U(t)=PX(t)$, nous en d\'eduisons que le syst\`eme $(1)$ avec les conditions initiales $u(0)=1$ et $v(0)=w(0)=0$ est un probl\`eme de Cauchy dont l'unique solution est 
$$
\Q\{
\eqalign{
u(t)&=1+\e^{4t}-\e^{4t}\cr
v(t)&=2-\e^{4t}+\e^{-4t}\cr
w(t)&=1+\e^{-4t}
}
\W.
\qquad (t\in\ob R).
$$

\sol PTanp.
\noindent
1. Nous notons $h_0$ la fonction constante d\'efinie par $h_\theta(t):=1$ pour $0\le t\le 1$. Lorsque $\theta>0$, nous notons $h_\theta$ la fonction d\'efinie sur $[0,1]$ par 
$$
h_\theta(0):=0\quad\hbox{et}\quad h_\theta(t):=t^\theta=\e^{\theta\ln(t)}\qquad (0<t\le1)
$$
Comme $h_\theta(t)=\e^{\theta \ln(t)}$ pour $t\in]0,1]$, cette fonction est de classe $\sc C^\infty$ sur $]0,1]$ en tant que compos\'ee de l'exponentielle et du logarithme. Par ailleurs, nous remarquons que 
$$
\lim_{t\to0^+} h_\theta(t)=\lim_{t\to0^+}\e^{\alpha \ln(t)}=0=h_\theta(0). 
$$
et nous en d\'eduisons que la fonction $h_\theta$ est continue en $0$ et donc sur $[0,1]$. Comme il en est de m\^eme pour la fonction constante $h_0$, la fonction $h_\theta$ est continue sur $[0,1]$ quel que soit $\theta\ge0$. \medskip\noindent
Lorsque $\alpha\ge0$ et $\beta\ge0$, nous remarquons que 
$$
g_{\alpha,\beta}(t)=h_\alpha(t)h_\beta(1-t)\qquad (0<t<1). 
$$
A fortiori, la fonction $g_{\alpha,\beta}$ peut \^etre prolong\'ee sur $[0,1]$ par la fonction $g_{\alpha,\beta}:t\mapsto h_\alpha(t)h_\beta(1-t)$ qui est continue sur l'intervalle $[0,1]$, en tant que produit de la fonction continue $h_\alpha$ avec la compos\'ee des fonctions continues $h_\beta$ et $t\mapsto 1-t$, \`a valeurs dans $[0,1]$. 
Nous remarquons alors que 
$$\eqalign{&
g_{\alpha,\beta}(0)=h_\alpha(0)h_\beta(1-0)=h_\alpha(0)*1=\Q\{
\eqalign{
0\hbox{ si }\alpha>0
\cr
1\hbox{ si }\alpha=0
}
\W.
\cr
&
g_{\alpha,\beta}(1)=h_\alpha(1)h_\beta(1-1)=1*h_\beta(0)=\Q\{
\eqalign{
0\hbox{ si }\beta>0
\cr
1\hbox{ si }\beta=0
}
\W.
}
$$
\medskip\noindent
b. Comme la fonction $g_{\alpha,\beta}$ est continue sur $[0,1]$, nous pouvons l'int\'egrer que cet intervalle. De sorte que les int\'egrales $I(\alpha,\beta)$ sont d\'efinies. Par ailleurs, nous remarquons que 
$$
I(\alpha,0)=\int_0^1t^\alpha\d t=
\Q[{t^\alpha+1\F\alpha+1}\W]_0^1={1\alpha+1}\qquad (\alpha\ge0). 
$$
c. En proc\'edant au changement de variable $u=1-t$ {\it (c'est un diff\'eomorphisme de classe $\sc C^1$ de $[0,1]$  dans $[0,1]$ car l'application $\phi:t\mapsto u=1-t$ est une bijection de $I=[0,1]$ dans $J=[0,1]$, de classe $\sc C^1$ sur $I$ et sa bijection r\'eciproque $\psi:u\mapsto t=1-u$  est de classe $\sc C^1$ sur $J$)}, nous obtenons alors que 
$$
I(\alpha,\beta)=\int_0^1t^\alpha\d t=\int_1^0(1-u)^\alpha\big(1-(1-u)\big)^\beta (-\d u)=\int_0^1u^\alpha(1-u)^\beta=I(\beta,\alpha).
$$
d. Lorsque $\theta\ge1$, nous avons montr\'e \`a la question 1 que la fonction $h_\theta$ est de classe $\sc C^1$ sur $]0,1]$ et qu'elle est continue sur $[0,1]$.  Comme nous avons de plus
$$
\lim_{t\to0^+} h_\theta'(t)=\lim_{t\to0^+}\theta t^{\theta-1}=\lim_{t\to0^+}\theta h_{\theta-1}(t)=h_{\theta-1}(0), 
$$
nous remarquons que la limite $\lim_{t\to0^+} h_\theta'(t)$ existe. A fortiori, il r\'esulte du th\'eor\`eme de prolongement $\sc C^1$ que la fonction $h_{\theta}$ est de classe $\sc C^1$ sur $[0,1]$. \medskip\noindent
En int\'egrant par partie les fonctions $h_{\alpha+1}:t\mapsto t^{\alpha+1}$ et $t\mapsto {(1-t)^{\beta+1}\F \beta+1}={1\F \beta+1}h_{\beta+1}(1-t)$, qui sont de classe $\sc C^1$ sur $[0,1]$ d'apr\`es le raisonnement pr\'ec\'edent, nous obtenons alors que 
$$
I(\alpha+1,\beta)=\int_0^1t^{\alpha+1}(1-t)^\beta\d t=\Q[t^{\alpha+1}{t^{\beta+1}\F \beta+1}\W]_0^1-\int_0^1(\alpha+1)t^\alpha{t^{\beta+1}\F \beta+1}\d t={\alpha+1\F\beta+1}I(\alpha,\beta+1).
$$
e. Pour $n\ge0$, prouvons par r\'ecurence la proposition
$$
\forall \alpha\ge0, \qquad I(\alpha,n)=n!\prod_{k=0}^n{1\F\alpha+k+1}.\leqno{(\sc P_n)}
$$
La proposition $\sc P_0$ est vraie car $I(\alpha,0)={1\F\alpha+1}=0!\prod_{k=0}^0{1\F\alpha+k+1}$ pour $\alpha\ge0$, d'apr\`es le r\'esultat de la question 1b. \pn
Soit $n\ge0$ tel que $\sc P_n$ soit vraie, prouvons $\sc P_{n+1}$. Nous fixons $\alpha\ge0$ et nous d\'eduisons de la relation trouv\'ee en 1c que  
$$
I(\alpha,n+1)={n+1\F \alpha+1}I(\alpha+1,n). 
$$
D'apr\`es la proposition $\sc P_n$ appliqu\'ee pour $\tilde\alpha=\alpha+1$, il suit 
$$
I(\alpha,n+1)={n+1\F \alpha+1}n!\prod_{k=0}^n{1\F(\alpha+1)+k+1}=(n+1)!_{k=0}^{n+1}{1\F\alpha+k+1}.
$$
En particulier, nous avons montr\'e que 
$$
\forall\alpha\ge0, I(\alpha,n+1)=(n+1)!_{k=0}^{n+1}{1\F\alpha+k+1}, 
$$
autrement dit que $\sc P_{n+1}$ est vraie. 
En conclusion, la proposition $\sc P_n$ est vraie pour $n\ge0$. 
2a. Soit $a>0$. Par composition avec le logarithme, la fonction $f_a$ est d\'efinie en tout nombre r\'eel $x$ pour lequel $x\neq0$ (on ne doit pas diviser par $0$) et $1-{a\F x}>0$, c'est \`a dire sur l'ensemble
$$
\sc Df_a=]-\infty,0[\bigcup]a,\infty[.
$$
2b. Comme je pr\'ef\`ere utiliser les int\'egrales plut\^ot que l'in\'egalit\'e des accroissements finis, je r\'edige cela de la fa\c con suivante :  \pn
Lorsque $0<a<x$, nous avons
$$
\ln(x)-\ln(x-a)=\Q[\ln(u)\W]^x_{x-a}=\int_{x-a}^x{\d u\F u}.
$$
En int\'egrant sur l'intervalle $[x-a,x]$ l'in\'egalit\'e ${1\F x}\le {1\F u}\le{1\F x-a}$ valable pour $0<x-a\le u\le x$, nous obtenons alors que 
$$
{a\F x}=\int_{x-a}^x{\d u\F x}\le \ln(x)-\ln(x-a)\le \int_{x-a}^x{\d u\F x-a}={a\F x-a}.
$$ 
c. En d\'erivant la fonction $f_a$, nous obtenons que 
$$
\eqalign{
\forall x>a, \qquad f_a'(x)&=\big(x\ln(x-a)-x\ln(x)\big)'=\ln(x-a)+{x\F x-a}-1-\ln(x)\cr
&={x\F x-a}-1-\big(\ln(x)-\ln(x-a)\big) }
$$
Il r\'esulte alors de l'in\'egalit\'e pr\'ec\'edente que 
$$
\forall x>a, \qquad   f_a'(x)\ge{x\F x-a}-1-{a\F x-a}=0.
$$
En particulier, l'application $f_a'$ est croissante sur l'intervalle $]a,+\infty[$. \pn
Comme $f_a(x)=x\ln(a-x)-x\ln(x)\sim_aa\ln(a-x)$, nous remarquons que $\lim_{x\to a^+}f_a(x)=-\infty$ et nous en d\'eduisons que la courbe $\sc C_a$ pr\'esente en $a^+$ une asymptote verticale d'\'equation $x=a$ (la courbe reste \`a droite de l'asymptote). 
Comme 
$$
f_a(x)=x\ln\Q(1-{a\F x}\W)=-a-{a^2\F x}+o_\infty\Q({1\F x}\W),
$$ 
la courbe $\sc C_a$ pr\'esente en $+\infty$ une asymptote horizontale d'\'equation $y=-a$ et elle reste en dessous de cette courbe au voisinage de $+\infty$. 
$$
\tikzpicture[smooth,variable=\x,scale=0.5,baseline=(current bounding box.north)]
	\draw[very thin,color=black!20,step=0.5] (-1.5,-8) grid (11,1.5);
	\draw[->] (-1,0) -- (10.5,0) node[above] {\eightpts$x$};
	\draw[->] (0,-7.5) -- (0,1.1) node[left] {\eightpts$y$};
	\draw[very thin line,color=black] (5,-1)--(11,-1);
	\draw[very thin line,color=black] (1,-8)--(1,-2);
	\draw[color=red,samples=437,domain=1.0005:11] plot ({\x},{\x*ln(1-1/\x)}) node[right] {$\sc C_1$};
	\draw[very thin line,color=black] (7,-2)--(11,-2);
	\draw[very thin line,color=black] (2,-8)--(2,-5);
	\draw[color=blue,samples=100,domain=2.05:11] plot ({\x},{\x*ln(1-2/\x)}) node[right] {$\sc C_2$};
	\draw[very thin line,color=black] (9,-3)--(11,-3);
	\draw[very thin line,color=black] (3,-8)--(3,-7);
	\draw[color=black,samples=100,domain=3.3:11] plot ({\x},{\x*ln(1-3/\x)}) node[right] {$\sc C_3$};
	\node [anchor=north] at (current bounding box.south) {Courbes $C_1$, $C_2$ et $C_3$};
\endtikzpicture
$$
e. Soit $n>a$ un nombre entier. Comme la fonction $f_a$ est croissante sur l'intervalle $]a,+\infty[$, nous obtenons que $f_a(n+1)\ge f_a(n)$ 
et par composition avec la fonction exponentielle, 
qui est croissante, que $y_{n+1}\e^{f_a(n+1)}\ge\e^{f_a(n)}=y_n$. A fortiori, la suite $(y_n)_{n>a}$ est croissante. Comme $y_n=\e^{f_a(n)}$ et comme la fonction 
$f_a(x)$ converge vers $-a$ en $+\infty$, il resulte de la composition que 
$$
\lim y_n=\e^{\lim_{n\to\infty}f_a(n)}=\e^{-a}.
$$ 
Soit $x\ge0$ et $n\ge1$. Comme l'application $u\mapsto (1-u/n)^nu^x$ est continue sur $[0,n]$, l'int\'egrale $F_n(x)$ est bien d\'efinie. 
En proc\'edant au changement de variable $t=u/n$, qui est un diff\'eomorphisme de classe $\sc C^1$ (application lin\'eaire inversible), nous obtenons que 
$$
F_n(x)=\int_0^n\Q(1-{u\F n}\W)^nu^x\d u=\int_0^1(1-t)^nn^xt^xn\d t=n^{x+1}I(x,n).
$$
b. Soit $x\ge0$. Alors, nous d\'eduisons de la croissance de l'int\'egrale que  
$$
\eqalign{
F_{n+1}(x)-F_n(x)&=\int_0^{n+1}\Q(1-{u\F n+1}\W)^{n+1}u^x\d u-\int_0^n\Q(1-{u\F n}\W)^nu^x\d u
\cr
&\ge\int_0^n\Q(1-{u\F n+1}\W)^{n+1}u^x\d u-\int_0^n\Q(1-{u\F n}\W)^nu^x\d u
\cr
&\ge\int_0^n\underbrace{\Q(\Q(1-{u\F n+1}\W)^{n+1}-\Q(1-{u\F n}\W)^n\W)}_{>0\smash{\hbox{ car la suite $y_n$ est croissante pour $a=u>0$}}}u^x\d u\ge0
}
$$
A fortiori, la suite $(F_n(x))_{n\ge1}$ est croissante pour chaque $x>0$ (pour $x=0$, c'est trivial). 
\medskip\noindent
i. Soit $x\ge0$ et $u>0$. Alors, nous remarquons que 
$$
u^{x+2}\e^{-u}=\e^{(x+2)\ln u-u}
$$
Lorsque $x$ est fix\'e, nous avons $\lim_{u\to+\infty}(x+2)\ln u-u=-\infty$. \pn
A fortiori, $\lim_{u\to+\infty}\Q(u^{x+2}\e^{-u}\W)=-\infty$ et donc  il existe un nombre r\'eel $A_x>0$ ($A_x$ d\'epend de $x$ mais ce n'est pas grave) tel que 
$$
\forall u\ge A_x, \qquad (x+2)\ln u-u\le0. 
$$
En reportant dans l'exponentielle, nous obtenons alors que 
$$
\forall u\ge A_x, \qquad u^{x+2}\e^{-u}=\e^{(x+2)\ln u-u}\le \e^0\le 1.
$$
De sorte que $\e^{-u}\le {1\F u^{x+2}}$ pour $u\ge A_x$. \medskip\noindent
ii. Fixons le nombre r\'eel $x>0$.Comme il a \'et\'e d\'emontr\'e \`a la question 2e que la suite $y_n$ est croissante de limite $\e^{-a}$, nous remarquons que 
$$
\Q(1-{u\F n}\W)^n\le\e^{-u}\qquad (0<u\le n)\leqno{(*)}
$$
et nous en d\'eduisons que 
$$
F_n(x)=\int_0^n\Q(1-{u\F n}\W)^nu^x\d u\le =\int_0^n\e^{-u}u^x\d u 
$$
En  utilisant Chasles pour $n\ge A_x$, nous obtenons alors que 
$$
F_n(x)\le\int_0^{A_x}\e^{-u}u^x\d u+\int_{A_x}^n\e^{-u}u^x\d u 
$$
Il r\'esulte alors de l'in\'egalit\'e $(*)$ que 
$$
\eqalign{
F_n(x)&\le\int_0^{A_x}\e^{-u}u^x\d u+\int_{A_x}^n{u^x\F u^{x+2}}\d u 
\cr
&\le\int_0^{A_x}\e^{-u}u^x\d u+\Q[-{1\F u}\W]_{A_x}^n
\cr
&\le\int_0^{A_x}\e^{-u}u^x\d u+{1\F A_x}
}
$$
Comme cette in\'egalit\'e est \'egalement v\'erifi\'ee lorsque $0\le n\le A_x$, nous avons bien d\'emontr\'e que 
$$
F_n(x)\le\int_0^{A_x}\e^{-u}u^x\d u+{1\F A_x}\qquad (n\ge1).
$$
iii. Lorsque l'on fixe $x\ge0$, il r\'esulte de l'in\'egalit\'e pr\'ec\'edente que la suite $(F_n(x))_{n\ge1}$ est major\'ee (le membre de droite de l'in\'egalit\'e \'etant un nombre ne d\'ependant que de $x$ qui a \'et\'e fix\'e et non pas de $n$). Or, la suite $(F_n(x))_{n\ge1}$ est croissante d'apr\`es ce qui a \'et\'e prouv\'e en 3a. A fortiori, pour chaque nombre r\'eel $x$, la suite $(F_n(x))_{n\ge1}$ converge vers une limite, que nous notons  $F(x)$. 
\medskip\noindent
D'apr\`es les relations trouv\'ees en 1d et en 3a, nous avons
$$
\eqalign{
F_n(x+1)&=n^{x+2}I(x+1,n)=n^{x+2}{x+1\F n+1}I(x,n+1)=(x+1){n\F n+1}n^{x+1}I(x,n+1),\cr
&=(x+1){n\F n+1}{n^{x+1}\F (b+1)^x}F_{n+1}(x).}
$$
Comme ${n\F n+1}{n^{x+1}\F (b+1)^x}$ converge vers $1$ lorsque $n$ tends $+\infty$, en faisant tendre $n$ vers $+\infty$ dans cette relation, nous obtenons alors que
$$
F(x+1)=(x+1)F(x)\qquad (x\ge0).
$$

\sol PTaql. 
\noindent 1. Pour $x\in\ob R$, les applications $t\mapsto{\sin^2(tx)\F t^2}$ et $t\mapsto{\sin^2(tx)\F t^2(1+t^2)}$ sont continues sur $]0, +\infty[$ en tant que quotient de fonctions continues dont le dénominateur ne s'annule pas sur cet intervalle. De plus, nous remarquons d'une part que 
$$
{\sin^2(tx)\F t^2(1+t^2)} = x^2 +o_0(1) = {\sin^2(tx)\F t^2}, 
$$
c'est-à-dire que ces deux fonctions sont prolongeables par continuité en $0$, et d'autre part que 
$$
\Q|{\sin^2(tx)\F t^2(1+t^2)}\W| \le \Q|{\sin^2(tx)\F t^2}\W| \le {1\F t^2} \qquad (t\ge1).
$$
Comme l'intégrale de Riemann $\int_1^\infty{\d t\F t^2}$ converge ($\alpha = 2>1$), nous concluons que $A(x)$ et $F(x)$ sont deux intégrales convergeantes. A fortiori, les fonctions $A$ et $F$ sont définies sur $\ob R$.  
\medskip\noindent
2. Lorsque $x\neq 0$, nous posons $\alpha := x/|x| \in\{-1, 1\}$ et nous procédons au changement de variable (difféomorphisme de classe $\sc C^1$) $t=u/x$ pour obtenir que 
$$
A(x):=\int_0^\infty{\sin^2(tx)\F t^2}\d t=\int_0^{\alpha\infty}{x^2\sin^2(u)\F u^2}{\d u\F x} = x\int_0^{\alpha\infty}{\sin^2(t)\F t^2}\d t.
$$
Comme l'application $t\mapsto {\sin^2(t)\F t^2}$ est paire, nous obtenons finalement que 
$$
A(x)= \alpha x \int_0^\infty{\sin^2(t)\F t^2}\d t = |x| A(1).
$$
Comme cette égalité est également vraie lorsque $x=0$, nous concluons que
$$
\forall x\in\ob R, \qquad A(x)=|x| A(1).
$$
3. Pour $x\in\ob R$, nous remarquons que 
$$
F(x)-A(x) = -\int_0^\infty{sin^2(tx)\F 1+ t^2}\d t
$$
En majorant en module à l'aide de la relation $|\sin(u)|\le 1\ (u\in\ob R)$, il vient alors
$$
\Q|F(x)-A(x)\W| =  \int_0^\infty {\sin^2(tx)\F 1+ t^2}\d t \le  \int_0^\infty {1\F 1+ t^2}\d t= \Q[\arctan t\W]_0^\infty = {\pi\F2}.  
$$
En particulier, nous concluons que 
$$
\forall x\in \ob R, \qquad F(x)=A(x) + O(1) = |x|A(1) + O(1).
$$
Comme $A(1)>0$ en tant qu'intégrale d'une fonction continue, positive, non identiquement nulle, il vient
$$
F(x) \mathop\sim\limits_{\pm\infty} A(1)|x|.
$$
4. Pour chaque $t>0$, l'application $x\mapsto {\sin^2(tx)\F t^2(1+t^2)}$ est de classe $\sc C^2$ sur $\ob R$ en tant que quotient de fonctions de classe $\sc C^2$ dont le dénominateur ne s'annule jamais. \PAR\noindent
Pour $x\in\ob R$ fixé, l'application $t\mapsto{\sin^2(tx)\F t^2(1+t^2)}$ et ses dérivées partielles  $t\mapsto {\sin(2tx)\F t(1+t^2)}$ et $t\mapsto {2\cos(2tx)\F 1+t^2}$ par rapport à $x$ sont continues sur $]0,+\infty[$, prolongeables par morceaux en $0$, et sont majorées par $2/t^2$ pour $t\ge1$. A fortiori, elles sont continues par morceaux et intégrables sur $]0,+\infty[$. 
\PAR\noindent
Enfin, pour $x\in\ob R$ et $t>0$, nous remarquons que 
$$
\Q|{2\cos(2tx)\F 1+t^2}\W|\le{2\F 1+t^2}
$$
avec $\int_0^\infty {2\F 1+t^2}$ qui est une intégrale convergente. A fortiori, il résulte du théorème de dérivation des intégrales impropres à un paramètre que l'application $F$ est de classe $\sc C^2$ sur $\ob R$ et que l'on peut intervertir dérivation et intégration (2 fois) pour calculer ses dérivées. Ainsi, on a 
$$
\forall x\in \ob R, \qquad F'(x) = \int_0^\infty{\sin(2tx)\F t(1+t^2)}\d t\quad\hbox{et}\quad F''(x)= \int_0^\infty{2\cos(2tx)\F 1+t^2}\d t
$$
5. Pour $x\in\ob R$, nous déduisons des calculs effectués aux questions 3 et 4 que 
$$
\eqalign{
F(x)-A(1)|x|&=F(x)-A(x)=-\int_0^\infty{sin^2(tx)\F 1+ t^2}\d t
={1\F 2}\int_0^\infty{cos(2tx)-1\F 1+ t^2}\d t
\cr
&={F''(x)\F 4}-{1\F2}\int_0^\infty{1\F 1+ t^2}\d t={F''(x)\F 4}-{\pi\F 4}.}
$$
6. En particulier, pour  $a:=-4A(1)$ et $b:=\pi$, l'application $F$ est solution de l'équation différentielle linéaire du second ordre avec second membre
$$
F''(x)-4F(x)=a|x|+b\qquad (x\in\ob R).
$$
7. Les fonctions $t\mapsto \e^{2t}$ et $t\mapsto\e^{-2t}$ sont trivialement solutions de l'équation homogène $y''-4y$. De plus, l'application $t\mapsto A(1)|t| - {\pi\F 4}$ est solution de l''équation différentielle avec second membre sur chacun des intervales $]-\infty, 0[$ et $]0, +\infty[$. Comme l'application $F$ en est également solution sur chacun de ces deux intervales, il existe des constantes $\lambda, \beta, \alpha, \beta$ dans $\ob R$, uniques, telles que 
$$
F(x)=A(1) |x| - {\pi\F 4}+ \cases{
    \alpha\e^{2x} + \beta\e^{-2x}& $si x<0$\cr
    \lambda\e^{2x} + \mu\e^{-2x}& $si x>0$\cr    
}
$$
Comme $F(x)\sim_{\pm\infty} A(1)|x|$, les relations $\beta = \lambda = 0$ sont nécessairement vérifiées de sorte que 
$$
F(x)=A(1) |x| - {\pi\F 4}+ \cases{
    \alpha\e^{2x} & $si x<0$\cr
    \mu\e^{-2x}& $si x>0$\cr    
}
$$
Par ailleurs, l'application $F$ \'etant continue sur $\ob R$ et plus particulièrement en $0$, il résulte de l'identité $F(0^-) =F(0^+)= F(0)=0$ que 
$$
0 = - {\pi\F 4} + \alpha = - {\pi\F 4} + \mu
$$
c'est-à-dire que $\alpha=\mu={\pi\F 4}$ et par conséquent que 
$$
F(x)=A(1) |x| - {\pi\F 4} + {\pi\F 4}\e^{-2|x|} \qquad (x\neq 0). 
$$
8) Comme l'application $F$ est de classe $\sc C^1$ sur $\ob R$ et plus particulièrement en $0$, il résulte alors de l'égalité $F'(0^+)=F'(0)=0$ que 
$$
A(1) - {\pi\F 2} = 0.
$$
En conclusion $\ds \int_0^\infty{\sin^2(t)\F t^2}\d t= A(1) = {\pi\F 2}$.

\endinput